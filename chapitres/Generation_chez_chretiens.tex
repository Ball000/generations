
\chapter{Les chrétiens et la génération}


 Les chrétiens sont apparus avant le milieu du premier siècle de notre ère%
% [1]
\footnote{Sources : Peter \fsc{BROWN}, \emph{Le renoncement à la chair, virginité, célibat et continence dans le christianisme primitif}, 2002. Alexandre \fsc{FAIVRE}, \emph{Naissance d'une hiérarchie, les premières étapes du cursus clérical}, 1977. Collectif, \emph{Aux origines du christianisme}, 2000. A.~\fsc{HAMMAN}, \emph{La vie quotidienne des premiers chrétiens}, p. 95-197, 1971. Aline \fsc{ROUSSELLE}, \emph{La contamination spirituelle, science, droit et religion dans l'antiquité}, 1998.}%
. Au tout début ils ne se distinguaient des autres juifs que par leur jugement sur la personne de Jésus : pour eux il était le Messie, qu'il n'y avait donc plus à attendre. Ils tenaient la bible juive \emph{(l'Ancien Testament)} pour leur référence. Ils ont mis plus d'une génération à réaliser qu'ils n'étaient plus des juifs comme les autres, et les romains aussi. Mais ensuite ils ont fermement refusé d'être confondus avec les premiers, et de sacrifier aux dieux des seconds, ce qui les a entrainés à subir des persécutions%
% [2]
\footnote{Jusqu'en 313 leurs croyances et leurs pratiques n'ont pas été reconnues par Rome et ils ont été l'objet de persécutions plus ou moins épisodiques et rigoureuses. À partir de la fin du \siecle{1}, se reconnaître chrétien et refuser de sacrifier aux dieux civiques était en effet un délit suffisant pour être mis à mort sans autre forme de procès : les chrétiens n'avaient pas obtenu la même reconnaissance que les juifs (que le versement du \emph{fiscus judaïcus} libérait de l'obligation de sacrifier à l'empereur et aux dieux des cité) et ils refusaient de se soumettre à ce tribut personnel. En le payant ils auraient évité d'être sanctionnés pour leur refus des idoles, mais ils tenaient à ne pas être confondus avec eux. Les pouvoirs publics ne les recherchaient pas systématiquement, mais ils ne pourchassaient pas non plus ceux qui les maltraitaient. Au contraire les magistrats instruisaient les dénonciations qui leur étaient transmises. La peine encourue en cas de refus de sacrifier aux dieux était la mort.}
en dépit desquelles ils se sont répandus jusqu'à devenir une des composantes incontournables de l'empire, même si elle restait minoritaire. 

 Dans le même temps ils se sont organisés hiérarchiquement et se sont créé un clergé permanent. Ils ont ajouté à la bible juive les livres du \emph{Nouveau Testament} : les quatre \emph{Évangiles}, les \emph{Actes des apôtres}, \emph{L'Apocalypse} et diverses \emph{épîtres} (celles de Paul de Tarse d'abord et surtout). Ils ont inventé la première théologie et un culte original. 

 Les premières communautés chrétiennes (les premières « églises%
% [3]
\footnote{... du mot grec \emph{ecclésia}, qui signifie « assemblée », ce mot désigne le rassemblement par convocation des citoyens libres, tandis que le mot \emph{sunagoguè}, de sens voisin et dont vient synagogue désigne plutôt une réunion organisée par accord mutuel des participants.}
») fonctionnaient sur le modèle des communautés juives (des synagogues). Leur représentation du monde, fidèle en cela aussi au judaïsme, empruntait ses traits essentiels à la famille : un seul dieu père, une communauté qui se définit comme une famille de frères et de sœurs, etc. Mais au fil du temps les différences sont devenues de plus en plus évidentes. Pour une part importante ces différences étaient liées à la façon dont les chrétiens abordaient la vie sexuelle et à la place qu'ils lui donnaient … ou ne lui donnaient pas. 


\section{Indissolubilité du mariage}

 Conformément au droit romain le mariage des chrétiens reposait sur la volonté des seuls époux%
% [5]
\footnote{Quand à Rome deux personnes voulaient se marier il leur suffisait de dire tous les deux leur intention de convoler en présence de témoins dans le cadre d'une simple fête domestique. Aucune cérémonie plus officielle n'était nécessaire, même si les mots prononcés étaient plus ou moins codifiés (très proches en fait de ce que nous disons aujourd'hui, puisque c'est de là que nous l'avons reçu), et si l'échange d'anneaux (nos alliances) était de règle.}%
. L'expression publique de leur choix réciproque suffisait. Il n'existait rien qui ressemble à un mariage « religieux », mais \emph{leur accord public avait des effets religieux}. Ils avaient la même horreur de l'inceste que les autres romains, qui refusaient le mariage entre cousins accepté par les orientaux%
% [6]
\footnote{... ce que prouvent les réactions au mariage de l'Empereur Claude (41--54 de notre ère) avec sa nièce. Même si cet exemple a été imité il heurtait le sentiment des romains, alors qu'en Orient cela ne faisait guère problème.}%
. Par contre ils tenaient pour valides des unions jugées impossibles ou même interdites par la loi romaine : entre sénateur et affranchie, entre citoyen et barbare, et même ils reconnaissaient la validité des mariages des esclaves entre eux ou avec des personnes libres,~etc.

 Les quatre évangiles sont d'accord pour dire que le Christ enseignait que l'union conjugale est indissoluble: \emph{S'approchant, des pharisiens lui demandèrent : « Est-il permis à un mari de répudier sa femme ? » C'était pour le mettre à l'épreuve. Il leur répondit : « Qu'est-ce que Moïse vous a prescrit ? -– Moïse, dirent-ils, a permis de rédiger un acte de divorce et de répudier. » Alors Jésus leur répliqua : « c'est en raison de votre caractère intraitable qu'il a écrit pour vous cette prescription. Mais à l'origine de la création Dieu les fit homme et femme. Ainsi donc l'homme quittera son père et sa mère, et les deux ne feront qu'une seule chair. Ainsi ils ne sont plus deux, mais une seule chair. Eh bien ! Ce que Dieu a uni, l'homme ne doit point le séparer. » Rentrés à la maison les disciples l'interrogèrent de nouveau sur ce point. Et il leur dit : « Quiconque répudie sa femme et en épouse une autre, commet un adultère à l'égard de la première ; et si une femme répudie son mari et en épouse un autre, elle commet un adultère. »} (Mc 10, 2-12). \emph{Si c'est elle \emph{[l'épouse]} qui se sépare de son mari et qui devient la femme d'un autre, elle commet un adultère.} (Mt 5,32b). \emph{Quiconque répudie sa femme et en épouse une autre commet un adultère, et celui qui épouse une femme répudiée par son mari commet un adultère.} (Lc, 16, 18) 

 Cette doctrine est si éloignée des conceptions de l'époque, juives ou autres, qu'elle a toutes les chances d'appartenir à l'enseignement du Christ le plus authentique et le plus affirmé. Il n'était pas possible pour ceux qui l'écoutaient de le tenir pour nul et non avenu quelles que soient les oppositions qu'il suscitait et les difficultés qu'il créait et créerait à l'avenir : \emph{... Les disciples lui dirent : « Si telle est la condition de l'homme envers la femme, il n'est pas expédient de se marier. » Et lui de leur répondre : « Tous ne comprennent pas ce langage, mais ceux-là seulement à qui c'est donné. Il y a en effet des eunuques qui sont nés ainsi du sein de leur mère, il y a des eunuques qui le sont devenus par l'action des hommes, et il y a des eunuques qui se sont rendus tels en vue du royaume des cieux. Comprenne qui pourra !} (Mt 19, 10-12). 
 
 Ceci dit pour le Christ le mariage n'est que pour cette terre : \emph{... ceux qui auront été jugés dignes d'avoir part à l'autre monde et à la résurrection des morts ne prennent ni femme ni mari ... car ils sont pareils aux anges ...} (Luc 20, 34-36)

 Pour Paul de Tarse le modèle du mariage était l'union du Christ avec son Église, union qui accomplissait l'alliance de Dieu et d'Israël (« ancienne alliance ») dont les prophètes avaient à maintes reprises dans le passé exprimé les fluctuations dans le langage de l'amour humain. Ce modèle identifiait l'homme au Christ et la femme à l'Église. Le mari se devait d'aimer son épouse : \emph{Maris, aimez vos femmes comme le Christ a aimé l'Église : il s'est livré pour elle, afin de la sanctifier en la purifiant par le bain d'eau qu'une parole accompagne ; car il voulait se la présenter à lui-même toute resplendissante, sans tache ni ride ni rien de tel, mais consacrée et sans reproche. De la même façon les maris doivent aimer leurs femmes comme leurs propres corps. L'amoureux de sa femme s'aime lui-même. Or nul n'a jamais haï sa propre chair ; on la nourrit au contraire et on en prend bien soin. C'est justement ce que le Christ fait pour l'Église : ne sommes-nous pas les membres de son corps ? « Voici donc que l'homme quittera son père et sa mère pour s'attacher à sa femme, et les deux ne feront qu'une seule chair » : ce mystère est de grande portée ; je veux dire qu'il s'applique au Christ et à l'Église. Bref, en ce qui vous concerne, que chacun aime sa femme comme soi-même, et que la femme révère son mari.} (Eph. 5,25-33)%
%  [4] 
\footnote{... que cette épître ait été écrite par Paul lui-même ou par un de ses disciples est aujourd'hui en débat. Pour notre objet l'important est qu'elle ait été reçue comme venant de lui, exprimant sa doctrine, et qu'elle ait été intégrée dans le Canon de l'Église, la liste officielle de ses textes de référence.}%
. 

 Pour Paul, Dieu est fidèle en dépit et au-delà de toutes les infidélités d'Israël. De la même façon le Christ a été fidèle jusqu'à la mort. C'est ainsi qu'il argumentait son refus de tout remariage tant qu'un ex conjoint était vivant : \emph{... que la femme ne se sépare pas de son mari -- en cas de séparation qu'elle ne se remarie pas ou qu'elle se réconcilie avec son mari -- et que le mari ne répudie pas sa femme.} (I Cor 7,10-12). L'union ne pouvait être dissoute que par la mort : \emph{La femme demeure liée à son mari aussi longtemps qu'il vit ; mais si le mari meurt elle est libre d'épouser qui elle veut, dans le Seigneur seulement} (c'est-à-dire parmi les membres de la communauté chrétienne). (I Cor 7,39). 
 
 Les devoirs des époux étaient réciproques et égaux. L'union d'un homme et d'une femme était exclusive, ce qui interdisait la polygynie : \emph{Que chaque homme ait sa femme et chaque femme son mari} (I Cor 7,2). En bon juif Paul de Tarse n'avait rien contre les relations sexuelles, à la condition qu'elles s'inscrivent dans le cadre bien tempéré d'une vie conjugale régulière, excluant les pratiques de nature à empêcher la fécondation, et les avortements. Quant à la sexualité hors mariage, hétérosexuelle ou homosexuelle, il ne s'y intéressait que pour la condamner sans appel, comme les moralistes juifs et stoïciens de son époque. La fidélité était donc exigée des hommes mariés, et tous leurs écarts étaient qualifiés d'adultères, désormais aussi coupables moralement que ceux des épouses, alors que jusque là ils n'étaient adultères selon la loi que s'ils avaient une relation sexuelle avec la femme légitime d'un autre homme. 
 
 Même si leurs obligations réciproques étaient identiques, Paul ne mettait aucunement en question la soumission des femmes aux hommes, pas plus qu'il n'était prêt à leur donner un quelconque pouvoir de représentation dans les assemblées. Pour lui comme pour toute l'antiquité, païenne ou juive, la famille était une institution hiérarchisée et non une démocratie, et c'était l'homme qui la dirigeait et non la femme : \emph{... Le chef de tout homme, c'est le Christ ; le chef de la femme, c'est l'homme ; et le chef du Christ c'est Dieu ...} (I Cor 1) 

 Dans sa première Apologie Justin résume la position des églises vers 155 après J.-C. : \emph{Voici ce qu'il \emph{(Jésus)} dit de la chasteté : « Quiconque aura regardé une femme pour la convoiter a déjà commis l'adultère dans son cœur. » Et : « Que si votre œil droit vous scandalise ; arrachez-le et jetez-le loin de vous ; il vaut mieux n'avoir qu'un œil et entrer dans le royaume des cieux, qu'avoir deux yeux et être jeté dans le feu éternel. » Et : « Celui qui épouse la femme répudiée par un autre homme commet un adultère. » Et : « Il y a des eunuques sortis tels du sein de leur mère ; il y en a que les hommes ont fait eunuques, et il y en a qui se sont faits eunuques eux-mêmes en vue du royaume des cieux ; mais tous n'entendent pas cette parole. » Ainsi ceux qui, selon la loi des hommes, contractent un second mariage après leur divorce, comme ceux qui regardent une femme pour la convoiter, sont coupables aux yeux de notre maître; il condamne le fait et jusqu'à l'intention de l'adultère ; car Dieu voit non seulement les actions de l'homme, mais même ses plus secrètes pensées. Et pourtant combien d'hommes et de femmes sont parvenus à plus de soixante et soixante-dix années, qui, nourris depuis leur berceau dans la foi du Christ, sont restés purs et irréprochables durant leur longue carrière ! Ce fait se retrouve dans les peuples de toute contrée ; je m'engage à le prouver.} 
 
 On a vu qu'à Rome tout témoin d'un adultère féminin devait dénoncer les coupables. Le mari d'une adultère devait la répudier sans tarder, et elle était condamnée à l'infamie. Sinon il était lui aussi condamné à l'infamie comme proxénète, ainsi que ses enfants à naître. En tant qu'infâme il perdait son autorité sur ses enfants déjà nés et une fraction importante de ses biens était confisquée. Son mariage était dissous même s'il continuait de cohabiter avec son épouse. Les premiers chrétiens ne pouvaient pas faire l'impasse sur des lois civiles aux effets aussi redoutables : passer outre aurait été héroïque, surtout si des enfants étaient impliqués. Au surplus ils est probable qu'ils pensaient comme tous leurs contemporains et que l'adultère féminin leur paraissait très grave et honteux, beaucoup plus grave et plus honteux que celui des maris. Il est donc probable qu'ils répudiaient eux aussi les épouses infidèles. D'ailleurs l'évangile de Matthieu accepte la répudiation de l'épouse en cas de « fornication » ou de « prostitution » \emph{(porneia)} : \emph{Il a été dit d'autre part : Celui qui répudie sa femme doit lui remettre un acte de divorce. Eh bien ! Moi je vous dis : quiconque répudie sa femme, hormis le cas de fornication, la voue à devenir adultère ; et si quelqu'un épouse une répudiée, il commet un adultère} (Mt 5,31-32). » \emph{Or je vous le dis : quiconque répudie sa femme -- je ne parle pas de la prostitution -- et en épouse une autre, commet un adultère.} (Mt 19, 9). 

 Mais le vrai problème n'était ni la répudiation ni le divorce, c'était le remariage. Il ne s'agissait pas de décider si l'adultère mettait fin à un mariage, de toute façon déjà dissous par la loi civile, mais s'il permettait au conjoint innocent de contracter validement un nouveau mariage. Si la loi romaine faisait aux maris des femmes adultères l'obligation de répudier celles-ci, elle ne les obligeait pas à se remarier. S'ils ne le faisaient pas ils subissaient les conséquences des Lois d'Auguste et ils étaient taxés comme célibataires. Ce n'est que s'ils n'avaient pas encore le quota réglementaire d'enfants que les conséquences devenaient sérieuses : ils devaient alors renoncer à hériter de personnes qui ne faisaient pas partie de leur famille. Ceci dit s'ils n'avaient pas d'amis aisés susceptibles de les coucher sur leurs testaments la perte n'était pas grande. Il était surtout problématique d'interdire toute vie sexuelle et peut-être toute descendance à des hommes encore jeunes au seul motif que leurs épouses leur avaient été infidèles. 

 Les discussions entre théologiens et évêques ont donc porté sur ce que l'on devait entendre par « fornication » et par « prostitution ». Au fil du temps, ils se sont mis d'accord sur l'idée que par ces mots, le Christ avait voulu désigner non pas l'infidélité de l'un ou de l'autre des époux, mais la transgression des interdits de mariage, c'est-à-dire tout ce qui se rapproche de l'inceste%
% [7]
\footnote{cf. A.-M.~\fsc{GERARD}, 1989, p. 878.}%
. Au troisième siècle au plus tard c'est cette interprétation qui avait pris l'ascendant, mais la discussion est restée ouverte jusqu'au milieu du moyen-âge. Même quand un remariage après divorce était admis, et cela semble n'avoir pas été rare jusqu'à la réforme grégorienne (milieu du Moyen-Âge), ce n'était qu'une concession faite en vue d'éviter de plus grands maux%
% [8]
\footnote{La lecture littérale de Matthieu n'a jamais été oubliée notamment en Orient, et c'est cette lecture que choisira la réforme protestante.}%
.

 En stricte doctrine chrétienne le viol ne déshonorait ni ne souillait la victime : \emph{rien de ce qui est hors de l'homme et qui entre dans l'homme ne peut le souiller ; mais ce qui sort de l'homme voilà ce qui souille l'homme} (Marc 7, 15). D'autre part le suicide comme la répudiation étaient interdits. Cela étant le ressenti des personnes concernées ne pouvait s'affranchir des représentations communes : ce qu'on nommait l'honneur de la victime et celui de sa parentèle étaient en jeu. Comment redonner une bonne réputation aux victimes et à leur famille en dépit de ces représentations sans employer les moyens radicaux qui avaient cours jusque là, sans faire disparaître les victimes ? Comment soigner et restaurer le regard des hommes sur les femmes violées ? (20 siècles après ces questions se posent toujours). 

 De même que les juifs les chrétiens cherchaient à marier leurs enfants à des membres de la communauté chrétienne, sans exclure formellement les mariages avec un non chrétien. Le seul cas où le remariage après divorce était autorisé par Paul concernait justement les unions où un conjoint non chrétien mettait des obstacles à la pratique religieuse de son conjoint chrétien (« Privilège Paulin »).


\section{Valorisation du célibat et de la continence}

 Les évangiles valorisaient le célibat de manière implicite en mettant en valeur deux célibataires : Jean le Baptiste et Jésus. Par ailleurs et surtout on y trouvait divers discours explicites en faveur du célibat, notamment le récit suivant qui est commun à trois évangiles sur quatre (Matthieu 19, 16-22 ; Marc 10, 17-22 ; Luc 18, 18-23) et qui fait donc partie du noyau de traditions et de paroles autour desquelles s'est articulée la prédication du premier demi-siècle de l'Église : \emph{Or voici qu'un homme s'approcha et lui dit : « Maître, que dois-je faire de bon pour posséder la vie éternelle ? » Jésus lui dit : « Qu'as-tu à m'interroger sur ce qui est bon ? Un seul est le Bon. Que si tu veux entrer dans la vie, observe les commandements. -- Lesquels ? » lui dit-il. « Eh bien », reprit Jésus : « Tu ne tueras pas, tu ne commettras pas d'adultère, tu ne voleras pas, tu ne porteras pas de faux témoignage ; honore ton père et ta mère, et tu aimeras ton prochain comme toi-même. » Le jeune lui dit : « Tout cela, je l'ai gardé ; que me manque-t-il encore ? -- Si tu veux être parfait, lui dit Jésus, va, vends ce que tu possèdes, donne-le aux pauvres, et tu auras un trésor aux cieux ; puis viens, suis-moi. » Quand il entendit cette parole, le jeune homme s'en alla contristé, car il avait de grands biens.} (Matthieu 19, 16-22). 

 La vie religieuse ou le célibat consacré des siècles futurs ont là leur origine. Appliquer à la lettre la suggestion de Jésus \emph{(vends ce que tu possèdes)} impliquait en effet de n'avoir plus d'héritage à transmettre et donc plus jamais d'enfants, sauf à manquer à tous les devoirs d'un père, ce qui est une position que nul exégète \emph{sérieux}%
% [10]
\footnote{... Jack \fsc{GOODY} note que dans son ouvrage \emph{Contra avaritiam}, Salvien, prêtre marseillais du \siecle{5}, à une période où le christianisme est devenu la religion d'état des populations romaines, conseille aux parents de laisser leurs biens à l'Église plutôt qu'à leurs enfants, car \emph{mieux vaut la souffrance des enfants en ce monde que la damnation des parents dans l'autre} (p. 107). Cette position antisociale n'est vraisemblablement qu'un médiocre artifice de rhétorique chez un auteur porté par ailleurs aux exagérations et à l'hyperbole : à la même époque Saint Augustin, dont l'autorité est sans commune mesure avec celle de Salvien, refusait formellement à l'Église le droit d'accepter tout legs fait au détriment d'un fils (Jack \fsc{GOODY}, p. 101). Il conseillait de (ne) léguer à l'Église (que) la part d'\emph{un} fils, ce qui diminuait d'autant plus le montant des legs qu'il y avait plus d'héritiers vivants. Le problème n'était pas seulement théorique : nombreuses ont été au fil des siècles les réactions des autorités civiles pour empêcher l'enthousiasme des plus fanatiques des dévots, ou la terreur de la damnation éternelle des mourants, ou leur désir de régler des comptes avec leurs enfants, de dépouiller leurs héritiers légitimes.}
n'a jamais prêtée à l'auteur de ce texte. 

 Lorsque les chrétiens valorisaient le célibat et la chasteté, ce n'était pas sans échos dans le monde gréco-romain des premiers siècles de notre ère : les philosophes stoïciens et les médecins d'alors étaient soucieux de ne pas donner au sexe plus de place qu'il n'en méritait et de maîtriser les passions, au premier rang desquelles la passion amoureuse. Les juifs aussi avaient leurs \emph{nazirs} et leurs \emph{esséniens}. 

 Pour Paul de Tarse les personnes continentes étaient moins exposées aux dangers moraux et aux angoisses que ceux et celles qui choisissaient le mariage : \emph{Pour ce qui est des vierges, je n'ai pas d'ordre du Seigneur, mais je donne un avis en homme qui, par la miséricorde du Seigneur, est digne de confiance. J'estime donc qu'en raison de la détresse présente, c'est l'état qui convient ; oui, c'est pour chacun ce qui convient. Es-tu lié à une femme ? Ne cherche pas à rompre. N'es-tu pas lié à une femme ? Ne cherche pas de femme. Si cependant tu te maries, tu ne pèches pas ; et si la jeune fille se marie, elle ne pèche pas. Mais ceux-là connaîtront des épreuves en leur chair, et moi, je voudrais vous les épargner.} (I Cor 7,25-28)

 \emph{Je voudrais vous voir exempts de soucis. L'homme qui n'est pas marié a souci des affaires du Seigneur, des moyens de plaire au Seigneur. Celui qui s'est marié a souci des affaires du monde, des moyens de plaire à sa femme ; et le voilà partagé. De même la femme sans mari, comme la jeune fille, a souci des affaires du Seigneur ; elle cherche à être sainte de corps et d'esprit. Celle qui s'est mariée a souci des affaires du monde, des moyens de plaire à son mari. Je vous dis cela dans votre propre intérêt, non pour vous tendre un piège, mais pour vous porter à ce qui est digne et qui attache sans partage au Seigneur.} (I Cor 7,32-35)

 Selon lui ceux qui supportaient la continence et qui en faisaient le choix étaient libérés de toute attache terrestre, et dégagés des soucis du monde : c'était un point de vue très stoïcien. Ils choisissaient « la meilleure part » d'où sa réticence devant les remariages, sauf pour les veufs et veuves jeunes et sans enfants. En effet en dépit de sa préférence pour la continence il ne pensait pas que celle-ci était faite pour tout le monde ni qu'elle était sans risques%
% [11]
\footnote{Cf. Blaise \fsc{Pascal} : \emph{L'homme n'est ni ange, ni bête, et le malheur veut que qui veut faire l'ange fait la bête.} \emph{(Pensées)}}
: \emph{J'en viens maintenant à ce que vous m'avez écrit. Il est beau pour l'homme de ne pas toucher à la femme. Toutefois en raison du péril d'impudicité, que chaque homme ait sa femme et chaque femme son mari. Que l'homme s'acquitte de son devoir envers sa femme, et pareillement la femme envers son mari. La femme ne dispose pas de son corps, mais le mari. Pareillement, le mari ne dispose pas de son corps, mais sa femme. Ne vous refusez pas l'un à l'autre, si ce n'est d'un commun accord, pour un temps, afin de vaquer à la prière ; puis reprenez la vie commune, de peur que Satan ne profite, pour vous tenter, de votre manque de maîtrise. Ce que je dis là est une concession, non un ordre. Je voudrais que tout le monde fût comme moi ; mais chacun reçoit de Dieu son don particulier, l'un celui-ci, l'autre celui-là. Je dis toutefois aux célibataires et aux veuves qu'il leur est bon de demeurer comme moi. Mais s'ils ne peuvent se maîtriser qu'ils se marient : mieux vaut se marier que de brûler.} (I Cor 7, 8-9)

 Il conseillait le célibat une fois satisfait le désir d'une descendance et obtenu le droit d'hériter qui en découlait pour les citoyens romains, une fois passées la jeunesse et ses orages, et on était vite vieux à une époque où les femmes commençaient souvent à avoir des enfants dès 13 ou 14 ans, où la moitié de ceux-ci mouraient avant leurs vingt ans, et où ceux qui atteignaient cet âge avaient de fortes chances d'être déjà orphelins de père. 


\section{Désacralisation de la fécondité, valorisation des enfants}

 Comme les juifs les chrétiens refusaient que le mariage ait pour fin la continuité du culte des ancêtres. Mais contrairement à eux ils refusaient d'accorder une valeur religieuse à la fécondité individuelle : ni les Évangiles ni les Épîtres retenues par le Canon des écritures chrétiennes n'en parlent. Nul, même marié, n'était à leurs yeux tenu de concevoir des enfants. C'est le peuple chrétien tout entier, et non chaque famille, qui devait croître et se multiplier. Cette position leur permettait de refuser le divorce et la polygamie, alors qu'il leur aurait été difficile de maintenir ces refus si la fécondité avait été posée comme un devoir pour chaque individu. La seule obligation à laquelle chaque chrétien devait se soumettre était de n'opposer aucune barrière à sa fécondité au cours des actes sexuels dans lesquels il s'engageait. 

 Comme les juifs, l'Église désapprouvait moralement l'abandon, mais elle interdisait aussi l'avortement quel qu'en soit le motif. 

 Dans sa première Apologie (vers 155) Justin (100--165) écrivait à l'empereur à propos des abandons et ventes d'enfants : \emph{Quant à nous, loin de commettre aucune impiété, aucune vexation, nous regardons comme un crime odieux l'exposition des enfants nouveau-nés ; parce que d'abord nous voyons que c'est les vouer presque tous, non seulement les jeunes filles, mais même les jeunes garçons, à une prostitution infâme ; car de même qu'autrefois on élevait des troupeaux de bœufs et de chèvres, de brebis et de chevaux, de même on nourrit aujourd'hui des troupes d'enfants pour les plus honteuses débauches. Des femmes aussi et des êtres d'un sexe douteux, livrés à un commerce que l'on n'ose nommer, voilà ce qu'on trouve chez toutes les nations du Globe. Et au lieu de purger la terre d'un scandale pareil, vous en profitez, vous en recueillez des tributs et des impôts !}

 \emph{... Quant à l'exposition des enfants, il est un motif encore qui nous la fait abhorrer. Nous craindrions qu'ils ne fussent pas recueillis, et que notre conscience restât ainsi chargée d'un homicide. Au reste, si nous nous marions, c'est uniquement pour élever nos enfants ; si nous ne nous marions pas, c'est pour vivre dans une continence perpétuelle}.

 Dans un plaidoyer en faveur des chrétiens adressé à l'empereur Marc-Aurèle, Athénagoras%
% [13]
\footnote{... cité par Albert \fsc{DUPOUX}, \emph{Sur les pas de Monsieur Vincent, 300 ans d'histoire parisienne de l'enfance abandonnée}, 1958, p. 5.}
exposait ainsi la position de l'Église : \emph{Nous tenons pour homicides les femmes qui se font avorter, et nous pensons que c'est tuer un enfant que de l'exposer.} 

 Deux générations plus tard Tertullien (160--245) écrivait que \emph{l'homme existe avant la naissance, de même que le fruit est tout entier dans la graine}. 

 La seule méthode acceptable pour limiter le nombre des enfants était donc la continence. Celui qui ne pouvait élever plus d'enfants qu'il n'en avait déjà se devait de « s'abstenir de sa femme ».
 
[1] \tempnote{Je n'ai pas trouvé l'appel de la note 9 : du coup je l'ai laissée ici...}
[2] 
[3] 
[4] 
[5] 
[6] 
[7] 
[8] 
[9] On remarquera que Paul ne donne pas d'explications supplémentaires, comme si ses arguments allaient de soi pour ses correspondants et ne pouvaient qu'emporter leur conviction. On ne voit pas ce que les anges viennent faire ici, sauf à supposer qu'ils seraient en danger de succomber aux charmes féminins et qu'il ne faudrait donc pas les tenter en leur laissant croire qu'elles sont libres et sans maître ? Croyait-on à cela à l'époque de Paul ? Nous ne sommes peut-être plus en mesure de comprendre ce qu'il voulait dire ? 
[10] 
[11] 
[13] 

