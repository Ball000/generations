% 28.02.2015 :
% haut Moyen Âge
% _, --> ,
% Antiquité
% ~etc.
% ~\%


\chapter{Familles de chair}


 À partir de Constantin les lois de l'Empire, puis celles des royaumes qui en Occident ont repris sa succession, se sont lentement alignées sur les conceptions chrétiennes du mariage et de la génération%
% [1]
\footnote{Cf. Georges \fsc{DUBY}, \emph{Le chevalier, la femme et le prêtre}, 1981.}% 
. Mais dans le même temps la vie familiale à la romaine était également mise à mal par les « barbares ». Ceux-ci ont introduit des pratiques différentes, principalement germaines%
%[2]
\footnote{Jean-Pierre \fsc{POLY}, \emph{Le chemin des amours barbares, Genèse médiévale de la sexualité européenne}, 2003.}% 
, sur les territoires de l'ancien Empire romain d'Occident. Le haut Moyen Âge est un temps de conflits, de coexistence et de compromis entre les droits et coutumes des royaumes « barbares » et le droit romain%
%[3]
\footnote{Cf. Pierre \fsc{PETOT}, \emph{La famille}, 1992.}% 
. On constate l'effacement progressif des traditions juridiques romaines, compensé dans une large mesure par l'élaboration (ou la résurgence) de pratiques non romaines, dites \emph{coutumières}, propres à chaque lieu et caractérisées d'abord par une très grande variété%
%[4]
\footnote{Mais lorsqu'il s'agira à partir du \siecle{12} de reconstruire un droit unifié et cohérent, à la fois dans le domaine civil \emph{(droit civil)} et dans le domaine religieux \emph{(droit Canon)}, c'est au \emph{Code de Justinien}, publié en 529 et 534, que les juristes savants vont se référer.}% 
. 

 Toutes ces réserves étant faites, et malgré une infinité de particularités tenant aux lieux et aux temps, les lignes de force du système articulant en \emph{chrétienté} les familles, les autorités civiles et les institutions d'Église sont demeurées les mêmes au-delà du Moyen Âge%
% [5]
\footnote{... et même au-delà, jusqu'à la Révolution Française, même si des évolutions très significatives ont eu lieu à partir de la Renaissance et des Réformes protestantes et catholiques. Le paradoxe c'est même que c'est aux \crmieme{17} et \crmieme{18} siècles que les familles se sont le plus étroitement conformées aux principes chrétiens, sous la garde conjointe, vigilante et de plus en plus efficace, des autorités religieuses et civiles.}% 
. Sur la délimitation de cette période de l'histoire et en ce qui concerne mon sujet je ne peux que constater que celle qui convient le mieux est celle du \emph{long Moyen Âge} de Jacques \fsc{LE GOFF} (\emph{Faut-il vraiment découper l'histoire en tranches ?} 2013). À bien des points de vue le Moyen Âge ne s'est achevé qu'avec la Révolution Française, même si la deuxième moitié du \siecle{18} (à partir de 1760 à quelques années près) participait déjà du siècle suivant, notamment sur le plan des idées, avec le mouvement européen des \emph{Lumières}. Pour schématiser on pourrait dire qu'il y a une unité dans la période qui va de Constantin à l'Encyclopédie. 

\section{Mépris de la chair ?}

 Les chrétiens sont-ils coupables d'avoir diabolisé les plaisirs de la chair ? La réponse à ces questions n'est pas simple. D'abord il faut dire avec Paul Veyne qu'ils \emph{n'ont rien réprimé du tout, c'était déjà fait}. Le monde patriarcal des cités antiques n'était en rien un monde de liberté sexuelle, sauf pour les hommes libres, et encore. Puis les philosophes stoïciens étaient passés par là pour exiger des hommes libres eux-mêmes qu'ils orientent leurs désirs vers la seule procréation. Quant aux médecins ils allaient eux aussi dans le sens d'une grande modération dans l'activité sexuelle. 

 Ceci étant dit il est vrai que les théologiens chrétiens des premiers siècles ont longtemps été tentés par les thèses \emph{encratites}%
% [6]
\footnote{Selon Encyclopedia universalis : Encratite est un \emph{terme signifiant « les continents » (du grec \emph{enkratès}) et désignant plusieurs sectes hérétiques de l'Église ancienne qui prônaient un rigorisme moral radical (interdiction du mariage, abstention de viande et de vin) fondé sur une condamnation de la matière et du corps considérés comme les œuvres d'un démiurge distinct du Dieu suprême. Tatien, d'abord disciple de Justin, à Rome, est traditionnellement tenu pour le fondateur, vers 170, de cette secte ascétique des encratites, probablement dans la région d'Édesse. L'encratisme fut alors proscrit sous ses diverses formes par de nombreux décrets de Théodose I\ier, à la fin du \siecle{4}, et de Théodose II, en 428.
La sévérité des mesures impériales suffirait à témoigner de l'importance de la secte à cette époque. L'encratisme s'est alors confondu avec le manichéisme et a trouvé des prolongements chez les Messaliens et les Bogomiles (et les Cathares). Le rigorisme que pratiquaient ses adeptes se voulait une négation de l'œuvre du démiurge. Les fondements doctrinaux de la secte consistaient dans le rejet de certaines parties des Écritures, en particulier de l'Ancien Testament, et dans un recours à des textes de la littérature apocryphe présentant des tendances ascétiques très marquées. Certaines positions doctrinales et liturgiques découlaient généralement de la conception encratiste de la création et de la matière : négation du salut d'Adam (Tatien), négation de la résurrection de la chair, docétisme en christologie, utilisation d'eau à la place du vin pour célébrer l'eucharistie. La ligne de démarcation entre l'encratisme et le gnosticisme est difficile à tracer : ce dernier est dans une large mesure marqué par un courant rigoriste, et l'encratisme semble avoir accueilli des spéculations d'origine gnostique}.
Richard \fsc{GOULET}}%
, proches de celles des manichéens qui soutenaient que la matière est mauvaise par nature, que l'âme préexiste au corps, et qu'avec la conception elle chute dans un monde matériel et charnel, lieu de l'esprit du mal%
%[7]
\footnote{... ce qui paradoxalement pouvait conduire les adeptes de ces doctrines à une licence effrénée puisque rien sous le ciel n'avait plus d'importance : « méprises et fais ce que tu veux ».}% 
. Le plus emblématique des théologiens encratites, Tatien (deuxième siècle après J.-C.), considéré comme hérétique par divers \emph{Pères de l'Église}, rejetait le mariage et condamnait l'usage de la viande et du vin comme de tout autre plaisir de la chair. Préconisant l'eau pour célébrer l'eucharistie à la place du vin, il recommandait de se garder de tout acte sexuel, et de ne pas faire d'enfants pour ne pas prolonger l'existence d'un monde qu'il jugeait mauvais. 

 Dans le même ordre d'idées, selon Robert Markus%
% [8] 
\footnote{Robert \fsc{MARKUS}, \emph{Au risque du christianisme, l'émergence du modèle chrétien (\siecles{4}{6})}, Cambridge University Press, 1990, réédition en Français, Presses Universitaires de Lyon, 2012.} 
de nombreux auteurs du \siecle{4} pensaient qu'Adam et Ève \emph{avaient été créés sans sexe, avec une innocence que certains comparaient à l'innocence des enfants. Si Adam et Ève n'avaient pas péché, les rapports sexuels n'auraient pas été requis pour augmenter et multiplier la race humaine}.

 Augustin, 
\tempnote{Vérifier les parties citées et celles qui les commentent, et la présentation de ces trois paragraphes : qu'est-ce qui est à Markus, à Augustin, à nous ?}%
qui avait longtemps été manichéen et qui avait donc partagé cette thèse, finit par la rejeter complètement en disant : \emph{Je ne vois pas pourquoi il ne devrait pas y avoir de mariage honorable au Paradis}. Il soutient \emph{que l'union sexuelle et la reproduction ne dérivent pas de la Chute à partir d'un état pré pubère de l'innocence, mais qu'elles font partie de l'intention originelle de Dieu envers les créatures}. En renonçant à ses convictions antérieures, Augustin rejetait ainsi un large consensus parmi les chrétiens de son époque. Ses nouvelles positions à l'égard de la sexualité sont exceptionnelles à la fin du \siecle{4}. Elles s'opposent aux conceptions d'autres Pères de l'Église comme Jérôme, Ambroise ou Grégoire de Nysse,~etc. 

 « Ce qui devait être expliqué n'était pas l'existence de la sexualité, mais plutôt son mode de fonctionnement et l'impact du péché d'Adam sur la sexualité de ses descendants. » « Les problèmes que posait la sexualité n'étaient ni plus ni moins les mêmes que ceux que posait l'homme. » Selon Augustin à la fin de sa vie : « Ce n'est pas la chair corruptible qui a rendu l'âme pécheresse, c'est l'âme pécheresse qui a rendu la chair corruptible. »

 De son point de vue selon Robert Markus « La tension que décrivait l'enseignement manichéen, entre deux natures différentes dans un conflit permanent, était maintenant transposée en terme de conflit interne avec soi-même... Ce qui est répréhensible et honteux dans la sexualité n'est pas son existence même, mais sa tendance à être hors de tout contrôle et à échapper à la raison... un tel constat impliquait une réhabilitation de la chair... Et au bout du compte, cela impliquait aussi une réhabilitation du mariage. » 

 C'est Augustin qui a théorisé le premier la notion de « péché originel ». Il faisait de la reproduction humaine le lieu de sa transmission. Compte tenu de son immense postérité intellectuelle jusqu'à la fin du Moyen Âge, ses idées ont eu une profonde influence. Elles impliquaient une ascèse à laquelle étaient conviés les époux au même titre que les religieux, ascèse aussi méritoire de son point de vue que celle à laquelle s'astreignaient ces derniers. 

 Les ambivalences d'Augustin n'étaient pas de nature à empêcher les moines, très influents durant tout le haut Moyen Âge où ils fournissaient l'essentiel des intellectuels, de tenir la virginité pour un état de vie plus proche de la perfection que celui des mariés, ni de dénigrer la féminité et l'exercice de la sexualité (de toutes les sexualités), ni d'essayer de lui imposer des limites. Il faudra arriver au \siecle{12} pour que ces points de vue soient en partie remplacés par une exaltation du mariage comme état de vie chrétien. Ce qui n'empêchait pas les laïcs de choisir parmi tous ces principes et toutes ces règles celles qui leur paraissaient les plus raisonnables ou les moins intenables. Au sein des sociétés chrétiennes il existait donc une tension permanente%
% [9]
\footnote{Jacques \fsc{ROSSIAUD}, \emph{Sexualités au Moyen Âge}, Éditions Jean-Paul Gisserot, Paris, 2012.}% 
. 
 

\section{Disparition de l'adoption}

 La première des donations à visée religieuse des païens était leur héritage. De droit c'est leur héritier qui était l'officiant de leur culte mortuaire. À partir du moment où les cultes païens ont été disqualifiés, puis interdits, il n'y avait plus de motif religieux de se procurer à toute force un héritier, puisqu'il n'y avait plus de culte des morts à assumer. L'adoption plénière, celle qui fabriquait des héritiers légitimes avec des étrangers, a donc presque totalement disparu de la scène, et cela pour quinze siècles. C'est l'Église, et non plus les familles, qui gérait le culte des défunts en même temps qu'elle veillait sur les corps rassemblés dans les cimetières et le sol des églises, ce pour quoi elle recevait des donations. Elle n'avait pas de raison de se soucier de la pérennité des lignées, au contraire, l'absence d'héritiers n'était de son point de vue qu'un malheur individuel, et seulement pour cette vie. Cette absence n'avait pas d'incidence sur le salut de l'âme des défunts après leur mort. On en revenait donc aux règles juives : pas de filiation « fictive ». Mais cela ne s'est pas fait du jour au lendemain et il y a fallu plusieurs siècles, d'autant plus que les familles détentrices d'un « honneur », d'une charge publique, à commencer par les rois et les \emph{domini}, les seigneurs, avaient impérativement besoin d'héritiers pour ne pas perdre leur position sociale, et n'étaient pas d'accord sur ce point avec les clercs. 


\section{Divorces et remariages}

 Le remariage après divorce du vivant du premier conjoint a longtemps été reconnu comme valide dans ses effets par les autorités civiles, notamment en ce qui concernait la légitimité des enfants à naître, alors même qu'il était sanctionné comme une faute par l'Église (excommunication, pénitences publiques...) ou par les autorités civiles elles-mêmes (amendes, exil, confiscation de biens...). 

 Si l'adultère d'une femme avec un esclave était depuis Constantin puni de la mort des deux complices, il avait prescrit que le mari d'une femme adultère, entremetteuse ou empoisonneuse pouvait la répudier tout en conservant sa dot, et pouvait se remarier. À défaut de condamnation plus grave elle était reléguée dans une île. Dans les autres cas une femme répudiée conservait sa dot, et si l'époux se remariait elle pouvait « envahir » sa maison et prendre possession de la dot de la nouvelle élue. L'épouse pouvait elle aussi répudier un époux coupable d'homicide, d'empoisonnement, de violation de sépulture, et s'en aller avec sa dot. Elle la perdait dans les autres cas. 

 L'empereur Honorius fixe pour chaque époux trois paliers%
% [15]
\footnote{J.-P.~\fsc{POLY}, \emph{Le chemin des amours barbares}, p. 42.}% 
. Le mari peut répudier sa femme pour « crime grave » (les motifs précisés par Constantin), et il gardera sa dot et pourra se remarier. S'il la répudie pour « faute contre les mœurs » il reprend la donation qu'il lui a faite en l'épousant, mais doit rendre sa dot, et attendre deux ans pour se remarier. S'il la répudie pour d'autres motifs il perd dot et donation, et il ne peut plus se remarier. L'épouse peut de même quitter son mari pour « cause grave » (toujours les motifs de Constantin) et se remarier après un délai de cinq ans.

 Le remariage après divorce, du vivant du premier conjoint, ne semble avoir été en droit totalement éradiqué d'Occident qu'après les réformes Grégoriennes du milieu du Moyen Âge%
% [16]
\footnote{Cf. Georges \fsc{DUBY}, \emph{Le chevalier, la femme et le prêtre}, 1981.}% 
. Il a fallu que les tribunaux de l'Église obtiennent vers le \crm{10}\ieme{} ou \siecle{11} le monopole sur les affaires concernant le mariage pour que le vieux mot latin \emph{divortium} prenne le sens de séparation sans droit au remariage du vivant de l'autre conjoint, sens qu'il a gardé jusqu'à la Révolution. 


\section{Phobie de l'inceste}

 Même si les mariages entre cousins germains, et entre nièce et oncle paternel, avaient fini par être autorisés pendant un temps sous l'empire, l'interdit de l'inceste était ressenti avec force à Rome. L'Église partageait cette horreur de l'inceste : elle avait même choisi de comprendre le mot \emph{Porneia} comme désignant exclusivement les unions incestueuses, considérant que la seule cause acceptable de nullité des mariages était la proximité excessive des époux. L'un de ses objectifs était d'écarter du sein des parentèles toute expression des désirs sexuels (hors couples mariés), avec les rivalités, jalousies et rancœurs qui les accompagnent, pour donner toute la place à la seule fraternité et à la \emph{caritas}%
% [10]
\footnote{\emph{Caritas} = amour désexualisé : souci du bien de l'autre, amour de l'autre pour lui-même (même s'il n'a rien d'aimable). Il est souvent rendu par « charité » (dérivé direct de \emph{caritas}), mot où nous ne percevons plus aujourd'hui beaucoup d'amour.}% 
. L'autre objectif était de renforcer le tissu social. Augustin d'Hippone formule ainsi sa pensée : \emph{L'union du mâle et de la femelle, pour autant qu'elle relève du genre humain, est une sorte de pépinière de charité. \emph{[...]} Une très juste raison de charité%
%[12] 
\footnote{\emph{caritas}}
invita les hommes \emph{[...]} à multiplier leurs liens de parenté ; un seul homme ne devait pas en concentrer trop en lui-même, il fallait les répartir entre des sujets différents ; ainsi leur grand nombre contribuerait à préserver plus efficacement les liens de la vie sociale. Père et beau-père sont, en effet, les noms de deux liens de parenté. Que chacun ait un homme pour père et un autre pour beau-père, la charité s'étend sur un plus grand nombre \emph{[...au lieu qu']} un seul homme eût été, pour ses enfants frères et sœurs mariés entre eux, père, beau-père et oncle \emph{[...]}, autres pour le même homme seront alors la sœur, l'épouse, la cousine ; autres le père, l'oncle, le beau-père ; autres la mère, la tante, la belle-mère. Ainsi, loin de se restreindre à un cercle étroit, le lien social s'étendra plus largement et sur plus de têtes par des alliances multiples%
%[13]
\footnote{Livre XV de \emph{la Cité de Dieu}, d'après la traduction de G.~\fsc{COMBES}.}% 
.}

 Plus l'on étend le périmètre de l'inceste plus il faut aller loin de sa famille de naissance pour trouver un conjoint. Cela diminue le risque que les descendants d'une personne (un homme dans le système patriarcal) ne deviennent si puissants, au moyen d'une endogamie stricte de sa descendance, qu'ils puissent menacer le reste de la société, n'ayant pas à composer entre des allégeances multiples. On peut à contrario évoquer les observations de Germaine \fsc{TILLON} dans \emph{le harem et les cousins}, 1966, et l'opposition qu'elle fait entre la « république des beaux-frères » et la « république des cousins ». L'Église s'opposait ainsi aux pratiques orientales, qui privilégiaient le mariage entre cousins (et même entre frères et sœurs en Égypte), comme aux coutumes germaniques qui favorisaient les unions préférentielles entre familles aux alliances redoublées de génération en génération%
% [11]
\footnote{Jean-Pierre \fsc{POLY}, \emph{Le chemin des amours barbares, Genèse médiévale de la sexualité européenne}, Perrin, 2003.}% 
. On ne peut pas dire qu'elle choisissait pour autant un système de parenté contre les autres, même si elle posait le couple entouré de ses enfants au centre de ses préoccupations. Elle mettait seulement une limite contraignante aux systèmes familiaux qui cherchaient à se fermer sur eux-mêmes.

 Au tournant entre le \crmieme{4} et le \siecle{5} les empereurs Théodose, Arcadius et Honorius, ont tenté d'interdire le mariage entre cousins germains, mais ces interdits ont été levés quelques années plus tard par les empereurs (d'Orient) suivants, même si en accord avec Ambroise de Milan et l'évêque de Rome, Augustin plaidait pour cet interdit : \emph{Qui peut douter qu'il ne soit aujourd'hui plus honnête d'interdire le mariage même entre cousins germains ?}. Il arguait de sa proximité excessive avec l'inceste fraternel, et du fait que même si les lois de l'Empire l'avaient effectivement autorisé, la coutume romaine n'y était pas favorable. Mais il n'en était pas de même en Orient, où le mariage entre cousins germains était traditionnellement tenu pour idéal, trop inscrit dans la culture pour que l'argumentation d'Augustin puisse y être entendue, et où il ne sera plus question de l'interdire aussi rigoureusement par la suite. En Occident la position d'Augustin, reprise siècle après siècle par l'église, finira néanmoins par triompher.

 À partir du \siecle{7} et surtout du \crmieme{11} au \crmieme{13} en Occident, l'Église entend la notion d'inceste de manière de plus en plus extensive, jusqu'au septième degré, comme le faisait le droit romain ancien. Par dessus le marché à partir du \siecle{7}, elle s'est ralliée progressivement, et non sans résistances même en son sein, à un mode de calcul de ces degrés qui excluait tous les descendants \emph{des arrière-grands-parents des arrière-grands-parents} du sujet concerné%
% [17]
\footnote{Cf. \emph{Histoire du droit civil}, Jean-Philippe \fsc{LEVY} et André \fsc{CASTALDO}, p. 93-95.}% 
, ce qui multipliait de façon exponentielle le nombre des personnes interdites, du moins dans les familles qui prétendaient connaître leurs ancêtres aussi loin dans le passé, celles dont la légitimité reposait sur leur ascendance. Les humbles n'avaient pas une telle prétention, et on n'attachait pas autant d'importance aux irrégularités formelles de leurs unions. En ce qui les concernait il suffisait que l'interdit porte sur toutes les personnes ressenties par eux comme faisant partie de leur parenté. 

 Non seulement les évêques et théologiens d'Occident y ont ajouté toutes les parentés par alliance, y compris beaux-frères et belles-sœurs, mais ils y ont aussi adjoint la \emph{parenté spirituelle} qui liait les parrains et marraines d'un même enfant, et la parentèle de ceux-ci, sans compter les \emph{parentés} illicites nées des rencontres extra conjugales. La phobie de l'inceste a donc conduit à des extrêmes absurdes qui créaient mécaniquement des situations impossibles dans les communautés étroites où les personnes se déplaçaient fort peu en dehors des familles aristocratiques, et où en l'absence de registres d'état-civil il était difficile ou impossible de faire des généalogies fiables. 

 La déliquescence des États et donc celle des cours de justice a fini par assurer à l'Église l'exclusivité du traitement des litiges touchant aux mariages entre le \crmieme{9} et le \siecle{12}. Au fur et à mesure que son influence sur le droit du mariage grandissait sa définition de l'inceste était bon gré mal gré intégrée par les familles dans leurs stratégies. A-t-elle été pour elle un outil de conquête du pouvoir ? L'extension des limites de l'inceste servait objectivement ses intérêts matériels et politiques en multipliant les risques de nullité et les demandes de \emph{dispense}%
% [18]
\footnote{En ce qui concernait ces interdits il s'agissait d'une question de discipline et non d'une règle de foi. L'Église se reconnaissait donc le droit d'en dispenser les fidèles qui en faisaient la demande, mais cela ne se faisait pas toujours sans frais.}% 
. C'est la thèse de Goody, et elle n'est pas invraisemblable (selon Poly elle est plausible, mais à partir du \siecle{11} seulement),

 ... mais il faut observer qu'au même moment les rois et les autres puissants s'appuyaient sur les mêmes principes pour s'immiscer dans les conflits au sein des familles de leurs dépendants, et pour gérer à leur convenance les transmissions des patrimoines, des héritages, et des fiefs. 

 C'est de la même façon que les autorités civiles se sont opposées à ce que les enfants illégitimes reçoivent le même traitement que les autres, notamment dans les héritages. Ils s'opposaient surtout à ce qu'ils puissent hériter des fonctions fournissant un surcroît d'\emph{honneur}, c'est-à-dire les fonctions de pouvoir. Rois et clercs ont mis des siècles à parvenir à cette fin, mais ils y sont parvenus. C'est qu'ils avaient des intérêts convergents dans l'affaire. Les puissants avaient intérêt à ce que les familles de leurs concurrents et de ceux qui dépendaient d'eux aient des difficultés à trouver un héritier, sachant qu'environ une femme sur cinq ayant l'âge de procréer n'était pas féconde, que suivant les « honneurs » à transmettre, les filles ne convenaient pas aussi bien qu'un garçon ou étaient exclues, suivant les législations ou les coutumes en vigueur (ex. la loi Salique chez les Francs), et que les enfants illégitimes ne pouvaient en hériter. Il était avantageux pour les puissants de plaider l'illégitimité des enfants de leurs ennemis promis à un riche héritage pour les en déposséder, et donner à un autre de leur choix l'honneur qui devait leur échoir. 

 Et l'extension à l'infini de l'inceste rendait paradoxalement plus facile, pour tous ceux qui pouvaient assumer un procès canonique, d'obtenir l'annulation d'un mariage pour inceste si une alliance plus profitable ou une femme plus désirable ou supposée plus féconde se présentait : si tout le monde était parent de tout le monde toutes les unions étaient incestueuses, et donc à la merci d'un procès gagné d'avance (en somme : « si vous ne voulez pas être piégé dans un mariage indissoluble épousez votre petite cousine »). 


\section{Enfants en trop, enfants « irréguliers »}

 Ce n'est pas parce qu'elle était interdite que l'exposition des enfants avait disparu. Les pauvres ont toujours eu recours à l'exposition et n'ont jamais été sanctionnés pour ce motif. Quant aux ventes d'enfants, interdites en principe, elles ne tombaient sous le coup de la loi que lorsqu'elles obéissaient à d'autres motifs que le dénuement%
% [19]
\footnote{... mais qui vendait son enfant pour d'autres raisons (sauf un enfant que le père de famille supposait né d'un adultère de son épouse) ?}% 
. Bien au contraire les acheteurs ont en réalité été encouragés par le fait que les parents qui abandonnaient étaient déchus de leurs droits. Il faudra que le servage disparaisse aux \crmieme{12} et \siecle{13} pour que les ventes d'enfants disparaissent aussi... Et c'est à partir de cette époque que le nombre des expositions de nouveaux-nés dans les villes va se mettre à poser de sérieux problèmes d'ordre public.

 En accord avec la Bible, l'Église a toujours interdit à ses fidèles l'avortement et l'infanticide, et Constantin a introduit cette interdiction dans le droit romain. Qu'en était-il en réalité ? Les avortements et les infanticides ont-ils d'un seul coup disparu ? Il est difficile de le croire. Les infanticides n'ont certainement pas disparu. Ainsi Grégoire de Tours (539-594) rapporte le cas d'une femme qui avait mis au monde un enfant monstrueux : \emph{Comme c'était pour beaucoup un sujet de moquerie de l'apercevoir, et qu'on demandait à la mère comment un tel enfant pouvait être né d'elle, elle confessait en pleurant qu'il avait été procréé pendant une nuit de dimanche. Et n'osant le tuer comme les mères ont coutume de le faire, elle l'élevait de même que s'il eût été conforme}%
% [20]
\footnote{... cité par D.~\fsc{ALEXANDRE-BIDON} et D.~\fsc{LETTE}, p. 27.}% 
. On croit en effet à cette époque que les naissances d'enfants mal conformés sont le résultat de relations sexuelles durant les périodes d'abstinence obligatoire, pendant le carême ou l'avent, pendant les règles%
%[21]
\footnote{Il en est de même pour la lèpre. Il est si difficile de ne pas savoir pourquoi le malheur vous frappe qu'on préfère encore s'en proclamer responsable.}% 
,~etc.

 Mais le plus suggestif c'est le « \emph{n'osant le tuer comme les mères ont coutume de le faire} ». Il faut remarquer la simplicité avec laquelle Grégoire de Tours rapporte ce qui est pour lui une évidence contre laquelle il ne s'indigne pas. Entre les règles morales, même celles qui étaient inscrites dans la loi, et les pratiques effectives, il y avait une marge, comme toujours, et l'infanticide est si aisé et si difficile à prouver. Les nouveaux-nés sont si fragiles, et il arrivait si souvent qu'ils soient étouffés par mégarde sans intention maligne lorsqu'ils partageaient le lit de leur mère, pour avoir plus chaud ou lui éviter de se relever la nuit,~etc. 

 Les avortements ont pu se raréfier en l'absence de médecins et de sages-femmes compétents et prêts à louer leurs services (à supposer que ces compétences se soient effectivement perdues chez les femmes d'expérience, ce qui est à prouver), mais les avortements n'ont jamais été ressentis comme des infanticides, et tout au plus comme des fautes lourdes. Les avortements précoces étaient d'autant moins culpabilisés que pour la plupart des théologiens du Moyen Âge comme pour ceux de l'Antiquité, l'animation du fœtus n'avait pas lieu au moment de la fécondation, mais bien plus tard, chacun défendant sa propre théorie (Cf. Maaike \fsc{VAN DER LUGT}, \frquote{L'animation de l'embryon humain et le statut de l'enfant à naître dans la pensée médiévale}, in \emph{L'embryon, formation et animation}, collectif, déc 2004, Paris, Vrin, p. 234-254). 

 Les enfants issus de simples mésalliances (sénateur--affranchie, femme libre--esclave,~etc.) ne posaient pas de problème religieux aux chrétiens, pas plus qu'aux juifs, même s'ils posaient des problèmes familiaux et sociaux, et même si le droit romain pourchassait ces mésalliances. Ceux dont l'Église réprouvait vraiment la naissance étaient ceux qui avaient été conçus dans le cadre d'une transgression de ses propres lois morales, les \emph{fruits du péché}. 

 Dans ce domaine, les règles de l'Église viennent presque intégralement des juifs. L'échelle de gravité des fautes est calquée sur l'échelle des \emph{mamzerim}. Y ont été ajoutés les enfants nés des personnes qui ont fait vœu de célibat, par analogie avec le sort des enfants illégitimes des prêtres du Temple de Jérusalem. 

 Les \emph{irrégularités de conception} étaient classées comme suit de la moins grave à la plus grave :
\begin{enumerate}
% a)
\item ceux qui ont été conçus dans le cadre d'un concubinage stable, monogame et sans interdit de mariage, et qui n'ont pas (encore) été régularisés par un mariage subséquent ;
% b)
\item ceux qui sont nés d'un rapport de hasard (fornication) ou d'un concubinage qui n'a pas duré ;
% c)
\item ceux qui ont été conçus alors que leur mère se prostituait (fornication) ;
% d)
\item ceux qui sont nés d'un adultère avéré (enfants adultérins) ;
% e)
\item ceux qui sont nés des relations coupables, consenties, d'un clerc ou d'une religieuse ayant fait vœu de célibat (sacrilège) ;
% f)
\item ceux qui sont nés du viol d'une femme mariée (sacrilège) ;
% g)
\item ceux qui sont nés du viol d'une vierge consacrée (sacrilège) ;
% h)
\item ceux qui sont nés d'un inceste. 
\end{enumerate}

 Tous ces enfants étaient illégitimes. Quand ils étaient le fruit des œuvres de leur père avec une servante ou une concubine, ils ont souvent été élevés dans la famille de leur père, au moins pendant le haut Moyen Âge : chez les Germains cela allait de soi. Par contre même s'ils étaient invités à pardonner à leurs épouses infidèles, les maris n'avaient pas l'obligation d'assumer les enfants adultérins de celles-ci. En ce cas l'abandon anonyme était un droit reconnu officiellement, même aux maris fortunés. Mais ils pouvaient aussi les assumer, comme en droit romain. Quant aux enfants nés d'un « sacrilège » ou d'un inceste il est vraisemblable qu'ils étaient le plus souvent traités comme des enfants abandonnés.


\section{Les éducations}

 Pour la plupart des enfants des villes (qui représentent peu de chose à l'époque) la petite enfance se passe à la campagne chez une nourrice. La mise en nourrice a concerné plus d'enfants que tous les internats éducatifs, collèges ou hôpitaux réunis, et de très loin. C'était en effet une nécessité absolue pour les femmes des villes qui exerçaient un métier : elles ne pouvaient consacrer à l'allaitement le temps nécessaire jusqu'au sevrage de l'enfant (à deux ans), et il n'y a eu jusqu'au \siecle{20} aucun substitut valable au lait féminin. La généralisation du nourrissage mercenaire reposait aussi sur la possibilité de gagner (à compétences égales) beaucoup plus d'argent en ville qu'à la campagne. Cela permettait aux citadines, même de ressources modestes, d'acheter le lait et le temps des paysannes. La croissance des villes a donc entraîné une augmentation massive du recours à la mise en nourrice.

 Cela n'a pu se faire aussi largement que parce qu'était peu ou pas perçue l'influence des premières relations de l'enfant avec sa mère ou un substitut sur la construction de sa personnalité : le bébé était censé n'avoir besoin que de lait. Le nourrissage mercenaire est donc une institution dont il a été fort peu parlé pendant des millénaires. Cette pratique n'était ni pensée, ni pensable. Elle était du côté des corps et de la nature, des réalités féminines, au même titre que la grossesse et l'accouchement, qui se faisaient aussi bien quand les hommes n'en parlaient pas, sinon mieux. Seul présentait de l'intérêt pour ces derniers ce qui commençait avec l'âge de raison (7 ans).

 Dès qu'ils ont l'âge de raison les enfants de ces temps ne sont plus regardés comme fondamentalement différents des adultes. Ils ne reçoivent aucune protection spéciale (protection du corps contre les gestes traumatiques, protection des yeux et des oreilles contre les spectacles traumatiques). L'éducation est rude et les sanctions sévères. Celui qui économiserait les verges et le fouet croirait mal faire. Les orphelins continuent d'être l'objet de toutes les attentions des autorités. Quant aux jeunes délinquants, condamnables dès 7 ans, ils perdent à 12 ans \emph{l'excuse de minorité}, qui de toute façon n'est pas automatique même avant cet âge. Ils sont en tout traités comme des adultes.

 Dans l'immense majorité des cas chacun apprend de son père le métier de son père. Dès 6 ans la plupart des enfants travaillent autant qu'ils le peuvent. A partir de cet âge un enfant de pauvre ne coûte plus guère. Sauf chez les riches et des puissants, dès 12 ans chacun gagne réellement le pain qu'il mange chez ses parents ou chez un maître. Le placement réciproque des adolescents chez des alliés des parents (oncles, surtout maternels, suzerain,~etc..) est un outil éducatif souvent employé (les jeunes nobles servent comme pages, les fils de paysans comme pâtres, les marins comme mousses,~etc.). Le placement en apprentissage chez un artisan (maître ès arts) n'est possible que si les parents paient l'apprentissage : c'est un luxe auquel les pauvres ne peuvent pas prétendre. Pour l'école il en est de même. 

 L'Antiquité grecque ou romaine avait élaboré à l'intention de ceux qui pouvaient se le payer un système complet d'enseignement (primaire, secondaire et supérieur). D'autre part un certain nombre de postes de professeur du secondaire étaient financés par les cités. L'état romain finançait des chaires d'enseignement supérieur (Augustin d'Hippone en est un représentant illustre : ses écrits le décrivent successivement élève, étudiant, enseignant et titulaire de chaire). Au \siecle{4} ce système continue de fonctionner, au \siecle{6} il est pratiquement en ruines en Occident, alors qu'il perdurera encore dix siècles à Byzance sans changements de structure. Les universités européennes ne relèveront le flambeau qu'à partir des derniers siècles du Moyen Âge. 

 En attendant seules résistent les écoles cathédrales et monastiques. Les premières écoles \emph{épiscopales} ou \emph{cathédrales} fleurissent au \siecle{4}. Elles ont pour principal objet de former les futurs clercs, mais les élèves peuvent à la fin du cursus refuser d'entrer dans le clergé. Le \emph{deuxième concile de Vaison} (529) prescrit à chaque prêtre chargé de paroisse de mettre en place une \emph{école paroissiale} à l'intention des jeunes les plus vifs d'esprit. Ce sont les premières \emph{petites écoles}. Elles sont d'abord destinées à alimenter \emph{l'école cathédrale} en sujets d'élite destinés à former le personnel ecclésiastique, et n'ont pas pour but d'apprendre à lire à tous comme c'est le cas chez les juifs : la vie religieuse du chrétien n'exige pas qu'il sache lire, il suffit qu'il sache entendre. Son activité professionnelle ne l'implique pas non plus : l'enseignement lettré (maîtrise du latin, langue de la culture et des savants) est inutile à qui ne sera pas clerc. Si l'on cherche le pouvoir il est alors plus efficace de connaître les armes que la rhétorique ou le droit. Les \emph{écoles monastiques} apparaissent au \siecle{4}, mais elles ne prennent en principe que des enfants destinés à devenir moines (« donnés » très jeunes à Dieu par leurs parents) et seuls apprennent le latin, les moines « de chœur », ceux qui chantent dans le chœur, ceux qui pourraient être ordonnés prêtres. Pourtant bien des écoles monastiques acceptent aussi quelques jeunes qui ne sont pas destinés à devenir moines, et Charlemagne leur en fera l'obligation.
 
 
 