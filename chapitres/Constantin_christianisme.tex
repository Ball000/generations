
 Constantin inscrit en 313 le christianisme dans la liste des cultes reconnus%
% [1]
\footnote{S'il a choisi le christianisme c'est sûrement pour étayer son pouvoir \emph{(hoc signo, vinces)}, c'est peut-être aussi par conviction personnelle. Pour \fsc{VEYNE} c'est d'abord par conviction personnelle (\emph{Quand l'empire romain est devenu chrétien}, 2007). Moins on prête d'effectifs aux chrétiens, et donc d'influence, et plus on doit prêter de conviction personnelle à Constantin. Il n'a été baptisé que peu de temps avant sa mort, comme il était fréquent de son temps : le baptême pardonnait toutes les fautes antérieures, mais une fois baptisé il n'y avait pas encore de dispositif de purification prévu pour les pécheurs irréductibles. }% 
. Son action a été poursuivie par ses successeurs, à l'exception de Julien%
%[2]
\footnote{Sources :
\\Danièle \fsc{ALEXANDRE-BIDON} et Didier \fsc{LETT}, \emph{Les enfants au Moyen-Âge, \siecles{5}{15}}, 1997.
\\Didier \fsc{LETT}, \emph{Famille et parenté dans l'Occident médiéval, \siecles{5}{15}}, 2000.
\\Peter \fsc{BROWN}, \emph{Genèse de l'Antiquité tardive}, 1978.
\\Peter \fsc{BROWN}, \emph{Pouvoir et persuasion dans l'antiquité tardive, vers un empire chrétien}, 1992.
\\Peter \fsc{BROWN}, \emph{Le renoncement à la chair, virginité, célibat et continence dans le christianisme primitif}, 2002.
\\Jean-Michel \fsc{CARRIE}, Aline \fsc{ROUSSELLE}, \emph{L'Empire romain en mutation, des Sévères à Constantin}, 192-337, 1999.
\\Christian \fsc{DELACAMPAGNE}, \emph{Une histoire de l'esclavage, de l'antiquité à nos jours}, 2002.
\\Jean \fsc{DURLIAT}, \emph{De l'antiquité au Moyen-âge, l'Occident de 313 à 800}, 2002.
\\Alexandre \fsc{FAIVRE}, \emph{Naissance d'une hiérarchie, les premières étapes du cursus clérical}, 1977.
\\Bertrand \fsc{LANCON}, \emph{Le monde romain tardif, \siecles{3}{7} ap. J.C.}, 1992.
\\Henri-Irénée \fsc{MARROU}, \emph{Décadence romaine ou antiquité tardive ? \siecles{3}{6}}, 1977.
\\Henri-Irénée \fsc{MARROU}, \emph{L'Église de l'Antiquité tardive}, 303-604, 1963.
\\Paul \fsc{PETIT}, \emph{Histoire générale de l'Empire romain, 3, le Bas-Empire (284-395)}, 1974.
\\Aline \fsc{ROUSSELLE}, \emph{La contamination spirituelle, science, droit et religion dans l'antiquité}, 1998.}% 
. Depuis Jules César, l'empereur cumulait sa fonction avec celle du \emph{grand pontife}, le prêtre le plus important de Rome. Ce dernier avait le pas sur tous les prêtres des cultes autorisés, et supervisait leur bon fonctionnement. Constantin a exercé cette fonction jusqu'à sa mort%
%[3]
\footnote{Les Empereurs chrétiens se dessaisiront de ce titre vers 380 et l'évêque de Rome en héritera un siècle plus tard.}% 
. Est-ce en tant que Grand Pontife ou en tant qu'empereur qu'il a permis le culte des chrétiens ? Toujours est-il qu'en faisant ce choix il orientait l'histoire de l'Occident jusqu'à aujourd'hui.

 À partir de cette date il a traité les évêques comme étaient traités les clergés des autres religions reconnues, c'est-à-dire comme un corps de magistrats religieux associés au pouvoir civil. Ils ont accepté cette reconnaissance comme un développement providentiel de l'histoire du salut, comme la reconnaissance de leurs droits. Pour eux comme pour tous leurs contemporains il était inimaginable que l'état se désintéresse des religions, c'est-à-dire des dieux, et qu'il les relègue dans la sphère privée, au risque que l'un d'entre eux ne se venge cruellement de cette négligence discourtoise. 

 L'Église bénéficie désormais de tous les droits des cultes reconnus par l'État : droit de recevoir dons et legs, inaliénabilité des biens fonciers, exemptions d'impôts,~etc. L'empereur la dote de bâtiments et de propriétés terriennes%
% [4]
\footnote{Selon les thèses (controversées mais séduisantes) de \fsc{DURLIAT} (2002), il pouvait s'agir de la \emph{propriété éminente} de \emph{villae}, qu'il faut distinguer de la propriété ordinaire, de la propriété \emph{utile} (c'est-à-dire le droit de vendre, et d'acheter la terre, le droit de la mettre en valeur et de jouir des fruits du sol). Selon lui la \emph{villa} était à cette époque \emph{une circonscription fiscale} peuplée notamment de cultivateurs (nommés \emph{colons} en langage administratif) qui pouvaient être propriétaires de leur exploitation agricole, ou simples tenanciers. A cette époque le terme villa pouvait aussi désigner un château et ses terres, ce qui crée des confusions. Celui qui possédait la propriété \emph{éminente} d'une villa fiscale, nommé le \emph{dominus}, était chargé d'y faire la collecte des impôts : le \emph{cens}, les droits de mutation,~etc. Ceux-ci correspondaient selon les estimations de \fsc{DURLIAT} à environ 20~\% de l'ensemble des revenus des contribuables, fournis en monnaie, en nature, ou en corvées, et dont une partie revenait au \emph{dominus} pour prix de ses services. Il était en quelque sorte le percepteur de cette \emph{villa}. C'était une charge lucrative et honorable tout à la fois. Les grandes fortunes de l'empire romain reposaient sur ces propriétés très particulières, d'ailleurs non exclusives de la propriété utile des mêmes domaines. On pouvait recevoir ces charges de l'empereur, ou les transmettre par héritage, les vendre et les acheter (comme la \emph{ferme des impôts} sous l'Ancien Régime). Le \emph{dominus} assurait le lien entre les colons concernés et les administrations de l'État. Il était en quelque sorte le Seigneur de cet espace. Par certains aspects cela préfigurait les \emph{Seigneuries} apparues à partir du milieu du Moyen-âge.}% 
. Pour ne pas les obliger à sacrifier aux dieux civiques, les clercs sont exemptés des charges curiales, c'est-à-dire de l'obligation pour les plus fortunés de participer à la curie, au conseil municipal de leur cité%
%[5]
\footnote{... ce qui était à la fois un honneur et un impôt, puisqu'ils devaient financer de leurs propres deniers certaines des dépenses de celle-ci. Cf. P. \fsc{VEYNE}, 1976.} 
. L'exemption des charges curiales (dont bénéficiaient aussi un certain nombre de prêtres des cultes civiques) leur permettait de consacrer leur temps et leur fortune à d'autres formes d'\emph{évergésies} que celles que devaient traditionnellement pratiquer les curiales. On comptait sur eux pour investir dans l'assistance aux pauvres, la construction d'hôpitaux, etc. 

 Il n'existait pas à Rome de pouvoir judiciaire indépendant des autres administrations impériales. Comme tous les autres magistrats de l'Empire, prêtres païens y compris, les évêques ont donc reçu le droit de régler les litiges qu'on leur soumettait et qui ressortaient de leurs compétences. C'était reconnaître officiellement le rôle de juridiction qu'ils exerçaient déjà depuis longtemps, notamment en matière familiale et doctrinale. Comme les prêtres des cultes reconnus jusque là ils ont reçu le pouvoir d'enregistrer les affranchissements d'esclaves : leurs actes écrits avaient désormais force de preuves, et leurs arbitrages devaient d'autant plus être respectés qu'ils avaient la faveur du prince, ce qui se voit entre autres preuves au fait qu'ils reçoivent le pouvoir de juger seuls des fautes des clercs.

 Constantin ne se borne pas à favoriser le culte chrétien, il initie aussi la séparation de l'État et des religions païennes. Celle-ci va s'effectuer par étapes entre 325 et 391. En 382 l'empereur supprime les privilèges des vestales et des prêtres païens et interdit aux cités de financer les temples païens (nomination des prêtres, entretien des bâtiments, fourniture des offrandes pour les sacrifices, achat d'encens,~etc.). Progressivement les biens et les terres de ceux-ci sont confisqués au profit du trésor public. En 391 les cultes non chrétiens sont interdits, à l'exception du judaïsme, tandis que le christianisme devient religion d'État. 

 Les évêques n'avaient pourtant pas tout pouvoir sur Constantin ni sur ses successeurs. C'était au moins aussi souvent le contraire. En fait rien ne changeait dans la nature des liens entre le pouvoir et la religion officielle. C'est Constantin qui en 325 a convoqué dans son palais de Nicée le premier concile œcuménique%
% [6]
\footnote{... c'est-à-dire qu'il rassemblait tous les évêques vivants. Jusqu'au \siecle{11} tous les conciles seront convoqués par les autorités civiles.} 
de l'Église, et c'est lui qui l'a présidé. A l'issue de ce concile c'est lui qui a donné force de lois à ses décisions en les contresignant. Il s'agissait d'un échange de légitimités entre empereur et évêques, d'un étayage réciproque, conforme aux traditions antiques de confusion de la religion et de la cité. Dans l'œuvre législative réalisée sous le règne de Constantin il faut donc tenir compte non seulement de l'influence de l'Église, évidente, mais aussi de sa marque personnelle, et de l'évolution générale des attitudes romaines face aux mœurs, au sexe, au couple et à la famille. Lorsque l'un des successeurs de Constantin aura pris fait et cause pour l'hérésie arienne ce seront les catholiques qui subiront la défaveur du prince.

 Selon son propre mot Constantin se considérait comme « l'évêque du dehors », l'évêque (\emph{episcopos} : le surveillant, l'inspecteur) des non chrétiens. Il estimait de son devoir de conduire l'ensemble de ses sujets, chrétiens ou non, à la vérité telle qu'il la concevait, et à défaut de les convertir tous, ce qui était du ressort des évêques, il entendait au moins mettre les lois en accord avec les principes chrétiens (avec ceux du moins qu'il approuvait) et créer ainsi un milieu de vie qui favoriserait les conversions.

 Mais l'évolution ne s'arrêtera pas à la simple reconnaissance du christianisme comme \emph{religio licita}. Le petit-fils de Constantin décrète en 391 que seule la religion chrétienne est désormais autorisée. Tous les sujets de l'empereur sont à partir de cette date fermement invités à professer publiquement la foi que celui-ci leur désigne : la doctrine définie par l'évêque de Rome%
% [7]
\footnote{... ce qui en soi est nouveau : il n'y avait pas de profession de foi consciente et articulée dans les religions antiques, du moins pas avant le triomphe du christianisme.}% 
. Tous les autres cultes sont interdits. Seule la dissidence juive continuera d'être tolérée, comme une espèce de « butte témoin » des temps pré chrétiens et de l'ancienne alliance, mais il lui sera interdit de faire des prosélytes, surtout parmi les chrétiens%
%[8]
\footnote{La conversion des chrétiens ou des païens au judaïsme était interdite, de même que les mariages mixtes. Si un juif faisait circoncire son esclave chrétien cela entraînait \emph{ipso facto} l'affranchissement de ce dernier, s'il le réclamait. Dès 313 Constantin condamnait à mort les juifs qui lapidaient ceux de leurs religionnaires qui se convertissaient au christianisme : encore une fois ce n'était un temps de liberté religieuse dans aucun des camps en présence.}% 
. 

 Si Constantin n'avait pas récusé le titre de \emph{Pontifex Maximus}, ses successeurs ne porteront plus ce titre, sauf Julien qui a tenté de remettre la religion traditionnelle à l'honneur. Malgré de fortes tentations et une sacralisation de leur fonction assez ambiguë, aucun empereur chrétien, aucun roi n'osera%
% [9]
\footnote{... jusqu'à Henri VIII, roi d'Angleterre.} 
se proclamer dignitaire de l'Église. C'est donc l'évêque de Rome qui héritera du titre de pontife. Cela ne les empêchait pas de se considérer comme les partenaires permanents et obligés des évêques, comme les soutiens de leur pouvoir et le bras armé qui défendait leurs enseignements. Soucieux d'ordre et d'unité les empereurs chrétiens et les rois qui vont leur succéder seront étroitement associés à l'Église dans le choix des évêques, les arbitrages théologiques et disciplinaires, la convocation des conciles, la définition de la discipline ecclésiastique, etc. Jusqu'au milieu du moyen-âge leurs édits, décrets et codes concerneront le fonctionnement interne des églises et la discipline ecclésiastique au même titre que celui des autres corps de la société, et les décisions des conciles n'auront force exécutoire que pour autant qu'ils les auront approuvées et appuyées de leur autorité. Le sacre des empereurs et l'onction des rois seront l'objet d'une valorisation qui en fera presque un sacrement analogique à l'ordination des clercs. À Constantinople l'Église et l'Empereur vont être inséparables pendant un millénaire. Il s'agissait de deux pyramides hiérarchiques symétriques, étayées l'une par l'autre et se chargeant à elles deux de régir l'empire. 

 Certes, au premier degré d'analyse l'évangélisation des populations a été l'œuvre de l'Église, mais elle a été très vigoureusement appuyée dans cette tâche par les pouvoirs publics, sans lesquels son action était vouée à l'échec : il s'agissait d'abord et avant tout de convertir les rois. Le reste suivait presque mécaniquement. En dépit de la doctrine ecclésiale qui voulait que les candidats au baptême agissent librement et de leur propre initiative, ils l'ont parfois fait sous la contrainte, et tous les évêques n'ont pas protesté contre les pressions que subiront les « païens », les « hérétiques » (ariens en particulier) et les juifs (notamment en Espagne)%
% [10]
\footnote{Sur ce sujet on peut se référer notamment au livre de Bruno \fsc{DUMEZIL}, \emph{Les racines chrétiennes de l'Europe, Conversion et liberté dans les royaumes barbares \siecles{5}{8}}, Fayard, 2005, Paris.}% 
. 

 L'Église se définissait elle-même comme un partenaire qui n'avait pas vocation à l'exercice du pouvoir temporel. Quand un prêtre, un évêque ou un pape prétendait gouverner les affaires terrestres, ce qui n'a jamais manqué de se produire, les responsables laïcs pouvaient le renvoyer à l'Évangile : \emph{rendez à César ce qui est à César, et à Dieu ce qui est à Dieu} (Mt 22, 17-21). Mais en contrepartie de ce renoncement les clercs prétendaient à l'exclusivité sur le culte, sur les dogmes qui servaient de cadre de pensée à tous, et de plus en plus sur l'enseignement, sur l'assistance, et sur les mœurs, notamment familiales. 

 Pourtant les décrets et rescrits promulgués par Constantin et ses successeurs n'ont pas fait disparaître les pratiques antérieures d'un trait de plume, ni transformé toutes les familles de l'empire romain, plus ou moins bien baptisées, ou pas baptisées du tout, en autant de \emph{saintes familles}%
% [11]
\footnote{Sources : Jean-Pierre \fsc{LEGUAY}, \emph{L'Europe des états barbares, \siecles{5}{8}}, Belin, Paris, 2002. Jean-Pierre \fsc{POLY}, \emph{Le chemin des amours barbares, Genèse médiévale de la sexualité européenne}, 2003. Stéphane \fsc{LEBECQ}, \emph{Les origines franques, \siecles{5}{9}, Nouvelle histoire de la France Médiévale}, 1990.}% 
. Les non chrétiens, encore majoritaires en 313, ont certes fait de la résistance face aux décrets de Constantin, mais les baptisés aussi. L'ordre public possède ses propres logiques : la société a toujours toléré ou soutenu bien des choses que les évêques réprouvaient. Inversement l'Église a défendu avec une persévérance millénaire des positions que la société n'a jamais cessé de considérer comme idéalistes, impraticables ou irresponsables. Les sociétés n'ont jamais été une pâte malléable dans les mains du clergé, qui était lui-même divisé sur bien des sujets de morale personnelle et familiale, et sur bien des points d'accord avec ses ouailles. En dépit du poids de l'Église, il n'y a jamais eu de coïncidence rigoureuse entre les lois civiles et les prescriptions religieuses. Face à celles-ci les autorités civiles ont suivant les cas adopté toutes les positions possibles : de la collaboration intéressée et active, et même pressante, quand cela allait dans le sens de leurs intérêts, à l'opposition franche, en passant par l'inertie sceptique.

