
\chapter{Mœurs juives}

\section{Le pur et l'impur}

 Les catégories du pur et de l'impur étaient investies par tous les peuples de l'antiquité, mais tout particulièrement par les hébreux. Diverses choses rendaient impur, « souillaient », et exigeaient donc un rite et/ou un délai de purification : ne pas respecter le Shabbat ; manger sans avoir fait préalablement les ablutions rituelles ; manger des aliments pour lesquels la dîme (un impôt du dixième de chaque production) n'avait pas été payée ; approcher les non circoncis (impurs par excellence) et surtout manger avec eux, comme eux, etc. Rendaient impur l'exercice de certains métiers sales ou malodorants (tanneur, vidangeur, etc.) ; le fait de s'approcher d'un cadavre d'homme ou d'animal ; de faire couler le sang ; de consommer du sang, de consommer des animaux impurs (porcs, poissons sans écailles, etc.), des animaux abattus de façon irrégulière, ou morts d'accident (non abattus rituellement), ou abattus depuis trop longtemps, ou cuisinés de manière irrégulière, etc. Étaient impurs tous les écoulement issus des organes génitaux (règles, écoulement séminal spontané ou non, maladies vénériennes, etc.), et toutes les maladies de peau (« lèpre », etc.). L'impureté c'était aussi « connaître » charnellement une femme ou un homme, même son conjoint légitime. Toucher l'autel du Temple sans être dans l'état de pureté convenable souillait l'offrande et invalidait le sacrifice. Toucher à mains nues les objets consacrés au culte du Temple « souillait les mains », toucher à mains nues les rouleaux de la Thora « souillait les mains », etc.

 Les prêtres et les lévites respectaient les mêmes interdits que les autres hébreux mais en outre ils n'avaient pas le droit de porter des armes, puisque celles-ci versaient le sang et étaient impures par nature. Ils avaient encore moins le droit de s'en servir. Verser le sang humain leur interdisait définitivement de remplir leurs fonctions cultuelles. Dans le même sens ils devaient rester éloignés de tout blessé susceptible de les souiller par son sang, et se tenir à l'écart de tout cadavre humain et animal, sauf celui de leurs parents les plus proches. C'est d'abord durant leur semaine de service au Temple que les prêtres devaient éviter toute impureté. Les actes sexuels leur étaient interdits durant les jours de purification préalable et durant tous leurs jours de présence à l'autel. Mais puisque seuls les fils de la Tribu de Lévi pouvaient servir le Temple, ils devaient \emph{se procurer une descendance} en dehors de leur temps de service. Plusieurs dizaines de siècles auparavant il en était de même en Égypte (Serge \fsc{SAUNERON}, \emph{les prêtres de l'ancienne Égypte}, Editions du Seuil, 1998 (première édition 1957); Cf. page 47 et suivantes) obligation de la circoncision, du rasage intégral de la tête et du corps, d'ablutions répétées à heures fixes, du respect de divers interdits alimentaires, de jeûnes, abstinence précédant de plusieurs jours et accompagnant les périodes de service au temple, interdiction de la polygamie, interdiction de porter certains tissus, etc. 

 Les règles de pureté contenaient de nombreuses contradictions : c'est que la notion de pureté est complexe : à un premier niveau, le plus archaïque sans doute, la pureté n'était comme partout qu'un problème de frontières. Le divin et le démoniaque, le surnaturel faste et le surnaturel néfaste étaient également opposés au monde sans danger, banal et familier des hommes ordinaires. En ce sens-là le contact à mains nues des rouleaux de la Tora était impur, puisqu'ils étaient sacrés. Dans le coït l'homme et la femme participaient de manière directe à l'œuvre du Créateur. C'est pourquoi le coït le plus conforme au « \emph{croissez et multipliez} » de la Genèse, le plus légitime, le plus innocent, le plus sain, pour ne pas dire le plus saint, rendait \emph{impur jusqu'au soir}. Dans ce cadre on pouvait résumer ainsi : pur = inoffensif, profane, normal, quotidien, vivant, animal conforme à la norme de sa classe d'animaux (ex. : poisson avec écailles, ruminant au sabot fendu, etc.). Impur = sacré, divin, démoniaque, prodige, sang, sperme, coït, naissance, mort, animal anormal, réunissant les caractéristiques anatomiques de plusieurs classes d'animaux, par exemple le porc au sabot fendu mais non ruminant, les poissons sans écailles, etc. 

 Selon une deuxième perspective la pureté et l'impureté étaient des propriétés des choses et des corps. Dans ce cadre de pensée on pouvait faire les oppositions suivantes : pur = vivant, propre, net, limpide, clair, beau, harmonieux, gracieux, habile, droit, sain, intègre, jeune, neuf, vierge, intact ; impur = mort, sale, trouble, louche, sombre, laid, difforme, gauche, maladroit, malade, lépreux, infirme, mutilé, vieux, usagé, usé, abîmé, cassé, défloré, marqué d'un sceau, d'une marque de propriété.

 Ces deux acceptions du sacré et du profane étaient également présentes dans toutes les religions de l'antiquité. Même si chacune d'entre celles-ci variait dans le détail de ses classifications elles définissaient toutes de la même façon les impuretés essentielles : le coït, le sang, le sperme, la naissance, la mort, etc. Mais à un troisième niveau les hébreux enrichissaient la notion de souillure de connotations morales d'une manière plus affirmée que ne le faisaient les autres peuples de l'antiquité : Pur = bon, bien, vrai, juste, droit, sincère, bienveillant, honnête, intègre, innocent, juste, équitable, désintéressé. Impur = mauvais, mal, faute, faux, pervers, menteur, malveillant, méchant, injuste, inéquitable, intéressé, coupable.

 Dans cet approfondissement les prophètes d'Israël avaient joué un rôle déterminant. En effet ils avaient interprété l'histoire des relations de leur peuple avec le Seigneur comme celle d'une relation amoureuse entre deux personnes. Dans leur bouche leur dieu parlait comme un époux épris, blessé par l'infidélité de son épouse au moins autant que comme un père ou un maître tout-puissant et jaloux. 
Les prophètes suivaient presque toujours le canevas suivant :
\begin{enumerate}
% A)
\item Il n'est ni facile ni drôle d'être le peuple élu,
 \begin{enumerate}
 % 1)
 \item parce que les règles de pureté auxquelles il s'est engagé le séparent des autres, et le rendent infréquentable, alors qu'il voudrait qu'on l'aime, comme tout le monde,
 % 2)
 \item parce que ces règles sont pour lui-même un joug et un carcan, et qu'il lui est impossible de les observer toutes,
 % 3)
 \item parce qu'elles lui interdisent les jouissances faciles et sans dangers des autres peuples,
 % 4°)
 \item parce que ses prétentions à l'exclusivité de l'élection divine le rendent odieux au reste du monde. 
 \end{enumerate}
% B)
\item Voilà pourquoi Israël finit toujours par regarder ailleurs. Il oublie constamment les promesses faites par ses pères et se montre infidèle. 
% C)
\item Mais Le Seigneur est un Dieu jaloux. Il laisse sa colère s'abattre sur son peuple. Il le châtie, c'est-à-dire qu'Il permet à ses ennemis de l'accabler. 
% D)
\item Face à ces maux Israël se repent et revient vers Lui, parce qu'il n'y a pas d'autre dieu.
% E)
\item Il est fidèle et tient ses promesses. Il ne sait pas résister à la prière de son peuple. Il le délivre et disperse ceux qui le tourmentaient.
\end{enumerate}

 À partir du moment où les rapports du Seigneur et de son peuple devenaient une histoire d'amour, les problèmes de pureté et d'impureté ne pouvaient plus être traités de manière rituelle. Ce qui comptait désormais c'était la réponse à la question : « \emph{est-ce que tu m'aimes ?} ». Cet amour poussait des personnes des deux sexes à faire le vœu de devenir \emph{Nazir}, « consacré ». Ils (elles) rasaient alors leur chevelure et durant un temps plus ou moins long. Ils s'abstenaient de certaines nourritures et boissons comme de toute relation sexuelle. Il a existé pendant au moins un siècle et demi des communautés dont le style de vie était quasi monastique, les Esséniens, et qui fuyaient toute impureté. 

 La véritable impureté c'était désormais l'infidélité et le refus de reconnaître la faute commise. Pour les prophètes la véritable offrande de réparation, la seule qui ne serait jamais refusée, ce n'était plus la bête de choix, c'était un « cœur brisé », c'est-à-dire un repentir sincère. Le paradoxe est que ce nouveau point de vue n'abolissait aucune des exigences rituelles. La propension des amoureux est au contraire d'en rajouter, d'en faire plus que ce que l'usage ne prescrit. Les exigences rituelles devenaient une façon de parler, une façon de prier. 

 À partir de la destruction du Temple de Jérusalem (en l'an 70 de notre ère), les sages (rabbins) vont plus ou moins soumettre l'ensemble du peuple aux règles de pureté qui jusque là n'étaient imposées qu'à la caste sacerdotale.


\section{Morale et société}

 La justice était inséparable de la religion. A côté des fonctions cultuelles ou de direction spirituelle la fonction de prêtre ou de sage impliquait aussi de dire le droit et d'arbitrer les conflits soumis par les fidèles. Même en diaspora les tribunaux des synagogues jugeaient les affaires que leurs membres voulaient bien leur soumettre. Cela posait problème si l'un des plaignants refusait le jugement. Les communautés de la diaspora n'avaient en effet aucun intérêt à ce que les autorités civiles se mêlent de leurs affaires. L'impossibilité où elles se trouvaient d'exercer une contrainte physique sur leurs membres devait donc être compensée par l'autorité morale de leurs juges. Il fallait que leur équité soit indiscutable. Lorsqu'il s'agissait de mesurer la gravité d'un acte l'intention du sujet était déterminante. Chacun ne répondait que pour lui-même. Il n'y avait pas de responsabilité héréditaire ou collective. Les châtiments s'appliquaient à la seule personne des criminels et à leurs biens. Parmi les délits on trouvait l'inceste, le meurtre, le vol, mais aussi la profanation du sabbat, les jurons contre le Seigneur, l'idolâtrie, la sorcellerie, l'adultère... Le meurtre était la faute la plus grave. Les sacrifices humains étaient prohibés depuis Abraham. Tout ce qui s'en rapprochait de près ou de loin l'était aussi. Les spectacles de gladiateurs étaient interdits à un double titre : d'une part comme meurtres, d'autre part comme sacrifices idolâtres, puisque leur origine se situait dans le cadre du culte des ancêtres. Même la chasse et les spectacles sanglants où étaient abattus des animaux étaient prohibés : en effet ils étaient impurs puisqu'on y versait le sang. 

 Maltraiter autrui était une \emph{faute contre YHWH}, et cette faute était aussi grave que de ne pas rendre un culte à celui-ci ou de se prosterner devant les idoles. La Tora revenait sans cesse sur le devoir d'aider les pauvres, et d'abord les veuves et les orphelins. Il était pieux et méritoire d'entretenir les orphelins (en grec \emph{orphanos} pouvait englober les enfants abandonnés sans parents. En était-il de même quand c'étaient des juifs qui parlaient ?) de les recueillir et de les élever, de doter les orphelines et de les marier. C'était le prototype de l'œuvre vraiment bonne (\fsc{COHEN}, 1980, p. 225). Ce faisant il ne s'agissait pas de les adopter, mais de les élever comme des \emph{alumnii}, comme des enfants choisis, comme des enfants nourriciers ou spirituels. 

 Les pauvres avaient droit à ce dont ils avaient besoin pour vivre. L'aumône (\emph{tzedaka} = justice) rétablissait l'équité, puisque selon la Tora les richesses des riches ne leur avaient été confiées par Dieu qu'en gérance et non en pleine propriété (la punition de celui qui s'en dispensait était laissée à la discrétion du Seigneur puisque ce n'était qu'une obligation morale). Certaines dîmes spéciales étaient affectées aux pauvres, ce qui impliquait une administration collective de l'assistance (caisses de secours, etc.). Il était interdit de saisir ce qui était nécessaire aux débiteurs pour vivre. 

 Quand aux étrangers, résidents ou de passage, la Tora prescrivait de les traiter comme avec équité et sans discrimination : les ancêtres des juifs étaient eux aussi des étrangers quand ils vivaient en Égypte. 

 Comme chez les autres peuples de l'antiquité l'hospitalité était un devoir pour tous et à l'égard de tous, à charge de revanche. Pas plus qu'ailleurs elle n'était illimitée. Il s'agissait d'un droit moral à un hébergement ponctuel (trois nuits, sauf maladie ou blessure). Au-delà le voyageur était invité à pourvoir lui-même à ses besoins en travaillant. 

 En diaspora la judéité se superposait à la citoyenneté locale. C'était une citoyenneté comme une autre, puisqu'elle avait des effets au regard de la loi civile, même romaine, et il est évident qu'il y avait une forte ressemblance entre la notion de mamzer et celle d'infâme, avec son caractère transmissible par contact relationnel et par hérédité, etc. Mais celui qui le désirait pouvait entrer dans le peuple juif, par la circoncision, ou par le mariage avec un juif, de même qu'il était possible d'en sortir à Rome en sacrifiant aux dieux, ce qui ne donnait pas pour autant la citoyenneté romaine, mais si on la possédait déjà on pouvait rejoindre les autres citoyens non juifs sans les contraintes des règles de pureté, alors qu'il était presque impossible de devenir citoyen d'une cité grecque autrement que par naissance.


\section{Exaltation de la sexualité conjugale}

 Dans les premières pages de la Genèse la Tora met en scène la différence sexuelle et la génération : \emph{Dieu créa l'homme à son image, à l'image de Dieu il le créa, homme et femme il le créa. Dieu les bénit et leur dit : Soyez féconds, multipliez, emplissez la terre et soumettez-la.} (Gn 1, 27-28). \emph{C'est pourquoi l'homme quitte son père et sa mère et s'attache à sa femme, et ils deviennent une seule chair.} (Gn 2, 24)

 Selon \fsc{CHOURAQUI}, aux yeux des hommes de la Bible%
% [1]
\footnote{A.~\fsc{CHOURAQUI}, \emph{La vie quotidienne des hommes de la bible}, 1978, p.153-155.}%
 : \emph{L'activité sexuelle normale et licite est un bien. Elle constitue même l'objet du premier commandement que Dieu donne à l'homme dans la Genèse au terme de la création du monde : « fructifiez, multipliez, remplissez la terre... ». La vie sexuelle n'est d'ailleurs pas dissociée du couple et aucun mot n'existe en hébreu pour la désigner comme telle...}

 \emph{Celle-ci est étalée au grand jour et fait l'objet d'une législation très stricte, très abondante et très détaillée qui prouve moins la vertu du peuple de la Bible que l'importance pour lui de ces problèmes. Les documents, faits divers ou lois, que la Bible nous lègue sur ce thème n'ont sans doute aucun parallèle dans aucune civilisation de l'Antiquité...}

 \emph{La femme mariée est consacrée, sanctifiée, mise à part pour son époux. De ce fait, la copulation provoque une impureté comme tout contact avec le sacré. Après le coït, le couple doit faire ses ablutions et se purifier : il restera impur jusqu'au soir. La loi est la même pour l'homme après toute copulation ou toute perte séminale. L'activité sexuelle, de quelle nature qu'elle soit, introduit l'homme dans l'univers du sacré. Il doit être purifié pour retrouver la plénitude des fonctions profanes. Aussi les mœurs tendent-elles à une ségrégation des sexes.}

 \emph{Toute activité sexuelle est prohibée avec une femme qui a ses règles, et tout contact direct ou indirect avec elle est également interdit. La perte du sang provoque l'impureté de la femme... Car le sang, c'est la vie, et la perte du sang menstruel, comme les suites d'un accouchement, placent la femme dans la zone redoutable et mystérieuse qui se situe entre la vie et la mort, entre les pôles du pur et de l'immonde, qui définissent les termes majeurs de la dialectique biblique. En fait l'acte sexuel n'est ainsi permis qu'aux époques de fécondité de la femme et interdit quand elle est stérile.}

 \emph{ Les interdits sexuels pleuvent dans la législation et les peines sont d'une redoutable sévérité : la mort par lapidation ou par « tranchement du peuple ». À l'opposé de la licence qui règne dans ce domaine dans toute l'Antiquité ... on constate dans la Bible un effort quasi désespéré qui tend à discipliner et à orienter l'activité sexuelle du couple.}

 \emph{L'homosexualité ... est qualifiée « d'abomination » ... Les prophètes et les législateurs hébreux sont sans doute les premiers à la prohiber avec une implacable sévérité.}

 \emph{La bestialité ... est, elle aussi, punie de mort ... la prostitution est condamnée par la loi, mais elle subsiste en fait ...}

 \emph{Les précautions prises pour définir l'activité sexuelle illicite mettent en relief le caractère profondément original de l'amour et de la vie du sexe selon les hébreux. Leur souci majeur a été de provoquer une démythisation, une démystification, une libération et une sacralisation de l'activité sexuelle du couple.} »

 La Tora glorifiait l'amour entre l'homme et la femme (cf. \emph{le Cantique des cantiques}) et l'amour des parents pour les enfants. Le célibataire n'était pas considéré comme un homme complet et le célibat définitif n'était accepté qu'en cas d'incapacité complète de procréer. En effet chacun se devait d'avoir une descendance, et la stérilité était un malheur. Les épouses de ceux qui étaient décédés sans enfants devaient leur en donner dans le cadre du \emph{Lévirat}, en se mariant avec un de ses frères. La répudiation d'une épouse avait souvent ce motif. Si la répudiation de l'épouse était autorisée, elle n'était pourtant pas bien vue ; \emph{Je hais la répudiation, dit le Dieu d'Israël} (Malachie, 2, 16)

 Les pratiques sexuelles qui ne peuvent aboutir à une conception étaient interdites. Les relations homosexuelles masculines étaient considérées comme abominables et en théorie punies de mort quel que soit le statut ou l'âge des partenaires. 

 L'exposition des nouveaux-nés était interdite, sauf en situation d'absolue détresse. De même il était interdit de mettre à mort un enfant quel qu'il soit. C'était la conséquence directe du « \emph{tu ne tueras point} » du Décalogue. Les avortements étaient considérés jusqu'à un certain point comme des assassinats, sauf risque pour la vie de la femme, préférée en cas de danger mortel à celle du fœtus. Ceci étant dit les juifs adhéraient comme l'ensemble des gens de l'antiquité à l'idée que le fœtus ne devenait humain qu'au bout d'une certaine durée, avant laquelle il n'était pas animé, ce qui autorisait l'avortement. Sur cette durée les opinions variaient : quarante jour pour les garçons ? Quatre vingt ou quatre vingt dix pour les filles ? 

 Ces prescriptions favorisaient les naissances et l'équilibre du ratio garçons -- filles. Les hébreux en ont-ils eu conscience ? Bien sûr, même si ce n'était pas leur objectif premier. Leurs contemporains étaient parfaitement conscients des effets à long terme de ces comportements natalistes et nous en ont laissé des témoignages%
% [2]
\footnote{Cf. les commentaires de Tacite, dans Aline \fsc{ROUSSELLE}, 2001.}%
.

 La polygynie était permise : à côté des épouses légitimes, celles qui avaient reçu une dot de leur père et dont le mariage avait fait l'objet d'un contrat écrit, on admettait des concubines : femmes libres sans dot ou bien esclaves acquises à prix d'argent ou reçues comme butin de guerre%
% [5]
\footnote{Selon le Talmud l'homme qui voulait avoir plusieurs épouses ou concubines devait obtenir l'accord de sa première épouse, pouvoir entretenir matériellement de façon convenable deux ou plusieurs foyers, et être en mesure de remplir son devoir conjugal comme il convenait avec chacune de ses femmes. Sinon les épouses délaissées étaient en droit de se plaindre à la justice, et d'obtenir le divorce à leur avantage.}%
. Une femme qui n'avait pas donné d'enfant à son mari pouvait lui suggérer de prendre une deuxième épouse ou une concubine : ainsi avait fait Sara, stérile, qui avait offert sa propre servante (esclave) Agar comme concubine à son époux Abraham, afin qu'ils aient tous deux une descendance. Comme à Rome les enfants des épouses et des concubines étaient légitimes s'ils étaient reconnus par leur père et si la fidélité de la mère n'était pas mise en doute. La monogamie n'en était pas moins le modèle comme chez les grecs et les romains, ainsi que le signifiait l'obligation faite au Grand Prêtre de n'avoir qu'une seule épouse, et aucune concubine%
%[6]
\footnote{Les œuvres les plus tardives de la littérature hébraïque ne mettent en scène que des couples monogames. Aucun rabbin du passé, même dans l'antiquité, n'est connu pour avoir eu plus d'une femme : à vrai dire les sociétés où la polygamie est autorisée ne peuvent jamais compter beaucoup de foyers polygames, ne serait-ce que parce qu'il est fort coûteux d'entretenir plus d'un foyer, surtout si on élève tous les enfants qui y naissent. À Rome il était usuel que les grossesses des concubines soient interrompues par un avortement, contrairement aux grossesses des épouses légitimes, du moins tant que celles-ci n'avaient pas donné le jour au nombre d'enfants légitimes souhaitable : l'usage le plus fréquent était donc de n'entretenir qu'un seul foyer.}%
 : la polygamie impliquait donc une moindre perfection ou une moindre pureté.

 La famille juive était aussi patriarcale que les autres familles de l'antiquité méditerranéenne, et ses règles de fonctionnement ne différaient guère. Le père avait tout pouvoir sur les siens : femme, enfants%
% [3]
\footnote{Selon la Tora et le Talmud la rébellion d'un enfant contre son père serait punie de mort (comme à Rome) \emph{si ce dernier le demandait}, mais il faudrait qu'un jugement en bonne et due forme approuve sa demande.}%
, esclaves. Pour l'héritage les fils aînés étaient privilégiés. La virginité des femmes avant le mariage était attendue. L'épouse surprise en flagrant délit d'adultère était lapidée avec son complice sauf si son mari préférait la répudier. Tout le monde s'attendait à ce qu'il le fasse, comme à Rome, et c'était une cause de répudiation sans appel. Une épouse ne pouvait pas prendre l'initiative de divorcer. Il fallait que son mari lui accorde le droit de le faire, comme à Rome au même moment%
%[7]
\footnote{Selon le Talmud en cas de conflit conjugal insoluble un homme pouvait néanmoins être conduit à divorcer par la pression des membres influents de sa communauté : cela était-il une pratique courante avant notre ère ?}%
. La démence de l'épouse interdisait à l'époux de la répudier (et réciproquement), mais non de prendre une concubine. 

 Aux hommes les femmes de tous les autres hommes étaient interdites, par contre toutes les femmes non mariées leur étaient permises, comme dans les autres sociétés antiques. La prostitution était interdite par la Tora, mais coucher avec une prostituée n'était qu'une faute morale sans gravité et non une infraction légale. Seules étaient strictement interdits les prostituées et prostitués sacrés des temples païens, avec qui coïter équivalait à sacrifier aux idoles%
% [4]
\footnote{La fidélité masculine est néanmoins présentée par la Tora (ex. : Proverbes 5, 15 et 20) ou le Talmud comme le modèle à atteindre[4] : l'infidélité de l'époux favorise et entraîne celle de l'épouse : \emph{lui parmi les fruits mûrs, elle parmi les plantes croissantes}.}%
. 

 Le viol d'une femme libre était puni de mort. Le fautif n'échappait au châtiment que s'il épousait sa victime (pour cela il devait être accepté comme gendre par le père de celle-ci), mais il perdait en ce cas le droit de la répudier. Une femme répudiée puis épousée par un autre homme ne pouvait plus être épousée à nouveau par son premier mari. 


\section{Naissances impures}

 Le \emph{Deutéronome} interdisait d'épouser un étranger ou une étrangère (non juif), ou un eunuque (incapable d'engendrer). L'appartenance à Israël était inscrite dans la chair des hommes par la circoncision, mais elle l'était d'abord par la filiation. Selon la Bible au retour de Babylone (cinquième siècle avant J.-C.) Esdras avait chassé du peuple toutes les femmes étrangères, tous leurs enfants, tous les hommes qui ne voulaient pas se séparer d'elles, et tous ceux dont les origines familiales étaient discutables ou qui ne pouvaient apporter la preuve du contraire (\emph{Esdras}, 10, 17). Il s'agissait de construire une nation au sang pur, à une période où les cités grecques renforçaient elles aussi le lien entre citoyenneté et descendance légitime. La pureté de l'ascendance était un brevet de valeur religieuse et sociale. Connaître ses ancêtres sur de nombreuses générations était un signe d'excellence. Comme les quartiers de noblesse de l'ancien régime français, ou la « pureté du sang » des siècles classiques d'Espagne et du Portugal, cette pureté-là se capitalisait au fil des générations. Elle permettait de s'allier avec d'autres familles à la pureté aussi préservée, aux alliances aussi bien choisies que les siennes. Elle s'accroissait par la gestion intelligente des alliances, et se perdait si on se laissait aller aux rencontres de hasard. 

 Les mariages mixtes étaient en principe interdits, mais un \emph{gentil} (non-juif) pouvait se convertir au judaïsme, et la Bible en fournissait de nombreux exemples. Comme c'est par la mère que se transmettait l'appartenance au peuple hébreu celui ou celle qui se convertissait ne pouvait en faire pleinement partie, puisque sa mère n'était pas juive. Il (elle) n'était au sens strict qu'un allié du peuple saint. La conversion ne pouvait complètement annuler le fait qu'une femme soit née non juive, donc « impure ». Seules les femmes juives nées de mère juive pouvaient donner le jour à des enfants à la légitimité incontestable (légitimité religieuse et non pas civile). Ce n'est qu'au niveau de ses enfants que la famille d'un(e) converti(e) serait juive de plein droit. Cela ressemblait à l'intégration d'un affranchi dans le peuple romain, qui lui non plus n'était jamais pleinement assimilé aux citoyens, au contraire de ses enfants. Cela décourageait les hommes de chercher leurs épouses et leurs concubines à l'extérieur de leur communauté, sauf quand ils ne pouvaient faire autrement.

 Il était interdit d'entretenir un culte familial : cela aurait été faire preuve de paganisme. Le Seigneur seul connaissait le destin des ancêtres \emph{(dans le sein d'Abraham)}, et il n'y avait pas à s'en occuper. Ce n'était donc pas l'existence d'un culte familial qui définissait la famille. Pour cette raison l'adoption d'un étranger à la famille pour en faire un héritier était sans objet et donc interdite. 

 Le \emph{Deutéronome} (Deut. 23, 3-4 ; 24, 4) interdisait enfin d'épouser un \emph{mamzer} (pluriel : \emph{mamzerim}). On appelait \emph{mamzerim} les \emph{impurs de naissance}%
% [8]
\footnote{in \emph{histoire de la famille, I}, p. 377-379. Cela ne concerne en principe que les enfants nés de deux parents juifs.}%
 :
\begin{enumerate}
% A) 
\item tous les enfants issus d'unions interdites :
 \begin{enumerate}
 % 1)
 \item les unions incestueuses, qui étaient l'objet d'une réprobation sévère. Les interdits semblent avoir été les mêmes que les interdits romains contemporains ;
 % 2)
 \item les unions adultérines (adultère féminin) ;
 % 3)
 \item le mariage ou le concubinage avec un non juif, légalement nul pour les juifs, mais légitime pour la loi romaine : les enfants étaient juifs si leur mère était juive, non juifs dans le cas contraire ;
 % 4)
 \item l'union d'un juif non mamzer avec un juif mamzer ;
 % 5)
 \item le remariage d'un homme avec une femme dont il avait d'abord divorcé, et qui s'était remariée entre temps avec un autre homme.
\end{enumerate}
% B)
\item les enfants nés de père inconnu et ceux nés de père et mère inconnus (c'est-à-dire tous les enfants trouvés). Les enfants nés d'unions interdites avaient évidemment plus de risques que les autres d'être abandonnés. Par précaution les enfants exposés étaient donc classés dans les mamzerim ;
% C) 
\item les enfants nés d'une femme adultère ou prostituée, dont le père ne pouvait pas être désigné avec certitude ;
% D)
\item les enfants des mamzerim : ce statut se transmettait en effet en principe à toute leur descendance. 
\end{enumerate}

 L'enfant d'une juive non mariée et d'un juif dont on connaissait l'identité, qu'il soit marié ou non, n'était un mamzer que si ses deux géniteurs étaient interdits de mariage. Il n'était donc ni nécessaire ni suffisant de naître en légitime mariage pour échapper au statut de mamzer, mais la reconnaissance par un homme non interdit de mariage avec la mère était toujours indispensable. Les enfants trouvés étaient présumés mamzerim, mais il n'était pas possible de les rejeter sans autre forme de procès puisqu'ils étaient présumés descendants d'Abraham comme les autres, et puisque toute vie humaine était sacrée. Si nécessaire la communauté prenait donc en charge les enfants exposés, mais elle les mettait au dernier rang.

 Les mamzerim étaient \emph{exclus de l'assemblée du peuple jusqu'à la dixième génération}, ce qui veut entre autre dire qu'ils ne pouvaient épouser ceux qui n'étaient \emph{pas} nés impurs, Aucune famille bien née ne pouvait s'allier aux mamzerim. \emph{C'était même un devoir religieux que de ne pas leur donner un de ses enfants comme époux.} Quand il y avait des failles, des lacunes ou des faux-pas dans la chaîne des générations, la sanction c'était donc de ne plus pouvoir se marier qu'entre familles d'impureté équivalente. Un mamzer ne pouvait se marier qu'avec d'autres mamzerim, ou des païens, impurs par définition, ou des convertis, nés impurs. 

 On note les ressemblances entre le statut des mamzerim juifs et celui des romains atteints par l'\emph{infamie}, ou des grecs frappés par l'\emph{atimie}. Il s'agissait là de l'expression d'une vision du monde commune à toutes les sociétés antiques. Même si chacune délimitait la \emph{mauvaise réputation} à sa façon le noyau des hontes était commun. 


\section{Clergé et familles}

 Il n'y avait qu'un seul temple pour tous les juifs, celui de Jérusalem. Les charges et dignités religieuses étaient héréditaires. Les prêtres et les lévites étaient choisis exclusivement dans la \emph{tribu de Lévi}. Les prêtres descendaient en ligne directe d'Aaron, frère de Moïse. La tribu de Lévi vivait du produit de la Dîme versée par les 11 autres tribus. Les communautés de la Diaspora contribuaient elles aussi par l'impôt spécial \emph{(fiscus judaïticus)} à l'entretien du Temple et de ses desservants. 

 Les prêtres se succédaient au Temple de semaine en semaine selon un tour de service (une semaine toutes les 24 semaines ?). Le reste du temps ils vivaient chez eux, pas toujours à Jérusalem. Durant leur semaine de service ils présidaient aux cérémonies. Sacrificateurs ils abattaient rituellement, ils « immolaient » les bêtes offertes au nom du peuple ou des particuliers. Ils les découpaient et les préparaient comme le rituel le prescrivait. Les lévites étaient chargés des tâches autres que les sacrifices. Ils n'approchaient pas de l'autel et ne le touchaient pas : c'est qu'ils ne descendaient pas d'Aaron en ligne directe, ou bien que leur généalogie présentait des irrégularités, des impuretés.

 De la conception à la mort les prêtres et lévites vivaient dans l'obsession de la pureté rituelle. Pour officier ils devaient être parfaitement sains de corps, sans aucune maladie, sans difformité physique et sans infirmité, et dans la force de l'âge (25 à 50 ans). Ils ne devaient pas avoir la voix embarrassée. Il ne devait pas leur manquer une seule phalange : une mutilation même minime (oreille ou phalange coupée, entorse mal réparée...) leur faisait d'autant plus sûrement perdre leur emploi qu'il y avait plus de candidats que de postes à pourvoir. 

 Les généalogies des membres de la tribu de Lévi, et surtout celles des prêtres, étaient d'autant plus impeccables que s'ils voulaient officier au Temple (et vivre des offrandes) ils ne pouvaient épouser ni une juive divorcée, ni la fille de deux parents convertis, ni une convertie, ni la fille d'un homme non juif et d'une mère juive, ni une veuve refusée par son beau-frère \emph{(halizah)} dans le cadre du \emph{lévirat}, etc. 

 Il leur était interdit de vivre dans l'adultère, eux, et tous ceux et celles qui vivaient sous leur toit. Ils devaient épouser une femme vierge, et surtout qui n'ait pas été prostituée. Si leur épouse était infidèle, elle était lapidée jusqu'à la mort. Leurs filles devaient rester vierges tant qu'elles vivaient auprès d'eux et qu'elles mangeaient donc les \emph{choses sacrées} provenant du Temple, les parts réservées au clergé sur les offrandes. 

 Le grand prêtre était au sommet de la hiérarchie cléricale. Sa personne était sacrée car il avait été consacré \emph{(oint)}. Il était enserré dans le réseau de règles le plus contraignant. Au contraire de ses concitoyens il n'avait droit ni à la polygamie, ni à une concubine : il n'avait droit qu'à une seule épouse, de pure origine juive, et épousée vierge. Ces contraintes qu'il subissait plus durement qu'aucun autre dans sa vie privée soulignent le cœur des conceptions juives en matière de morale matrimoniale.
 
 