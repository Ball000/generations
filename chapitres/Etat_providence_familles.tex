
\chapter{L'État, providence des familles ?}


 À côté des drames qui en ont fait aussi une période noire (deux guerres mondiales, diverses guerres locales, les guerres de décolonisation et au moins une grande crise économique) le demi-siècle qui va de 1910 à 1960 a vu la fin silencieuse d'une civilisation rurale millénaire (ce qui a représenté un drame d'une autre nature pour bien des gens) et révolutionné la vie quotidienne : création des média de masse, construction de banlieues concentrationnaires, progrès fulgurants de la lutte contre les maladies infectieuses, début de la croissance explosive qui a caractérisé les « trente glorieuses »,~etc. Il a vu l'essor du salariat, celui de multiples caisses d'assurances sociales et de retraite (déjà initié à petite échelle dès la fin du \siecle{19}) puis leur extension à l'ensemble de la population. Il a vu la création des allocations familiales. Il a vu le début de la démocratisation des enseignements secondaire et supérieur, qui étaient à l'époque des outils indiscutables d'ascension sociale. 

 Ces années étaient marquées par la conviction qu'il était possible d'aller vers un monde meilleur, lorsque les forces du mal seraient vaincues (guerres mondiales, guerres coloniales, capitalisme, communisme, etc.) et ce monde semblait alors à portée de main. Apparue après la seconde guerre mondiale, l'expression \emph{État Providence} (en Angleterre le « \emph{welfare state} », état de bien-être par opposition à l'état de guerre) exprimait un aspect de ce projet : à défaut de faire descendre {\emph{ici-bas}} le paradis, procurer à tous au moins un solide filet de sécurité contre l'indigence et l'abandon social. 

 Cet effort de longue haleine, commencé dès le \siecle{19}, a obtenu des résultats très significatifs, grâce auxquels à partir de 1945 l'ensemble de la population, et en particulier les travailleurs les plus pauvres, a bénéficié d'assurances sociales et de retraites par répartition qui mutualisaient les risques. Ces systèmes imposés par les états rendaient en principe inutile le recours à l'assistance et à la bienfaisance comme la prise en charge des indigents par leur propre parentèle : il s'agissait de l'immense majorité de la population, et d'abord des salariés. 

 Le montant des aides financières accordées à tous les parents pour la prise en charge de leurs enfants a connu son apogée entre 1945 et 1965. Elles protégeaient de l'indigence les enfants des pauvres mieux qu'on ne l'avait jamais fait jusqu'alors. 

 Les mesures d'assistance en faveur des femmes en couche et des familles nécessiteuses, mais aussi les allocations ouvertes à toutes les familles (allocations familiales, salaire unique, allocations prénatales, etc.), les consultations de nourrissons, les crèches, et les améliorations progressives des conditions de travail, tout cela a facilité la vie des familles et permis que depuis le début du siècle le nombre des abandons décroisse régulièrement et massivement.

 Depuis le milieu du \siecle{19}, l'administration pouvait verser aux mères seules une aide afin qu'elles placent elles-mêmes leur enfant chez une nourrice de leur choix. Dans le même esprit, le placement chez une nourrice directement salariée par l'assistance publique a de plus en plus souvent été perçu comme une forme de secours, et non plus comme le remplacement d'une famille par une autre. On aidait la mère en lui fournissant une nourrice, là où les familles citadines non indigentes de l'époque se la procuraient elles-mêmes. À partir de 1924, l'assistance publique de Paris a commencé de placer en nourrice des enfants non abandonnés, dont les parents n'étaient pas déchus de leurs droits, des enfants qui n'étaient pas des pupilles. 

 Pour cela, il avait fallu franchir une barrière psychologique et oser placer en nourrice pour une durée indéterminée, l'enfant qu'une femme pourrait reprendre un jour. Cela allait contre les pratiques antérieures de l'assistance publique, mais (et ce n'est sans doute pas un hasard) c'est également à partir de l'année 1924 que la loi a permis d'adopter les enfants mineurs. C'est à partir de cette date que le service a la possibilité de procéder à des adoptions d'enfants abandonnés. Face à la réalité d'adoptions authentiques (quel que soit leur nombre réel, quelques centaines par ans semble-t-il), l'illusion que le placement dans une famille nourricière salariée était une espèce d'adoption ne pouvait plus tenir. Il ne pouvait plus être question pour un « nourricier » de prendre la place d'un parent dans le cœur de l'enfant, mais seulement de fournir à ce dernier une assistance pendant un temps plus ou moins long. 

 Au motif que les liens avec leurs parents n'étaient pas coupés, l'administration pouvait laisser les petits enfants concernés en collectivité (on a vu que c'était sa position traditionnelle face aux enfants qui avaient des parents), mais :
\begin{enumerate}
%a)
\item elle priverait alors autant de nourrices de leur emploi alors que les régions pauvres où elles vivaient avaient besoin de ces emplois et que le nombre des enfants abandonnés avait déjà beaucoup baissé depuis le début du siècle,
% b)
\item les nourrices coûtaient moins cher que les internats,
% c)
\item d'autre part, et surtout, on savait qu'en collectivité l'état de santé des petits enfants (0~à 4 ans) se dégrade inexorablement et rapidement au fil du temps, comme l'expérience l'avait régulièrement démontré depuis plusieurs siècles. Dès que le placement courait le risque d'être durable (plus de quelques semaines), il fallait autant que possible éviter aux plus jeunes les « dépôts » des enfants de l'Assistance Publique et les « orphelinats ».
\end{enumerate}
 Les jeunes concernés ont donc assez systématiquement été placés en famille d'accueil. On a rapidement observé, comme il était prévisible au vu de l'histoire antérieure de l'assistance, que le placement en famille d'accueil des tout-petits donnait de très bons résultats, et d'abord en ce qui concerne la santé physique qui était la préoccupation première à cette époque. Cette observation a assuré le succès de cette formule. Peu à peu les âges d'admission ont été assouplis. Les enfants qui ont des parents et qui gardent un lien avec eux ont pu entrer en famille d'accueil à un âge de plus en plus avancé, et tous ont fini par bénéficier de cette formule. 

 Le droit des parents au « dépôt » de leurs enfants à l'A.P. a été élargi par la loi du 15 avril 1943. Cette loi ouvrait un droit aux secours (dont le placement est l'une des formes possibles) aux enfants \emph{qui ont un père}, même quand celui-ci est valide et donc capable de travailler. Elle impliquait qu'un homme qui ne peut subvenir financièrement aux besoins de sa famille n'était pas pour autant disqualifié comme époux et comme père. Il n'avait pas pour autant à être sanctionné comme un débiteur insolvable à écarter de sa partenaire et de ses enfants. Il convenait plutôt de l'assister.

 On peut supposer que les séparations familiales et les privations de cette période de rationnement avaient facilité cette décision. L'un de ses objectifs a pu être de fournir une aide aux femmes et aux enfants des prisonniers de guerre retenus en Allemagne (valides certes, mais enfermés au loin et travaillant sans rémunération dans la tradition plurimillénaire de l'esclavage des vaincus). Mais il s'agissait aussi de reconnaître l'évolution des pratiques réelles des services. Preuve en est que cette réglementation n'a pas été abrogée après la guerre. 

 C'était un corollaire (paradoxal) du fait qu'on sortait peu à peu de la représentation patriarcale du monde qui dominait les siècles précédents : si les hommes n'étaient plus les patriarches tout-puissants qu'on avait pensé qu'ils étaient (ou voulu qu'ils soient), leur impuissance financière ne justifiait plus leur éviction.
 
 
