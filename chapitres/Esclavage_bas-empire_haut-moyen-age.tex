% 28.02.2015 :
% haut Moyen Âge
% _, --> ,
% Antiquité
% ~etc.
%~\%


\chapter[L'esclavage chez les chrétiens de l'Antiquité tardive et du haut Moyen Âge]{L'esclavage chez les chrétiens\\de l'Antiquité tardive\\et du haut Moyen Âge}


 Aucun auteur antique n'est allé jusqu'à une condamnation de l'esclavage en tant que système. À cette époque, un monde sans esclaves n'était pas pensable. Seules certaines sociétés arriérées et misérables d'alors se passaient réellement d'esclaves, et elles n'avaient rien de désirable pour les autres. 

 Selon Jean \fsc{ANDREAU} et Raymond \fsc{DESCAT} (\emph{Esclave en Grèce et à Rome}, 2006, p. 220) : \emph{Celui qui est allé le plus loin dans la condamnation de l'esclavage reste Grégoire de Nysse, au \siecle{4}. Non seulement il estimait que, devant Dieu, les esclaves sont les égaux des hommes libres, mais il regardait la possession d'esclaves comme un péché et un très grave péché. En effet, quoique toutes les créatures soient au service de Dieu et appartiennent à Dieu, le propriétaire d'esclaves s'est approprié certaines de ces créatures, ce qui revient à défier l'ordre divin et à revendiquer un droit qui ne peut être que celui de Dieu ...} (Homélie IV sur l'Ecclésiaste, 2, 7) \emph{... Grégoire a-t-il libéré tous ses esclaves ? Nous n'en savons rien. Mais ... même lui n'a pas milité pour l'abolition de l'esclavage}. Selon les mêmes auteurs, deux groupes dissidents juifs, les Esséniens et les Thérapeutes, étaient opposés à l'esclavage, mais n'ont pas non plus milité en ce sens : \emph{aucun penseur antique ne l'a fait, et il n'y a jamais eu, dans l'Antiquité, de mouvement abolitionniste}...

 Saint Augustin interprète l'existence de l'esclavage comme une conséquence du péché originel, et pour lui comme pour les stoïciens le fait d'être esclave du péché était bien plus grave que celui d'être esclave d'un maître. Même si l'Église a toujours soutenu le caractère méritoire des affranchissements elle ne s'est donc pas attaquée à l'institution de l'esclavage. Elle a reçu sans états d'âme des esclaves en donation, elle en a acheté, elle en a employé. Elle a même fait obligation aux évêques et prêtres trop bienveillants de lui racheter les esclaves qu'ils voulaient libérer ou de les remplacer par d'autres esclaves de valeur équivalente au moyen de leur fortune personnelle, pour compenser la perte matérielle qu'un affranchissement pourrait faire subir aux biens (inaliénables) de l’Église dont la gestion leur avait été confiée (ex. conciles espagnols \crm{4}--\siecles{5}{6}). 

 Cela étant dit il y a des choses que l'Église ne supportait pas :
\begin{enumerate}
% 1°)
\item Qu'un esclave ne puisse pas devenir chrétien alors qu'il le désirait, et qu'il soit empêché de satisfaire aux prescriptions d'une vie chrétienne régulière (culte dominical, jeûnes,~etc.). Elle était contrainte par les lois civiles de demander l'autorisation expresse des maîtres avant tout baptême d'esclave, mais elle vivait comme une persécution le fait qu'ils la lui refusent.
% 2°)
\item Qu'un ou une esclave soit contraint à des pratiques contraires au Décalogue, entre autre dans le domaine sexuel. Elle n'acceptait pas qu'un esclave soit condamné à vie à un célibat non choisi, ou à une vie de promiscuité sexuelle, ou à des avortements ou à l'exposition de ses enfants,~etc.
% 3°)
\item Elle défendait le droit au mariage des esclaves. Pour « unir » deux esclaves, il suffisait que le maître permette que soit organisée une cérémonie interne à la \emph{familia} où toutes les personnes présentes, libres et esclaves, étaient les témoins des conjoints et faisaient la fête avec eux, mais cet acte était infra juridique et purement domestique. Cette union (\emph{contubernium}, ou compagnonnage de chambrée) n'avait pas la valeur d'un authentique mariage au-delà des murs du domaine, et le maître n'était pas tenu de la respecter. Elle ne donnait pas aux intéressés de droits parentaux. A contrario l'Église reconnaissait le mariage des esclaves du moment qu'il était monogame, fidèle et inscrit dans la durée, sans distinguer leur union de celle des personnes libres.
% 4°)
\item L'Église ne supportait pas qu'on sépare les couples d'esclaves, qu'on les vende chacun de son côté, ou qu'on leur rende la vie commune impossible. Elle ne supportait pas que les enfants des esclaves soient séparés de leurs parents, confiés à d'autres personnes contre leur gré, et encore moins vendus de leur côté. Le corollaire du droit au mariage, sans lequel ce droit n'a aucun sens, est en effet qu'il soit garanti à ceux qui se marient un minimum de maîtrise sur le temps à venir et de droits sur leur conjoint et sur leurs enfants. 
\end{enumerate}

 L'Église n'acceptait donc l'esclavage que pour autant que le statut des esclaves soit aménagé, de même que les juifs n'acceptaient l'esclavage d'un coreligionnaire que s'il était traité en mercenaire (ou en gagé pour dettes) et non asservi à perpétuité. Si l'on acceptait ces exigences, le statut des esclaves se rapprochait de celui des hilotes grecs et de divers autres statuts de dépendants, dans lesquels la force de travail de ceux-ci appartenait à leurs maîtres de manière héréditaire, mais pas leurs corps ni leurs droits parentaux. Les serfs du Moyen Âge \emph{n'étaient pas des esclaves}, mais le latin ne possède qu'un seul mot pour désigner ces deux statuts \emph{(servi)}, ce qui plaide en faveur d'une évolution progressive de l'un vers l'autre sur plusieurs siècles.

 Les esclaves constituaient une part relativement importante de la population du haut Moyen Âge. Ils étaient toujours l'objet d'achat et de vente, et leurs enfants appartenaient toujours au maître de leur mère. La réduction des chrétiens en esclavage par la force avait été interdite par les derniers empereurs, de même que depuis des siècles il n'était pas permis d'asservir des citoyens grecs ou romains ... mais rien n'empêchait personne de se vendre soi-même. À qui n'avait ni alliés ni capitaux ni culture ni savoir-faire rare, il n'était pas plus facile qu'auparavant de trouver de quoi vivre. Était toujours à bon droit traité comme esclave celui qui se reconnaissait comme tel. L'interdiction d'asservir les chrétiens ne concernait pas les tribunaux, libres de condamner des coupables à l'esclavage. S'il était interdit d'asservir des chrétiens nés libres, rien n'obligeait à libérer les chrétiens qui étaient nés esclaves, même si leur affranchissement était un acte si louable que les évêques s'y impliquaient personnellement. Et les interdits consécutifs au fait d'avoir été un esclave (interdits qui constituent la \emph{marque servile}) frappaient toujours les affranchis, montrant la persistance des anciennes représentations. D'ailleurs les institutions ecclésiastiques possédaient leurs propres esclaves sans y voir aucun mal. Le principal souci des évêques était d'empêcher que les esclaves chrétiens ne soient soumis à des maîtres païens ou juifs susceptibles de les détourner de la foi chrétienne, et qu'ils ne soient déportés dans des contrées non chrétiennes. 

 Si la réduction en esclavage de chrétiens libres (« ingénus ») était un crime pour l'Église et pour les pouvoirs civils, il n'en était pas de même en ce qui concernait les autres (juifs, païens, hérétiques et schismatiques divers...), qui pouvaient être asservis sans problème. A fortiori n'étaient pas affranchis non plus les païens capturés à la guerre ou à la chasse aux esclaves (parmi les attraits de la guerre, celui d'y faire des esclaves demeurait aussi important que par le passé) même si leurs maîtres chrétiens avaient le devoir moral de les faire baptiser, en vertu de leur autorité sur la totalité des membres de leur maison. D'ailleurs ne valait-il pas mieux, comme toujours, asservir les vaincus plutôt que de les passer au fil de l'épée, eux, leurs femmes et leurs enfants, lorsqu'ils n'avaient pas les moyens de payer une rançon ?

 Le commerce des esclaves, notamment non-chrétiens, a donc perduré bien au-delà du Moyen Âge, alimenté par des circuits divers (\emph{slave} vient de « esclave » ), même si à partir du milieu du Moyen Âge, l'esclavage en tant que tel n'a plus joué en Europe un rôle important, sauf exception locale (Espagne, pourtour de la Méditerranée). 

 Le statut des esclaves s'est pourtant insensiblement modifié au fil des siècles : la plupart des esclaves ruraux ont été \emph{chasés}, c'est-à-dire installés dans une \emph{casa}, une maison, avec la pièce de terre plus ou moins étendue que le maître y adjoignait, et une concubine attitrée prise dans sa \emph{familia}, comme les \emph{colons esclaves} de l'Antiquité romaine. Mais dans le même temps le statut de beaucoup des tenanciers libres, \emph{colons libres} d'un propriétaire, ou propriétaires indépendants \emph{(alleutiers)}, sans oublier les affranchis, s'est dégradé au fil du temps pour se rapprocher de celui des esclaves. Pour une part de la population, plus ou moins grande selon les lieux, le résultat de ces deux mouvements a été la généralisation du statut de \emph{serf}, qui attachait chacun de manière héréditaire à une terre ou à un office, et l'assujettissait au seigneur \emph{(dominus)} de cette terre ou de cet office. 

 Ils étaient possédés par leur emploi, ils n'étaient pas totalement libres d'employer leur temps et leurs forces à leur gré. Ils ne pouvaient ni s'en aller ni se soustraire aux ordres reçus. Ils devaient se marier sur le domaine. Une partie de leurs droits personnels appartenaient au seigneur. Par contre et contrairement aux esclaves, ils jouissaient du reste de leurs droits personnels, notamment conjugaux et parentaux ... mais certaines terres, certains offices au service des puissants valaient parfois qu'on s'asservisse pour eux. Certains « postes » de serfs étaient jugés très enviables, au même titre que certains esclaves de personnages puissants pouvaient provoquer des jalousies.

 Le servage était une promotion pour les esclaves, mais une régression pour les personnes libres. En acceptant leur dépendance, celles-ci voyaient s'aliéner une part très significative de leur liberté. En contrepartie, elles faisaient partie d'une communauté villageoise. Les gens des villages, serfs ou libres, n'étaient pas toujours incapables de faire bloc et d'exercer une pression sur leur seigneur, qui avait besoin de leur prospérité matérielle autant qu'ils avaient besoin de sa protection. Ils pouvaient dans une certaine mesure intervenir en tiers entre un serf et lui. En droit comme en fait, il était assez difficile au \emph{dominus} de chasser un serf de sa terre. 
 
 En France l'esclavage a disparu au profit du servage vers le \siecle{8}, et le 3 juillet 1315 Louis Le Hutin a décidé que sont libres tous les esclaves chrétiens qui posent le pied sur le territoire français. 

 Cela n'empêchera pas les Français de recourir à l'esclavage dans leurs colonies avec l'approbation des autorités civiles (cf. \emph{le code noir}) : il suffira d'empêcher les esclaves (et aussi \emph{tous} les noirs) de toucher le sol de France. En effet la survie de l'esclavage sur les terres européennes (surtout dans l'Europe du sud, et d'abord en Espagne) a préparé les esprits a recourir à partir de la Renaissance à l'esclavage des indiens, puis des noirs d'Afrique, pour exploiter des deux Amériques et les autres colonies européennes (Surinam,~etc.). 

 Ce n'est qu'à partir de la fin du \siecle{18} qu'une part significative des intellectuels sont tombés d'accord pour condamner l'esclavage sans aucune circonstance atténuante (cf. \emph{L'Encyclopédie}).
