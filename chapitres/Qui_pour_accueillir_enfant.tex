L2 QUI POUR ACCUEILLIR L'ENFANT ?
 Les lois autorisant le divorce par consentement mutuel, la contraception et l'avortement ont dénaturalisé le modèle de famille traditionnel si bien que même si rien ne pourra jamais empêcher personne de croire à la valeur éthique, éducative ou civique de la « \emph{sainte famille} », fondée sur un couple hétérosexuel, monogame et indissoluble, élevant lui-même les enfants nés de ses œuvres, ce n'est plus qu'un choix parmi d'autres également légitimes. Dans la mesure où ce n'est plus qu'une possibilité parmi d'autres, comme c'était le cas avant les décisions de Constantin, et dans la mesure où cette possibilité n'est plus étayée par la loi et la puissance publique, comme elle l'était par l'Ancien Régime ou le Code Napoléon, la « sainte famille », la famille "traditionnelle" ne peut plus être identifiée avec ce que j'ai appelé la « famille constantinienne ». Il s'agit seulement de l'une des façons (post constantiniennes) d'organiser la reproduction humaine, à côté de diverses autres, dont la paternité, la maternité et l'adoption célibataires, le concubinage, le PACS ou le mariage hétéro ou homosexuel, etc.
 
 On peut chercher à explorer d'autres voies, comme le fait Agnès ECHENE, dans un article de 2004 intitulé \emph{« Quelle alternative au patriarcat ? Valoriser un modèle social non conjugal ».}Elle désigne dans le couple hétérosexué le lieu privilégié d'expression et de transmission de la violence machiste, et cela trop souvent avec la complicité (masochiste) féminine[10]. C'est pourquoi : « \emph{Ce n'est qu'en valorisant le modèle social non conjugal qu'une société peut se défaire du patriarcat. Il importe donc de favoriser une sexualité libre et variée, tout en étant discrète et protégée, surtout chez nos propres enfants ; peu importe dès lors qu'elle soit ardente ou paisible, monotone ou changeante, homophile ou hétérophile, dès l'instant qu'elle reste une affaire personnelle dont nul ne se mêle. Une telle évolution nécessite également une reconsidération du modèle familial qui doit se refonder sur des liens d'appartenance utérine et non pas consanguine ; cela remet en cause dès lors la paternité génitale qui doit laisser place à une paternité germaine : il faut en effet que ce soit les frères, oncles et cousins}(de la mère)\emph{ qui assument les enfants des femmes ; de nombreux signes avant-coureurs montrent qu'ils sont prêts à le faire et qu'il ne manque qu'un déclic. Mais il faut aussi que les femmes renoncent à obliger les géniteurs à être pères ; il faut qu'elles abandonnent toute velléité de recherche de paternité, de pension, partage, alternance, etc., et se tournent résolument vers leurs frères, oncles et cousins pour »donner » des pères à leurs enfants, qui ne s'en porteront pas plus mal.} » 
 
 On pourrait imaginer bien d'autres configurations, notamment la parentalité célibataire. Un nombre grandissant de femmes, cadres surtout, mais pas seulement, élèvent désormais leurs enfants toutes seules, sans reconnaître un père. Il est probable que la GPA permet et permettra à bien des hommes célibataires (quelles que soient par ailleurs leurs préférences sexuelles) d'en faire autant.
 Va-t-on vers des foyers constitués d'une femme et des enfants qu'elle a mis au monde, autour desquels graviterait la nébuleuse de ses amants et ex-amants ? Dans cette hypothèse les hommes de demain auraient des enfants de plusieurs femmes, enfants vivant ordinairement chez leurs mères, si bien que leur autorité sur chacun d'eux serait pratiquement nulle ? Ce serait l'inverse de la situation du pater familias romain, qui pouvait demander à plusieurs femmes des enfants sur lesquels lui seul avait autorité. Cela se rapprocherait de la configuration familiale matrifocale du modèle « antillais » ou « caraïbe »[8], dont l'origine se situe dans l'histoire du peuplement des Antilles. On a vu que les esclaves n'ont par définition aucun des attributs juridiques d'un père ou d'une mère sur les enfants dont ils sont les géniteurs et génitrices : seuls les propriétaires des génitrices possèdent des droits sur les enfants de celles-ci. C'étaient ces propriétaires qui faisaient d'elles des mères lorsqu'ils leur confiaient la garde des enfants qu'elles avaient portés, quel qu'en soit le géniteur. Il y avait une espèce d'alliance de fait[9] entre les génitrices et leurs maîtres pour élever les enfants qu'elles avaient mis au monde, tandis que leurs partenaires sexuels n'avaient pas droit à la parole et étaient réduits à n'être que des donneurs de sperme. 
 Faut-il comprendre qu'à l'avenir ce pourrait être l'État qui assumerait le rôle de tiers traditionnellement dévolu aux pères, à l'aide de ses services judiciaires et socioéducatifs ? tiers qui soutient matériellement et psychologiquement la mère dans sa tâche éducative, et qui introduit les exigences du monde extérieur au sein de la dyade mère-enfant.
 
 A rebours du modèle constantinien qui télescope sur le couple des seuls géniteurs toutes les dimensions de la conjugalité et de la parentalité (juridique, biologique, affective et éducative) et qui frappe tout le reste d'illégitimité, hormis le cas du décès de l'un ou de l'autre, le modèle de la filiation élective, volontaire, adoptive, est aujourd'hui valorisé. 
 Ainsi BORRILLO[1] écrit que : \emph{« … La filiation peut certes tenir compte du fait naturel, mais, en tant que dispositif d'agencement parental, elle répond à des règles propres, affranchies de la nature…}(La filiation)\emph{ …n'existe que lorsqu'elle est établie dans les conditions et selon les modes prévus par la loi. Autrement dit, la filiation est déterminée par la norme juridique et non par la nature. Ce lien juridique se tisse à partir de quatre fils principaux : la biologie (filiation par le sang), la volonté (adoption), la présomption (paternité supposée du mari de la mère) et le vécu (appelé en droit » possession d'état » ). »}
 \emph{« …En Occident, la coutume des barbares (ordalies) et le droit canonique (copula carnalis, coït charnel) ont eu en commun la vérité du corps comme fondement du lien juridique, contrairement à la civilité romaine pour laquelle la volonté constituait la clé de voûte du système juridique. Le droit moderne des personnes physiques qui s'esquisse à partir du XVIIIe siècle va opérer un retour aux règles du droit civil romain, en accordant à l'autonomie de la volonté une place centrale dans l'établissement des liens de filiation et une place éminente à la fiction juridique (fictio legis), créatrice de droits.}
 \emph{Ce qui compte ce ne sont plus tant les racines naturelles ou surnaturelles d'institutions intangibles que l'efficacité et la plasticité d'instruments juridiques procurant tel ou tel résultat (par exemple la paix des familles ou la solidarité des générations) ».}
 Selon lui :
 \emph{« …Fondée sur la volonté, l'adoption est une institution plus apte que la vérité biologique à assurer la stabilité des liens familiaux... » 
 « …Si l'adoption, et non la capacité reproductrice, était retenue comme modèle universel de la filiation, cela permettrait de fonder la parenté sur la responsabilité et sur un projet parental réfléchi, et d'éviter ainsi toutes les figures malheureuses des maternités ou des paternités non désirées… » 
 « …Ainsi la pure contractualisation des liens familiaux permettrait de laisser dans les mains des principaux intéressés la limitation et le contenu de cette communauté des affects, de volontés désirantes qui est l'essence de la famille. » 
 « Ainsi, la vie de couple cesserait d'être limitée à deux personnes de sexe différent, et l'ordre générationnel ne serait plus borné aux lignées masculine et féminine mais serait ouvert à la parenté unisexuée... »
 « …La contestation actuelle de l'ordre familial » naturel » n'est en définitive que la radicalisation de l'idéologie individualiste moderne, selon laquelle la volonté et non la différence des sexes constitue la base de l'institution matrimoniale et parentale. Une filiation dissociée de la reproduction permettra de justifier un système juridique fondé non pas sur la vérité biologique, mais sur le projet parental responsable. De ce point de vue, peu importe l'agencement familial (traditionnel, monoparental, homoparental, recomposé...), si les prémisses du contrat (égalité dans l'alliance et dans la filiation) sont respectées jusque dans leurs moindres effets. L'État devrait donc traiter sur un plan d'égalité l'ensemble des familles et les autres formes d'intimité… »
 » La coexistence du mariage, du Pacs et du concubinage pour tous les couples répondrait à cette exigence tout comme l'ouverture du droit à l'adoption, à l'A.M.P. et aux maternités de substitution au-delà des cas de stérilité... » 
 « …Contrairement à la filiation charnelle, la filiation choisie trouve son principe dans la liberté non seulement d'accueillir les enfants des autres, mais également d'abandonner ses propres enfants biologiques, ce qui est pour l'heure uniquement possible pour les femmes (accouchement sous X), mais devrait pouvoir s'élargir aussi aux hommes à travers une déclaration formelle de renoncement à la paternité. La généralisation de la filiation adoptive (y compris pour ses propres enfants biologiques) permettrait aussi de mettre la volonté au cœur du dispositif parental. Celui-ci reposerait exclusivement sur la volonté du ou des géniteurs qui donnent l'enfant et celle du ou des adoptants qui l'accueillent. De surcroît, l'adoption est une institution conçue à partir du droit de l'enfant à avoir une famille, contrairement à la filiation biologique qui apparaît plutôt comme un dispositif du droit à l'enfant… »
}Selon Daniel BORRILLO « \emph{Les progrès scientifiques... – congélation de sperme, d'ovules ou d'embryons, insémination artificielle, fécondation in vitro, identification génétique des parents... – ont provoqué… une » panique morale} » … » face aux possibilités vertigineuses de dissociation entre sexualité et reproduction qui se sont ouvertes en à peine une génération. 
 D'où une tendance des théoriciens et des praticiens du droit à valoriser de manière à ses yeux excessive, sinon exclusive, les liens biologiques parents enfants : \emph{ » …La biologie commença à devenir ainsi le soubassement réel ou symbolique[2] du système de parenté, à rebours d'une science juridique qui avait plutôt instauré la volonté au cœur de ce système. » « À partir des années 1990, l'expertise biologique s'est imposée dans les procès en contestation de paternité, la recherche des origines est revendiquée socialement comme droit fondamental de la personne, la différence de sexe est devenue une valeur... » « La nouvelle place prépondérante de la vérité biologique dans l'établissement du lien filial fut confirmée en France par la Cour de cassation[3]... Par là, la distinction traditionnelle entre reproduction (fait biologique) et filiation (fait culturel), fondement du droit civil moderne, se trouvait questionnée... non pas à partir d'arguments classiques provenant du droit canonique, mais par une rhétorique qui, d'une part, fera de la différence des sexes une condition sine qua non de la filiation, et, d'autre part, placera l'expertise sanguine et la preuve d'ADN au cœur du dispositif juridique de la parenté…} » 
 \emph{« …La vérité biologique apparaît comme l'argument fondamental non seulement pour s'opposer à la filiation homoparentale, mais aussi pour créer une sorte de hiérarchie des filiations par référence à la procréation naturelle, et finalement pour désigner les familles monoparentales ou recomposées comme cause de dysfonctionnements individuels et sociaux.} »
 On peut souligner le caractère provocateur de l'idée d'une\emph{ « déclaration formelle de renoncement à la paternité »,}en miroir du droit reconnu aux femmes à \emph{l'accouchement sous X.}
 Faudra-t-il en arriver là pour que les hommes ne se sentent plus manipulés par les femmes dans le domaine de la reproduction ?
 
[1] Les enjeux de la parentalité, Daniel Borrillo, \emph{Encyclopedia Universalis}
[2] Le \emph{biologique} comme fondement \emph{symbolique} du système de parenté !?
[3] …dans un arrêt du 28 mars 2000 définissant que\emph{}« \emph{L'expertise biologique est de droit en matière de filiation, sauf s'il existe un motif légitime de ne pas l'ordonner} » Civ. 1ère, 28 mars 2000, Bull. n° 103 ; Defrénois, 2000-06-30, n° 12, p. 769, note J. Massip ; Dalloz, 2000-10-12, n° 35, p. 731, note T. Garé ; JCP 2000-10-25, n° 43/44, conclusions C. Petit et note M.C.Monsallier-Saint-Mieu.