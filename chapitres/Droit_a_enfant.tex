
\chapter{Droit à l'enfant ?}


 À l'exception des orphelins sans famille, il n'existe pas d'enfant adoptable qui n'ait d'abord été abandonné. Du fait de la généralisation des recours aux procédés anticonceptionnels et aux interruptions volontaires de grossesse, les abandons de nouveaux-nés sont de plus en plus rares dans les pays développés : depuis bien longtemps il y en a beaucoup moins que de demandeurs. L'adoption des enfants plus âgés n'est pas simple et elle peut être terriblement éprouvante pour le narcissisme des adoptants : le nombre des enfants qui ont été adoptés et qui sont par la suite abandonnés par leurs parents adoptifs n'est pas négligeable. En fait les enfants adoptés risquent malheureusement plus que les autres enfants d'être abandonnés à cause des difficultés de tous ordres rencontrées avant et après l'acte d'adoption, par eux-mêmes, par leurs parents de naissance et par leurs parents adoptifs. Tout le monde n'est pas prêt à prendre de pareils risques.

 Dans ces conditions, comment ceux et celles qui ne peuvent ou ne veulent pas procréer, et qui ne renoncent pas pour autant à leur désir d'enfant, qui n'est après tout ni plus ni moins légitime que celui des autres, se procureront à l'avenir les enfants qu'ils désirent ? Pour le moment il demeure possible d'adopter les bébés des pays sous-développés qu'abandonnent les \tempuwave{pauvres} qui ne peuvent prévenir leurs grossesses autant qu'ils le voudraient, mais cela ne durera qu'un temps : et après ? 

 On pourrait dire cyniquement qu'il est toujours possible, dans les zones de non-droit, de faire disparaître des parents pour prendre leurs petits enfants, ou d'enlever les enfants qui sont les moins bien surveillés. C'est même très lucratif. Mais ces procédés criminels, dont les exemples contemporains ne manquent pas, ne peuvent donner à long terme de bons résultats : que répondre lorsque les anciens bébés enlevés demandent d'où ils viennent ? Si on leur dit la vérité, ils ne peuvent que prendre parti pour leurs parents de naissance, poussés \tempuwave{qu'ils sont} par la nécessité vitale de sauvegarder leur propre narcissisme. Et même si on la leur cache ils \tempuwave{flairent} le mensonge avec une sûreté (inconsciente) imparable, et ce mensonge empoisonnera toutes leurs relations jusqu'à sa levée (au moins).

 Mais il n'est pas nécessaire de recourir à des méthodes criminelles : tant que dureront les énormes inégalités de revenu observables sur cette planète, les plus fortunés pourront toujours louer le ventre des plus belles et des plus saines des filles des pauvres, de la même façon que les riches romains achetaient les plus jolies des jeunes esclaves afin qu'elles leur fassent des enfants bien à eux qu'ils n'auraient à partager ni avec un partenaire égal à eux en dignité, ni avec une belle famille aussi puissante que la leur. Le recours à des « mères porteuses » est dans la logique des évolutions libérales actuelles. Il est d'ores et déjà légalement possible dans plusieurs pays développés. Est-il appelé à se généraliser ? Comment refuser ce recours aux hommes homosexuels si l'on accorde l'assistance médicale à la procréation (PMA) aux femmes homosexuelles, et comment le refuser à tous les autres, hommes et femmes, si on l'accorde aux hommes homosexuels ? Et comment le refuser à qui que ce soit si des femmes (souvent pauvres et vivant dans des pays sous-développés) sont volontaires pour prêter leur ventre et abandonner leur enfant nouveau-né contre une indemnité suffisante. 

 C'est le seul moyen de mettre les hommes à égalité avec les femmes dans l'accès à l'enfant, ou plutôt de corriger l'inégalité que leur corps leur impose dans ce domaine, mis à part bien sûr le mariage traditionnel, monogame et indissoluble, dont c'était l'une des finalités. Lorsque leur mariage était rompu les pères romains gardaient leurs enfants : ils n'avaient donc pas particulièrement intérêt à ce que les unions soient indissolubles. Par contre leurs épouses avaient de bonnes raisons de craindre d'être répudiées et séparées de leurs enfants. Elles ont peut-être trouvé bon d'être mieux protégées de ce risque à partir du IVème siècle. Aujourd'hui où leur autonomie financière et les lois leur permettent de prendre l'initiative de quitter leurs maris sans quitter leurs enfants la situation se retourne et ce sont les hommes qui peuvent commencer de craindre d'être séduits puis abandonnés. 

 Plutôt que de se retrouver un jour contraints de continuer de payer pour leurs enfants sans plus les avoir auprès d'eux, tandis que souvent un autre qu'eux les éduque, les hommes pourraient choisir, quelle que soient par ailleurs leurs préférences sexuelles, de commencer par payer pour les posséder sans partage afin que personne ne puisse jamais les leur contester. Sur quels arguments fonder le refus d'une pareille évolution ? Elle ne serait au fond que le miroir de celle qui voit des femmes choisir en toute connaissance de cause de faire un enfant toutes seules. Si les humains ne diffèrent en rien de significatif en dehors de leurs caractéristiques biologiques, si les femmes n'ont pas besoin d'un homme pour élever un enfant, alors les hommes n'ont pas non plus plus besoin d'une femme pour assumer l'éducation de leurs propres enfants.

 En dépit de la pression des demandes individuelles et du modèle fourni par les pays où cette pratique est autorisée, le recours aux mères porteuses pourrait être interdit s'il était admis qu'il implique la réduction d'un humain au statut d'instrument de la volonté d'un tiers jusque dans son corps, s'il était reconnu que c'est inacceptable, même si cette personne a donné son accord, parce que cela fait de l'enfant à naître le produit d'un contrat commercial, toutes choses qui sont au cœur de l'esclavage. Mais refuser ce recours impliquerait aussi d'accepter l'idée qu'il n'existe pas de droit à l'enfant, c'est-à-dire que chacun peut être irrémédiablement privé d'enfant en dépit de ses désirs les plus authentiques et les plus légitimes. 

 Le mouvement des pratiques depuis un demi-siècle ne va pas dans ce sens.
 
 
