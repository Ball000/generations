
 Si les jeunes gens pouvaient se marier validement sans l'accord de leurs parents, en bonne logique ils avaient aussi le droit de ne pas se marier. Devant le désir d'un jeune de devenir religieux l'autorité du père s'arrêtait : pour les garçons à partir de 14 ans, pour les filles à partir de 12 ans. Le choix de la vie religieuse émancipait ceux qui le faisaient avant l'âge de leur majorité, et les mettait sous la protection de leur évêque : cela reposait évidemment sur une reconnaissance par les autorités civiles de la validité des vœux religieux. Cette reconnaissance leur a été accordée par les empereurs chrétiens et s'est maintenue jusqu'à la Révolution française.

 Ceux qui se sentaient attirés par une vie de célibataire consacré pouvaient proposer à une communauté religieuse de les coopter. Ce choix de vie entraînait des incidences légales importantes et définitives. En effet en prononçant ses vœux (pauvreté, chasteté, et surtout obéissance) le moine ou la religieuse se mettaient sous la puissance du responsable de la communauté, comme s'ils s'étaient fait adroger. Ils étaient juridiquement exclus de leur famille de naissance, et de tout héritage à venir. Comme des mineurs ils ne pouvaient plus rien faire de leur propre initiative. S'ils ne pouvaient signer aucun contrat en leur nom propre, ils pouvaient toujours, de la même façon qu'un esclave ou qu'un fils en puissance de \emph{pater familias}, exercer au nom de leur supérieur(e) tout mandat qu'il lui convenait de leur confier. Une fois entrés dans la communauté, c'était en principe pour toujours. Ils ne pouvaient plus sortir de leur état. Ils pouvaient dans une certaine mesure changer de monastère et même d'ordre religieux, mais les \emph{gyrovagues} qui erraient de couvent en couvent étaient mal vus. 

 Aucun religieux ne possédait rien qui lui soit personnel, et pourtant beaucoup d'entre eux avaient reçu de leurs parents une part d'héritage sous forme d'argent, de terre, etc. À leur entrée dans la communauté ils en avaient fait don (eux ou leurs parents) à la communauté, qui en contrepartie s'était engagée à les prendre en charge jusqu'à leur mort. Chaque communauté vivait de son travail et des revenus des biens qu'elle avait reçus en don : dots des religieux vivants \emph{et décédés}, loyers, récoltes, rentes et autres dons reçus de bienfaiteurs. Tous les biens appartenaient au monastère et celui-ci possédait le droit de posséder et d'exercer des actes juridiques en son nom propre. Si les religieux se succédaient de génération en génération, le monastère en tant qu'entité n'en persistait pas moins dans son être, unissant les morts et les vivants dans le même ensemble intemporel. Le modèle familial ainsi mis en œuvre était accepté en toute connaissance de cause ainsi que le montre l'emploi très précoce du vocabulaire de la famille : « père » (\emph{abba} = père en araméen = abbé), « mère », « frère », « sœur », etc. 

 ... Mais cette famille n'avait pas d'héritiers à pourvoir et ses biens étaient inaliénables et insaisissables, protégés par le statut de la \emph{mainmorte}. Ce mot a deux sens :
\begin{enumerate}
% a)
\item c'est le droit du seigneur de prendre les biens de son serf à sa mort. Les biens font \emph{échute}, c'est-à-dire réversion au seigneur qui en hérite. En ce sens les serfs sont \emph{gens de mainmorte}. Ce n'est pas le sens du mot \emph{mainmorte} qui nous concerne ici%
%[1]
\footnote{... même si un bon nombre des derniers serfs (fin \siecle{18}) appartenaient à des communautés religieuses de l'Est de la France, et si à cette époque les religieux ont eux aussi été nommés \emph{gens de mainmorte}, parce qu'incapables de transmettre des biens à des héritiers, non comme serfs d'un seigneur, mais comme religieux.}
 ;
% b)
\item on appelle aussi \emph{biens de mainmorte} ceux qui appartiennent à une personne juridique : ce sont les biens des collectivités qui ont le privilège de pérennité et n'ont pas à transmettre leurs biens à des héritiers. Cela conduisait à l'enrichissement progressif des communautés bien gérées... jusqu'au jour où leurs richesses devenaient trop tentantes pour les puissants du moment et leur étaient (re) prises par l'un d'entre eux : de ce point de vue l'histoire de la plupart des monastères est celle d'une suite de périodes d'accumulation et de moments de spoliation.
\end{enumerate} 

 À la fin de l'antiquité il était admis que dès leur plus jeune âge (6 ou 7 ans...) les parents puissent faire don à un monastère d'un ou plusieurs de leurs enfants, légitimes ou non%
% [2]
\footnote{Sources : Marc \fsc{BLOCH}, \emph{La société féodale}, Paris, 1939, 1994. Georges \fsc{DUBY}, \emph{Féodalité}, Paris, 1996, 1999. Collectif, \emph{L'homme médiéval}, Paris, 1989.}% 
. Ils accompagnaient le « don » de l'enfant d'un cadeau, souvent un bien foncier, qui devait permettre de subvenir à son entretien. Si l'enfant \emph{à Dieu donné} découvrait un jour que ce mode de vie ne lui convenait pas, il lui était extrêmement difficile d'en sortir. Le droit civil et les enseignements de l'Église se liguaient pour lui prêcher la résignation et lui barrer tout retour. Le jeune \emph{donné} à un monastère n'était d'ailleurs pas forcément plus contrarié dans ses choix que les jeunes esclaves, ou que les jeunes gens qui au même moment étaient mariés par leurs familles sans tenir compte de leur avis, ou qui devaient reprendre le métier de leur père. D'autre part, le \emph{don à Dieu} côtoyait des situations contemporaines par rapport auxquelles il représentait un progrès relatif (cf. {Boswell}) : abandon anonyme, infanticide, vente par les parents comme esclave, etc. 
 Les jeunes « donnés » ont pu à certaines périodes représenter une proportion importante de l'effectif des monastères, mais ceux-ci fournissaient aussi à ceux et celles qui n'étaient pas ou plus attirés par le mariage un moyen de l'éviter, alors que le célibat non consacré était mal accepté par la société civile. Cela permettait aux veuves d'échapper à la nécessité de se mettre sous la protection d'un mari. Cela donnait une chance aux femmes les plus douées de jouer un rôle public auquel elles n'auraient jamais pu rêver autrement. C'était le seul moyen pour les filles d'esquiver un mari grossier, mesquin ou brutal, et/ou d'éviter de risquer leur vie dans les grossesses et les accouchements%
% [3]
\footnote{... qui à l'époque faisaient mourir (en hôpital) une femme sur dix environ si l'on en croit les mémoires de \fsc{TENON}, ce qui ne représente pas une naissance sur dix, bien évidemment, mais d'une naissance sur 30 à une naissance sur 120 suivant les temps et les lieux (p. 242 et suivantes). Ce chiffre avait de quoi angoisser les jeunes filles, surtout celles de santé fragile, ou celles qui présentaient une malformation. Jacques Tenon était chirurgien à l'Hôtel-Dieu de Paris avant la Révolution. À la demande des autorités il a rédigé ses \emph{mémoires sur les hôpitaux de Paris} édités en 1788. Nous le citerons souvent.}%
. C'était un refuge pour les jeunes gens mal conformés ou de santé trop fragile. 

 D'un autre point de vue l'entrée en religion d'un enfant légitime diminuait mécaniquement le nombre des petits-enfants à naître (qu'il faudrait « établir » un jour sur le capital familial). C'était donc une forme indirecte de contrôle des naissances. C'est l'une des raisons, sinon la première, pour lesquelles les seigneurs grands et petits ont créé tant de monastères : ils avaient un intérêt direct à disposer d'institutions où placer l'excédent de leur progéniture dans un cadre conforme à la dignité de leur famille, et sans contrevenir aux lois de l'Église, qu'ils avaient à peu près intériorisées. D'ailleurs si la politique familiale l'exigeait (par exemple si les enfants privilégiés dans un premier temps décédaient) il n'était pas impossible de relever de ses vœux et de faire sortir du cloître une fille ou un fils, à la condition qu'il n'ait pas été ordonné prêtre (mais le plus souvent les moines ne l'étaient pas : pour le droit canon, c'étaient des laïcs, sauf ceux qui étaient ordonnés).

 Par ailleurs même quand on était un personnage public puissant il était difficile de faire d'un fils illégitime un successeur. D'autre part il était interdit d'ordonner prêtres les garçons illégitimes : pour eux le clergé séculier n'était un débouché envisageable qu'au prix de dispenses coûteuses. Au contraire les monastères ne manifestaient pas de réticences à les accueillir, tout comme ils accueillaient les enfants abandonnés. La vie des religieux est conçue comme une vie de purification. De plus c'est une vie cachée à l'écart du monde. Dans l'esprit du temps cela convenait parfaitement aux pécheurs et pécheresses repentis, aux natures perverties par le péché, et donc aux « impurs de naissance » ou aux clercs séculiers punis pour fautes graves. En outre, on considérait qu'ainsi les enfants illégitimes pouvaient racheter la faute de leurs parents. Il paraissait donc très louable de les vouer à la vie religieuse. Par ailleurs cela les excluait des jeux de pouvoir dont le monde profane est le théâtre. Il n'y avait plus à craindre de les voir parasiter les politiques familiales. C'est pour cela aussi que tant de filles de rois, légitimes ou non, qui ne pouvaient être données en mariage à des aristocrates inférieurs en dignité à leur beau-père au risque de donner naissance à des garçons d'ascendance royale, susceptibles de menacer les héritiers du trône, se sont retrouvées abbesses d'abbayes royales, jusqu'au \siecle{18}. C'est ainsi que les parents pouvaient estimer s'en sortir \emph{par le haut} du casse-tête créé par leurs enfants illégitimes.

 La part d'héritage (un bien foncier, une somme acquise définitivement par le couvent dès la profession, la \emph{dot}, etc.) donnée par leurs familles aux futurs religieux était fonction de la fortune familiale et du prestige de la maison religieuse où ils entraient. Chaque ordre et chaque monastère possédaient une « cote » sur le marché des valeurs de prestige, ce qui justifiait un coût (et inversement). En règle générale la part d'héritage de celui ou celle qui entrait dans les ordres était bien plus petite que celle des autres enfants de sa famille. Il convenait en effet qu'une fois tout réglé il reste aux pères un bénéfice financier à faire entrer des enfants en religion. 

 Selon la plupart des règles une communauté vivait non seulement de ses rentes, mais aussi du travail de ses membres. Mais souvent seuls les \emph{convers}, enfants de pauvres reçus gratuitement sans part d'héritage (ou adultes qui se donnaient eux-mêmes ainsi), travaillaient de leurs mains : sauf dons intellectuels ou spirituels éclatants ils n'étaient instruits que superficiellement et ils effectuaient l'essentiel des besognes matérielles, tandis que les religieux mieux dotés par leurs parents étudiaient, apprenaient à lire et à écrire, apprenaient le latin, chantaient au chœur, copiaient les livres, enseignaient, etc. Il y avait là une évidente sélection par l'argent et par la naissance. Pendant très longtemps il semble que nul n'y ait vu matière à scandale. C'est que depuis l'empire romain (dès Caracalla, sinon avant) les sociétés civiles contemporaines étaient très inégalitaires avec des castes et des hiérarchies civiles justifiées par l'idéologie de l'hérédité (du « sang »), à laquelle les barbares adhéraient autant sinon plus que les Romains. Il est vrai aussi que les exhortations de Paul de Tarse (entre autres) à demeurer à la place où Dieu vous a mis ne favorisaient pas la mise en question de l'ordre établi%
% [4]
\footnote{Il faudra attendre les ordres mendiants à partir du \siecle{12} pour que ces discriminations par l'argent au sein des ordres religieux soient dénoncées, mais non supprimées. François d'Assise, fondateur des franciscains, était fils de bourgeois, non d'aristocrate, ce qui lui avait sans doute donné un autre regard sur le caractère « naturel » des discriminations de caste. Elles ne semblent pas avoir été vécues comme insupportables avant le \crmieme{18} ou \siecle{19}, du moins pour la plupart des religieux qui avaient droit à l'écriture et qui ont laissé des témoignages : mais ce n'étaient pas eux qui étaient ainsi humiliés.}% 
. 

 Si les familles des bienfaiteurs et des fondateurs entretenaient des liens étroits avec « leur » monastère pour conserver la possibilité d'y placer des enfants, elles le faisaient aussi et au moins autant parce qu'elles comptaient sur les prières des religieux, sur leurs messes et sur leurs autres dévotions, offices divers qu'elles « fondaient » contre donation pour garantir le salut éternel des âmes de leurs membres. Chacun de ceux qui le pouvaient affectait une part de ses biens à ces donations comme leurs ancêtres pré chrétiens avaient affecté une part de leurs biens (un tiers selon Goody ?) aux sacrifices à faire après leur mort et aux objets qu'ils emportaient avec eux dans la tombe. C'est pourquoi jusqu'à la fin du moyen-âge presque tous les testaments contenaient des donations pour le repos de l'âme du défunt, faites à une institution religieuse et/ou d'assistance, ce qui à l'époque était indiscernable : toutes les œuvres d'assistance étaient aussi des « \emph{œuvres pieuses »}. Cela a contribué à produire en quelques siècles un quasi-monopole de l'Église dans le domaine des testaments et des conflits qui y sont liés.
 
 
 