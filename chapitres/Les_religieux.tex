
 Le mouvement monastique s'est développé depuis les premiers ermites qui ont fui le monde dès le \siecle{3} dans les déserts d'Égypte, et les premières veuves et vierges consacrées qui en ont fait autant à l'ombre des cathédrales, sous la protection des évêques. Il continuera de se développer à un rythme soutenu jusqu'au foisonnement de la fin du Moyen-Âge. Il prouvait par sa floraison que la continence \emph{perpétuelle} était possible%
% [2]
\footnote{... même si elle doit parfois s'appuyer sur \emph{l'impuissance de famine}, cf. Aline \fsc{Rousselle}, 1998, p. 203 - 224}% 
, et cela non seulement pour les femmes, de qui depuis toujours on l'exigeait au gré des besoins de leur famille, mais aussi pour les hommes. Saint Augustin, évêque de la fin du \siecle{4} et du début du \siecle{5}, vivait en communauté avec ses collaborateurs immédiats, communauté d'où sortiront un jour les chapitres de chanoines présents dans toutes les cathédrales. Au même moment les hôpitaux s'organisaient dans l'esprit des monastères. Ils étaient construits comme des églises dans lesquelles seraient logés des malades et si l'on en croit le concile de Nicée leur personnel était recruté parmi les religieux.

 S'appuyant sur les lettres de Paul de Tarse et les paroles du Christ, l'Église défendait le droit des jeunes de consacrer volontairement et librement leur vie à Dieu, alors qu'ils étaient encore \emph{dans la main} de leur père. Dans ce cas elle défendait leur droit de recevoir leur part d'héritage sans pour autant suivre la voie prévue par leurs parents, part d'héritage sans laquelle leur liberté de choix serait restée formelle. Cela leur permettait de s'engager dans le monastère ou l'hôpital de leur choix en faisant don à leur communauté de leur part d'héritage%
% [3]
\footnote{C'est ainsi qu'était mis en pratique la proposition de donner tous leurs biens aux pauvres faite par le Christ à ceux qui voulaient choisir la perfection (Parabole du « jeune homme riche ») : en effet leur nouvelle famille spirituelle n'était constituée que de membres qui avaient fait vœu de pauvreté.}% 
. On peut supposer que ce n'est pas par hasard qu'en 320 Constantin avait abrogé les lois d'Auguste qui exigeaient d'avoir engendré trois enfants et d'être marié pour recevoir les héritages venant de personnes éloignées, et qu'il avait posé des limites au droit des pères de déshériter un enfant. Contrainte par sa propre logique, et fidèle sur ce point au droit romain, l'Église plaidait pour le consentement mutuel des fiancés et contre l'idée que celui de leurs parents était nécessaire pour que leur mariage soit valide%
%[4]
\footnote{Là aussi elle allait contre l'autorité des pères. Cet enseignement-là restait en travers de la gorge de bien des pères, mais aussi des ecclésiastiques eux-mêmes pour autant qu'ils s'identifiaient aux intérêts temporels de leur famille d'origine, cf. les avanies subies par Abélard, alors qu'il était encore laïc et donc épousable, du fait de l'ecclésiastique qui était oncle et tuteur d'Héloïse.}% 
.

 Les revenus des monastères, des évêchés et des hôpitaux étaient fondés sur des propriétés, terres, domaines, etc., provenant des dons et des legs. Grâce aux rentes sur la terre%
% [5]
\footnote{Ressentie de l'antiquité à la fin du moyen-âge (au moins) comme le seul bien qui ne fait jamais défaut, et dont les fruits permettent de survivre quelle que soit la catastrophe économique qui puisse arriver (Paul \fsc{Veyne}, \emph{La société romaine}, chapitre).} 
et les immeubles (en nature ou en argent) il était possible sans recourir à l'impôt de « fonder » (en principe une fois pour toutes) des emplois \emph{(bénéfices)} de clercs, des écoles, des hôpitaux, des monastères, etc. Ce mode de financement était hérité de l'antiquité pré chrétienne. C'était déjà celui des temples païens. S'ajoutaient à ces revenus des contributions régulières notamment les différentes \emph{dîmes} versées par les fidèles, d'abord volontaires, puis obligatoires. Ainsi les institutions ecclésiastiques étaient autonomes et auto-suffisantes, sans courir les risques du marché, ni dépendre étroitement de généreux donateurs ou des pouvoirs locaux. Ce système ne faisait peser aucune charge récurrente sur le budget de la puissance publique et donnait aux institutions un maximum de liberté face aux pressions des pouvoirs publics. 

 Jusqu'à la fin du Moyen-Âge une part de presque tous les héritages était donnée aux pauvres (c'est-à-dire à leur protectrice officielle : l'Église) pour \emph{le salut de l'âme} des donateurs. Il existait déjà chez les anciens des fondations identiques auprès des temples païens. Quant aux barbares ils admettaient comme les Égyptiens, les Celtes et les Germains que chaque mort emporte dans son tombeau des biens pour l'au-delà, ce qui du point de vue des chrétiens ou des juifs était un signe de superstition. Cette part des biens du mourant qu'il comptait emporter avec lui (jusqu'à un tiers de sa fortune ?) l'Église lui proposait d'en faire meilleur usage, en l'investissant dans les \emph{œuvres pies} (pieuses). 

 Selon Raymond Goody il y avait un lien entre la défense par l'Église de la liberté de choix de vie des jeunes, celle du droit des jeunes à une part d'héritage même en cas de désaccord paternel, celle des chrétiens à faire des donations (notamment dans leur testament) et le financement des institutions religieuses qui fournissaient les lieux où chercher la perfection. Selon lui la nécessité de trouver des ressources pour faire vivre les paroisses, monastères et hôpitaux a exercé une pression déterminante sur la définition même des règles du droit de la famille. Elle aurait contribué à ce que le droit de l'Église mette des limites au droit des pères à imposer leur volonté à leurs enfants. Elle aurait aussi et surtout contribué à étendre les degrés de parenté interdisant les mariages. Même si cette thèse paraît un peu extrême, comme toute thèse qui attribue à une cause unique un mouvement observable sur plus de dix siècles, elle n'en contient pas moins une part de vérité significative. 

 En dehors du travail de leurs membres, qui exigeait lui-même un minimum d'outils de production et d'abord de terres, le financement des monastères reposait sur les \emph{dots} des postulants, notamment dans les monastères féminins qui ne pouvaient bénéficier comme les monastères d'hommes des honoraires de messes offertes pour le repos de l'âme des défunts. Au décès du religieux sa dot demeurait acquise au monastère (du moins tant qu'elle a consisté en un capital et non en une rente). Celui-ci avait donc des chances de voir grossir peu à peu son capital. Cela permettait (dans les meilleurs cas) d'accepter les postulants sans le sou et de consacrer le superflu au service des pauvres et des malades.
 
 
 