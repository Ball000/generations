
\chapter{Les devoirs des pères de l'Ancien Régime}


 Présenté par l'idéologie officielle, et sans doute vécu par (presque) tous ses sujets comme un père, le roi s'identifiait à son tour à tous les pères de famille. Leurs images étaient directement articulées avec celle de « Dieu le Père ». Chacune de ces images soutenait les autres et les justifiait. Jusqu'au milieu du \siecle{18} l'image paternelle partagée par l'immense majorité était globalement positive, et on estimait que l'exercice que pères et rois faisaient de leurs pouvoirs était bénéfique, alors que leur absence était un mal.

 Même s'ils n'étaient en aucune façon tout-puissants, on continuait pourtant de reconnaître aux pères, en contrepartie de leurs devoirs, une grande part de la puissance que détenaient les citoyens romains sur leurs enfants et leurs épouses. Dans les pays de droit écrit (comme le sud de la France) revenus à la fin du Moyen-Âge à une application stricte du droit romain, leur puissance ne cessait qu'avec leur mort. Partout leur mission éducative impliquait le \emph{droit de correction}. On considérait que c'était un devoir moral et social que de corriger les enfants \emph{et les épouses indisciplinées}. Jusqu'au \siecle{18} (au moins) il était admis qu'une tendresse excessive était plus dommageable pour l'enfant, et donc plus coupable, qu'une sévérité excessive : « {qui aime bien châtie bien} ». Montaigne nous dit qu'il fut placé de la naissance à l'âge de quatre ans chez des bûcherons, puis mis en pension en collège à partir de six ans. Il dit s'être trouvé mieux de cette enfance loin de sa famille... parce qu'il lui semblait que son père était « trop tendre%
% [1] 
\footnote{En tous cas en justifiant la décision de son père par son « excès de tendresse » Montaigne nous fournit un bel exemple de ce qu'on désigne aujourd'hui sous le nom de « fidélité » ou de « loyauté » des enfants, et des trésors de compréhension dont ils sont prêts à faire preuve face à toutes les décisions, quelles qu'elles soient, que leurs parents ont pu prendre.} 
» !

 Le roi soutenait l'autorité des pères sur leurs épouses et sur leurs enfants, et leur prêtait main-forte s'ils le demandaient, entre autres moyens par les \emph{lettres de cachet} ordonnant sans jugement l'incarcération de l'enfant récalcitrant, mineur ou majeur, ou de l'épouse indigne, volage ou frivole ou de mauvais caractère, etc. S'il le jugeait nécessaire, il pouvait se substituer de sa propre initiative%
% [2] 
\footnote{De même que lorsqu'il s'agit de ses enfants un père n'attend pas d'être saisi : par définition il parle en leur nom et à leur place (et en droit romain même quand ils sont adultes).} 
aux pères défaillants dans leur fonction de faire régner l'ordre dans leurs familles. 

 Mais il se devait aussi de contrôler qu'ils n'abusaient pas de leurs pouvoirs : leur droit de correction n'était pas un droit de vie ou de mort. Jamais les parents n'ont été autorisés à estropier leurs enfants, et l'appui donné par la force publique à leurs décisions n'était pas automatique.

 Selon les juristes et les théologiens du Moyen-Âge, être père (c'est-à-dire parent%
% [3]
\footnote{Quand on parlait des devoirs des pères on parlait aussi de ceux des mères, puisque si le père disparaissait la mère héritait de la quasi totalité de ses devoirs et pouvoirs.}% 
) impliquait trois devoirs : 
\begin{enumerate}
% a)
\item %Le premier devoir était celui de 
faire vivre l'enfant. L'avortement et l'infanticide étaient des délits. Celui qui ne pouvait ou ne voulait pas reconnaître l'enfant qu'il avait engendré était tout de même astreint à le nourrir, sauf quand il s'agissait d'un enfant adultérin. A fortiori un père devait-il nourrir son enfant légitime jusqu'au jour où celui-ci serait capable de subvenir lui-même à ses besoins. Il n'était pas moralement acceptable de l'abandonner, et encore moins de le vendre, sauf en cas d'extrême indigence ;
% b)
\item %Le deuxième devoir était celui d'
éduquer : par ses paroles et par ses exemples le père devait apprendre à ses enfants (garçons) tout ce qu'il faut savoir pour être un homme. Son épouse se chargeait de l'éducation de leurs filles, sous son autorité. Le plus souvent il leur apprenait son métier, son savoir, ses tours de main, il transmettait son expérience. Il pouvait déléguer sa mission d'enseignement à des tiers plus compétents ou plus disponibles que lui (alliés ou parents, précepteurs, enseignants, maîtres d'apprentissage, etc.). Dès l'antiquité on connaissait le contrat par lequel un homme confiait contre paiement son fils à un artisan expérimenté afin qu'il lui apprenne son métier, contrat dans lequel il déléguait à ce maître d'apprentissage son autorité sur son fils ;
% c)
\item %Le troisième devoir était celui de 
transmettre : il s'agissait pour le père de transmettre ce qu'il avait reçu de ses parents et ce qu'il avait acquis, les biens, les outils de production, etc. C'était le devoir du père d'établir dans la vie tous ses enfants, c'est-à-dire de leur fournir les moyens de subsister : un domaine, un outil de travail ou un savoir-faire, un métier. Selon le droit, le père \emph{carent}, c'est celui qui ne transmet aucun bien. En pays de droit écrit la tendance était à la transmission de l'ensemble de l'héritage à un seul héritier, choisi librement par le testateur, ce qui renforçait considérablement l'autorité paternelle. Cette tendance s'est encore durcie du \crmieme{12} au \siecle{18}. En pays de droit coutumier la tendance était au contraire (et est restée) à la succession \emph{ab intestat}, avec égalité entre tous les enfants, sans testament, sans liberté pour le testateur. Mais de petit pays à petit pays on pouvait rencontrer d'innombrables cas de figure. Dans presque tous les cas les filles qui avaient été dotées pour se marier étaient exclues de l'héritage, de même que les garçons et les filles entrés dans les ordres religieux : ils et elles avaient déjà reçu leur part d'héritage. Les veuves connaissaient un sort variable suivant les lieux. 
\end{enumerate}

\section{Montée en puissance de l'enseignement}

 Dans le même temps que le fonctionnement des familles semble s'être rigidifié sous l'autorité des pères, le domaine scolaire connaissait des évolutions importantes, toujours sous le contrôle étroit des évêques qui en assuraient pour l'essentiel le fonctionnement%
% [4]
\tempnote{J'ai tenté cette présentation de la note, pour l'éclaircir...}%
\footnote{Sur ce sujet :
\\\fsc{FURET} et \fsc{OZOUF}, \emph{Lire et écrire, l'alphabétisation des français de Calvin à Jules Ferry}, 1977.
\\Maurice \fsc{CAPUL}, \emph{Internat et internement sous l'ancien régime, contribution à l'histoire de l'éducation spéciale}, Thèse d'état, CTNERHI-PUF, Paris, 1983-1984.
\\Martine \fsc{SONNET}, {« Une fille à éduquer », in \emph{Histoire des femmes en Occident}, III, \siecles{16}{18}}, Collectif, sous la direction de Georges \fsc{DUBY} et Michelle \fsc{PERROT}, 2002, Chapitre 4, p. 131 à 168.
\\Georges \fsc{MINOIS}, \emph{Les religieux en Bretagne sous l'Ancien Régime}, 1989 ...}% 
, et pour une grande part le financement.

 À partir des derniers siècles du Moyen-Âge, de \emph{petites écoles} (écoles primaires) existaient dans toutes les villes épiscopales. Elles étaient les héritières directes des \emph{écoles cathédrales} du haut Moyen-Âge, et en tant que telles placées sous l'autorité du Chapitre. Au sein de celui-ci c'était le \emph{Chantre} de la Cathédrale qui avait la responsabilité de leur contrôle. C'est lui qui jusqu'à la fin de l'ancien régime agréait les candidats à l'enseignement, agrément sans lequel ils n'avaient pas droit d'enseigner. 

 À côté des connaissances profanes (lire, compter, parfois écrire) on enseignait aussi la religion, les disciplines du corps et de l'esprit, les bonnes manières de se conduire. La mission des petites écoles était en effet d'éduquer autant que d'enseigner. L'instruction, étant entendu qu'elle se devait d'inclure la religion, était considérée comme la meilleure défense contre l'envie de mal faire. Curés (pasteurs), parents et autorités locales étaient d'accord sur ce point. D'autre part les citadins voyaient aussi en elle la meilleure arme pour trouver et pour garder un travail, ce qui avait à la fois un intérêt économique et un intérêt social. À partir du Concile de Trente cette foi en l'enseignement s'est traduite par un véritable apostolat, dont témoignent les innombrables créations d'ordres enseignants. 

 Les petites écoles s'adressaient en priorité aux enfants des pauvres, c'est-à-dire, dans le langage d'alors, de tous ceux dont les ressources étaient précaires, de ceux qui n'avaient pas de rentes, de quelque nature qu'elles soient, et qui devaient travailler de leurs mains. Il s'agissait donc de l'essentiel de la population des villes. Mais les petites écoles étaient ordinairement payantes, comme les universités d'alors, qui étaient elles aussi sous le contrôle du Chapitre cathédral. Elles étaient sans doute à la portée des bourgeois aisés, commerçants et d'artisans, mais pas forcément à celle des autres. Lorsque les parents ne pouvaient pas payer et lorsque les paroisses ne pouvaient pas les subventionner (ce qui allait de pair : les paroisses étaient désargentées lorsque les paroissiens étaient pauvres, puisqu'il n'y avait pas de péréquation entre elles), elles ne pouvaient vivre que grâce à la modicité des salaires des enseignants. Dans ce contexte, seuls des ordres religieux pouvaient les prendre en charge, puisqu'ils bénéficiaient d'une sécurité financière et d'une surface sociale que ne pouvaient avoir des particuliers ou des communes pauvres. Les ordres enseignants ont donc fondé beaucoup de petites écoles gratuites, parfois ouvertes à côté d'un pensionnat géré par la même congrégation, qui faisait ainsi payer les riches pour les pauvres. 

 Malgré la gratuité de certaines écoles, beaucoup d'enfants n'étaient pas scolarisés, même dans les villes : leurs parents avaient trop besoin du produit de leur travail, ou bien ils ne voyaient aucune utilité à un apprentissage scolaire. Il n'était d'ailleurs pas évident pour ceux qui envoyaient leurs enfants à l'école qu'il faille que ceux-ci soient scolarisés avec assiduité pendant plusieurs années. Beaucoup se contentaient des quelques mois ou années nécessaires pour apprendre à lire et à écrire.

 Quant à l'instruction des paysans, jusqu'à la fin de l'ancien régime elle n'était pas jugée nécessaire. L'illettrisme n'avait pas d'incidence sur la productivité des travailleurs agricoles, étant donné le niveau des techniques alors en usage. D'autre part les maîtres et seigneurs craignaient qu'une instruction même minime ne les rende « raisonneurs » et « arrogants ». Voltaire, représentatif de sa classe et de son temps, était lui aussi de cet avis.

 Les écoles épiscopales, à la différence des petites écoles, enseignaient le latin sans lequel, selon les conceptions du temps, on restait un illettré, c'est-à-dire un non lettré. Créées à la fin de l'Antiquité pour fournir l'Église en clercs, elles ont toujours reçu un certain contingent d'élèves de familles relativement aisées promis à la vie civile. La croissance des villes a fait croître la demande d'instruction supérieure. Les collèges ont été créés sous l'autorité épiscopale, durant les derniers siècles du Moyen-Âge, sur le modèle monastique (comme le collège qui deviendra la Sorbonne) et se sont généralisés à partir de la Renaissance.

 Du point de vue des pédagogues d'alors, l'idéal était l'internat car selon eux il mettait les jeunes dans des conditions éducatives plus favorables. Pour les mêmes raisons, la formule homologue du \emph{couvent} se généralisait au même moment à l'intention des filles de familles aisées, bien qu'il semble que la durée de leurs séjours dans les couvents ait été très inférieure à celle de leurs frères dans les collèges, sauf quand il s'agissait d'un prélude à un noviciat, et que l'enseignement qui leur était dispensé était bien moins poussé (exclusion habituelle du latin, etc.). Entre les petites écoles (sans internat) et les internats des collèges, l'écart des coûts était énorme. Il fallait être vraiment à l'aise financièrement pour envoyer un ou plusieurs de ses enfants en pension%
% [5]
\footnote{Selon Martine \fsc{SONNET}, la pension d'un seul enfant représentait presque la totalité du salaire d'un ouvrier (« une fille à éduquer », Chapitre 4 de \emph{l'Histoire des femmes en Occident}, III, \siecles{16}{18}, p. 146). C'est pourquoi en 1760 les internats parisiens n'accueillaient que 13~\% de la population scolaire.}% 
. Mais pour les parents des collégiens l'éducation était un investissement familial, du moins quand il s'agissait des garçons, et même quand ils se destinaient à devenir des clercs (ce qui a été le cas de beaucoup ou même de la plupart des pensionnaires jusqu'au \siecle{18}). Cela justifiait des sacrifices. Aux familles qui ne pouvaient payer les frais d'une pension, seul l'externat était possible, en vivant en ville chez un membre de sa famille ou chez un logeur peu exigeant. En réalité dans les collèges il y avait beaucoup d'externes, sans doute plus que de pensionnaires au total, et souvent il n'y avait pas d'internat du tout. Mais le prestige de l'internat et la nécessité pratique de regrouper les collégiens en un seul lieu, souvent loin de leur domicile, a fait que la floraison des internats s'est poursuivie longtemps. Leur réseau n'a été achevé qu'au début du \siecle{20} (le nombre de places d'internat semble s'être maintenu ensuite sans grands changements jusqu'aux années soixante où il a commencé à baisser). 

 Avec l'internat il s'agissait de protéger les « enfants de famille » contre les « mauvaises influences » qui les « pervertissaient », tout en préservant les « filles honnêtes », la « paix des ménages », et la « tranquillité publique » des « débordements de la jeunesse ». Une fois pensionnaires, les jeunes ne perturbaient plus la vie de la cité où ils étudiaient, comme pouvaient le faire des bandes d'externes de tous âges vivant loin du contrôle de leurs parents et sur lesquels les logeurs comme les maîtres n'avaient guère d'autorité. Leurs parents n'avaient plus à craindre qu'ils se portent tort à eux-mêmes dans une vie d'écolier trop irrégulière, ni qu'ils tombent sous des influences « pernicieuses ». Quant aux jeunes filles, la clôture des couvents empêchait toute rencontre avec les jeunes gens de leur âge et protégeait donc leur « vertu » et leur réputation en attendant qu'elles soient mariables. L'internat était une garantie contre les erreurs de jeunesse qui réduisaient à néant les stratégies familiales. Dans un cadre totalement maîtrisé (lieux, relations, temps), le (la) jeune interne était coupé de ses parents d'une manière rigoureuse. Il était contenu fermement (murs, grilles, portier, clôture, clés, etc.) à l'abri des tentations et mauvaises influences du monde extérieur. Il s'agissait de donner rudement de saines habitudes au corps et à l'esprit%
% [6]
\footnote{À partir de 1640 et jusqu'au \siecle{19}, le Jansénisme a remporté un très important succès d'estime dans le clergé séculier et dans les milieux bourgeois cultivés. Pour ceux qui lui étaient favorables, l'une des fins du collège était de châtier en l'élève les appétits désordonnés et l'inclination au péché, dont l'emblème était le désir sexuel. Les jésuites étaient leurs principaux adversaires. Cela coïncidait dans une grande mesure avec le clivage entre gallicans et ultramontains.}%
. 

 Pour ce faire les collégiens étaient confiés soit à des clercs recrutés sur place, de niveau universitaire inégal et parfois médiocre (le plus souvent ceux-ci, sans compétences particulières autre que leurs titres de lettrés, gagnaient leur vie en enseignant tout en attendant de se voir attribuer un « bénéfice » ecclésiastique plus lucratif) soit à l'un des ordres religieux spécialisés à partir de la Renaissance dans l'enseignement (Jésuites surtout, mais aussi Oratoriens, Dominicains, etc.). Les collèges étaient ouverts à la ville dans les murs de laquelle ils étaient établis. S'ils étaient bien tenus et de qualité ils en formaient ordinairement l'un des fleurons les plus prestigieux. Ils y entretenaient une vie intellectuelle et mondaine active et d'autant plus valorisée que les autres sources de distraction étaient rares. 

 Pour prix de leur liberté perdue, il était proposé aux collégiens de s'investir dans la découverte du savoir, et celui-ci était ressenti par leurs enseignants et leurs parents comme quelque chose qui en valait la peine. Ils entraient dans une aristocratie de l'esprit. La première chose que réclamaient les parents était qu'on enseigne le latin à leurs fils (pas à leurs filles). À l'époque l'enseignement secondaire et supérieur se faisait en latin dans toute l'Europe%
% [7]
\footnote{Dans toute l'Europe les thèses seront encore soutenues en latin durant la plus grande partie du \siecle{19}.}% 
. C'était la langue vivante, la langue de communication des communautés intellectuelles du temps. 

 Mais depuis l'\emph{ordonnance de Villers-Cotterêts} (1539) qui imposait le français comme langue administrative du Royaume, il n'était plus possible de tenir un \emph{office} public si on ne le maîtrisait pas suffisamment. La langue française n'était encore que le parler de l'Île-de-France, domaine du roi. Partout ailleurs c'était une langue étrangère qui allait mettre très longtemps à déloger les langues locales des places et des marchés. L'enseignement du français reçu dans les collèges était donc incontournable pour entrer dans les professions libérales, la fonction publique ou le clergé. 

 Le collège était par conséquent un moyen sûr de promotion individuelle et familiale. Pour les gens ordinaires c'était la seule voie d'accès aux emplois prestigieux et qualifiés. Le fils de famille confronté à l'épreuve du collège continuait de dépendre de ses parents. Ils payaient sa pension : il continuait de manger leur pain. Il continuait de correspondre avec eux. Il les retrouverait aux prochaines vacances s'ils ne venaient pas le voir avant. Les perspectives d'avenir qu'ouvrait l'internat étaient intéressantes. Quand la discipline lui pesait trop il pouvait se dire avec assez de vraisemblance qu'il était en train d'acquérir à ce prix les moyens d'atteindre un statut personnel valorisant, et qu'il s'inscrivait harmonieusement dans le projet de ses parents. En acceptant de se soumettre il pouvait espérer devenir un membre puissant et respecté de sa communauté d'origine : cela présentait l'allure d'une épreuve initiatique. 

\section{La correction paternelle}

 Tous les jeunes de famille n'entraient pas docilement dans les projets parentaux. Certains d'entre eux entraient en conflit ouvert avec leurs parents au-delà des normes reçues (éminemment variables suivant les siècles et les lieux) : vagabonds, fugueurs, jeunes aux fréquentations suspectes, exclus pour indiscipline de collèges successifs, fauteurs de vols domestiques ou d'actes « d'inconduite sexuelle », d'insultes et de voies de faits, « libertins », c'est-à-dire jeunes rétifs à toute mesure éducative, etc. 

 À la demande de leurs parents, ces jeunes peuvent être traités en tout comme les délinquants condamnés. Pour les enfants difficiles des familles aisées il y avait des solutions payantes dans les sections des collèges et internats contemporains affectés à la « correction ». Ceux qui n'en avaient pas les moyens étaient internés avec les délinquants condamnés, dans les sections « de force » des hôpitaux, où s'effectuaient les peines de prison. Leurs parents payaient une pension qui tenait compte de leurs ressources. 

 À partir de la fin du \siecle{17} et de plus en plus souvent au fil du \crmieme{18}, les \emph{enfants de famille}, garçons et filles mineurs \emph{et majeurs}, qui ont commis de vrais actes de délinquance, mais aussi ceux qui donnent simplement du mécontentement à leurs parents par leurs fréquentations, leur mauvaise conduite, leur indocilité, leur violence aveugle ou leur absence de sens commun (« insensés »), leurs dépenses inconsidérées, ou leurs dettes de jeu, peuvent, sur la demande de ces derniers exerçant ainsi leur droit de correction, faire l'objet d'une \emph{lettre de cachet}, c'est-à-dire d'une \emph{décision administrative d'internement} dans un hôpital, une prison, une forteresse, un couvent, un collège, ou même leur déportation aux colonies. Les lettres de cachet, qui ont une origine très ancienne, bien antérieure au \siecle{17}, peuvent aussi être accordées à l'encontre de conjoints aux comportements répréhensibles (cette mesure a beaucoup plus souvent frappé les femmes que les hommes). 

 L'autorité publique n'est pas obligée d'accorder satisfaction aux demandes qui lui sont faites, et reste seule juge de l'opportunité de la mesure. Elle est surtout sollicitée à Paris, notamment par les couches populaires, contrairement aux provinces où l'internement administratif est moins facile à obtenir et où les couches populaires n'y ont guère recours. Même si au fil du temps les lettres de cachet font l'objet de critiques de plus en plus virulentes et si les autorités publiques y répugnent de plus en plus, les demandes se font de plus en plus nombreuses au fil du \siecle{18}. 

 En effet les familles sollicitent ces lettres comme une grâce : cela leur évite la honte causée par la publicité du recours à la justice, le coût d'un procès, et aussi la publicité de la mesure d'enfermement. La réputation du jeune (ou de l'adulte) ainsi placé peut s'en relever plus facilement. Cela évite le contrôle par la justice de la nature exacte des faits et de la proportionnalité des sanctions aux dommages et délits constatés, ce qui permet à d'authentiques délinquants bien nés d'échapper à peu de frais aux conséquences normales de leurs actes. 

 Mais cela permet aussi aux parents abusifs d'exercer des pressions sur leurs enfants rétifs à leurs projets (ce qui explique les critiques de plus en plus virulentes des lettres de cachet), à une époque où le consentement des parents est exigé à tout âge et pour tout mariage sous peine d'exhérédation, et où bien des entrées en religion sont imposées par eux sans tenir compte des désirs du ou de la jeune concerné. 

\section{Enfants « adoptifs »}

 On a vu que dans le but de défendre le mariage monogame et indissoluble, l'Église a tout fait depuis l'antiquité pour que les enfants illégitimes ne puissent pas devenir des héritiers de plein exercice. C'est pour cette raison que l'adoption était interdite, et pourtant... De l'antiquité à la fin de l'ancien régime, on peut observer en nombre non négligeable des situations plus ou moins proches d'une adoption, où une personne, souvent un ecclésiastique (cf. \hbox{Villon}, adopté par un chanoine), souvent aussi un couple sans enfants, exerçaient la puissance paternelle sur un enfant qui n'était pas né d'eux et l'élevaient jusqu'à sa majorité. C'était par exemple le cas à Lyon, où les recteurs de l'Hôtel-Dieu « adoptaient » ainsi des orphelins. 

 Ces situations d'\emph{alumnii} (adoptions simples) étaient parfois sanctionnées par des actes juridiques où les nourriciers faisaient un legs à l'enfant devant un procureur fiscal, et où ils s'engageaient à l'élever, instruire et établir matériellement à leurs frais comme leur propre enfant. Pour autant cela ne faisait pas de lui un membre de leur famille ni un héritier. 

 En principe seul un enfant légitime sans parents pouvait bénéficier de ce dispositif. Souvent, probablement le plus souvent, il était orphelin, mais des enfants légitimes pouvaient aussi être abandonnés solennellement par leurs parents, qui reconnaissaient par écrit qu'ils renonçaient à leur puissance paternelle, et à l'héritage de leur enfant s'il décédait. Pour autant ce dernier ne changeait ni de parenté ni de nom. Quand il possédait des biens, l'adoptant, tel un tuteur, les gérait jusqu'à sa majorité et il était responsable sur ses propres biens de sa gestion. 

 Les enfants abandonnés, nés de parents inconnus, ont très longtemps été exclus de ce genre de prise en charge%
% [8]
\footnote{À Lyon jusqu'en 1765. Ensuite ils y ont été traités comme les autres. Ce n'est que dans les dernières années avant la Révolution que les idées ont changé sur ce point : un signe de l'évolution qui s'est faite dans les esprits au fil du \siecle{18} et qui est apparue au grand jour à partir des années 1760-1770.}%
. Pourtant il était courant que des personnes accueillent pour l'élever un enfant abandonné à eux confié par un hôpital ou par une paroisse, qu'elles refusent d'être rémunérées pour l'élever, qu'elles le gardent jusqu'à sa majorité et qu'elles l'établissent dans la vie, ce qui en fait ressemblait beaucoup à la situation des enfants nés légitimes et juridiquement « adoptés ». Si aucun de leurs héritiers légitimes ne s'y opposait, elles faisaient de lui l'un de leurs héritiers. Mais il n'était pas question pour cet enfant d'hériter d'une fonction impliquant l'exercice public du pouvoir. 

 Derrière les mots employés il n'est pas toujours facile de reconnaître les situations réelles : adoption simple ? tutelle ? parrainage ?%
% [9]
\footnote{Cf. Jean-Pierre \fsc{GUTTON}, \emph{Histoire de l'adoption en France}, 1993.} 


