% 28.02.2015 :
% haut Moyen Âge
% _, --> ,
% Antiquité
% Le 24.02.2015 :
% ~etc.
% Moyen-Âge
% ~\%


\chapter{Progrès ou régressions ?}


 De même que chaque époque possède ses propres représentations sur le bien, le mal, le désirable et l'insupportable, chaque époque secrète aussi sa propre vision de l'avenir (cette vision éclaire en général beaucoup mieux sur l'époque elle-même que sur l'avenir). Dans un texte publié le 29 janvier 2013 sur \href{http://www.slate.fr}{Slate.fr}, dans le cadre des polémiques préalables au vote ouvrant le mariage aux personnes de même sexe, Jacques \fsc{ATTALI} prolonge les évolutions des mœurs actuelles de manière rigoureusement logique, et avec l'intention de ne porter aucun jugement d'aucune sorte, ni moral, ni d'opportunité. Ce texte me semble exposer clairement les idées que l'on se fait aujourd'hui de ce que seront les pratiques à venir dans le domaine familial et reproductif :
 
« \emph{Comme toujours, quand s'annonce une réforme majeure, il faut comprendre dans quelle évolution de long terme elle s'inscrit.}
 
« \emph{Et la légalisation, en France après d'autres pays, du mariage entre deux adultes homosexuels, s'inscrit comme une anecdote sans importance, dans une évolution commencée depuis très longtemps, et dont on débat trop peu : après avoir connu d'innombrables formes d'organisations sociales, dont la famille nucléaire n'est qu'un des avatars les plus récents, et tout aussi provisoire que ceux qui l'ont précédé, nous allons lentement vers une humanité unisexe, où les hommes et les femmes seront égaux sur tous les plans, y compris celui de la procréation, qui ne sera plus le privilège, ou le fardeau, des femmes.} 
 
« \emph{\primo La demande d'égalité. D'abord entre les hommes et les femmes. Puis entre les hétérosexuels et les homosexuels. Chacun veut, et c'est naturel, avoir les mêmes droits: travailler, voter, se marier, avoir des enfants. Et rien ne résistera, à juste titre, à cette tendance multiséculaire. Mais cette égalité ne conduit pas nécessairement à l'uniformité: les hommes et les femmes restent différents, quelles que soient leurs préférences sexuelles.}
 
« \emph{\secundo La demande de liberté. Elle a conduit à l'émergence des droits de l'homme et de la démocratie. Elle pousse à refuser toute contrainte ; elle implique, au-delà du droit au mariage, les mêmes droits au divorce. Et au-delà, elle conduira les hommes et les femmes, quelles que soient leurs orientations sexuelles, à vouloir vivre leurs relations amoureuses et sexuelles libres de toute contrainte, de tout engagement. La sexualité se séparera de plus en plus de la procréation et sera de plus en plus un plaisir en soi, une source de découverte de soi, et de l'autre. Plus généralement, l'apologie de la liberté individuelle conduira inévitablement à celle de la précarité ; y compris celle des contrats. Et donc à l'apologie de la déloyauté, au nom même de la loyauté : rompre pour ne pas tromper l'autre. Telle est l'ironie des temps présents : pendant qu'on glorifie le devoir de fidélité, on généralise le droit à la déloyauté. Pendant qu'on se bat pour le mariage pour tous, c'est en fait le mariage de personne qui se généralise.}
 
« \emph{\tertio La demande d'immortalité, qui pousse à accepter toutes mutations sociales ou scientifiques permettant de lutter contre la mort, ou au moins de la retarder.}
 
« \emph{\quarto Les progrès techniques découlent en effet de ces valeurs et s'orientent dans le sens qu'elles exigent: en matière de sexualité, cela a commencé par la pilule, puis la procréation médicalement assistée, puis la gestation pour autrui. Ces questions de bioéthique ne découlent évidemment pas des demandes d'égalité venant des couples homosexuels et concernent toutes les formes de reproduction, y compris -- et surtout -- « hétérosexuelles ». Le vrai danger viendra si l'on n'y prend garde, du clonage et de la matrice artificielle, qui permettra de concevoir et de faire naitre des enfants hors de toute matrice maternelle. Et il sera très difficile de l'empêcher, puisque cela sera toujours au service de l'égalité, de la liberté, ou de l'immortalité.}
 
« \emph{\FrenchEnumerate{5} La convergence de ces trois tendances est claire: nous allons inexorablement vers une humanité unisexe, sinon qu'une moitié aura des ovocytes et l'autre des spermatozoïdes, qu'ils mettront en commun pour faire naitre des enfants, seul ou à plusieurs, sans relation physique, et sans même que nul ne les porte. Sans même que nul ne les conçoive si on se laisse aller au vertige du clonage.}
 
« \emph{\FrenchEnumerate{6} Accessoirement, cela résoudrait un problème majeur qui freine l'évolution de l'humanité: l'accumulation de connaissances et des capacités cognitives est limitée par la taille du cerveau, elle-même limitée par le mode de naissance: si l'enfant naissait d'une matrice artificielle, la taille de son cerveau n'aurait plus de limite. Après le passage à la station verticale, qui a permis à l'humanité de surgir, ce serait une autre évolution radicale, à laquelle tout ce qui se passe aujourd'hui nous prépare. Telle est l'humanité que nous préparons, indépendamment de notre sexualité, par l'addition implicite de nos désirs individuels... » }

 On ne voit pas bien en quoi les divers types d'enfance que ces manipulations en tous sens de la biologie et des relations interpersonnelles seraient susceptibles d'offrir aux individus nés dans leur cadre seraient plus désirables que les enfances « traditionnelles » ni en quoi il s'agirait d'un progrès, mais ces manipulations n'en font pas moins rêver bien des adultes d'aujourd'hui.
 
 Dans \emph{L'avenir d'une illusion} (1927) \fsc{FREUD} se demande jusqu'où une société humaine peut se permettre d'être souple et tolérante étant donnée la violence des pulsions, désirs et angoisses qu'elle a pour tâche d'humaniser. Il répond qu'une grande dose de répression est inévitable, et que c'est même une des conditions de l'élaboration d'œuvres culturelles de valeur.

 Dans une société donnée, aussi élaborée soit-elle, sont inapparents, inconscients, ou plutôt innommables et innommés, déniés, les traits de sauvagerie qu'elle n'a pas suffisamment élaborés, les blocs de sauvagerie qu'elle n'a pas su penser. Pourtant bien en évidence aux yeux de tous, ces duretés et ces cruautés font d'autant moins problème qu'elles paraissent aussi inexorables que le jour et la nuit, aussi naturelles que le soleil et la pluie. Ce sont des points aveugles dans la représentation qu'elle se donne d'elle-même. Ils sont toujours involontaires et personne ne les a jamais désirés de manière consciente. 

 Le plus bel exemple de ce fait est que depuis des milliers d'années on a admis comme un fait établi et ne souffrant pas la discussion que les femmes étaient inférieures aux hommes, faites pour leur obéir et les servir, et qu'il était donc indispensable qu'une part plus ou moins grande de leurs droits soient détenus et exercés par des membres de leur entourage. 

 L'histoire des enfants sans parents est elle aussi marquée par plusieurs de ces points aveugles, à commencer par la dureté du sort fait partout, depuis toujours et jusqu'aujourd'hui en toute bonne conscience aux enfants de naissance illégitime, quelles qu'aient été les manières successives de définir en quoi leur naissance était illégitime, c'est-à-dire inopportune. Quoi de plus barbare que la croyance en une impureté ou une infamie de naissance ? Quoi de plus arbitraire et déraisonnable que l'idée qu'être né d'un ou d'une esclave interdisait irrévocablement de prétendre à des postes à responsabilité ? Quoi de plus étrange pour nous que la valeur religieuse du sang, ou la « pureté » d'une généalogie ? Quoi de plus absurde que de disqualifier moralement les « enfants du péché » tout en absolvant ceux qui avaient commis le « péché » dont ils étaient nés ? 

 Tout se passe comme si les conceptions archaïques du pur et de l'impur avaient continué d'être tenues pour vraies jusqu'à nos jours alors que le caractère moralement insatisfaisant de ces représentations avait été dénoncé il y a deux mille ans par les stoïciens aussi bien que par les évangiles, dont pourtant les thèses ont été méditées sans interruption depuis lors. Jusqu'au début du \siecle{20} chacune de ces propositions, en théorie insoutenables, du point de vue même de ceux qui s'y conformaient, a été tenue pratiquement pour vraie par tous ou presque tous, ou par chacun presque tout le temps. Jusqu'à Vincent de Paul on n'appelait pas négligence le sort qui était fait aux nouveaux-nés abandonnés, parce que les exclure du monde des familles légitimes paraissait être la façon correcte de les traiter et qu'on n'en imaginait pas d'autre. Lui a su le premier ou l'un des premiers, voir en eux autre chose que des êtres religieusement impurs qu'il était moralement indifférent de laisser mourir du moment qu'ils étaient baptisés. C'est sur les représentations de ses contemporains qu'il a travaillé et non sur l'art d'accommoder les bébés séparés de leur mère (cet art ne posait pas plus de problèmes à la majorité des jeunes femmes de son époque qu'à celles d'aujourd'hui). 

 Il y a moins d'un siècle les mineurs vagabonds étaient encore considérés et traités comme des délinquants : la criminalisation de leurs errances avait commencé à la fin du Moyen Âge : auparavant on les assimilait aux pèlerins et on se recommandait à leurs prières. 

 De même il n'y a guère plus d'un demi-siècle on regardait encore avec méfiance les rencontres entre les enfants placés en institution et leurs parents. 

 Et il n'y a pas trente ans qu'on a pris la mesure des dégâts psychologiques que provoquent les sévices sexuels perpétrés par les adultes sur les enfants, surtout quand ils ont autorité sur eux. 

 Il n'est donc pas impossible qu'aujourd'hui même s'étalent sous nos yeux des malheurs et des souffrances que nous ne voyons pas, des maltraitances que nous tolérons ou que nous produisons en toute bonne conscience. Si c'est réellement le cas, alors dans un siècle, ou dans dix, on nous reprochera de les avoir méconnus, sans comprendre que nous ne pouvions pas les voir, aveuglés que nous étions par nos théories, nos croyances ou nos intérêts inconscients, de la même façon que nous sommes scandalisés par la brutalité, l'insensibilité, l'aveuglement et les aberrations des logiques de nos prédécesseurs. 

 Est-ce que les lois et les pratiques qui encadreront à l'avenir la conception des enfants et l'art de les accommoder produiront moins de souffrances et de troubles que celles du passé chez les enfants et chez leurs parents ?

 Le recours à la prévention des naissances, à la pilule anticonceptionnelle, à la pilule » du lendemain » et à l'avortement permet en principe qu'il ne naisse plus d'enfants non désirés. Mais suffit-il que ceux qui naissent aient été désirés par leurs géniteurs ou par leurs parents adoptifs pour que disparaissent les problèmes qu'ils posent ou ceux qu'ils rencontrent ? Nul ne peut garantir qu'à l'avenir il y aura moins d'enfants mal assumés que par le passé. 

 Si la pauvreté matérielle n'est plus depuis longtemps un motif suffisant à lui seul pour séparer les enfants de leurs parents, est-on assuré pour autant qu'il n'existe et n'existera plus jamais d'enfants privés de l'un ou de l'autre de leurs parents alors que ceux-ci sont disponibles, volontaires pour les élever et suffisamment compétents ? L'absence de l'un des deux parents pour d'autres raisons que la maladie ou la mort devient au contraire quelque chose de plus en plus fréquent. 

 L'évolution de la législation et des mœurs ne va pas dans le sens du renforcement des capacités éducatives des familles. Elle multiplie le nombre des situations où la fonction éducative de l'un ou de l'autre des parents est plus ou moins disqualifiée ou empêchée, tandis que le remplacement au quotidien de l'un des parents de naissance par le partenaire sexuel et affectif de l'autre n'est pas toujours accepté par les jeunes concernés et ne présente pas toujours l'efficacité éducative nécessaire. 

 Le nombre s'élève donc des parents qui face à leurs enfants sont plus ou moins seuls. Lorsque la prise en charge éducative de ceux-ci fait problème, notamment à l'adolescence, des appuis extérieurs sont souhaitables mais ceux que propose la collectivité ne sont pas gratuits (ex. internats scolaires, assistance éducative, placement en famille d'accueil,~etc.). On passe de « l'auto production » familiale des activités éducatives à leur « externalisation » et à leur « professionnalisation ». Comme c'est un domaine où il n'y a pas à espérer de gain de productivité cela accroît les coûts éducatifs de manière très sensible. Jusqu'où peut-on aller dans cette voie avant d'estimer que c'est trop cher payé ? Enfin rien ne garantit que l'efficacité des diverses aides éducatives apportées aux parents soit supérieure à ce qu'en d'autres circonstances ils auraient pu assumer eux-mêmes : ce serait déjà bien si on pouvait être assuré qu'elle ne soit pas moindre. 
 
 Est-ce que le recours à une adoption ou à une mère porteuse est aussi satisfaisant du point de vue des enfants que du point de vue de leur(s) parent(s) ? Le désir de connaître leurs « origines » est de plus en plus fermement affirmé par beaucoup d'adultes nés d'une insémination artificielle avec donneur (IAD), tout comme celui des jeunes et des adultes nés sous X de connaître leur génitrice, ne peuvent que rendre dubitatif. L'évacuation par les parents légaux (parents « réels », « actuels ») des parents de naissance, des géniteurs, même si elle est assez ordinairement souhaitée (et on peut humainement comprendre ce souhait) n'est pas possible. Les parents de naissance font irrémédiablement partie de la relation entre les parents légaux et leurs enfants, même si c'est seulement de façon imaginaire. A défaut de pouvoir exiger d'être élevés par leurs deux parents de naissance, les enfants veulent au moins les connaître. Et même si on le leur récuse ils continuent de le réclamer. Et au nom de quel droit supérieur pourrait-on le leur dénier ? Ils ont ce droit au moins autant que tout adulte a droit de vouloir un enfant. 
 
 Dans le modèle de famille juif ou chrétien l'accueil de tous les enfants conçus est un devoir. Dans ce cadre à celui qui demande pourquoi il est né on répond : « Dieu t'a voulu », ce qui clôt toute discussion. Des générations d'enfants ont trouvé cette explication suffisante : leur narcissisme en était suffisamment étayé. Un droit absolu à l'existence leur était ainsi reconnu quoi qu'il arrive, et cela même s'ils ne correspondaient pas totalement, ou pas du tout, aux attentes de leurs parents.
  
 Le droit à l'interruption de grossesse (IVG) a changé la donne. Dans certaines circonstances précisées par la loi l'embryon ou le foetus a perdu la protection que la loi lui accordait depuis Constantin. Il a perdu le droit de devenir un jour la personne qu'il est \emph{en potentiel}. Selon le droit actuel il n'acquiert de personnalité juridique qu'à la naissance. Jusque là il n'est qu'un \emph{objet} juridique. Tant qu'il n'est pas né il n'est en quelque sorte qu'une partie du corps de sa mère (cf. le droit romain). L'argument de fond c'est qu'un individu qui n'est une personne qu'en puissance a moins de droits que celui qui est actuellement une personne accomplie. Le foetus n'est pas une personne accomplie, contrairement à sa mère : {\emph{Le mot « personne » évoque l'idée d'une présence ou d'une absence « humaine », et surtout d'un échange réciproque avec l'autre, tandis qu'individu est utilisé pour désigner l'un, en tant qu'indivisible, d'une espèce}} (Wikipédia). 

 Les cas où la santé physique de la mère est sérieusement menacée par la grossesse ne posent guère de problème moraux, pas plus que ceux où le foetus est atteint de troubles gravissimes. On sait d'ailleurs que les médecins sont amenés de temps en temps à abréger la vie des nouveaux-nés non viables : la Hollande l'a reconnu dans le cadre du \emph{protocole de Groeningen}. La Belgique s'est également engagée dans cette voie. 
 
 Face aux autres cas, lorsque c'est à première vue le bien-être de la mère ou celui de sa famille qui sont visés, les enfants pourraient entendre qu'on attend d'eux de n'être pas une gêne et de ne pas coûter d'efforts. Ils pourraient comprendre que c'est dans la réalité, et non dans leurs fantasmes les plus archaïques, que leurs parents ont sur eux droit de vie ou de mort.

 Les tenants de l'interdit de l'avortement se scandalisent qu'on tue des enfants non nés puisqu'il n'y a rien qui, selon eux, les différencie radicalement des nouveaux-nés. D'autres moralistes \emph{s'appuient sur le même constat}pour demander au contraire que soit reconnu aux parents le droit de supprimer les nouveaux-nés dont ils ne veulent pas, même viables, et notamment ceux qui présentent des problèmes biologiques non détectés au cours de la grossesse (ex : trisomie 21,~etc.). D'autres vont encore plus loin. Dans un article du 2 mars 2012 publié dans le \emph{Journal of Medical ethics}, Alberto \fsc{Giubilini} et Francesca \fsc{Minerva} proposent à la suite de Peter \fsc{Singer} d'étendre le droit à l'avortement au-delà de la naissance (ce qu'ils nomment \emph{avortement post-natal}). Voici un extrait de cet article (traduction personnelle) :

« \emph{Le droit prétendu des individus (tels que foetus et nouveaux-nés) de développer leurs potentialités, droit que certains défendent, cède devant l'intérêt de ceux qui sont actuellement des personnes (parents, famille, société) de rechercher leur propre bien-être, parce que, comme nous venons de le démontrer, ceux qui sont seulement des personnes potentielles ne peuvent pas être lésés par le fait de ne pas être introduits dans l'existence. Le bien-être des personnes actuelles \emph{[c'est-à-dire le bien-être actuel des humais parvenus au stade de personnes en acte, ce que ne sont pas encore les nouveaux-nés, selon leur point de vue]} pourrait être affecté par de nouveaux enfants (même en bonne santé), réclamant de l'énergie, de l'argent et des soins, toutes choses dont la famille peut manquer. Parfois cette situation peut être évitée par un avortement, mais parfois cela n'est pas possible. Dans ces cas du moment que les non-personnes n'ont pas de droit moral à vivre, il n'y a pas de raisons de refuser l'avortement post-natal. Nous avons certes un devoir moral envers les futures générations alors qu'elles n'existent pas encore. Parce que nous tenons pour garanti que ces personnes existeront (quelles qu'elles soient) nous devons les traiter comme des personnes actuelles du futur. Cet argument, cependant, ne s'applique pas à tel ou tel nouveau-né en particulier, parce que nous ne pouvons pas tenir pour garanti qu'il deviendra une personne un jour. Est-ce qu'il existera \emph{[en tant que personne en acte]} dépend en fait de nous et de notre choix}.

« \emph{L'adoption peut-elle être une alternative à l'avortement post-natal ?}

« \emph{On pourrait nous objecter que l'avortement post-natal ne devrait être pratiqué que sur les personnes potentielles qui ne pourront jamais avoir une vie digne d'être vécue. Dans cette hypothèse les individus en bonne santé et capables d'être heureux devraient être donnés à l'adoption lorsque leur famille ne peut pas les élever. Pourquoi devrions-nous tuer un nouveau-né en bonne santé alors que le confier à l'adoption ne grèverait les droits de personne mais au contraire accroîtrait le bonheur des personnes impliquées (adoptant et adopté) ?}

« \emph{Notre réponse est la suivante : nous avons précédemment examiné l'argument de la potentialité (potentialité des êtres de devenir une personne) et montré qu'il n'est pas suffisamment puissant pour contrebalancer l'intérêt de ceux qui sont actuellement des personnes. En réalité combien minces puissent être les intérêts d'une personne actuelle, ils seront toujours supérieurs à l'intérêt (hypothétique) d'une personne en puissance de devenir une personne réelle, parce que ce dernier est égal à zéro. Dans cette perspective ce sont les intérêts des personnes actuelles qui ont de l'importance, et parmi ces intérêts nous devons en particulier considérer les intérêts de la mère qui peut souffrir psychologiquement si elle donne son enfant en adoption. On observe souvent que les mères de naissance rencontrent des problèmes psychologiques sérieux à cause de leur incapacité à élaborer leur perte et à surmonter leur chagrin. Il est vrai que le chagrin et le sentiment de perte peuvent accompagner l'avortement et l'avortement post-natal aussi bien que l'adoption, mais nous ne pouvons pas affirmer que pour la mère de naissance celle-ci est la moins traumatique. Par exemple, « ceux qui pleurent un décès doivent accepter l'irréversibilité de la perte, mais souvent les mères naturelles rêvent que leur enfant va revenir vers elles. Cela rend difficile pour elles d'accepter la réalité de la perte parce qu'elles ne peuvent jamais être tout à fait certaines que cette perte est irréversible. »}

« \emph{Nous ne cherchons pas à suggérer que ce sont des arguments décisifs contre la validité de l'adoption comme alternative à l'avortement post-natal. Cela dépend beaucoup des circonstances et des réactions psychologiques. Ce que nous sommes en train de suggérer c'est que si l'intérêt des personnes actuelles doit prévaloir, alors l'avortement post-natal doit être considéré comme une option permise aux femmes qui pourraient souffrir de donner leur nouveau-né à adopter.} »

 Il s'agit clairement de promouvoir le \emph{droit à l'infanticide}, très largement répandu dans le monde entier, mais supprimé par Constantin. Même si elle fait penser à Jonathan \fsc{Swift} et à son « \emph{Humble proposition pour empêcher les enfants des pauvres en Irlande d'être à la charge de leurs parents ou de leur pays et pour les rendre utiles au public} » (1729), cette demande est formulée sans le moindre humour. Notre société lui opposera-t-elle un veto inflexible ou ira-t-elle jusqu'à reconnaître aux parents ce droit nouveau ? 

 On peut se poser les mêmes questions (droit des « personnes » opposé à celui des individus qualifiés de « non-personnes ») pour les grands vieillards trop dépendants (démences séniles...) et pour tous les malades physiques ou mentaux aux capacités de relation irréversiblement dégradées. Leur euthanasie active soulagerait bien évidemment leurs familles et les systèmes d'assistance médicale et sociale des multiples problèmes qu'entraîne leur prise en charge.
 
 
