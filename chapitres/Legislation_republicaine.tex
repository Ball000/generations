J1 LEGISLATION REPUBLICAINE (1880-1946)
 Depuis la fin du XIXème siècle il s'est produit énormément d'évènements significatifs. Je me limiterai à pointer à la lumière de la longue histoire ce qui a eu l'impact le plus décisif, directement ou indirectement, sur les familles, sur l'éducation, et donc également sur les enfants dont les parents n'ont pas pu s'occuper, et pour commencer voici les dates et les décisions essentielles[1] :
 
 1880 : création d'un enseignement secondaire public pour les filles, calqué sur le modèle de celui des garçons. 
 1881 : obligation pour chaque commune de mettre à la disposition de ses administrés une école gratuite et laïque.
 1881 : Création du \emph{Service des enfants moralement abandonnés.} Il a pour objet de recevoir les jeunes de 12 à 16 ans sans support familial, pénalement mineurs, non secourus puisque le service ne recevait pas de nouveaux entrants après l'âge de douze ans, et n'ayant que la mendicité et le vol pour subsister : mineurs arrêtés pour « \emph{vagabondage et autres menus délits, et aussi ceux que leurs parents se montraient incapables de diriger} ».
 1882 : obligation scolaire pour les garçons et filles de 6 à 12 ans. Avant leurs 12 ans (11 ans s'ils ont le certificat d'études) il est interdit aux pères de placer leurs enfants chez un employeur, ou de les employer eux-mêmes à plein-temps.
 1884 : Réouverture du droit au divorce, (uniquement pour faute, comme en 1804).
 1886 : laïcisation du personnel enseignant des écoles publiques. 
 1889 : La loi du 24 juillet « \emph{sur la protection judiciaire des enfants maltraités et moralement abandonnés} » précise les conditions de la déchéance de la puissance paternelle. Cette déchéance totale est prononcée par un Juge : 1°) facultativement pour inconduite des parents, 2°) facultativement en cas de mauvais traitements ou de délaissement de l'enfant, 3°) et de plein droit dans le cas de certaines condamnations infamantes (ce qui était le cas depuis l'antiquité). 
 1889 : la loi du 24 juillet 1889 « \emph{sur la protection judiciaire des enfants maltraités et moralement abandonnés} » confie à l'administration (c'est-à-dire à l'Assistance Publique) la tutelle des enfants maltraités, victimes de crimes, ou de délits, ou délaissés. Le service les prend en charge même s'ils sont âgés de plus de 12 ans à leur entrée. Ces enfants sont retirés autoritairement à leurs parents et deviennent des pupilles comme les autres. Ils sont traités à l'instar des autres enfants du service. Quel que soit leur âge, autant que faire se peut ils seront placés en nourrice, pour de longues durées, et dans tous les cas ils seront coupés de leurs parents déchus.
 À partir de 1896 : « \emph{les familles indigentes mises devant la nécessité d'abandonner (sont) autorisées après enquête à correspondre directement avec enfants et nourriciers} ». D'autre part les « \emph{enfants de parents internés} [sont désormais] \emph{considérés comme n'ayant pas été abandonnés volontairement} », et les correspondances directes entre parents et enfants sont autorisées.
 1893 : les femmes séparées de corps ont la pleine capacité civile : elles récupèrent les droits qu'elles avaient quand elles étaient célibataires.
 1897 : les femmes mariées peuvent être témoins dans les actes civils et notariés.
 1901 : Loi sur les associations à but non lucratif. Leur fondation est libre, basée sur la notion de contrat entre personnes. Elles ne peuvent recevoir ni dons ni legs.
 1904 : dénonciation unilatérale du Concordat de 1802.
 1904 : autorisation donnée aux amants condamnés pour adultère de s'épouser après leur(s) divorce(s) ou le décès du conjoint trompé.
 1904 : Loi du 12 avril : majorité pénale à 18 ans au lieu de 16, élargissement de l'excuse de minorité, affirmation de la nécessité de faire passer l'éducatif avant le répressif pour les mineurs pénaux.
 1907 : La loi du 13 juillet permet aux femmes mariées de toucher et de gérer elles-mêmes leur propre salaire, au lieu qu'il soit remis à leur mari comme c'était la règle durant tout le XIXème siècle. 
 1912 : autorisation des recherches en paternité naturelle. Les enfants naturels peuvent demander des aliments à chacun de leurs géniteurs : ce texte vise essentiellement les pères, et les mères peuvent agir au nom de leurs enfants. 
 1912 : La loi du 22 juillet créée des tribunaux spéciaux pour enfants et adolescents. Elle pose les premiers jalons de la liberté surveillée
 1913 : Mesures d'assistance en faveur des femmes en couche nécessiteuses, et des familles nombreuses nécessiteuses.
 1917 : une femme peut être nommée tutrice et siéger au conseil de famille
 1920 : une femme mariée peut adhérer à un syndicat sans l'autorisation de son mari.
 1920 : Toute forme de propagande anticonceptionnelle ou de publicité pour des instruments de lutte anticonceptionnelle est interdite (préservatifs, pessaires, diaphragmes, etc. qui restent néanmoins disponibles en pharmacie).
 1921 : la loi ouvre la possibilité de prononcer une déchéance partielle de l'autorité paternelle. 
 1923 : L'adoption des enfants abandonnés (sans limite d'âge inférieure) est ouverte aux couples mariés. Nommée légitimation adoptive elle n'annule pas le passé de l'enfant.
 1924 : Identité complète des programmes d'études dans le secondaire féminin et masculin.
 1925, l'A.P. commence à placer en nourrice les jeunes enfants (âgés de moins de quatre ans, dans un premier temps) placés \emph{en dépôt} par leurs parents et elle les y laisse grandir. 
 1932 : \emph{Allocations familiales} (pour tous les enfants)
 1931 : Les femmes peuvent être nommées (élues ?) juges.
 1935 : Le décret-loi du 30 octobre sur « \emph{la correction paternelle et l'assistance éducative} » institue l'assistance éducative à domicile. 
 1935 : Le \emph{vagabondage} des mineurs cesse d'être un délit, (contrairement à la mendicité et au racolage qui demeurent des délits). 
 1938 : La femme mariée acquiert certains des droits des femmes célibataires : droit à une carte d'identité, à un passeport, à ouvrir un compte en banque sans l'autorisation de son époux.
 1941 : Allocation de salaire unique (\emph{parmi les textes promulgués sous l'Occupation ne comptent que ceux qui ont été confirmés à la Libération : plusieurs d'entre ces derniers avaient été préparés bien avant la guerre}). 
 1941 : ouverture des hôpitaux à tous, quels que soient leurs revenus : depuis longtemps déjà les hôpitaux recevaient des malades qui payaient leur séjour et les soins qui leur étaient dispensés. Ils payaient eux-mêmes ou c'est un tiers qui le faisait : militaires (dès l'ancien régime), accidentés du travail (1897), assurés sociaux (1928), etc. En 1900 cela contribuait pour 20 % aux recettes des hôpitaux. En 1940 40 % des hospitalisés donnaient lieu à un remboursement. Le 21 décembre 1941 il est décidé d'étendre cette possibilité à tout le monde, sans maintenir d'exclusive. Comme bien des décisions de cette époque ce n'était que la mise en œuvre de décisions préparées dés 1938, c'est pourquoi cette orientation n'a pas été remise en question à la libération.
 1943 : la loi du 15 Avril 1943 donne un droit aux secours aux enfants \emph{qui ont un père, même quand celui-ci est valide}. Le droit des parents au « dépôt » volontaire de leurs enfants à l'Assistance Publique est élargi.
 1944 : octroi du \emph{droit de vote} aux femmes. 
 1945 : L'Ordonnance du 2 février crée le corps des juges pour enfants, pour les jeunes de moins de 18 ans. Elle crée l'éducation surveillée à l'intention des mineurs délinquants.
 1946 : Création des \emph{allocations prénatales}.
 1946 : La constitution déclare\emph{ égaux}les droits des hommes et des femmes.
 
[1] Sources : \emph{L'Assistance Publique en 1900}, ouvrage collectif de l'Administration générale de l'Assistance Publique, écrit à l'occasion de l'Exposition Universelle de Paris de 1900, composé et imprimé par les pupilles de la Seine de l'école d'Alembert à Montévrain, Paris, 1900. Consultable au Musée Social, Paris, VIIème. Collectif, sous la direction de Michel CHAUVIERE, Pierre LENOËL, Eric PIERRE, Protéger l'enfant, Raison juridique et pratiques socio-judiciaires (XIXème et XXème siècles), Presses Universitaires de Rennes, 1996. Collectif, sous la direction de Jean DELUMEAU et Daniel ROCHE, \emph{Histoire des pères et de la paternité}, Larousse, 1990, édition 2000. Collectif, sous la direction de Jean IMBERT, \emph{Histoire des hôpitaux en France}, Privat, 1982, 559 p. Collectif, sous la direction d'Alain BURGUIERE, Christine KLAPISH-ZUBER, Martine SEGALEN, Françoise ZONABEND, \emph{Histoire de la famille, 3, Le choc des modernités}, Armand Colin Éditeur, Paris, 1986. BROUSOLLE Paul, \emph{Délinquance et déviance, brève histoire de leurs approches psychiatriques}, Privat, Toulouse, 1978. CUBERO José, \emph{Histoire du vagabondage du moyen-âge à nos jours}, Imago, Paris, 1998. DONZELOT Jacques, \emph{L'invention du social, essai sur le déclin des passions politiques}, Seuil, Paris, 1994. DUPOUX Albert, \emph{Sur les pas de Monsieur Vincent, 300 ans d'histoire parisienne de l'enfance abandonnée}, Édité par la Revue de l'Assistance Publique, Paris, 1958. PINELL Patrice, ZAFIROPOULOS Markos, \emph{Un siècle d'échecs scolaires (1882-1982}), Les éditions ouvrières, Paris, 1983.