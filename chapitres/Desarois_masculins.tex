
 Notre retour sur l'histoire montre à quel point la situation actuelle est révolutionnaire. Le mariage avait pour but, essentiel sinon unique, de fabriquer des pères et de leur donner des enfants. L'ancien Droit romain semblait réduire les femmes à n'être que des ventres au service des hommes. Ceux-ci répudiaient celles qui ne leur avaient pas donné les héritiers qu'ils voulaient, ou prenaient des concubines. En cas de séparation, le Droit leur attribuait systématiquement tous leurs enfants%
% [1]
\footnote{... ce qu'il a fait jusqu'au milieu du \siecle{19} dans les pays anglo-saxons.}% 
. Ils pouvaient assumer eux-mêmes leur éducation, ou les confier à leur propres parents. Ils pouvaient même les confier à leurs ex épouses, mais toujours sous leur propre autorité et à leurs frais.

 On a vu que « l'obligation de résultat », l'obligation de fécondité, qui pesait sur les seules femmes mariées a été supprimée par Constantin, qui a exclu la stérilité des motifs de divorce. La loi impériale romaine a ensuite confirmé l'interdit fait aux chrétiens de se remarier après divorce. À l'obligation de fécondité des épouses s'est substituée une obligation (religieuse) de moyens pour chacun des deux époux de ne pas mettre d'obstacle autre que l'abstinence%
% [2] 
\footnote{En France les relevés démographiques montrent l'érosion progressive du respect de cette obligation, et l'extension depuis trois siècles des pratiques anticonceptionnelles : ce que les anciens moralistes nommaient les « \emph{funestes secrets} ».} 
à une conception et à une naissance. Si les femmes mariées ont ainsi été protégées contre la répudiation et contre la privation de leurs enfants, par contre la loi ne les autorisait pas plus qu'avant à se dérober au « devoir conjugal » lorsque leur mari l'exigeait, ni aux grossesses qui en découleraient, et à leurs risques, sauf à demander une séparation. 

 Depuis 1967, même si leurs maris le désirent, les femmes mariées ne sont plus tenues par la loi de laisser libre cours à leur fécondité. Aujourd'hui le corps des femmes est à elles, y compris l'embryon ou le fœtus, qui juridiquement en fait partie depuis 1975, comme c'était le cas dans le droit romain antique. 

 La loi ne se soucie plus de soutenir le désir masculin en ce domaine. Même si elles sont leurs épouses, même s'ils sont les géniteurs de l'enfant qu'elles portent, même si elles avaient été d'accord pour le concevoir avec eux, les hommes n'ont plus le droit d'exiger des femmes qu'elles leur donnent cet enfant. Elles peuvent l'abandonner à la naissance contre le gré du père de l'enfant. On est au plus loin du droit du \emph{pater familias} romain de faire surveiller la grossesse et l'accouchement de son épouse (ou ex épouse), pour qu'elle ne puisse pas lui dérober un enfant né de ses œuvres.

 Dans le même temps ont été supprimées toutes les limites légales qui pouvaient interdire le rattachement d'un enfant naturel à son géniteur, à l'exception des inséminations artificielles avec donneur, ou IAD. Une mère qui le demande recevra toujours l'appui de la justice pour rechercher le géniteur de son enfant, quelle que soit la situation personnelle de cet homme (comme sous l'ancien régime) et cela se fera avec une efficacité désormais imparable. Aucun père n'est plus « \emph{incertus} ». Vivant ou mort son ADN le désignera, sauf lorsque la mère veut cacher son identité à l'enfant ou aux tiers (mais si une mère qui accouche « sous X » refuse de laisser à son enfant des renseignements sur sa propre identité, elle le peut). Si la mère le veut, le géniteur sera contraint d'assumer financièrement un enfant qui héritera de lui à part entière, contrairement à ce qui se passait jusqu'au \siecle{19}. Mais cela ne lui donnera pas forcément le moindre droit sur l'éducation de l'enfant : en ce sens cela n'en fera pas un père.

 Pour l'essentiel, on peut donc dire que la maîtrise de la génération est passée du côté des femmes. La famille monoparentale d'aujourd'hui, c'est assez ordinairement la famille \emph{moins} le père. Dans la majorité des séparations (85~\%) ce sont les mères qui gardent les enfants. Est-ce pour ces raisons que l'initiative des divorces vient beaucoup plus souvent des épouses que des maris ? Beaucoup d'hommes ont plus à perdre que leurs femmes au divorce, et surtout les plus pauvres. 

 Les mères ont toujours eu une place de choix dans les représentations : elles sont traditionnellement du côté de l'accueil de la vie et de son entretien, de l'intime, de la tendresse, du cœur. Mais aujourd'hui cette idéalisation n'est plus contrebalancée par l'idéalisation symétrique des pères des siècles classiques. Aujourd'hui la déploration des déficiences des pères, de leurs fragilités et de leur irresponsabilité, est un passage obligé de tout discours sur la famille, tandis que l'idée qu'ils puissent exercer une force ou une puissance dans leur relation à leurs enfants (à tout enfant) renvoie à des représentations de violence et de maltraitance. Quand on parle sans les spécifier des violences conjugales ou intra familiales, il va de soi qu'il s'agit des violences masculines, alors que l'observation montre que les femmes sont très capables de concurrencer les hommes dans ce domaine aussi. 

 D'ailleurs maintenant que le capital le plus utile c'est le capital intellectuel, maintenant que l'avenir des enfants se prépare à coup d'études longues, financées en grande partie par la collectivité, sous la houlette de professionnels de l'enseignement et sous le contrôle de l'État, qu'est-ce qu'un père pourrait bien transmettre à ses enfants (à part ses biens) sans menacer leur autonomie ?

 Dans l'effritement de l'autorité des pères, Françoise \fsc{HURSTEL} pointe trois moments clé : la loi de 1889 contre les « \emph{parents indignes} », la loi de 1935 abolissant le droit de « \emph{correction paternelle} » et la loi de 1938 abolissant la « \emph{puissance maritale} ». Ont été abolies toutes les dispositions juridiques sur lesquelles était fondé dans le passé l'exercice masculin d'un rôle patriarcal. Le résultat est que « [...] \emph{nous ne savons plus ce qu'est la place d'un père et ce que sont ses fonctions} », et que « \emph{ce ne sont pas des petits bouts de la paternité qui ont changé, mais l'ensemble du système a muté avec la mort du \emph{pater familias}.} »%
% [3]
\footnote{Françoise \fsc{HURSTEL}, « Penser la paternité contemporaine dans le monde occidental : quelles places et quelles fonctions du père pour le devenir humain, sujet et citoyen des enfants ? », in \emph{Neuropsychiatrie de l'enfance et de l'adolescence}, 53 (2005) 224-230.} 

 Autrefois (jusqu'aux années 60 du siècle dernier ?) c'est l'excès de présence et de poids des pères qui faisait problème. Aujourd'hui on déplore qu'ils ne soient jamais assez présents, ou jamais là où il faut. Françoise \fsc{HURSTEL} soutient que cela est l'effet de ces changements, et non leur cause. Si les lois suivaient l'évolution des mœurs, alors la promulgation d'une loi serait le signe que les esprits sont prêts à l'accueillir. Dans cette hypothèse, pendant les années précédant la promulgation de chacune des lois ci-dessus, on aurait dû observer un mouvement de l'opinion publique stigmatisant les parents indignes, le recours abusif au droit de correction paternelle, ou le scandale que constitue l'existence d'une puissance maritale. Selon elle ce n'est pas ainsi que cela s'est passé, au contraire. Ce n'est qu'à partir de la promulgation de la loi de 1889 que la presse aurait commencé de dénoncer les carences des pères « indignes%
% [4]
\footnote{« \emph{alcoolique, pauvre, inculte et violent} », Françoise \fsc{HURSTEL}, \emph{la déchirure paternelle}, p. 113.} 
 ».

 Et de même ce n'est que vers 1942 que les spécialistes de l'éducation auraient commencé de dénoncer les pères sans autorité, tandis que la notion de carence n'aurait envahi les écrits qu'à partir de 1950 : « \emph{C'est donc quelques années après la promulgation de ces lois faisant disparaître des textes juridiques les termes de puissance (maritale) et ceux de correction paternelle tout en maintenant ceux de chef et d'autorité (paternelle), qu'est décrite cette figure d'un père manquant d'autorité et de sévérité ; et que les spécialistes admonestent les pères d'une position qui est bien celle de chef de famille.} » 

 Selon elle, l'opinion publique n'aurait donc appelé aucune de ces lois de ses vœux. Ces réformes n'auraient été imaginées, réclamées, et parfois discrètement expérimentées que par les seuls experts, médecins, administrateurs, juges et travailleurs sociaux directement intéressés à leur mise en œuvre. Pour Françoise \fsc{HURSTEL}, tous les discours sur les déficiences des pères actuels ne sont que des productions imaginaires qui coexistent avec des réalités qui n'ont pas grand-chose à voir avec eux. En effet, les enquêtes sur le terrain ne montrent rien qui permette de croire que les pères d'aujourd'hui seraient dans l'ensemble moins attentifs et moins présents que ne l'étaient ceux du passé%
% [5]
\footnote{... mais cela exige d'éviter les biais méthodologiques. Il faut notamment que ces enquêtes ne se placent pas consciemment ou inconsciemment du seul point de vue des mères. Cf. Germain \fsc{DULAC}, « La configuration du champ de la paternité : politiques, acteurs et enjeux », in \emph{Politiques du père, numéro spécial de Lien social et politiques}, (n° 37) 1997, p. 133-142.}%
. Certes il y a des pères qui sont incompétents, irresponsables ou délinquants, mais cela n'a rien de nouveau, et rien ne permet d'affirmer qu'il y en ait plus qu'autrefois. Les discours ne portent pas tant sur ce que font réellement les pères que sur ce qu'ils devraient faire dans l'idéal pour être de bons pères. 

 Pour elle, il s'agit, à l'aide de ces discours, d'asseoir l'autorité de ceux qui prétendent savoir ce qu'est un bon père et qui sont les bons pères : « \emph{du point de vue de la paternité les hommes de la période contemporaine n'auront pas été gâtés. Je propose une image pour illustrer ce que peut être la notion de carence : lorsqu'un homme devient père, il endosse un pardessus plein de trous et de soupçons..., plus précisément une image de plus en plus dévalorisée, et cela quelle que soit la valeur personnelle de l'homme qui assume une telle fonction. Et ce qui les caractérise est un discours dévalorisant des spécialistes ; tellement dévalorisant qu'il apparaît, en fait, comme un discours de l'exclusion des pères... au profit du super père spécialiste. Si les pères peuvent être dits carents \emph{[en droit le père « carent », c'est celui qui ne laisse rien à ses enfants, qui ne leur laisse aucun héritage]}, c'est parce qu'ils sont relégués à cette place par ceux-là mêmes qui normalisent les pratiques autour de l'enfant. Nous dirons que ces pères carents sont en fait d'abord des pères exclus par les théoriciens de l'éducation.}%
% [6]
\footnote{Françoise \fsc{HURSTEL}, \emph{la déchirure paternelle}, p. 112-113.} 

 « [...] \emph{Ainsi les signifiants inscrits dans la loi produisent des effets imaginaires qui se repèrent dans les représentations collectives, les modèles normatifs du père et les pratiques sociales.} 

 « \emph{Je ferai ici un pas de plus et avancerai ceci : non seulement les signifiants des lois produisent des effets imaginaires, mais encore les lois elles-mêmes ne sont connues que par le biais de ces productions...}

 « \emph{Les figures du père carent semblent bien avoir une fonction sociale et idéologique importante, celle d'être l'une de ces fonctions sociales qui rendent compte et qu'il y a du père dans notre société (au sens du père symbolique et de la fonction paternelle) et qu'il y a du changement dans les montages qui instituent le père... bref, elles seraient un mode d'historicisation d'une structure.}

 « \emph{Mais en retour cet imaginaire du père marquera chaque homme ayant à assumer la fonction paternelle, chaque mère appelée à reconnaître qu'il y a du père pour son enfant%
% [7]
\footnote{Idem, p. 113-115.}% 
}. » 

 Il est ordinaire et au fond assez normal que les adolescents, garçons et filles, soient en état d'incertitude identitaire, avec tous les malaises que cela implique, mais ils supportent encore moins bien les incertitudes identitaires de leurs adultes de référence que les leurs propres. Ce n'est pas un hasard si ce sont les garçons qui expriment aujourd'hui le plus durement leur désarroi : violences contre les personnes et les biens, prises de risques inconsidérées, désinvestissement scolaire, etc. Ils ont besoin que les adultes (hommes et femmes) reconnaissent que c'est une puissance valeureuse qui croît en eux et non une violence erratique, brutale et destructrice, juste bonne à être périodiquement sacrifiée en holocauste aux dieux de la guerre.

 Puisque le patriarcat est mort et que les femmes ne retourneront plus dans des gynécées, sinon contraintes et forcées (par qui ? Pour quoi ?), et puisque dans le domaine familial aussi le droit à l'égalité s'impose comme le principe de base indiscutable, il faudra inventer (ou découvrir, ou redécouvrir) pour les hommes une place qui soit aussi désirable que celle des femmes : des points de vue et des désirs spécifiquement masculins sur les enfants sont-ils acceptables ? Mais dans un environnement allergique à tout ce qui ressemble à du paternalisme, qu'est-ce qu'un homme est autorisé à désirer concernant des enfants ? Les hommes sont-ils fondés à dire quelque chose sur les enfants ? Sont-ils fondés à dire quelque chose aux enfants ? Il faudra sans doute commencer par admettre qu'il existe des valeurs masculines, ou une manière masculine de faire vivre les valeurs universelles.

 Derrière le problème de la paternité se profile la question « à qui appartient l'enfant ?%
[5]\tempnote{Il y a un appel à la note 5 ici... est-ce normal ? Dois-je l'adapter à \LaTeX{} ?}%
 ». Il ne s'appartient pas à lui-même, sauf à supprimer le statut de mineur. On ne peut pas plus dire qu'il n'appartient à personne. Du point de vue des enfants, n'appartenir à personne (ou appartenir à une institution) c'est être abandonné. Depuis très longtemps les enfants n'appartiennent plus aux seuls pères. Est-ce qu'ils appartiennent désormais aux seules mères ? ... ou bien aux deux (géniteurs) parents, comme le dit la loi ? ... ou bien à l'ensemble de ceux qui les élèvent en leur donnant leur argent et leur temps, dont les beaux-pères et belles-mères ? ... ou bien encore à l'État ?


