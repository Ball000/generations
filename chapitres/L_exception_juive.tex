
 Les romains nommaient \emph{juifs} (originaires de \emph{Judée}) ceux que les grecs nommaient \emph{hébreux}%
%[1]
\footnote{Ils seront présentés ici tels qu'ils étaient après leur intégration à l'Empire romain, mais alors que le Temple de Jérusalem était encore en fonction (avant l'an 70 de notre ère).
Sources : André \fsc{CHOURAQUI}, \emph{La vie quotidienne des hommes de la Bible}, 1978. A.~\fsc{COHEN}, \emph{Le Talmud, exposé synthétique du Talmud et de l'enseignement des rabbins sur l'éthique, la religion les coutumes et la jurisprudence}, 1980. Alain \fsc{STEINSALTZ}, \emph{Introduction au Talmud}, 1976, Paris 2002. Collectif, \emph{Aux origines du christianisme}, 2000. André-Marie \fsc{GERARD}, \emph{Dictionnaire de la Bible}, 1989. Alan \fsc{UNTERMAN}, \emph{Dictionnaire du judaïsme, Histoires, mythes, traditions}, 1997.}%
.

 Aux alentours du début de notre ère, et jusqu'aux deux « guerres juives » (\hbox{66-70} et \hbox{132-135} après J.-C.), il y avait un fort noyau de juifs en Judée et en Galilée : ils représentaient semble-t-il la majorité des habitants (2 millions ?) de ces territoires. Mais la plupart d'entre eux étaient dispersés sur toute la terre alors connue : c'était la \emph{diaspora} (la « dispersion »). Toutes les grandes villes antiques abritaient des communautés juives, parfois plus nombreuses que la population de Jérusalem (Alexandrie, Antioche...)%
% [2]
\footnote{Il semble qu'au moins un habitant sur dix de l'Empire romain était juif : de 6 à 8 millions pour un nombre total compris entre 50 et 60 millions ? Dans la partie orientale de la méditerranée c'était un habitant sur cinq ?}%
. Et il existait aussi des communautés juives en dehors du monde romain (Mésopotamie, Arabie, Éthiopie, etc.).

 C'est souvent comme esclaves que les juifs avaient voyagé vers la terre de leur exil ou de leur déportation. Cela avait commencé dès 722 avant J.-C., avec la fin du Royaume du nord, celui des 10 tribus, écrasé par les Assyriens. Cela avait continué en 586 avant J.-C., avec la destruction du Royaume de Juda et du premier Temple de Jérusalem, et la déportation des élites vers Babylone. Cela s'était poursuivi avec la guerre des Macchabées (deuxième siècle avant notre ère). Ces tribulations culmineront avec la grande révolte de 70 après J.-C., avec la destruction du Temple d'Hérode et la déportation des survivants de Jérusalem. En 135 après J.-C., la diaspora deviendra l'unique lieu de vie des juifs, qu'ils soient descendants d'immigrés ou de déportés nés en Judée ou en Galilée, ou d'autochtones convertis à leur religion. 

%UN DIEU A PART
\section{Un dieu à part}

 Les juifs étaient un cas particulier dans le paysage religieux de l'empire romain. Ils avaient en effet réussi à faire admettre qu'en termes de religion civique ils se borneraient à faire des sacrifices et des prières à leur propre dieu, pour les cités, pour l'État. Les autorités romaines leur avaient garanti ce droit quand ils avaient fait appel à eux (guerres des Macchabées) parce que la Judée et de la Galilée valaient bien une entorse à leurs propres croyances et coutumes. En contrepartie la diaspora versait un impôt spécial au Temple de Jérusalem pour payer les sacrifices offerts pour l'État. C'est pourquoi l'administration romaine protégeait comme une œuvre d'intérêt public la collecte et le transport de cet impôt, le \emph{fiscus judaïcus}. Le fait que les romains acceptaient leur refus de sacrifier ne veut pas dire qu'ils appréciaient leur religion et leurs manières de vivre. Au contraire ils les trouvaient détestables et ne se privaient pas de le dire. Si les citoyens romains qui étaient juifs étaient exemptés du service militaire (c'étaient les seuls), cette exemption n'était pas un privilège mais une exclusion. Leur mode de vie était trop contraire à celui des militaires romains et interdisait leur intégration dans une unité à la vie rythmée par les cérémonies de la religion civique, et leurs règles de pureté interdisaient leur intégration dans une chambrée de \emph{gentils} (... un membre des \emph{gens}, des peuples autres que le peuple juif). 

 Les juifs se désignaient comme un « peuple saint », le peuple de leur Dieu : YHWH. Par respect pour sa transcendance ils évitaient de prononcer son nom, et le désignaient par un autre mot (le Nom, le Seigneur, le Saint...). 

 L'audace des juifs n'était pas qu'ils affirmaient avoir un dieu bien à eux, ni qu'ils prétendaient que leur dieu était le plus puissant ou le meilleur de tous. Cela, c'est ce que pensaient \emph{toutes} les cités de l'antiquité. Ce que croyaient les juifs de la fin de la République romaine c'est que leur dieu était \emph{le seul} dieu existant, et donc \emph{le dieu de tous} les hommes. Depuis le retour de leur exil à Babylone les autres dieux n'étaient plus pour eux que des images, des simulacres, des \emph{idoles} impuissantes, derrière lesquelles se cachaient des démons malfaisants, qui cherchaient à tromper les hommes. C'est pour ce motif que les juifs étaient qualifiés de « {sans dieux} », d'athées. 

 Ils se définissaient aussi eux-mêmes comme le peuple de la \emph{Tora}. Au sens large celle-ci est une collection de 24 écrits (\emph{ta biblia} en grec : les livres) de plus ou moins grande ancienneté, et au sens strict ce sont les cinq livres du \emph{Pentateuque}, attribués à Moïse%
%[3]
\footnote{Au sens large la \emph{Tora} recouvre à très peu près ce que les chrétiens appellent \emph{Ancien Testament} (« ancien témoignage »). La plupart de ces textes sont très largement antérieurs à notre ère, mais la date de mise par écrit de chacun d'eux fait encore l'objet de débats entre spécialistes : du huitième siècle avant notre ère au dernier ? La \emph{Septante} est la version grecque de la Tora, traduite pour les juifs de la diaspora grecque aux troisième et second siècles avant notre ère.}%
.

 La déportation de l'élite du peuple à Babylone et à partir de 586 avant notre ère l'interruption forcée des sacrifices à Jérusalem, avaient provoqué une révolution dans leur pensée et dans leurs pratiques religieuses. S'ils s'étaient crus autorisés à se construire des temples à Babylone, comme le faisaient les autres déportés qu'ils y côtoyaient, peut-être auraient-ils oublié qu'ils étaient des exilés, peut-être auraient-ils fini par s'y sentir chez eux, comblés d'être l'une des minorités influentes de son riche et puissant Empire ? Mais à leurs yeux les sacrifices ne pouvaient être valides qu'à Jérusalem. Ils avaient refusé l'interprétation de leur situation que leurs contemporains considéraient comme évidente. Ils avaient refusé de voir dans la ruine du Temple et leur propre déportation la défaite et l'assujettissement de leur dieu national par les dieux de Babylone. Contre le sens commun ils avaient choisi de comprendre ces tribulations comme une punition à eux infligée en raison de leurs infidélités : le Seigneur s'était servi des étrangers pour punir Israël, ce qui prouvait une fois de plus qu'il était plus puissant que tous leurs dieux. Les exilés avaient choisi de croire que leur retour vers leur dieu se traduirait par la fin de leur déportation : le temps de l'exil avait été un temps de ferveur et d'approfondissement de leur foi. 

%LE PROBLEME DU MAL 
\section{Le problème du mal}

 Si le dieu des juifs était le seul vrai dieu, il n'était que trop visible qu'eux-mêmes étaient loin d'être puissants : jamais riches, rarement souverains, et plusieurs fois écrasés par leurs puissants voisins et déportés. Ceux qui suivaient les prescriptions de la loi de Moïse (les \emph{dix commandements}) n'étaient pas plus épargnés que les autres par les soucis et les malheurs. Des méchants prospéraient avec insolence, tandis que des justes étaient accablés. Comment l'idée d'un dieu unique, créateur, tout-puissant et bon pouvait-elle tenir face à tout ce mal ? Si vraiment il savait et pouvait tout, pourquoi permettait-il l'injustice ? Pouvait-il exiger des faibles hommes une justice impeccable et ne pas la pratiquer Lui-même ? Quel était l'intérêt de passer alliance avec Lui ? 

 Pour répondre à cette question peu à peu l'hypothèse s'est faite jour chez les prophètes que si le juste (et/ou Israël) est accablé de maux et de souffrances, c'est peut-être que le Seigneur le met à l'épreuve, pour tester sa foi et sa détermination. 

 D'autre part (avec le \emph{Second Isaïe} surtout) est apparue l'idée que le juste peut souffrir non seulement du fait des pécheurs mais \emph{à leur place :}« \emph{Le rôle du serviteur de YHWH dans le procès du salut est souligné ici pour la première fois : on sait quel destin eut cette idée dans la littérature religieuse des Juifs et des Chrétiens. La souffrance a une valeur expiatrice, rédemptrice et salvatrice. Telle est la nouvelle réponse que le prophète apporte au problème du mal. Israël, serviteur de YHWH, souffre non pas à cause de ses fautes mais pour expier celles des peuples qui le haïssent, le persécutent et le pillent. Sa souffrance est féconde puisqu'elle fera prendre conscience aux nations des crimes qui les souillent. La grandeur du serviteur se situe ainsi dans son rejet, sa déchéance et sa souffrance ; il accepte de les subir afin d'amener la rédemption du monde} »%
% [4]
\footnote{\fsc{CHOURAQUI}, 1978, p. 281.}%
. Le Juste souffrirait donc pour racheter la faute des autres, pour payer pour eux la dîme%
%[5]
\footnote{Rédimer, d'où vient le mot \emph{rédemption}.}%
. C'était une interprétation nouvelle des très anciens rites du « \emph{bouc émissaire} » et de la « \emph{victime propitiatoire} » : c'était justement parce que la victime, bouc émissaire y compris, était innocente, que sa souffrance était en mesure de racheter les fautes des méchants. 

 L'idée que ceux qui souffraient payaient pour les autres faisait appel à une économie de la douleur et des mérites humainement incompréhensible sinon scandaleuse : comment cela pouvait-il se faire ? Selon quelle comptabilité sinistre ou obscène ? Se pouvait-il que le Seigneur jouisse (sadiquement) du spectacle de la souffrance humaine ? Ces questions sont au cœur du \emph{Livre de Job}. La réponse de l'auteur de celui-ci est que Dieu est si transcendant, si au-dessus de toute comparaison avec l'homme, qu'il n'est pas possible à celui-ci de comprendre ses intentions. Il est donc vain de lui demander des comptes et de lui faire des reproches. Malgré l'épreuve du mal injustement, absurdement subi, il convient au contraire de lui faire confiance et de continuer de croire en sa bonté. 

 On ne pouvait commencer à trouver humainement cohérente une telle doctrine, \emph{ce qui n'est pas la même chose que la comprendre}, que si l'on croyait que le Seigneur répondrait par une rétribution \emph{post mortem} à l'injustice subie et à toutes les peines endurées par l'innocent. La plupart des courants du judaïsme des derniers siècles avant J.-C. adhéraient à la croyance en la survie des morts, avec un jugement individuel portant sur la totalité des actes de chaque individu, et en conséquence un paradis éternel de jouissance pour les justes, notamment ceux qui mouraient \emph{à cause de} leur fidélité au Seigneur (cf. les \emph{martyrs} du \emph{Deuxième livre des Macchabées}) et une éternité de tourments pour les méchants%
% [6].
\footnote{La notion d'une survie éternelle des morts ou de leur principe vital (leur « âme »), avec un jugement rétrospectif de la vie du mort, avait été élaborée par les égyptiens bien avant que les juifs n'y adhèrent. Durant les derniers siècles avant notre ère c'était une notion presque universelle dans le monde antique, mais la qualité de vie promise par la plupart des « enfers », ou lieux de survie des « ombres » des morts, laissait encore beaucoup à désirer. Pour Kant la morale implique logiquement un jugement et une sanction \emph{post mortem}. Les hommes sont ainsi faits et ne peuvent penser autrement. À ses yeux cette exigence de la pensée n'est d'ailleurs ni la preuve qu'il existe une divinité bonne et juste ni la preuve qu'elle n'existe pas.}%
.

 Si la solution biblique au problème du mal n'avait rien de raisonnable elle entrainait néanmoins des conséquences positives pour ceux qui souffraient :
\begin{enumerate}
 % a)
\item ni une santé prospère, ni une vie amoureuse et conjugale réussie, ni des affaires florissantes, ni une nombreuse progéniture, ni le pouvoir conquis, ni les victoires sur l'ennemi ne pouvaient plus être considérés comme des preuves de vertu ;
% b) 
\item si la maladie, le malheur, la souffrance physique, les deuils, les persécutions, l'exil ni la mort n'étaient pas la sanction des fautes du sujet qui les subissait ou de ses ascendants, il n'était plus nécessaire de croire qu'il les avait mérités ;
% c) 
\item il n'était pas nécessaire de dénier l'existence de ces maux, pour protéger la perfection ou la bonté du Seigneur, ni de minimiser leur poids : le mal restait un mal.
\end{enumerate}

 De cela il découlait que :
\begin{enumerate}
%  a)
\item celui qui souffrait n'avait pas à croire que le Seigneur lui en voulait ni qu'il le punissait ;
% b)
\item rien n'autorisait les autres à le mépriser ;
% c) 
\item il était même possible qu'il soit en train de payer à leur place leurs dettes morales ;
% d)
\item leur devoir le plus élémentaire était donc de prendre leur part de son fardeau en lui apportant aide et assistance;
% e)
\item sinon c'est eux qui seraient un jour dans le malheur après leur mort, tandis qu'il serait glorifié comme Job, s'il endurait ses maux sans perdre confiance dans la justice du Seigneur%
% [7]
\footnote{Cf. dans l'évangile de Luc la parabole de Lazare et du « mauvais riche » : Luc 16, 19-31.}%
.
\end{enumerate}

 Dans cette perspective le premier devoir de chacun c'était de tout mettre en œuvre pour soulager celui qui était dans la peine. La preuve de la sincérité de l'attachement au Seigneur, c'était le service des pauvres, des malades, des malheureux de toutes sortes :

\begin{verse}
 « {Quel est le jeûne que je veux ? \\
 C'est briser les chaînes du crime, \\
 Délier le harnais et le joug, \\
 Renvoyer libre l'opprimé \\
 Et déposer le joug. \\
 Partager ton pain avec l'affamé, \\
 Ramener chez toi le pauvre des rues, \\
 Couvrir celui que tu vois nu, \\
 C'est ta propre chair que tu ne fuis plus.} » \\
 (Isaïe, chap. 58, 6--7)
\end{verse}

 
%UN CULTE SPIRITUEL
\section{Un culte spirituel}

 À Babylone les exilés avaient procédé à une relecture de leurs traditions, en vue d'adapter leurs prescriptions à de nouvelles conditions de vie au milieu de peuples étrangers. Le point clé de leur révolution cultuelle et culturelle, c'est qu'ils avaient mis la pratique des bonnes œuvres à égalité avec les sacrifices du Temple. D'autre part, pour remplacer ces derniers et toutes les fêtes grâce auxquels les autres peuplent renouvelaient leur communion, ils avaient institué l'obligation de l'étude personnelle et collective des textes sacrés. C'est à ce moment-là qu'aurait commencé le processus de mise par écrit des livres de la \emph{Tora}. Cette bibliothèque devait en quelque sorte remplacer le temple et le pays perdus. À l'encontre des religions contemporaines, la pratique religieuse des simples fidèles incluait désormais l'étude et la réflexion. 

 Pour un homme l'étude de la Tora était à la fois un devoir et une prière. Les parents avaient pour premier devoir d'initier leurs fils à la Tora et de tenir fils et filles à l'écart des séductions du monde païen pour en faire des adultes fidèles d'Israël. L'enseignement devait veiller à ne laisser à l'écart ni les orphelins ni les indigents. Dès l'âge de six ou sept ans on apprenait à lire et écrire dans le texte de la Tora%
% [8]
\footnote{Chez les grecs et les romains non plus personne n'aurait imaginé à cette époque-là un enseignement « primaire » non imbibé de religion : les textes des premiers exercices scolaires grecs et romains étaient les vies des divers dieux.}%
. L'enseignement de la langue grecque était accepté mais la littérature et surtout la philosophie grecques étaient récusées. 

 Les synagogues étaient les instruments les plus visibles du nouveau culte. Chacune d'elles était à la fois école primaire, maison d'étude, maison de prière, centre communautaire, lieu d'assemblée, restaurant et lieu de réception%
% [9]
\footnote{De la même façon les temples grecs et romains louaient des salles fermées ou des salles à manger en plein air pour ceux qui voulaient recevoir plus de personnes qu'ils ne pouvaient loger dignement chez eux.}%
, hôtellerie pour les coreligionnaires de passage, et tribunal pour les mariages, répudiations, et autres conflits de tous ordres entre coreligionnaires. Aucune autorité centrale ne les créait ni ne les contrôlait. Tout juif adulte (mâle) pouvait en diriger le culte.

 Afin d'achever de se différencier définitivement des autres peuples auxquels ils étaient mêlés les « sages » avaient voulu que pour chaque activité humaine il y ait une manière juive de procéder. Selon le Talmud%
% [10]
\footnote{\fsc{Cohen}, 1980, p. 227. Le Talmud est la deuxième grande œuvre des juifs, après la Tora. Il a fixé par écrit à partir du deuxième siècle de notre ère la « Loi orale » transmise de sage en sage, de rabbin en rabbin, à côté de la Tora. En ce qui concerne la question de l'existence d'écoles \emph{pour tous} au tout début de notre ère, l'information donnée par le Talmud est pourtant d'autant plus vraisemblable qu'à l'époque où nous nous situons les cités grecques et romaines finançaient elles aussi des institutions scolaires pour leurs jeunes citoyens.}
ils avaient planté autour du peuple la \emph{haie} des prescriptions de la Tora, dressée comme un rempart qui le séparait des autres peuples et le gardait \emph{pur} de toute contamination. La réaction de rejet des observateurs de l'antiquité devant la « superstition » des juifs s'expliquait en grande partie par la rigidité du cadre dans lequel ces derniers s'étaient corsetés. Ils trouvaient que les juifs étaient infréquentables, et « ennemis du genre humain » : de toutes leurs bizarreries les plus discourtoises étaient leur refus d'assister à tout sacrifice aux dieux des autres et de manger avec aucun incirconcis, en un temps où il n'y avait pas de vraie cérémonie publique sans sacrifice aux dieux et pas de sacrifice sans repas en commun. 

 Et pourtant, malgré la coupure avec le monde ordinaire qu'impliquait le mode d'existence juif, les communautés de la diaspora exerçaient une attraction certaine sur leur environnement, et le nombre des « prosélytes » était relativement important. Beaucoup se satisfaisaient d'être des « \emph{craignant Dieu}%
% [11]
\footnote{La « \emph{Crainte de Dieu} » désignait l'attitude d'adoration respectueuse du Seigneur et la volonté de respecter ses commandements : moins une attitude de peur (encore qu'elle n'en soit pas exempte) que de révérence.}%
 », non circoncis%
% [12]
\footnote{La circoncision était douloureuse et non sans risques, et surtout très mal vue chez les grecs et les romains, quand elle n'était pas interdite par ces derniers à tous les hommes libres comme toute autre mutilation.}%
. Les plus courageux ou les plus convaincus se faisaient circoncire, ce qui en faisait de nouveaux juifs. Si la judéité découlait normalement de la naissance, elle pouvait aussi être le fruit d'un choix délibéré par amour pour le Seigneur (un amour qui pouvait aller jusqu'à la mort si les autorités civiles exigeaient quelque chose de contraire à la Tora). Comme le fait remarquer Paul \fsc{VEYNE}, c'était radicalement différent des cités contemporaines qu'on ne choisissait jamais par amour pour leurs dieux.

