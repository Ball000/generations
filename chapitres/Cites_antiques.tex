
\chapter[Cités antiques, inégalitaires et patriarcales]{Cités antiques,\\inégalitaires et patriarcales}

Les cités grecques et romaines n'étaient des démocraties que pour
leurs propres citoyens, et ceux-ci ne représentaient qu'une faible fraction de leur
population. Côte à côte vivaient des personnes aux statuts personnels
différents et hiérarchisés, sans parler des citoyennes, traitées comme des
mineures, réduites au silence sur les affaires publiques, et plus ou moins
cloîtrées chez elles, même à Rome où les femmes honnêtes ne sortaient
de chez elles que voilées de la tête aux pieds.

À côté des citoyens vivaient les étrangers domiciliés et les étrangers
de passage. On les nommait \emph{métèques} chez les grecs, \emph{pérégrins} chez les
romains, c'est-à-dire voyageurs, étrangers en déplacement (d'où est venu
notre \emph{pèlerin}). Leur situation était précaire. Ils devaient apporter des avantages
à la cité et ne lui faire souffrir ni dépense ni désagrément. En Grèce
ils devaient payer des taxes de séjour. Ils n'avaient pas le droit de vote. Ils
n'avaient pas le droit d'épouser des citoyens ou des citoyennes. Ils ne
pouvaient hériter d'un citoyen ni posséder terres ou maisons. Ils ne pouvaient
porter d'armes, ni faire partie des soldats, et en cas de guerre on
les tenait en suspicion. Ils étaient tenus à l'écart des cérémonies du culte
des dieux de la cité. Ils ne pouvaient porter plainte ni se défendre eux-mêmes.
Ils ne pouvaient être entendus directement par les tribunaux. Ils
devaient trouver parmi les citoyens honorables un « patron » qui les représente
et les défende en justice, mais ce « patronage » n'était pas gratuit.
Les peines qu'ils subissaient étaient beaucoup plus sévères que celles
que subissaient les citoyens. Ils ne pouvaient devenir citoyens qu'avec
beaucoup de difficultés, surtout dans l'aire grecque où ils pouvaient demeurer
métèques de père en fils dans la même cité pendant plusieurs siècles%
\footnote{Ainsi
les juifs d'Alexandrie, exclus pendant trois siècles de la citoyenneté de cette cité grecque et bloqués dans le statut de métèques. Ils ont fini par être totalement éliminés par un dernier pogrom, à la suite d'une dernière révolte.}.

À côté de l'ensemble des personnes libres, citoyens et étrangers,
s'activait la masse des esclaves, parfois plus nombreuse que celle des citoyens :
 trente à quarante pour cent de la population, peut-être plus, dans
les périodes où l'esclavage était le plus prospère. L'esclavage colorait la
totalité du fonctionnement et des représentations des sociétés antiques,
et notamment du fonctionnement des familles.

À divers signes vestimentaires chacun pouvait repérer si une
femme était célibataire ou si elle vivait en couple, si une jeune fille était
libre ou esclave. De la même façon le corps des jeunes garçons de famille
comme celui des autres hommes libres étaient protégés par des signes visibles.
Le vêtement permettait de savoir quel(le) esclave était prostitué(e)
au public, quelle autre était réservée par son maître à son service exclusif.
Chacun savait ainsi où il était autorisé à désirer et où il ne l'était pas, où il
pouvait porter la main et où il lui fallait absolument s'abstenir.

Seuls les hommes libres avaient la plénitude des droits civiques. Ils
étaient les chefs de leur famille, qui étaient toutes « patriarcales ».
Ils étaient de droit les prêtres du culte des ancêtres et de celui des dieux
de la maison sur l'autel familial. Ces cérémonies étaient si importantes
que leur bonne exécution était surveillée par les cités. C'est eux qui exerçaient
la justice dans leur maison. Ni leurs concitoyens ni les lois
n'étaient fondés à s'immiscer dans les relations qu'ils entretenaient avec
leurs dépendants (épouse, enfants, esclaves, clients), sauf s'ils excédaient
les limites de manière scandaleuse ou sacrilège : si leur ivresse de pouvoir
ou de colère dépassait la mesure \emph{(hubris)} ou si leurs pratiques incestueuses
risquaient d'attirer les foudres du ciel et d'atteindre indirectement leurs
voisins.

Pour Aristote, entre maître et esclaves, entre père et enfants, il n'y
a pas de rapports de justice à strictement parler, mais seulement une
justice \emph{domestique}, exercée de façon monarchique par le père de famille \emph{et
son épouse}. Si cette dernière est en position seconde par rapport à son
époux, elle ne lui est pas pour autant assujettie%
\footnote{{\emph{Il est clair que les grands textes classiques de la Grèce conçoivent le mariage comme une association établie en vue de la bonne gestion du patrimoine et de la procréation des enfants pour la continuation de la famille et le peuplement de la cité. Toutefois, l'idée d'une ségrégation des femmes dans le gynécée, la mainmise de l'homme libre sur toutes les formes de gouvernement risqueraient d'occulter les exigences éthiques que Xénophon, Platon, Aristote imposaient aux époux. La femme, objet d'échange entre un père et celui qui devient son gendre, acquiert, une fois mariée, des droits et des privilèges. Les rapports des époux relèvent de la vertu de justice. Et, si la femme commet une injustice en refusant de se soumettre à son mari, celui-ci a le devoir de la former, afin qu'elle partage son pouvoir sur la maison. L'épouse ne saurait se confondre ni avec une esclave ni avec une enfant. Le rapport que l'époux exerce sur elle n'en est pas moins de domination.}}
Marie-Odile \fsc{Métral-Stiker} : \emph{Mariage et couple} © Encyclopædia Universalis, 2004.}.
Elle a droit
d'émettre des objections, elle a droit à des explications, elle peut
réclamer d'être convaincue et chercher elle-même à convaincre : elle a
droit à la parole. Selon le philosophe les deux époux sont donc dans
une relation \emph{politique}. 

Pourtant les citoyennes du temps des républiques antiques vivaient sous la
tutelle des hommes, en position de perpétuelles mineures. C'étaient les pères qui seuls avaient pleine autorité
sur leurs enfants. En cas de divorce ou de répudiation ils en avaient la
garde, et ils les confiaient à qui ils voulaient, éventuellement à leur ex
épouse, mais sous leur propre contrôle. Elles pouvaient
être répudiées au moyen d'une simple lettre, sans que leur mari ait
à se justifier, tandis qu'elles-mêmes ne pouvaient pas demander le divorce,
puisque leur statut de mineures légales leur déniait la capacité de poser des actes juridiques. Pour le même motif elles ne pouvaient
pas intenter d'action en justice ni gérer leurs biens propres
(leur dot et/ou leurs autres biens personnels). Le travail salarié
n'existait pas pour elles, ce qui ne veut pas dire qu'elles ne faisaient
rien, mais qu'elles travaillaient quasi exclusivement
à domicile sous l'autorité de leur père ou de leur mari.  Elles ne pouvaient
diriger une entreprise ni un commerce. Veuves, elles ne pouvaient
exercer la tutelle de leurs propres enfants. 

Les hommes adultes et libres n'étaient jamais astreints à la continence
et les maris n'étaient pas astreints à la fidélité%
\footnote{Parmi les relations sexuelles hors mariage on distinguait le concubinage (un homme, marié ou non, et une
femme non mariée, une liaison stable avec à Rome obligation de fidélité pour la concubine), le stupre (un homme 
marié, une femme non mariée, pas de stabilité, liaison sans lendemain), la fornication (un homme et une femme 
non mariés, pas de stabilité, liaison sans lendemain), ou l'adultère (un homme, marié ou non, une femme mariée, 
et une liaison sans lendemain).}
(à Rome ils pouvaient
même entretenir des maîtresses ou des amants sous le toit conjugal).
Au contraire, la vie sexuelle des femmes était toujours contrôlée par
un homme (père, frère, mari, concubin, etc.), et les épouses se devaient
d'être fidèles à leurs époux quoi qu'ils fassent avec d'autres (filles ou garçons)
qu'elles.

Sous la République romaine et selon la tradition la plus ancienne,
les filles quittaient leur propre famille en se mariant et entraient dans celle
de leurs maris. Elles étaient placées par leurs pères sous l'autorité de
ceux-ci, comme s'ils les adoptaient. Juridiquement, leur statut ressemblait
à celui de leurs propres fils et filles : \emph{comme si} elles étaient les sœurs de
leurs enfants. En cas de décès de leur mari, elles étaient d'ailleurs comptées
au nombre de ses héritiers, à égalité avec chacun de leurs enfants
communs. Mais bien avant la fin de la république romaine, elles ont le
plus souvent été confiées à leurs époux \emph{sine manu}, sans « la main », c'est-à-dire
que leurs pères conservaient le pouvoir de les diriger (\emph{manu}) et
qu'elles restaient juridiquement étrangères à la famille de leurs enfants.
Leurs pères pouvaient de leur propre initiative rompre les mariages de
leurs filles s'ils y trouvaient intérêt ou (plus souvent ?) si elles le leur demandaient
et s'ils y consentaient. Ces dispositions juridiques n'empêchaient
pas les jeunes romains de savoir qu'ils étaient nés de l'union de
deux familles, ni d'être fiers de la famille de leur mère et de l'aimer autant
que celle de leur père.

Les femmes mariées de Rome avaient droit au \emph{manteau des matrones},
qui enveloppait tout leur corps et ne laissait voir que leur visage à l'exclusion
des cheveux. Les femmes du monde grec étaient plus strictement
recluses dans leurs gynécées et exclues de l'espace public que les romaines.
Quant aux femmes d'Asie (Asie Mineure), certaines d'entre elles
étaient depuis toujours entièrement voilées, visage y compris. Le
« manteau des matrones », le voile, protégeait celles qu'il recouvrait
contre les regards et surtout les gestes masculins. C'était en effet un signal :
les agressions sexuelles contre une femme étaient punissables
comme des crimes si (et seulement si) la victime portait le manteau des
matrones ou la tunique des jeunes filles libres.

Les enfants légitimes sont ceux qui peuvent hériter de la fortune
de leur père même sans testament, qui peuvent de droit lui succéder et le
remplacer (les garçons au moins) dans toutes ses charges et prérogatives.
Pour les grecs et les romains n'étaient légitimes que les enfants nés d'une
mère mariée \emph{en justes noces} avec leur père au moment de leur conception.

Les noces n'étaient \emph{justes}, un mariage n'était valide, que si les règles
suivantes étaient respectées :
\begin{enumerate}
%a) 
\item Il fallait que les liens de parenté ne l'interdisent pas%
\footnote{Cf. p. 92, \emph{Histoire du droit civil}, Jean-Philippe \fsc{Levy}, André \fsc{Castaldo}.}.
Le mariage
était toujours interdit en ligne directe jusqu'à l'infini (parents,
grands-parents, enfants, petits-enfants, etc.). Par contre le monde hellénistique
autorisait le mariage entre frères et sœurs, ce qui était surtout
pratiqué en Égypte. Les grecs autorisaient le mariage entre demi-frères et
sœurs. À Athènes le mariage était possible entre demi-frères et sœurs de
même père (consanguins). À Sparte il était autorisé entre demi-frères et
sœurs de même mère (utérins). Quant aux romains ils ont toujours interdit
le mariage entre frères et sœurs. Jusqu'au \siecle{3} avant notre
ère le droit romain interdisait le mariage jusqu'au sixième degré, aussi
bien du côté des agnats (famille du père) que des cognats (famille de la
mère). Mais du \siecle{3} av. J.-C. au Haut Empire Romain, on constate
une tendance au relâchement des interdictions, ce qui évitait la
dispersion des patrimoines. Le mariage est interdit entre bru et beau-père,
belle-fille et beau-père, gendre et belle-mère, ou beau-fils et belle-mère,
mais est autorisé entre beaux-frères et belles-sœurs. La \emph{sobrina}
(cousine issue de germains) puis la \emph{consobrina} (cousine germaine) sont
autorisées. Le mariage entre tante et neveu est interdit, mais le mariage
entre l'oncle paternel et la nièce a été autorisé sous l'Empire Romain.
%b) 
\item Il fallait que le statut personnel des deux conjoints autorise un
mariage. L'union d'un citoyen avec une métèque ou une pérégrine (et inversement)
ou une esclave (idem) n'était pas valide, ce qui ne veut pas dire
qu'il n'y en avait pas, mais que les enfants qui en naissaient étaient illégitimes,
et avaient ordinairement le statut de leur mère, même quand ils
étaient reconnus par leur père. À Rome les patriciens, nobles, ne pouvaient
se marier en dehors de leur caste.
%c)
\item La fille ne devait avoir été ni enlevée ni séduite contre l'accord
de son père (ou de son tuteur). Il fallait qu'il l'ait donnée à son époux, et
de bon cœur. Il ne devait pas l'avoir vendue contre de l'argent, ni échangée
avec des cadeaux ou des avantages personnels, comme cela se passait
pour les concubines.
\end{enumerate}
Les hommes ne supportaient pas l'idée de mourir sans héritier, ne
serait-ce que pour assurer le culte des morts familial : leur propre avenir
post mortem était en jeu. Ils accordaient donc un soin jaloux au choix de
leur héritier, et il n'était pas question qu'on leur en impose contre leur
gré. Il s'agissait aussi et en même temps d'assurer la prospérité de leur
maison et la sécurité de leurs vieux jours. Pour y parvenir, ils disposaient
d'un certain nombre d'outils :

\begin{enumerate}
%a) 
\item La fidélité des épouses : La plus grande des fautes que celles-ci
pouvaient commettre était d'introduire \emph{subrepticement} dans la famille de
leur mari un héritier qui ne serait pas né de ses œuvres%
\footnote{En l'absence des connaissances sur la physiologie de la reproduction qui sont les nôtres aujourd'hui les représentations étaient incertaines et infiltrées de projections. On se demandait si le corps des femmes coopérait à la 
fécondation à égalité avec celui des hommes, s'il existait en quelque sorte un « sperme féminin » dont la présence 
était nécessaire, ou s'il n'était qu'un réceptacle passif pour le sperme masculin, seul actif (cf. Aline \fsc{Rousselle}, 
1998). Les modèles en présence oscillaient de celui où le corps de l'homme et celui de la femme coopèrent à la fécondation à celui où seul l'homme compte, mais ils n'écartaient pas absolument celui où les précédents amants de 
la femme comptaient aussi (mais pas les amantes de l'homme). On supposait parfois que le corps d'une femme 
était marqué par la semence de chacun de ses partenaires (en se mêlant à son sang), et que cette imprégnation jouait
un rôle obscur mais réel
dans la fécondation de ses
enfants légitimes,
conçus des œuvres
de son mari. Cette 
croyance redoublait la faute des femmes adultères et augmentait l'importance de la virginité des jeunes filles. Elle 
aggravait encore la gravité des viols qu'elles pouvaient subir, les rendant irréparables.}.
À Rome (mais pas en Grèce) s'il était consentant nul n'avait plus rien à y redire.
%b) 
\item La répudiation : la stérilité des couples (20 \% des unions jusqu'au
\siecle{19} ?) était presque toujours imputée à la femme, c'est
pourquoi le renvoi de celle-ci, suivi d'un remariage, était la solution normale,
attendue.
%c) 
\item Compte tenu du caractère pré scientifique des idées du temps
sur la grossesse il y avait une relative indistinction entre les méthodes anticonceptionnelles
et les méthodes abortives. On estimait que la vie ne
commençait qu'à partir de la coagulation de la semence (masculine seule,
ou masculine et féminine, suivant les auteurs) coagulation dont la durée
pouvait être longue puisque c'est le mouvement du fœtus qui en fournissait
la preuve. Aristote (384-322 avant notre ère) pensait que les garçons
étaient animés au bout de quarante jours de grossesse, les filles au bout
de quatre-vingt (cette opinion sera tenue pour la plus vraisemblable jusqu'à
la fin du Moyen Âge en raison de l'immense autorité de son auteur).
Les mesures anticonceptionnelles et abortives (entendues à partir du
moment où les mouvements du fœtus étaient perceptibles) n'étaient
frappées d'aucun interdit. Plus que des médecins c'était d'abord le domaine
des femmes d'expérience, dans le secret des gynécées. Dans ce
domaine la faute pour une femme c'était d'aller contre le désir de son
époux ou de son maître, qui avaient le droit de la contraindre à avorter,
comme celui de lui interdire de le faire.
%d) 
\item L'infanticide : ceux des enfants qui naissaient mal formés ou
trop chétifs semblent avoir été aussitôt noyés ou étouffés par les sages-femmes,
avec l'assentiment de tous. Selon Sénèque : {\emph{les enfants, s'ils sont
débiles ou difformes, nous les noyons}%
\footnote{\fsc{Seneque}, \emph{De ira}, Livre I, chapitre VI}%
}.
Pour \hbox{Soranos}, médecin grec du
\siecle{2} de notre ère, la puériculture est l'art de décider {\emph{quels sont les
nouveaux-nés qui méritent qu'on les élève}}. L'infanticide était aussi le sort ordinaire
des esclaves nouveaux-nés et de tous les enfants nés de conceptions
irrégulières, embarrassantes ou scandaleuses (inceste, adultère...).
%e) 
\item L'abandon : le père de famille faisait abandonner {\emph{au bon cœur
des inconnus}%
\footnote{John \fsc{Boswell}, 1993.}%
} les nouveaux-nés dont il ne voulait pas. Tant qu'il le faisait
dans les premiers jours, avant de leur avoir donné un nom, personne
n'avait rien à y redire. Lorsqu'il souhaitait qu'ils vivent il les faisait exposer
en un lieu public, connu de tous, et où se rendaient régulièrement
ceux qui étaient à la recherche d'un nourrisson sans parents (à tel endroit
du forum, entre telle et telle colonne de tel temple, etc.). Selon un auteur
de comédie du \siecle{3} avant notre ère c'étaient d'abord les filles
qu'on exposait : {\emph{un garçon on l'élève même si on est pauvre, une fille, on l'expose
même si on est riche%
\footnote{Cité par Jean-Nicolas \fsc{Corvisier} et Wieslaw \fsc{Suder}, \emph{la population de l'antiquité classique}, Que sais-je, PUF, 2000.}%
}} (boutade qui ne nous dit malheureusement pas
quels étaient les pourcentages d'abandons effectifs).
%f) 
\item La vente : longtemps les pères ont eu le droit de vendre leurs
enfants à tout âge, dès leur naissance. Ils pouvaient a fortiori les mettre
en gage chez un prêteur. Assez tôt les cités méditerranéennes ont pourtant
interdit la réduction en esclavage de leurs propres citoyens (mais pas
celle de leurs métèques et pérégrins) autrement que par un jugement public
en bonne et due forme : une sanction pénale. Mais comment punir
un père d'avoir vendu son enfant si c'était le seul moyen qu'on lui avait
laissé pour que cet enfant et lui-même puissent survivre (cf. chapitre~\ref{vente-parent} à la page~\pageref{vente-parent}) ?
%g) 
\item Le don d'un fils à un autre citoyen : un citoyen suffisamment
pourvu en garçons pouvait donner l'un d'eux à un concitoyen dépourvu
de fils. Le fils donné perdait tout droit à l'héritage de son père, mais devenait
l'héritier de son père adoptif. Quant à son père de naissance il
pouvait ainsi donner une part d'héritage plus importante aux héritiers qui
lui restaient.
%h) 
\item Le choix des gendres : les filles ne choisissaient pas leur mari.
Les pères (et à défaut les tuteurs) donnaient leurs filles aux hommes de
leur choix, et à certaines conditions ils pouvaient aussi rompre leur mariage
à leur gré si une autre alliance paraissait préférable.
%i) 
\item L'adoption d'un garçon pour en faire son héritier : qu'il soit célibataire
ou marié un homme pouvait adopter des enfants ou des adultes
pour en faire ses héritiers et successeurs. Seul un citoyen pouvait hériter
d'un autre citoyen, afin que ses biens (notamment la terre de la cité, toujours
ressentie comme sacrée) ne tombent pas aux mains d'étrangers :
seuls des citoyens étaient donc adoptables par les citoyens. C'étaient les
seuls dont les successions avaient une importance politique.

En Grèce un homme ne pouvait adopter qu'en l'absence complète
d'héritier mâle, et il fallait que l'adopté soit choisi au plus près possible
dans sa parentèle mâle. Seuls pouvaient être adoptés des enfants ou des
jeunes gens légitimes (nés en justes noces de deux parents eux-mêmes
nés libres, citoyens et légitimes). Si l'adoptant n'avait qu'une fille et aucun
fils légitime (fille \emph{épiclère}), le garçon qu'il adoptait se devait d'épouser cette
fille (sa cousine germaine dans beaucoup de cas) sauf à perdre la jouissance
de l'héritage de celle-ci au profit de ses enfants dès qu'ils seraient
en âge de le revendiquer, à leur majorité.

Ni à Rome ni en Grèce une femme ne pouvait adopter. Elle pouvait
prendre en charge un enfant jusqu'à l'âge adulte et l'établir dans la
vie, si elle obtenait de son mari ou de son tuteur le droit de le faire. Mais
elle ne pouvait pas pour autant en faire son successeur, et elle n'avait jamais
l'autorité d'un père sur lui. De même aux yeux de la loi une épouse
n'était-elle pas la mère légale des enfants \emph{adoptés} par son mari.
%j) 
\item La légitimation des enfants des concubines : \emph{dans l'aire grecque}
l'enfant d'une concubine, même née libre, et d'un citoyen était un citoyen
illégitime et ne pouvait hériter. Quant à l'enfant né d'un citoyen et d'une
esclave, il ne pourrait même pas devenir un citoyen, ni lui ni sa postérité
après lui. Les enfants des concubines, libres ou esclaves, ne pouvaient
donc jamais succéder à leurs pères. Les laisser vivre ne présentait pour
ces derniers que des inconvénients.

\emph{À Rome par contre} les esclaves devenaient des citoyens adoptables à
la condition d'être affranchis dans les règles : seul un citoyen romain
pouvait transmettre ce statut, non une citoyenne ni un pérégrin. Il pouvait
déclarer dès leur naissance comme libres les enfants de ses esclaves,
ce qui faisait d'eux des citoyens. Il pouvait alors les reconnaître pour
siens et en faire ses héritiers. Si une concubine lui donnait un garçon
c'était une assurance contre le risque de se retrouver un jour sans descendance,
aussi n'avait-il aucune raison de s'interdire d'avoir une ou plusieurs
concubines, libres ou esclaves. Les concubines, affranchies ou libres,
étaient astreintes à la fidélité au même titre que les épouses, et elles
avaient droit elles aussi au manteau des matrones. Le citoyen romain
pouvait aussi laisser dans le statut d'esclave un enfant né de ses relations
avec une de ses esclaves. Il pouvait ainsi en obtenir une obéissance plus
exacte, et se donner le temps de vérifier s'il était digne de lui succéder. À
la différence de la déclaration faite dès la naissance l'affranchissement ultérieur
ne pouvait pas faire de l'affranchi un fils légitime, ni effacer la
marque servile qui limitait ses droits civiques et qui lui interdisait la plupart
des « honneurs », mais elle lui permettait quand même d'hériter. Il
suffisait pour cela que le père affranchisse son enfant par testament et en
fasse son légataire universel.
%k) 
\item À côté de l'adoption les anciens connaissaient la prise en charge
d'un \emph{alumnus}, d'un enfant nourricier. Cela bénéficiait à des garçons ou des
filles sans parents. Celui qui avait accueilli un enfant sans connaître son
statut de naissance (ce qui était la règle avec les nouveaux-nés exposés)
l'élevait à son gré comme un esclave ou comme un libre. C'était une espèce
d'adoption incomplète. \emph{L'alumnus} n'entrait pas dans la famille de
l'adoptant et n'avait aucun droit à hériter. Les romains pouvaient néanmoins
choisir leur héritier parmi leurs \emph{alumnii}. Cette pratique ne reposait
que sur la bonne volonté de l'adoptant et sur l'affection mutuelle, parfois
teintée d'une dose de pédophilie. Puisqu'il ne s'agissait pas de faire des
\emph{alumnii} des successeurs les femmes étaient sur ce terrain à égalité avec les
hommes, et il n'y avait pas de limites au nombre d'enfants qu'ils ou elles
pouvaient ainsi accueillir. Ils ou elles se sentaient généralement le devoir
moral d'établir dans la vie ceux de ces jeunes qu'ils avaient élevés comme
des personnes libres.
\end{enumerate}


\section{Le \emph{pater familias} romain}

Il ne faut pas confondre le patriarcat et la prééminence des hommes
sur les femmes, même s'il y a bien entendu un lien significatif entre
ces deux éléments. Le patriarcat c'est la prééminence du principe paternel
dans l'ordre de la parenté. Ce n'est pas la prééminence du masculin, qui
se retrouve régulièrement même là où les familles sont matriarcales. C'est
le fait que la famille paternelle a plus d'importance que la famille maternelle.
Le cas le plus significatif c'est celui des sociétés où l'enfant appartient
entièrement à la famille de son père, et où il n'a rien à attendre en
terme de protection ou d'héritage de la famille de sa mère.

C'était le cas des familles de la Rome républicaine qui présentaient
des traits patriarcaux très purs, au moins sur le plan du droit.

Le \emph{pater familias} de Rome n'était pas un père banal : en fait il n'avait
même pas besoin d'être père. Un père pouvait n'être pas le pater familias
de ses enfants, tandis que quelqu'un qui n'avait aucun enfant pouvait être
pater familias.

Les dépendants d'un pater familias s'adressaient à lui en l'appelant
\emph{dominus}, « seigneur », « maître », et une \emph{familia} n'était pas ce que nous
appelons une famille, mais l'ensemble des personnes dépendantes d'un même
\emph{dominus} : d'abord l'ensemble de ses esclaves, et accessoirement tous ses
autres \emph{familiers}, enfants y compris.

Par contre son épouse n'en faisait partie que si son propre père
avait donné à ce dernier sa puissance sur elle : c'était le mariage \emph{cum manu}
(« avec la main »), en voie de désuétude à la fin de la République.

L'autorité d'un \emph{pater familias} s'éteignait à sa mort, ou bien quand il
était condamné à une peine infamante, ou bien quand il devenait esclave,
quel qu'en soit le motif, ou bien quand il renonçait de lui-même à sa
propre puissance sur un enfant en l'émancipant, ou bien en le vendant,
ou bien en le donnant à un autre pater familias pour que celui-ci l'adopte.
Pour posséder le titre de \emph{pater familias} il fallait être homme \emph{(vir)}, et citoyen
romain. Il fallait avoir vécu soi-même sous l'autorité d'un autre pater familias.
Il fallait ne plus être sous sa puissance (\emph{sub manu}, « sous sa main »).
Cette autorité se transmettait donc comme un témoin qu'on se passe
d'homme à homme, à la façon d'un adoubement ou d'une onction, à la
suite d'une initiation, ou encore comme une espèce de grade militaire.

Un homme ne devenait pater familias qu'à la mort de son propre
père, même s'il était chargé d'enfants depuis longtemps : à Rome il n'y
avait de majorité civile (à partir de 25 ans) que pour les orphelins de père.
Seuls ces derniers étaient autonomes : \emph{sui juris}.

Tant qu'un pater familias était vivant aucun de ses enfants ne possédait
rien en propre et ne pouvait prendre aucune initiative susceptible
de diminuer sa fortune (gérer des affaires personnelles, emprunter, se
marier, affranchir des esclaves, etc.). Rien ne pouvait le contraindre de
couvrir les dettes contractées par un de ses enfants : nul n'était donc assez
imprudent pour prêter de l'argent à celui qui n'était pas encore orphelin
de père, quel que soit son âge. Le statut juridique d'un homme en
puissance de père était donc en partie celui d'un mineur, même s'il avait
femme et enfants, même s'il était d'âge mûr, même s'il était magistrat ou
s'il dirigeait des campagnes militaires, etc. Ceci dit l'âge au mariage des
hommes, (autour de 25 ans, et souvent plus) et les taux de mortalité antiques
faisaient qu'à 20 ans la plupart des fils étaient orphelins de père.

La consommation sexuelle n'était pas nécessaire à la validité du
mariage romain. De droit le pater familias était le père de tous les enfants
de son épouse, sauf de ceux dont il contestait la paternité, ce qu'il était le
seul à pouvoir faire. Du moment que les choses se passaient avec son accord
ou sur son ordre un eunuque était légalement le père des enfants de
son épouse, et nul n'avait rien à y redire.

On a vu que les femmes commettaient une faute très grave si elles
introduisaient l'enfant d'un autre homme dans la famille de leur mari à
l'insu de celui-ci. Elles en commettaient une tout aussi grave si elles le
privaient volontairement d'un héritier, par avortement, abandon, infanticide,
etc.

Le pater familias de la république romaine pouvait abandonner ses
enfants, les vendre, les donner, les faire mourir à leur naissance, ou les
chasser (= premier sens de l'émancipation). Sans aller jusque là il pouvait
les déshériter et léguer ses biens à un enfant adopté%
\footnote{Il fallait tout de même qu'il ait un motif de mécontentement suffisant pour choisir cette solution sinon
la jurisprudence voulait que la justice casse le testament.}.

Celui qui avait vendu un de ses enfants n'en récupérait pas moins
sa puissance \emph{(potestas)} sur cet enfant si celui-ci était affranchi par son maître
(encore fallait-il qu'il le sache). Selon la loi romaine des XII tables, table
IV, (vers 450 avant notre ère) il pouvait néanmoins perdre définitivement
sa puissance. {\emph{Si pater filium ter venum duit, filius a patre liber esto}} :
si un père vend trois fois son fils, ce fils sera émancipé de la puissance de
ce père. Ce cas de figure implique un père qui utiliserait son fils (qui
pouvait parfaitement être un homme adulte) comme une source de revenus,
au lieu de le laisser libre de trouver des revenus lui-même à défaut
de pouvoir subvenir lui-même à ses besoins.

Du moment qu'il pouvait prouver sa paternité, le père qui avait
abandonné un nourrisson ou un enfant (abandonné et non vendu, ce qui
aurait éteint ses droits tant que l'enfant n'était pas affranchi) pouvait
n'importe quand le reprendre à celui qui l'avait recueilli, sans avoir à lui
verser aucune indemnité. Sa puissance n'avait pas été éteinte par l'abandon.

À Rome quel que soit son âge un citoyen qui n'avait pas de \emph{pater
familias} pouvait se mettre volontairement sous l'autorité d'un autre citoyen :
c'était \emph{l'adrogation}. Le résultat était le même qu'une adoption. Dans
les deux cas il fallait un écart d'âge minimum entre les partenaires, pour
mimer la nature. Une fille aussi pouvait bénéficier de l'adrogation. Pas
plus qu'un adopté un adrogé n'était \emph{ipso facto} héritier principal et successeur
de son \emph{pater familias}. Il fallait pour cela un testament en bonne et
due forme.


\section{Le patriarcat}

Les sociétés méditerranéennes des derniers siècles avant notre ère
(Romaine, Grecque, Juive...) étaient patriarcales. La famille romaine
semble être le modèle même de la famille patriarcale, si l'on s'en tient à
l'image que nous en donne le droit du temps de la République. Peut-on
dire pour autant que l'organisation patriarcale est la forme la plus ancienne
de la famille ? À cette question Emmanuel \fsc{Todd}%
\footnote{Emmanuel \fsc{Todd}, \emph{L'origine des systèmes familiaux}, Tome I, L'Eurasie, NRF,
essais, Gallimard, Paris, 2011.}
répond par
la négative (il récuse d'ailleurs tout autant l'idée qu'il ait existé des familles
authentiquement matriarcales).

Selon lui le patriarcat aurait été inventé en Mésopotamie durant le
troisième millénaire, et en Chine à la fin du deuxième millénaire avant
notre ère. Il se serait ensuite très lentement diffusé de proche en proche
dans le monde ancien. Le patriarcat donne aux hommes la prééminence
et l'autorité et met les femmes en une position inférieure, assujettie. Là
où il a réussi il aurait selon \fsc{Todd} supplanté la famille nucléaire qui
donne la même importance aux parentés maternelles et paternelles et où
femmes et hommes ont une valeur et une autonomie comparables.

Le patriarcat n'aurait atteint la cité romaine qu'au cours du dernier
millénaire avant notre ère, bien après son arrivée en Grèce, et il n'aurait
pas réussi à se propager plus loin vers le nord de l'Italie, vers la Gaule et
la Germanie... Le droit romain de la famille serait donc le témoin de la
pointe la plus extrême de la marche vers l'ouest du patriarcat durant
l'antiquité.

Mais toujours selon \fsc{Todd} les signes que la situation réelle vécue
au sein des familles romaines ne reflétait pas la pureté du modèle patriarcal
sont nombreux. Ce modèle aurait été adopté par les romains avec
l'enthousiasme des néophytes, témoin le droit romain, mais ils auraient
continué de ressentir et de penser en termes de famille nucléaire,
d'égalité des familles maternelle et paternelle et d'égalité des époux en dignité,
et ils auraient valorisé l'amour conjugal de manière bien peu patriarcale.
C'est d'abord les familles les plus riches en patrimoine,
l'aristocratie, qui aurait intériorisé les normes patriarcales, tandis que celles
qui avaient peu à transmettre y auraient été peu sensibles. La Grèce
aurait au contraire effectué une conversion plus complète.

À partir de la fin de la république et pour diverses raisons les romains
(c'est-à-dire toutes les personnes libres de l'empire dès le troisième
siècle) seraient revenus à une situation moins déséquilibrée, reconnaissant
de plus en plus à égalité la double parenté paternelle et maternelle.

