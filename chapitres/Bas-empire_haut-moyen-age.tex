% 28.02.2015 :
% haut Moyen Âge
% _, --> ,
% ~etc.
% Antiquité
% ~\%


\chapter[Les sociétés du Bas-Empire et du haut Moyen Âge]{Les sociétés du Bas-Empire\\et du haut Moyen Âge}


 Nous ne pouvons lire que les textes et inscriptions qui nous sont parvenus, or l'écriture était à la fin de l'Antiquité et au début du Moyen Âge un privilège et une distinction. Compte tenu de la diminution progressive du nombre des laïcs cultivés, les clercs, et surtout les moines, en devenaient peu à peu les spécialistes. C'est par leur truchement, c'est à travers leurs yeux que nous sommes aujourd'hui condamnés à regarder leur monde. Beaucoup d'entre eux étaient eux-mêmes issus des familles de guerriers, de l'aristocratie de la naissance. Les serfs et les esclaves ne pouvaient pas être ordonnés, sauf à être affranchis au préalable, ce qui les déliait de leur dépendance à leur maître. Leur société était l'héritière de l'empire de Caracalla dans lequel, si tous les hommes libres étaient devenus citoyens romains, seuls les nobles \emph{(clarissimi)} jouissaient de la totalité des droits autrefois garantis par la citoyenneté. Que tous les baptisés (hommes et femmes, esclaves ou libres ...) aient la même valeur aux yeux de Dieu, ce qu'ils enseignaient, n'était pas une raison suffisante pour que l'égalité soit recherchée sur cette terre. Au contraire, les hiérarchies leur paraissaient naturelles, nécessaires, crées par Dieu en vue du bien commun. Leur point de vue était conforté par les écrits de Paul de Tarse ou ceux d'Augustin.

 Dans la lignée de l'Antiquité grecque et romaine, et donc du mépris des hommes libres pour les tâches serviles, ils pensaient que l'activité intellectuelle avait plus de valeur que le travail manuel, juste bon pour ceux qui ne possédaient ni revenus fonciers ni savoirs, ce qui allait de pair en l'absence d'écoles gratuites. À leurs yeux étaient associés, sauf exception dûment soulignée, le \emph{sang vil}, la lâcheté et l'incapacité à tenir parole, le paganisme (religion des \emph{pagani}, des paysans) et la sorcellerie, la servilité morale et les \emph{tâches serviles}... Ils trouvaient naturel que coïncident le \emph{sang noble} et les \emph{tâches nobles}, telles que l'étude et le \emph{service divin} (prêtres, évêques, moines de chœur chantant les offices en latin), le sang des aristocrates et l'aptitude à prêter serment, à dire le vrai, à tenir sa parole, à s'engager par contrat : si l'on en croit les \emph{Vies de saints} écrites au haut Moyen Âge, rares étaient ceux d'entre ces derniers qui \emph{n'étaient pas} issus de haute noblesse. Les moines qui les rédigeaient étaient \emph{presque} incapables d'imaginer qu'un personnage digne d'être mis sur les autels puisse ne pas être né d'un puissant seigneur et d'une noble et pieuse dame. 

 Chez les Germains comme chez les Celtes, c'est la naissance qui déterminait la valeur. À leurs yeux la société reposait sur le \emph{sang}, c'est-à-dire les ascendants, la lignée, l'hérédité. Il existait quelques lignées nobles, descendantes en partie de l'aristocratie romaine, en partie des aristocraties barbares, et de plus en plus des deux à la fois, distinguées de toutes les autres, celles des multitudes de personnes au sang vil, sans parents dignes de mémoire. La société s'organisait en un système qui serait un jour théorisé (par des clercs) comme l'union de ceux qui prient (et qui prêchent et enseignent), de ceux qui combattent (et qui dirigent), et de ceux qui nourrissent tout le monde (ceux qui transpirent et œuvrent de leurs mains et qui paient taxes et dîmes). Ces derniers étaient d'abord les « vilains », ceux qui habitaient les \emph{villas}, c'est-à-dire les paysans : 95~\% de la population d'alors.

 Les plus humbles n'ont laissé de traces directes que pour les archéologues. On ne peut donc savoir quelles étaient leurs propres représentations. Jusqu'où avaient-ils la possibilité ne pas s'identifier à l'image que les savants de leur époque, tous clercs, avaient d'eux-mêmes ?

 Il faudra attendre le \siecle{12} pour que la croissance des villes, celle des populations et celle des économies permettent une renaissance des civilités, sous des formes inconnues de l'Antiquité, mais aussi brillantes dans certains domaines. 
 
 Les chrétiens apportaient une philosophie de l'histoire, une explication totale du monde et une morale pour tous les instants. Comme l'avait voulu Constantin, ils fournissaient une idéologie unificatrice à l'empire. Mais à la fin de l'Antiquité celle-ci était encore loin d'avoir imprégné la culture et les mœurs. Pour \fsc{DUBY} il faudra attendre le \siecle{12} pour qu'elle soit véritablement intériorisée par l'ensemble des populations. Pourtant dès le \siecle{6}, « chrétien » désignait une identité (une « ethnie » en langage médiatique actuel) au même titre que « Romain », et les deux identités tendaient à se confondre. 

 De son côté l'Empire romain influençait profondément les chrétiens, qui avaient calqué leur organisation territoriale sur lui, avec une hiérarchie religieuse parallèle à la hiérarchie civile. Le christianisme avec ses représentations entrait en résonance avec les conceptions des empereurs, de la même manière que le système impérial lui-même exprimait sans doute le \emph{style de communication} des gens de cette époque%
%[1]
\footnote{Peter \fsc{BROWN}, 1999}%
. Au \siecle{4}, dans les vastes basiliques offertes par Constantin, le nombre des participants, la structure hiérarchique de l'assemblée et le style des homélies, la minutie des rituels, la pompe, le décorum, les luminaires, l'encens, la musique et les chants, tout rappelait les splendeurs des cérémonies des temples romains, grecs ou égyptiens. Au même moment et pour encore un tout petit peu de temps ces derniers continuaient de déployer leurs fastes immémoriaux. 

 Les évêques étaient assez régulièrement issus des familles de sénateurs ou de chevaliers, qui fournissaient ses magistrats à l'Empire, élus par leur clergé et par les membres importants de leur église locale. Certains avaient été eux-mêmes de hauts fonctionnaires avant leur ordination (cf. Ambroise de Milan, ancien préfet). Les évêques parlaient comme les préfets, avec la même conscience de la grandeur de leur mission et de leur légitimité, et la même rhétorique particulière du Bas Empire. Le ton de leurs écrits était en consonance avec celui des mandements et rescrits impériaux. Ils enseignaient et admonestaient leurs ouailles, ils écrivaient et géraient leurs églises avec la même logique intellectuelle, le même esprit juridique, la même conscience professionnelle et le même sens de la grandeur de leur tâche et de leur fonction que les magistrats et fonctionnaires d'alors. Ils présidaient le culte chrétien comme leurs pères avaient présidé les sacrifices des religions civiques dans le cadre de leur \emph{cursus honorum}. 

 Pendant les siècles du haut Moyen Âge, en raison de l'effondrement du système d'enseignement public antique, provoqué en grande partie par la déchéance des villes désertées par l'essentiel de leurs citoyens et passées aux mains des rois barbares, ce sont les clercs qui ont tenté avec plus ou moins de succès de maintenir les traditions littéraires et administratives romaines. De plus en plus souvent ils sont devenus les seuls experts de l'écriture, de la littérature et de l'éloquence, capables d'occuper les emplois de lettrés, et à ce titre ils peuplaient les chancelleries des grands.

