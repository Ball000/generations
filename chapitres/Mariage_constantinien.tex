
\chapter{Le « mariage constantinien »}


 J'appelle « mariage constantinien » le mariage romain tel qu'il a été modifié par Constantin et ses successeurs pour l'accommoder aux conceptions chrétiennes, sans pour autant le modeler sur elles. En effet jusqu'à la Réforme Grégorienne (\siecle{11}) l'Église n'avait pas le monopole du droit familial, et les autorités civiles ne se sentaient pas obligées d'appuyer toutes ses prétentions dans un domaine aussi critique pour la transmission du pouvoir. Dans les deux ou trois siècles où l'Église détiendra le monopole du droit familial les rois ne cesseront de le lui contester, avec de plus en plus de succès. Pendant la longue durée de « chrétienté » les laïcs n'ont jamais suivi la discipline ecclésiastique sans une certaine dose d'ambivalence. Les écarts entre le droit religieux (droit \emph{canon}) et les lois civiles n'ont jamais été nuls, pour ne pas parler de l'\emph{à peu près} avec lequel ces lois étaient respectées. Ce que j'appelle le mariage constantinien est donc un modèle qui n'a jamais été pleinement réalisé, et surtout pas sous Constantin. Pourtant ce modèle tendra peu à peu à s'incarner dans les pratiques et les représentations, et il ne sera peut-être jamais aussi respecté que durant les derniers siècles de notre ancien régime.

 Dans le mariage constantinien plusieurs fonctions distinctes sont télescopées sur une seule personne : un époux est à la fois le détenteur des droits juridiques de son épouse (son curateur), son amant, le géniteur de ses enfants, le détenteur des droits de ces enfants mineurs, et le responsable de leur éducation (c'est-à-dire leur père légal). Symétriquement, une épouse est la seule femme capable de donner à son époux des enfants légitimes, des héritiers, quel que soit le nombre de ses concubines. Chaque enfant légitime ne peut être que l'enfant biologique de ses parents légaux (leur enfant « naturel » au sens antique du terme). Seuls les enfants légitimes ont droit à une part d'héritage, par définition.

 Le Bas-Empire et le haut Moyen-Âge continuent de reconnaître sans discussion la validité des concubinages stables monogames non incestueux, et la légitimité civile et religieuse des enfants qui en naissent : Justinien les autorise à hériter, mais il ne fait ainsi que rappeler une règle de droit ancienne. Dans la pratique du Bas-Empire le concubinage monogame est une forme de mariage souvent (presque toujours ?) employée par les personnes qui ne possèdent pas de patrimoine significatif et ne voient donc pas la nécessité de s'unir en public et solennellement ni de passer devant un notaire. Dans le même sens Augustin d'Hippone enseigne qu'une concubine qui se veut fidèle à son concubin lui est mariée devant Dieu de manière aussi légitime qu'une épouse en titre.

 Pour l'Église c'est le mariage qui fonde la famille et non la présence des enfants, même si elle met l'accueil des enfants au premier rang des « fins du mariage ». De son point de vue, le mariage crée en effet \emph{dès sa célébration} une parenté nouvelle \emph{que les époux soient féconds ou non.} Cette parenté « par alliance » a donc des effets directs sur les membres des parentèles des époux (frères, sœurs, etc.) : elle étend le cercle des partenaires qui leur sont désormais définitivement interdits, même si l'un des époux décède.

 Selon la doctrine chrétienne, identique sur ce point au droit romain, ce sont les époux qui s'unissent l'un à l'autre : cela implique qu'ils soient capables de discernement (âge suffisant, santé mentale) et libres de leur personne : célibataires ou veufs, non esclaves, non engagés par contrat dans une entreprise qui empêcherait la vie commune, à l'abri de toute pression, libres de tout vœux religieux, sexuellement aptes au mariage. L'Église a toujours soutenu contre les parents que les jeunes gens ont le pouvoir de se marier validement sans leur accord, même si elle admettait qu'en leur désobéissant ces jeunes gens les déliaient de leur devoir de les établir dans la vie. 

 Contrairement au droit romain l'Église en est progressivement venue à ne reconnaître la réalité juridique d'un mariage que lorsqu'il a été consommé, assez probablement parce que chez les barbares (cf. chapitre suivant), même christianisés, les unions se construisaient en plusieurs étapes séparées par de très longs intervalles, les premières étapes (dont les promesses de fiançailles) ayant parfois lieu alors que les futurs époux étaient encore de très jeunes enfants. Pour les besoins des procès en nullité de mariage il a fallu trouver un critère décisif dans cette progression, et c'est la consommation du mariage qui a été retenue. 

 Selon les évêques et théologiens chrétiens, le célibat non consacré est licite, mais chez les jeunes gens sans enfants, en bonne santé et disposant de moyens matériels suffisants, il est suspect d'égoïsme, de libertinage ou de désirs « contraires à la nature » (homosexualité notamment dont la mise en acte a toujours été condamnée moralement, même si elle n'a été semble-t-il que rarement sanctionnée). Quels que soient les préférences individuelles la copulation n'est légitime que dans l'état de mariage monogame. D'autre part le mariage est le seul moyen légitime de répondre à l'ordre divin (\emph{croissez et multipliez} de la Genèse). C'est donc l'état normal de tous ceux qui ne sont pas ordonnés à un ministère ou engagés dans la vie religieuse. Mais comme la fin première du mariage est la procréation d'enfants légitimes les remariages sont déconseillés (quoique autorisés) quand cette fin est à priori inatteignable étant donné l'âge ou l'état de santé des conjoints. 

 À partir du \siecle{4} dans l'empire romain, ce n'est plus d'abord et avant tout par la relation de pouvoir qu'il exerce sur les membres de sa maison que le père est juridiquement défini. En effet, il est soumis au devoir de \emph{piété}%
% [2] 
\footnote{La piété était l'affection réciproque et le respect mutuel entre les divers membres de la famille nucléaire, y compris le devoir d'assistance.} 
à l'égard de ses enfants au même titre qu'ils le sont à son égard, et autant qu'eux. D'autre part chez les romains (mais pas chez tous les barbares alliés à Rome) en cas de décès du père, c'est la mère qui, à partir de 390, exerce la tutelle de ses enfants mineurs (et d'eux seuls) si elle a cinquante ans et plus, et du moins tant qu'elle ne se remarie pas, ce qui est le cas général, indépendamment même des réserves ecclésiastiques face au remariage des veuves dotées d'enfants. À partir de 390 une femme n'est plus considérée comme incapable par nature de représenter juridiquement une autre personne qu'elle-même. Le fait qu'à partir de ce moment elle puisse exercer, de droit, la tutelle de ses enfants (même si c'est sous le contrôle éventuel et plus ou moins étroit de la famille de son mari), manifeste que les droits et les devoirs dits « paternels » sont en réalité dès ce moment ceux du couple parental, même si tant qu'il vit c'est le mari qui représente le couple face au monde extérieur%
%[3]
\footnote{... et ce sera le cas jusqu'aux années 60 du \siecle{20}.}% 
. Au fil des siècles, la mise en pratique de ce principe a varié de pays en pays en fonction de nombreux facteurs. Il est probable que plus l'héritage était mince et la famille de petite importance, plus le droit de la veuve non remariée à exercer en toute liberté la tutelle de ses enfants mineurs lui était reconnu, et inversement. Ainsi dans les familles riches et puissantes, il pouvait y avoir tellement d'intérêts matériels ou politiques en jeu que la veuve n'avait pas forcément beaucoup d'impact sur l'éducation de celui de ses fils qui devait prendre la succession de son mari dans ses fonctions publiques. 

 Jusqu'à Constantin la fécondité de chaque femme mariée appartenait sans limites à son mari. Désormais elle ne lui appartient plus. Il n'est plus permis de se débarrasser des enfants non voulus par l'avortement ou par l'infanticide. Sauf indigence extrême il n'est pas non plus permis de s'en débarrasser par l'exposition ni la vente. Une femme n'a donc plus autant à craindre qu'auparavant qu'on ne l'oblige contre son gré à avorter ou à abandonner son nouveau-né. Mais sa fécondité ne lui appartient pas non plus. Pas plus que son mari elle n'a droit de vie ou de mort sur l'enfant qu'elle porte. Chacun des époux reconnaît aussi à l'autre un droit sur son propre corps. Les deux époux se doivent réciproquement fidélité : c'est un devoir \emph{moral} pour l'homme autant que pour sa femme, et même si ses propres infidélités ne sont pas sanctionnées par la loi tout est fait pour qu'il n'ait aucun intérêt à entretenir des maîtresses%
% [1]
\footnote{Cela ne l'empêche évidemment pas d'avoir des rapports avec des prostitué(e)s, rapports qui par nature ne s'inscrivent pas dans la durée.}% 
. Chacun des deux époux a l'obligation de satisfaire autant qu'il est en son pouvoir les désirs sexuels de l'autre, ce qui veut dire que l'épouse doit accepter les étreintes de son mari, quoi qu'elle puisse penser des risques de grossesse et de santé à quoi cela l'expose, et quels que soient ses propres désirs. Ceci dit la modération est prêchée aux maris, qui se voient prescrire la continence de nombreux jours par an. Le \emph{devoir conjugal} n'est par ailleurs exempt de faute morale que si aucun obstacle n'est mis à la fécondation. Les seuls moyens de contrôle des naissances autorisés par l'Église sans restriction ni réticences sont le retard de l'âge au mariage, le célibat et la continence. 

 Il n'est plus possible en principe (mais ce principe a mis de nombreux siècles à s'imposer en dépit de la lutte constante de l'Église) de répudier une épouse présumée stérile (en cas de stérilité dans un couple, c'est celle de la femme qui est toujours suspectée en premier). Ou bien les hommes ont la chance de vivre un mariage fécond et de voir au moins l'un de leurs fils légitimes atteindre l'âge adulte pour leur succéder, ou bien ils doivent renoncer à tout héritier direct tant que vit leur épouse%
% [4]
\footnote{Pour les maris les moins patients il ne restait plus que le « divorce à la carolingienne », c'est-à-dire l'assassinat de l'épouse. Cela ne pouvait se faire que si les institutions policières étaient faibles et les parents de l'épouse moins puissants que ceux du mari. Plus l'État était déliquescent, plus il était possible, comme toujours, de prendre des libertés avec toutes les règles de droit, à la condition de disposer de la force.}% 
. Les couples stériles (dont le nombre n'était pas du tout négligeable jusqu'à l'avènement de la médecine moderne, 20~\% environ) et ceux dont aucun enfant n'a atteint vivant l'âge adulte, sont invités à consacrer aux bonnes œuvres, aux pauvres et à l'Église les ressources qu'ils auraient transmises à leurs héritiers s'ils en avaient eus.

 Tous les enfants nés hors mariage sont pénalisés. En principe il n'est plus possible pour un homme de se faire des héritiers sans se marier, même si la prise en charge d'\emph{alumnii} et leur installation dans l'existence reste une bonne œuvre. La \emph{légitimation par mariage subséquent} est désormais la seule exception de plein droit%
% [5] 
\footnote{... jusqu'au \siecle{20}. Les enfants irréguliers légitimés par les empereurs, les rois ou les papes, ne l'étaient pas de plein droit mais à la faveur d'une grâce, qui pouvait toujours être refusée sans justification, et n'allait pas sans contreparties coûteuses.} 
à la pénalisation des enfants nés hors mariage, et ses conditions sont strictes. Chacun, quelque puissant qu'il soit, doit savoir que s'il a l'imprudence de faire un enfant hors mariage ou dans un mariage contesté par son curé, ou par son évêque, ou par son seigneur, par le roi ou par sa propre parentèle, il ne pourra pas le faire reconnaître comme un de ses héritiers sans combat ou sans procès. Cet enfant ne pourra sans doute pas lui succéder. L'exhérédation totale ou partielle des enfants illégitimes est restée jusqu'à la fin du \siecle{20} le premier frein apporté au désir des hommes de se procurer une descendance ailleurs qu'avec leur épouse légitime, même si d'innombrables exemples montrent que cette règle a mis des siècles à s'imposer.

 Et la perspective de se retrouver avec un enfant à charge, seule, sans aucun espoir d'une légitimation (ni même d'une aide significative venant du père de l'enfant lorsqu'il était déjà marié puisque aucune donation au-delà des frais d'éducation n'était plus autorisée depuis Constantin) a été un obstacle majeur à l'exercice d'une sexualité féminine en dehors du mariage ou avant le mariage. 

 Mais les épouses savent aussi qu'il est devenu, sinon impossible, du moins difficile de les chasser de leur maison ou de leur imposer de cohabiter avec une concubine%
% [6]
\footnote{... même si pour ceux dont la puissance excède de beaucoup celle du commun des mortels, aristocrates, rois, la question peut se présenter différemment, et si les amours ancillaires sont de tous les temps.}% 
. Elles sont à peu près assurées que les infidélités de leur époux n'entraîneront de conséquences graves ni sur elles, ni sur leurs enfants, ni sur le futur héritage de ceux-ci. Tout au plus des « aliments » devront-ils être versés aux enfants nés de leurs maîtresses, mais cela ne portera que sur d'assez petites sommes et seulement jusqu'à ce qu'ils soient mis au travail : 8-10 ans. Il n'est plus question de financer leur établissement dans la vie. 

 À l'occasion du sac de Rome et des horreurs qu'il a entraînées, Augustin reprend à son compte l'idée (banale à son époque) que le viol ne « souille » pas la victime, mais seulement son auteur. Il en tire la conclusion qu'il n'est donc pas question que la victime soit punie pour un acte auquel elle n'a pas consenti : \emph{La sainteté du corps ne consiste pas à préserver nos membres de toute altération et de tout contact... Ainsi donc, tant que l'âme garde ce ferme propos qui fait la sainteté du corps, la brutalité d'une convoitise étrangère ne saurait ôter au corps le caractère sacré que lui imprime une continence persévérante... Nous soutenons que lorsqu'une femme, décidée à rester chaste, est victime d'un viol sans aucun consentement de sa volonté, il n'y a de coupable que l'oppresseur... J'admire beaucoup cette parole d'un rhéteur qui déclamait sur Lucrèce : « Chose admirable ! » s'écriait-il ; « ils étaient deux ; et un seul fut adultère ! » \emph{[celui qui viola Lucrèce]} Impossible de dire mieux et plus vrai...}

 \emph{Mais d'où vient que la vengeance est tombée plus terrible sur la tête innocente \emph{[sur Lucrèce qui s'est suicidée]} que sur la tête coupable? Car Sextus \emph{[son violeur]} n'eut à souffrir que l'exil avec son père, et Lucrèce perdit la vie. S'il n'y a pas impudicité à subir la violence, y-a-t-il justice à punir la chasteté ? ... Quant à nous, pour réfuter ces hommes étrangers à toute idée de sainteté qui osent insulter les vierges chrétiennes outragées dans la captivité, qu'il nous suffise de recueillir cet éloge donné à l'illustre Romaine : « Ils étaient deux, un seul fut adultère ». On n'a pas voulu croire, tant la confiance était grande dans la vertu de Lucrèce, qu'elle se fût souillée par la moindre complaisance adultère. Preuve certaine que, si elle s'est tuée pour avoir subi un outrage auquel elle n'avait pas consenti, ce n'est pas l'amour de la chasteté qui a armé son bras, mais bien la faiblesse de la honte. Oui, elle a senti la honte d'un crime commis sur elle, bien que sans elle. Elle a craint, la fière Romaine, dans sa passion pour la gloire, qu'on ne pût dire, en la voyant survivre à son affront, qu'elle y avait consenti. À défaut de l'invisible secret de sa conscience, elle a voulu que sa mort fût un témoignage écrasant de sa pureté, persuadée que la patience serait contre elle un aveu de complicité. Telle n'a point été la conduite des femmes chrétiennes qui ont subi la même violence. Elles ont voulu vivre, pour ne point venger sur elles le crime d'autrui, pour ne point commettre un crime de plus, pour ne point ajouter l'homicide \emph{(d'elles-même)} à l'adultère; c'est en elles-mêmes qu'elles possèdent l'honneur de la chasteté, dans le témoignage de leur conscience ; devant Dieu, il leur suffit d'être assurées qu'elles ne pouvaient rien faire de plus sans mal faire, résolues avant tout à ne pas s'écarter de la loi de Dieu, au risque même de n'éviter qu'à grand-peine les soupçons blessants de l'humaine malignité.} (\emph{La Cité de Dieu} livre 1, chapitres 18 et 19) 

 Augustin s'y prend de manière paradoxale pour défendre les victimes de viol. Il reprend d'abord l'idée, banale à son époque (cf. la parole du rhéteur), que la pénétration de leur corps contre leur gré ne les a pas moralement souillées, et qu'elles n'ont à se sentir coupables de rien. Mais la honte d'avoir subi le viol sans pouvoir l'empêcher est toujours ressentie par les victimes de quelque âge et sexe qu'elles soient. « Faire honte » est d'ailleurs l'un des buts de certains agresseurs, indépendamment du plaisir sexuel qu'ils peuvent prendre dans ces actes. Augustin qualifie cette honte de « faiblesse », de sentiment compréhensible sinon évitable, mais non fondé en raison, et contre lequel il convient de lutter. Ce faisant il reprend les argumentations traditionnelles. À cela il ajoute l'interdiction du suicide, que celui-ci soit de honte, de protestation ou de désespoir. 

 Le suicide a toujours été condamné par l'Église, comme il l'était en général par la Tora : \emph{comme un péché grave, sauf chez les « fous » ou les victimes d'un « grand chagrin » selon le \emph{premier concile de Braga} qui s'est tenu vers 561. Il s'agissait alors pour l'Église de marquer une différence avec la mentalité héritée de la civilisation romaine qui voyait dans le suicide une mort comme une autre pour le désespéré et une voie honorable, un moyen de rachat pour le criminel}. (Wikipédia). 

 En refusant que le suicide soit une solution acceptable face à la détresse et à la douleur morale éprouvées par les victimes de viol, Augustin posait sur leurs épaules un fardeau qui a pu être insupportable à certaines. Pourtant en leur faisant un devoir de survivre, il déniait aussi aux victimes collatérales (époux, enfants, parentèles, voisins) le droit de les tuer ou de les pousser au suicide pour apaiser leur propre honte de n'avoir pas été capables, eux non plus, d'empêcher le forfait.
 
 