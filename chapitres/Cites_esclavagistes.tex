
\chapter{Cités esclavagistes}

Un esclave, c'est quelqu'un qu'on achète et qu'on vend comme un
objet ou un animal : un {\emph{instrument animé}}, un {\emph{animal
domestique qui parle}}. Il n'a pas droit à un nom propre. Il n'est compté ni au nombre des citoyens,
ni au nombre des étrangers. Il n'est enregistré nulle part et sa
mort ne compte que pour son propriétaire. Son corps n'est pas protégé
par la loi. Jusqu'à la fin de la République romaine, un maître a droit de vie
et de mort sur ses esclaves. A fortiori peut-il légitimement user de leur
corps comme il l'entend : le battre, le blesser, le torturer, le violer, le
prostituer, le castrer, l'avorter,~etc. Dans la mesure où un esclave n'est
pas considéré comme une personne, mais comme un simple prolongement
de la personne de son maître (une « prothèse »), ses rapports
sexuels avec celui-ci n'ont pas le sens d'une relation interpersonnelle.
Un esclave ne peut pas porter plainte. S'il est à bout de souffrances
il n'a comme recours que de chercher asile dans un temple, en espérant
être revendu à un maître moins dur. C'est le maître qui est responsable
de tout ce que l'esclave fait, dit ou subit, qui porte plainte et qui
touche les dommages et intérêts s'il est blessé ou tué, qui supporte les
frais et les amendes s'il cause un accident, et c'est encore lui qui le punit,
sauf à l'abandonner à la partie adverse comme compensation \emph{(nexus)}.

Un esclave n'a pas droit à la parole (publique) : on ne peut faire
crédit à ses paroles que s'il éprouve une crainte plus forte que celle que
son maître lui inspire, ce qui implique de le torturer. En son nom propre
il ne peut ni signer un contrat, ni s'engager par serment, ni se marier, ni
exercer une autorité sur personne. S'il engendre un enfant il n'est pas son
parent légal, même si son maître lui en confie l'éducation. Si une esclave
donne naissance à un enfant celui-ci appartient à son maître : s'il le laisse
vivre il peut le lui enlever pour le vendre ou le confier à quelqu'un d'autre.
Par contre il peut exercer par délégation un pouvoir sans limites autres
que le désir de son maître. Il n'est qu'un instrument, mais si son maître
est un grand personnage sa puissance effective peut être très grande.

Seul le maître subvient aux besoins de l'esclave. Il est admis qu'il
en a le devoir même quand ce dernier est malade, estropié, trop vieux
pour servir, etc. mais jusqu'à la fin de la République romaine ce n'est
qu'une élégance morale. L'affranchir parce qu'il est devenu incapable de
fournir un travail (ce qui le réduit à mendier et/ou à mourir plus ou
moins vite de faim et de misère) n'est pas un délit, tout juste une inélégance
morale.

Tout se passe donc comme si l'esclave ne faisait pas partie du genre
humain, mais personne n'est dupe de cette convention juridique. Sans
cela on ne lui ferait pas la morale. Sans cela on se contenterait de le menacer
des pires supplices s'il venait à porter la main contre son maître (ce
qu'on ne manque pas de faire aussi). Sans cela on ne traiterait pas comme
un assassin le meurtrier d'un esclave (sauf si c'est son maître). Mais la
meilleure preuve qu'un esclave n'est pas une chose, c'est qu'il peut acquérir
ou retrouver les droits des personnes libres. À partir de ce moment il
reçoit un nom propre (jusque là il n'en avait pas), bâti (à Rome) sur celui
de son maître et non sur celui qu'il possédait éventuellement avant de
devenir esclave. C'est une espèce de renaissance sociale. À partir de ce
moment l'ancien esclave est compté dans les recensements. Il peut porter
plainte lui-même si on lui fait tort. En dépit de quelques limites à ses
droits, qui constituent la \emph{marque servile}, il peut prêter serment, se marier,
devenir enfin le père légal de ses enfants (s'il a pu les racheter) et pas seulement
leur géniteur, comme de ceux qu'il aura à l'avenir.

Le maître peut donner à son esclave les droits d'une personne libre,
ou les lui vendre : très souvent l'affranchi doit en effet verser à son
maître une somme au moins égale à sa propre valeur vénale, prélevée sur
le « \emph{pécule} » qui lui est laissé pour encourager ses efforts. Si le maître peut
donner ces droits, cela signifie qu'il les détenait. Un esclave n'est donc
pas exactement quelqu'un qui n'a aucun droit, c'est quelqu'un dont tous
les droits appartiennent à un autre que lui-même, ce qui est le statut des
personnes mineures. D'ailleurs le latin désigne les esclaves et les enfants
du même mot, \emph{puer}.

Les mineurs ne peuvent parler pour eux-mêmes \emph{(in fans)}. Ce sont
leurs parents qui détiennent tous leurs droits. Ils les commandent en
tout, et les sanctionnent s'ils leur désobéissent. Ce sont aussi les parents
qui portent plainte si leurs enfants sont lésés par un tiers. Et c'est encore
eux qui reçoivent l'argent des dommages et intérêts. Mais au cours de
leur croissance les enfants prennent progressivement possession de leurs
droits en apprenant à les exercer dans le cadre et les limites fixés par la
loi et par leurs parents. Leur soumission n'a qu'un temps, contrairement
à ce qui se passe pour l'esclave, et ils n'ont pas à racheter leur statut
d'homme libre.

%Lorsqu'ils sont affranchis les esclaves restent dans la mouvance de
Affranchis, les esclaves restent dans la mouvance de
leur maître, à qui comme des enfants ils doivent leur liberté (même
quand ils la lui ont achetée, puisqu'il aurait pu refuser de la leur vendre),
qui devient leur « patron » : \emph{patronus} est dérivé de \emph{pater}. Ils portent son
nom. Ils lui doivent respect et assistance. Ils renforcent son prestige et
son influence avec plus de docilité et de gratitude obligée que des fils selon
la chair, des gendres ou des beaux-frères.


\section{Un esclave, pour quoi faire ?}

\subsection{Une main-d'œuvre à bon marché}

On peut appeler \emph{servile} n'importe quel travail, s'il est fait par un esclave
sur l'ordre d'un maître. Son statut ne lui interdit en aucune façon
d'être compétent ou cultivé, et parfois bien plus que son maître. Il peut
exercer la médecine ou enseigner une langue étrangère ou la philosophie,
etc. C'est ainsi que Rome a été intellectuellement conquise par les grecs
qu'elle avait réduits en esclavage. Ils lui ont fourni ses précepteurs, ses
professeurs, ses artistes, et ses médecins. Un maître avisé faisait en sorte
que ceux de ses jeunes esclaves qui étaient doués d'un esprit vif ou d'un
talent particulier soient bien instruits. Cela lui permettait de louer plus
cher leurs compétences, de les établir à son compte dans un atelier, une
boutique, un cabinet médical, une exploitation agricole, à charge pour
eux de lui verser une contribution régulière.

Mais à l'exception de ces relatifs privilégiés c'est de manière brutale
qu'on exploitait la force de travail des esclaves sur les domaines agricoles,
dans des ateliers artisanaux ou industriels, sur les galères, etc. D'autres
travaillaient dans les mines ou les chantiers de travaux publics : au
cinquième siècle avant notre ère plusieurs dizaines de milliers d'esclaves
travaillaient dans les mines publiques du \emph{Laurion} (mines d'argent), et
remplissaient ainsi les caisses d'Athènes, qui grâce à eux a pu à diverses
périodes se passer de lever des impôts tout en équipant ses armées et sa
puissante flotte de guerre. À côté d'autres ressources matérielles et intellectuelles,
l'esclavage a ainsi puissamment concouru au « Miracle Grec ».
Parfois loués à la journée par leur propriétaire (comme dans les entreprises
d'intérim actuelles), ces esclaves vivaient dans des brigades à l'organisation
militaire, encasernés comme les bagnards français, les déportés du
Troisième Reich ou ceux de tous les Goulag(s).

L'objectif n'était ni leur rééducation ni leur mort lente, pourtant la
vie de ces esclaves-là n'en était pas moins une épreuve, parfois pire que la
mort. Ils travaillaient d'ailleurs à côté des délinquants condamnés à l'esclavage
(ex : les condamnations romaines \emph{ad minas}, aux mines, substituts
à la peine de mort). On attendait des esclaves un rendement productif.
Cela impliquait qu'ils demeurent en bonne santé : les premiers lieux de
soins collectifs, les premières infirmeries, pour ne pas dire les premiers
hôpitaux, semblent être nés dans les exploitations industrielles ou agricoles
qui employaient des esclaves par centaines ou par milliers. Les premiers
professionnels de la médecine ont peut-être été des esclaves ou des
affranchis grecs formés dans ces hôpitaux-là ? Une clientèle captive et
peu considérée socialement est en effet une aubaine pour l'entraînement
à la pratique de la médecine et de la chirurgie.

\subsection{Un corps sans défenses}

Pour celui qui est devenu un esclave, ni ses parents, ni son milieu
d'origine, ni la ville à laquelle il a peut-être été arraché n'existent plus. Ils
n'ont pas su, pas voulu ou pas pu le défendre contre sa dégradation. Ils
l'ont peut-être même chassé. Si un individu se vend lui-même c'est qu'il
accepte l'idée que sa famille ne peut plus lui apporter de soutien. Il
l'abandonne autant qu'il s'en voit abandonné. Son passé est disqualifié,
réduit à rien, et son nom avec. Il est désormais seul, hors parenté, coupé
de ses racines et disponible pour toutes les nouvelles impressions.

L'esclave participe au culte des ancêtres du maître, c'est-à-dire qu'il
adopte pour ancêtres ceux de son maître, avec ordre d'oublier les siens,
qui n'ont pas su lui porter chance. Au dehors de la maison du maître il
est désigné par le nom collectif de la \emph{familia}. En quelque sorte il porte sa
livrée, son « logo ».

En lui refusant un nom propre on lui dénie le droit à une identité
personnelle. Il est la chose de son maître, aux intérêts et désirs duquel il
faut qu'il s'identifie s'il veut survivre. C'est cette dépersonnalisation (cette
aliénation) de l'un des deux et cette exaltation de l'autre qui est au cœur
de l'intérêt que présente l'esclave pour son maître. Tout se passe comme
si seul le maître avait un narcissisme.

Particulièrement typique du travail servile est le travail au contact
et au service du corps d'autrui (l'aider à s'habiller, lui laver les pieds, l'accompagner
aux thermes et porter ses affaires de bain, le raser, se soumettre
à ses désirs sexuels, etc.). C'était une position traditionnellement désignée
comme « féminine » ou (et) infantile. L'esclave mâle était dévirilisé
par son statut. Les esclaves \emph{domestiques} vivaient sous le toit du maître :
portiers, femmes de chambre, concubines, cuisiniers, cochers et palefreniers,
valets, intendants, secrétaires, précepteurs... Ils le connaissaient
personnellement et pouvaient en être remarqués et favorisés. Ils étaient
\emph{comparativement} privilégiés et vivaient plutôt plus confortablement que les
autres esclaves. Mais il allait de soi \emph{dans ce cadre de pensée} que le maître ne
pouvait pas abuser de ses esclaves, puisque ceux-ci n'avaient le droit de
se refuser à aucune de ses exigences. On peut aussi bien dire que le maître
avait le droit d'abuser de tout esclave et que celui-ci ne pouvait s'en
plaindre à personne. Les esclaves pouvaient être mis au service du public,
et on en trouvait dans tous les types de services à la personne. Le plus
constant de ces services était la prostitution. Elle reposait sur des bataillons
d'esclaves des deux sexes, jeunes filles et jeunes garçons (parfois castrés
pour conserver les attributs de la juvénilité). Il n'y avait pas d'âge
pour commencer. C'est le marché qui commandait. Le statut d'esclave
permettait de satisfaire en toute légalité la demande des pédophiles et autres
amateurs de pratiques de toutes espèces.

Pour les ambitieux l'esclave était un outil qui offrait de grands
avantages. Il n'obéissait qu'à son maître et il était de par la loi dans sa «
main » plus qu'aucun collaborateur libre : c'était un outil docile. Il pouvait
être corrigé sans ménagements, sans qu'aucune famille ne puisse le
défendre ou venger l'affront subi. D'autre part il ne faisait pas partie des
citoyens, il ne pouvait donc pas briguer la position sociale de son maître
ni lui porter ombrage. S'il était promu par ce dernier à un poste enviable
il continuait de tout lui devoir, et pouvait toujours être remis à la place
où il était avant sa promotion. Si son maître faisait de mauvaises affaires
ou tombait en disgrâce, lui-même se retrouvait sur le marché aux esclaves
et tout était pour lui à recommencer. Même si les esclaves n'étaient
pas attachés à leurs maîtres par l'affection, ils leur étaient liés solidement
par la juste appréhension de leurs intérêts réciproques. Chaque puissant
personnage avait intérêt à posséder beaucoup d'esclaves. Il lui était même
possible de les organiser en milices privées pour combattre contre celles
de ses concurrents, les armes à la main. Le latin nommait les esclaves et
les enfants du même mot, \emph{puer} : en effet l'esclave était bloqué comme un
éternel mineur dans une position infantile. Souvent il s'agissait d'ailleurs
d'un enfant, plus ou moins promis à être affranchi une fois adulte. L'esclavage
réalisait alors une espèce de transfert du père au maître : une espèce
d'adoption « au petit pied ». Lorsque les esclaves étaient affranchis
ils restaient dans la mouvance de leur maître, qui devenait leur patron. Ils
portaient son nom. Ils pouvaient même (à Rome) être adoptés. Ils venaient
ainsi renforcer leur \emph{dominus} avec plus de docilité et de gratitude
obligée que des fils selon la chair, des gendres ou des beaux-frères.

Mais c'est justement pour cela que l'intérêt d'un dirigeant qui était
parvenu au sommet du pouvoir, du Prince, n'était jamais de voir se multiplier
les esclaves de ses concurrents potentiels. D'autre part l'esclavage «
interne », celui de ses propres concitoyens, celui des individus issus de
son propre peuple, lui retirait des sujets mobilisables et imposables, et
renforçait le pouvoir de ses rivaux potentiels. C'est pourquoi selon Alain
\fsc{Testart} lorsque ceux qui parvenaient au pouvoir suprême étaient avisés
et qu'ils avaient suffisamment de force ou d'autorité :
\begin{enumerate}
% A)
\item leur première
démarche était d'interdire de réduire en esclavage aucun de leurs
propres sujets, quel qu'en soit le motif. Cette interdiction leur valait la reconnaissance
du petit peuple, et obligeait leurs rivaux potentiels à recourir
exclusivement à des étrangers s'ils voulaient des esclaves ;
% B)
\item leur seconde
initiative était de se poser en protecteur de tous les esclaves, s'immisçant
ainsi en tiers au sein de la relation maître -- esclave pour affaiblir
le lien de soumission absolue qui reliait le second au premier. L'objectif
ultime du Prince était de se faire des esclaves des sujets dévoués ;
% C)
\item une
troisième solution était qu'il se réserve le monopole de la possession des
esclaves, qu'il garde donc pour lui et ses vassaux l'exclusivité de ceux que
la guerre lui procurait. En constituant ses propres esclaves comme une
armée à sa dévotion, et en se réservant l'exclusivité de ce type d'outil il se
donnait le moyen de désarmer les citoyens libres (et d'abord ses concurrents
directs) et de faire régner son ordre.
\end{enumerate}

On trouvait donc des esclaves publics dans l'entourage de bien des
dirigeants. Parfois une partie ou la totalité de l'administration était constituée
d'esclaves, jusqu'au sommet y compris. Les républiques antiques
n'ont jamais eu recours à des armées d'esclaves pour lutter contre leurs
ennemis extérieurs, mais seulement à des citoyens libres, égaux (en principe)
et dotés du droit à la parole. Par contre dès l'apogée des cités grecques
des municipalités possédaient des esclaves (publics) qui assumaient
les tâches des employés municipaux d'aujourd'hui. Ils étaient ouvriers
communaux, employés à la bibliothèque, au tribunal, à l'assemblée, au lycée,
ou chargés de la police de la ville,~etc.%
%[1]
\footnote{À Athènes dès le \siecle{5} avant J.-C., ils peuvent loger où ils veulent et vivre en famille avec une concubine. Ces esclaves n'ont pas le souci du lendemain : c'est à la ville, à l'État (leur propriétaire), de s'en charger, sauf à les libérer ou les vendre à quelqu'un d'autre. De vrais fonctionnaires titulaires de leur poste ? Il se peut que cela ait été vécu ainsi, au moins par endroits et par moments. Cela dit le système répressif qui leur était réservé ne leur permettait pas de prendre à la légère leurs attributions. Quinze ou vingt siècles plus tard en Turquie le corps des Janissaires était constitué de soldats esclaves du sultan. Ils avaient été « prélevés » (enlevés) encore enfants, au titre du tribut, sur les populations non musulmanes de l'empire (populations soumises au statut de dhimmi. À partir de ce jour ces enfants n'entretenaient plus de relations avec leur famille. Convertis d'autorité à la religion de leur maître ils n'avaient plus d'autre parent que lui.}

En conclusion le sort concret des esclaves n'était pas homogène,
et les gens de l'antiquité en avaient bien conscience, au point peut-être de
moins percevoir les ressemblances que les différences ? Chaque cas
concret devait être repéré sur les axes suivants : du travail manuel au travail
intellectuel ; du travail d'exécution le moins qualifié aux tâches de
commandement d'équipes de travail, à la gestion d'une entreprise, ou à
l'expertise (artisans, artistes, enseignants, médecins) ; de l'intimité charnelle
la plus étroite avec le maître à la distance la plus impersonnelle. Le
seul point commun à tous était le statut des esclaves. C'est ce qui fait dire
à Paul \fsc{Veyne} que {\emph{l'esclavage est, tantôt un lien juridique archaïque qui s'appliquait
au rapport de domesticité, tantôt esclavage de plantation, comme dans le sud
des États-Unis avant 1865. Dans l'Antiquité, la première forme est de très loin la
plus répandue. L'esclavage de plantation, qui seul concerne les forces et rapports de
production, est une exception propre à l'Italie et à la Sicile de la basse période hellénistique,
de même que l'esclavage de plantation était une exception dans le monde du \siecle{19} :
la règle en matière agraire pour l'Antiquité était... la paysannerie libre
ou le servage. Spartacus, après avoir détruit le système de l'économie de plantation,
aurait évidemment admis, comme toute son époque, l'esclavage domestique.}} (in Paul
\fsc{Veyne}, \emph{Comment on écrit l'histoire}, chapitre VIII, Éditions du Seuil, Paris,
1971).


\section{La fabrique des esclaves}

Les esclaves que consommaient les sociétés antiques venaient de
plusieurs sources. La violence armée en fournissait un nombre qui variait
selon les lieux et les époques, mais il existait aussi d'autres filières auxquelles
on pense moins, imbriquées étroitement dans le fonctionnement
ordinaire des familles antiques, et qui suivant les périodes ont pu être au
moins aussi « productives ».

\subsection{La violence}

Selon les lois de la guerre unanimement acceptées le vaincu appartenait
corps et âme au vainqueur, lui, sa ou ses femmes, ses parents, ses
enfants, ses trésors, ses esclaves et ses terres. Le vainqueur pouvait à son
gré le tuer ou l'épargner. Il pouvait lui faire payer une rançon, exiger de
lui un tribut régulier en signe de soumission, ou encore en faire un esclave.
Il était ordinaire que les combattants soient tués tandis que les jeunes,
les femmes et les esclaves capturés étaient traités comme un butin, distribués
ou vendus. De là à faire la guerre pour se procurer des esclaves il
n'y avait qu'un pas, souvent franchi.

Aristote l'admettait d'ailleurs sans détours : si ceux qui sont faits
pour obéir, c'est-à-dire les barbares, acceptaient de se soumettre sans faire
d'histoire à ceux qui sont faits pour commander, c'est-à-dire les grecs,
il ne serait pas nécessaire de leur faire la guerre, mais comme ils ne sont
pas raisonnables il faut bien aller les chercher les armes à la main, dans
leur propre intérêt (leur intérêt, de son point de vue, étant qu'ils laissent
les plus raisonnables diriger leurs vies).

Les pirates et les bandits de grand chemin savaient eux aussi faire
des esclaves en s'en prenant aux voyageurs solitaires, aux maisons isolées,
aux bateaux de pêche et de commerce, en enlevant les enfants mal
gardés, etc. Dès que les autorités baissaient leur garde ce mode de production
des esclaves reprenait de l'importance, et cela durera jusqu'à la
fermeture du dernier marché aux esclaves.

\subsection{La sanction pénale}

Les prisons antiques n'étaient pas des lieux d'exécution des peines.
Avant jugement elles servaient à mettre les prévenus à la disposition de la
justice et à l'abri des vengeances privées, et après jugement à attendre
l'exécution de la peine. Par contre la grande majorité des maîtres possédaient
une prison privée, l'\emph{ergastule}%
%[2]
\footnote{Cela est si vrai qu'à différentes reprises les autorités romaines ont inspecté tous les ergastules pour délivrer les personnes libres qui y étaient injustement retenues, ou pour en débusquer les citoyens qui se soustrayaient ainsi à l'incorporation dans l'armée.}%
, sans laquelle ils ne pouvaient
conserver leur personnel ni dormir en sécurité. Les esclaves y demeuraient
enfermés, entravés ou non, en dehors de leur temps de travail.
Dans ce cadre la sanction des délits commis par des hommes libres, citoyens
ou métèques, pouvait être leur réduction au statut d'esclave. Un
condamné pouvait être vendu à un marchand d'esclaves, à un organisateur
de combats de gladiateurs, ou condamné aux mines et carrières de
l'État. Une condamnée pouvait être vendue à une maison de prostitution,
au même titre que n'importe quel(le) autre esclave, ou employée au
service du personnel masculin des mines, des carrières, etc. Dans ces bagnes
vivaient enfermés des milliers d'esclaves, dont un bon nombre avait
été condamné par un tribunal. Les condamnés à mort avaient eux aussi le
statut d'esclaves, {\emph{esclaves de la peine}}, durant l'intervalle entre leur
condamnation et leur exécution.

\subsection{Le sur-endettement}

Mais à côté de ces portes d'entrée dans l'esclavage il existait
d'autres mécanismes qui conduisaient à l'esclavage sans faire appel à la
violence armée, ni à une condamnation légale. Le plus efficace était
l'usure : le crédit à la consommation était à très court terme (ex : une semaine),
les taux d'intérêt étaient exorbitants (ex : 20 ou 30 \%), et les intérêts
produisaient eux-mêmes des intérêts. Si le remboursement n'était pas
effectué sans délai, la dette accrue des intérêts accumulés atteignait en
quelques mois des niveaux vertigineux. Cela mettait très vite ceux qui
étaient contraints d'emprunter, paysans pauvres et travailleurs libres
payés à la tâche, les \emph{mercenaires}, dans l'impossibilité de rembourser.

Même si l'entretien des esclaves ne coûtait que la nourriture la
plus grossière et la plus économique, comme ils mangeaient même s'ils
ne travaillaient pas, il y avait avantage à les maintenir occupés tant qu'il y
avait de l'ouvrage, quelle que soit la rémunération de cet ouvrage. De ce
fait les travailleurs libres ne trouvaient d'embauche que lorsqu'il y avait
trop à faire pour les esclaves et c'étaient eux qui absorbaient les irrégularités
du marché du travail. Dès qu'il était possible de s'en passer les mercenaires
ne pouvaient plus entretenir élever leur progéniture. Une fois
épuisées leurs économies ils étaient contraints de mendier pour acheter à
manger, ce qui était hasardeux, ou d'emprunter, sauf à être les clients
d'un patron \emph{(patronus)} d'autant plus généreux qu'ils avaient peu de chances
de pouvoir lui rendre un service à la hauteur de ses secours.

Rome connaissait la \emph{prison pour dettes}. La prison en question était
une prison privée (l'ergastule du créancier). À la condition d'être prêt à
prouver la réalité de sa créance devant un magistrat s'il lui en faisait la
demande le créancier s'assurait lui-même, avec l'aide de ses clients et de
ses esclaves, de la personne physique du débiteur. Celui qui ne pouvait
pas rembourser était remis en tant que \emph{nexus} (mot dont vient {\emph{annexion}})
à son créancier, qui pouvait faire de lui ce qu'il voulait, le faire travailler à
son profit, en abuser de différentes façons, le vendre (à Rome) comme
esclave au delà du Tibre, ou le tuer (aux temps anciens).

Les prêteurs exigeaient des gages. Un jour venait où l'emprunteur
n'avait plus d'autre gage à donner que l'un de ses dépendants ou lui-même.
Celui qui était \emph{gagé} vivait chez le prêteur, incarcéré avec les esclaves
de la maison, et aussi rigoureusement qu'eux. Il travaillait comme eux
et avec eux. Si l'emprunt était remboursé, il récupérait de plein droit sa
liberté. Dans le cas contraire il était inévitable que vienne un jour où il
serait abandonné au prêteur pour éteindre la dette (\emph{nexus}). Si l'on admettait
que son travail remboursait jour après jour une fraction de la dette,
alors échapper à la servitude demeurait encore imaginable mais si son
travail n'était considéré que comme la contrepartie de sa nourriture, alors
le capital emprunté devait être remboursé par un tiers. Les usages et les
lois ont varié sur ce point.

\subsection{La naissance}

Sauf initiative de leur maître les enfants des esclaves avaient le statut
de leur mère. Ceci dit elles étaient le plus souvent achetées pour travailler
durement et non pour faire des enfants. Les grossesses et les allaitements
les épuisaient d'autant plus qu'on refusait de tenir compte de
leur état. Les accouchements n'étaient pas sans danger, surtout pour celles
qui avaient été mal nourries et pour celles qui étaient mal conformées,
ce qui pouvait être le cas de celles d'entre elles qui avaient subi une enfance
de misère. D'autre part l'élevage des nouveaux-nés était risqué :
beaucoup mouraient bien avant d'avoir rendu le moindre service, et une
fois grands tous n'étaient pas employables avec profit, surtout si on avait
négligé leur formation physique, intellectuelle et professionnelle. Au lieu
de cela les esclaves vendus au marché étaient inspectés par un médecin,
garantis par le vendeur, et immédiatement rentables. Il était donc économiquement
préférable d'acquérir de grands enfants déjà élevés par
d'autres plutôt que de les produire soi-même. Les maîtres avaient donc
en général intérêt à empêcher les esclaves d'avoir des relations sexuelles :
les quartiers des femmes étaient séparés de ceux des hommes et soigneusement
verrouillés pour qu'ils ne puissent pas se rencontrer. Quant aux
prostituées, les maîtres les faisaient avorter lorsqu'elles étaient enceintes.

Si un maître donnait à l'un de ses esclaves l'exclusivité sexuelle de
l'une de ses esclaves et s'il lui permettait d'élever les enfants qui naissaient
de leurs rapports, c'est qu'il voulait le récompenser et le motiver.
C'est que c'était un esclave de confiance. Pourtant ce dernier ne possédait
aucun des enfants qu'il avait engendrés. S'il était affranchi il lui faudrait
encore les racheter à son maître.

\subsection{L'abandon}

L'abandon d'un enfant à sa naissance lui confère plusieurs des
traits de l'esclave : il n'est enregistré nulle part au nombre des citoyens ; il
n'a ni parents ni famille ; il est totalement dépendant de celui qui veut
bien le prendre en charge, qui n'a de comptes à rendre à personne s'il le
maltraite, et qui a le droit d'en faire son esclave. Son corps n'est pas protégé
par la loi : les auteurs de l'antiquité tardive tiennent pour assuré que
tous les garçons et filles bien conformés abandonnés à la naissance trouvent
preneur et sont élevés pour être prostitués dès qu'ils trouveront des
amateurs, bien avant leur puberté. Dans le même ordre d'idées rien dans
les lois de la République de Rome n'interdit aux mendiants professionnels
de mutiler un enfant abandonné, non citoyen, pour exciter la
pitié des passants : c'est mal vu mais la loi ne s'en mêle pas.

\subsection{La vente par un parent}

\label{vente-parent}

En période de chômage ou de disette, les plus pauvres sont acculés
à vendre leurs enfants ou à mourir de faim avec eux. Certes la vente suspend
(mais ne fait pas disparaître) leurs droits parentaux, mais son produit
procure au parent de quoi manger pendant un certain temps. Quant
aux enfants ainsi vendus, leurs maîtres ont l'obligation de les nourrir, ou
de les revendre, ou de leur rendre leur liberté (ce qui les rend à l'autorité
de leur parent, s'il ne les a pas perdus de vue entre temps). Il semble vraisemblable
que sous la pression de la nécessité, les parents soient souvent
contraints de vendre leurs enfants bien en dessous du montant qu'ils ont
déjà investi dans leur éducation.

Ceci étant dit les ventes d'enfants par leurs parents n'ont pas forcément
une signification unique. Il existe en effet des ventes \emph{temporaires},
des ventes du seul travail de l'enfant, et non de son corps, pour une durée
déterminée, parfois très longue, 15 ans, et même 25 ans. Dans ces cas
la vente a vraisemblablement le sens d'un contrat de travail archaïque, ou
d'un louage de service, ou d'un apprentissage ?

C'étaient évidemment les enfants mal investis, mal protégés, mal
tenus, mal contenus, par des parents pauvres, malades ou psychologiquement
défaillants, qui étaient particulièrement prédestinés par leur histoire
au statut de \emph{mineur à vie sans famille} qu'est le statut d'esclave. C'étaient les
petites filles qui couraient le plus de risque d'être vendues (celles qui
n'avaient pas été abandonnées dès leur naissance). C'étaient celles dont
les pères et mères se séparaient les premières.

\fsc{Testart} soutient%
%[3] 
\footnote{Alain \fsc{Testart} : 2002, p. 176 à 193.} 
que ce sont les mêmes sociétés et les mêmes
groupes sociaux qui acceptent qu'on puisse être réduit en esclavage
pour dettes et que le père de la fiancée s'enrichisse en mariant sa fille.
Quand le \emph{prix de la fiancée}, (versé un peu partout, sous des formes variables,
par le fiancé à la famille de celle-ci) n'est pas contrebalancé par une
dot (versée au couple par le père de la fiancée) d'un montant équivalent
ou supérieur, le père de la fiancée a financièrement intérêt à marier sa fille :
{\emph{pour admettre que l'on puisse vendre sa fille en esclavage, il faut d'abord admettre
que l'on puisse la vendre en quelque sorte à son mari. Pour admettre que l'on puisse
vendre son fils, il faut d'abord que l'on admette que l'on puisse vendre sa fille. Pour
admettre qu'un père puisse se vendre en esclavage, il faut déjà admettre qu'il puisse
vendre ses enfants. Et pour admettre que l'on puisse réduire en esclavage un membre
libre de sa communauté et l'exploiter comme esclave, il faut d'abord qu'on admette que
l'on puisse le faire pour le plus démuni et le plus fragile d'entre eux}%
%[4]
\footnote{Alain \fsc{Testart} : 2002, p. 193.}%
.}

{[...] \emph{cette condition -- tenir pour légitime pour un père de tirer profit de ses filles --
nous apparaît comme une condition nécessaire (mais nullement suffisante) pour que
l'on tienne plus généralement pour légitime de réduire en esclavage un des membres de
la société pour des raisons uniquement financières (esclavage pour dettes ou vente)}%
%[5]
\footnote{Alain \fsc{Testart} : 2002, p. 176 à 193.}%
.}

\subsection{La vente par soi-même}

Une fois vendus leurs enfants il ne reste aux indigents qu'à se vendre
eux-mêmes. S'il est interdit à Rome comme à Athènes d'asservir un
citoyen, sauf condamnation pénale, sa mise en vente est juridiquement
inattaquable s'il a un âge suffisant pour être considéré comme responsable
de ses actes (vingt ans) et s'il a encaissé au moins une partie du
produit de la vente, ce qui prouve qu'il a été complice de son propre asservissement.
Celui qui se vend désavoue sa propre famille, dont il reconnaît
publiquement l'incapacité à lui apporter du secours. Il coupe
tous ses liens juridiques avec ses parents, son épouse, ses enfants, etc. Il
démissionne de sa liberté et de ses responsabilités de citoyen. Il ne peut
plus hériter (en général il n'avait guère d'espoir de ce côté-là). S'il meurt
sous les mauvais traitements, ou d'un accident du travail (travaux publics,
mines, gladiateurs, etc.) sa famille ne peut demander de compensation. Il
ne peut plus se prévaloir de son état de liberté antérieur. Même s'il est un
jour affranchi il gardera à vie la marque servile et ne retrouvera jamais la totalité
de ses droits antérieurs.
Des jeunes gens bien formés et ambitieux mais désargentés sont
amenés à se vendre à un employeur%
%[6] 
\footnote{Cf. P. \fsc{Veyne}, \emph{La société romaine}.}
pour tenir des emplois de confiance.

Quant à ceux qui ne sont pas citoyens, ils font ce qu'ils veulent.
Personne ne se formalise qu'ils soient asservis, de leur plein gré ou non,
et ils peuvent vendre tous leurs enfants si cela est leur intérêt. Dans le
monde régenté par Rome, à cette époque en expansion continue, les
peuples soumis constituent un gisement d'esclaves achetés de manière
tout à fait légale à leurs parents.


\section{Qui peut-on asservir légitimement ?}

On a vu ci-dessus que durant l'antiquité chacun court le risque
d'être asservi par la violence dès qu'il s'éloigne de sa cité d'origine. Celle-ci
peut être de petite ou de grande taille : la cité (et la citoyenneté) romaine
n'a cessé de grandir jusqu'à devenir universelle en 212, mais longtemps
cela n'a pas été le cas. Les guerres entre cités étaient très fréquentes
et jusqu'à l'avènement de la « paix romaine » il fallait toujours être prêt
à défendre sa liberté les armes à la main.

Il n'y avait que dans sa propre cité qu'un citoyen recevait une protection
légale contre le risque d'être asservi, à la condition toutefois de ne
pas tomber sous le coup de ses lois pour crime ou délit : s'il n'était pas
condamné à mort il pouvait en effet être condamné à être vendu comme
esclave.

Les cités antiques ont longtemps accepté la réduction en esclavage
pour dettes de leurs concitoyens libres : Athènes jusqu'à \hbox{Solon} (vers
594-593 avant J.-C.), Rome jusqu'à \hbox{Appius} \hbox{Claudius} \hbox{Caecus} (entre 326 et 312
avant J.-C.). Au fil du temps elles ont eu tendance a interdire de garantir
les créances sur la personne des débiteurs et de vendre un concitoyen libre, mais ces lois n'ont pas pu empêcher les pauvres de se vendre eux-mêmes,
au contraire, puisqu'elles les empêchaient de trouver des créanciers prêts
à leur faire crédit. 

Malgré tout l'opinion commune n'en était pas moins que celui
qui est \emph{naturellement} destiné à l'esclavage c'est l'étranger, celui qui est différent,
celui qui ne parle pas comme les siens, autrement dit le barbare, selon
le point de vue exprimé par Aristote.
