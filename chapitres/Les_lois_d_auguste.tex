%C1 LES LOIS D'AUGUSTE
 Au moment où s'installait l'Empire de Rome, durant le dernier siècle avant notre ère%
%[1]
\footnote{Sources : \fsc{CARRIE} Jean-Michel, \fsc{ROUSSELLE} Aline, \emph{L'Empire romain en mutation, des Sévères à Constantin}, 192-337, 1999. \fsc{DELACAMPAGNE} Christian, \emph{Une histoire de l'esclavage, de l'antiquité à nos jours}, 2002. Collectif, \emph{Religions de l'Antiquité}, 1999, 592 p. \fsc{LE ROUX} Patrick, \emph{Le Haut-Empire romain en Occident, d'Auguste aux Sévères}, 1998. \fsc{PETIT} Paul, \emph{Histoire générale de l'Empire romain, 3, le Bas-Empire} (284-395), 1974. \fsc{PUCCINI-DELBEY} Géraldine, \emph{La vie sexuelle à Rome}, 2007. \fsc{ROUSSELLE} Aline, \emph{La contamination spirituelle, science, droit et religion dans l'antiquité}, 1998. \fsc{SARTRE} Maurice, \emph{Le Haut-Empire romain, Les provinces de Méditerranée orientale d'Auguste aux Sévères}, 1991. \fsc{VEYNE} Paul, \emph{La société romaine}, 1991 ; \emph{Le pain et le cirque, sociologie historique d'un pluralisme politique}, 1976.}
les citoyens romains avaient tendance à délaisser le mariage avec les citoyennes et à lui préférer d'autres manières de se procurer des héritiers. Cet état de fait était favorisé par bien des facteurs dont les flots d'esclaves que les conquêtes militaires déversaient alors. 

 Même si le mariage romain reposait comme le nôtre (son héritier direct) sur le consentement des époux, et même si les maris conservaient une autorité sans partage sur leurs enfants en cas de divorce, à cette période ils n'avaient plus qu'une autorité limitée sur leurs épouses. Celles-ci restaient sous l'autorité de leur père ou de leur tuteur, même quand elles étaient mariées (mariage « \emph{sine manu} », sans \emph{la main}, l'autorité). Et elles avaient le droit de divorcer de leur propre initiative même si dans ce cas elles devaient abandonner une part ou la totalité de leur dot à leur mari --- comme dédommagement ? Comme contribution aux frais d'élevage de leurs enfants communs ?

 Par contre les esclaves n'avaient juridiquement plus de parents, plus de lien juridique avec une famille, donc aucun recours face à leur maître, qui disposait à leur encontre d'un puissant arsenal de punitions et de récompenses. Les esclaves concubines ne pouvaient donc se permettre de faire mauvaise figure à leur maître, ni de le quitter contre son gré. S'il se lassait d'elles il n'avait aucune dot à rendre à un beau-père mécontent : il les vendait au marché ou les donnait à un de ses esclaves ou de ses clients. S'il les affranchissait il était de droit leur patron et leur tuteur jusqu'à sa mort. Elles ne pouvaient donc jamais fuir son autorité. Si elles passaient d'un amant à un autre, il avait tous les moyens de les sanctionner.

 Mais il est possible aussi que le nombre de citoyennes nubiles soit devenu insuffisant pour que chaque fils de famille puisse trouver une épouse parmi les citoyennes de son rang. Les citoyens abandonnaient-ils un nombre de fillettes excessif ? C'est qu'une fille légitime devait être dotée si son père voulait la marier dans les règles, et il perdait la face s'il s'y refusait. Ne pas doter une fille selon son rang eut été avouer qu'on n'en avait pas les moyens. On pouvait aussi se demander si elle présentait un défaut caché, si planait sur elle une suspicion d'illégitimité, ou encore si son père ne l'aimait pas de manière excessive, pour lui-même (les romains n'avaient aucune complaisance envers les désirs incestueux, dont la réalisation leur paraissait un sacrilège pouvant entraîner la colère des dieux contre tous les humains). Il est donc tout à fait imaginable que beaucoup de citoyens aient refusé de s'encombrer de filles s'ils prévoyaient qu'ils ne pourraient pas les marier de manière profitable pour leurs alliances politiques et économiques%
% [2]
\footnote{En Asie les mêmes causes produisent \emph{aujourd'hui} encore les mêmes effets. En Inde la nécessité de doter les filles, en Chine le fait qu'elles sortent de leur famille en se mariant et ne peuvent donc plus prendre en charge leurs parents âgés (la plupart ne jouissant pas d'une retraite), conduisent avec l'aide des échographies (mais l'infanticide à la naissance aboutit au même résultat) à un sex-ratio très déséquilibré, avec un fort excès de garçons, et à des trafics de femmes jeunes et pauvres.}%
. 

 Or, à Rome comme dans les cités grecques, seuls les citoyens de naissance \emph{légitime} pouvaient exercer des magistratures et seul les citoyens \emph{ingénus} (nés libres) pouvaient porter les armes. Les autres, les affranchis, ceux qui avaient commencé leur vie comme esclaves, et a fortiori ces derniers, étaient considérés comme sans courage, prêts à choisir de vivre en esclave plutôt qu'à mourir pour la cité. 

 L'empereur était d'abord un chef de guerre : du temps de la république romaine le titre d'\emph{Imperator} désignait le « \emph{commandant en chef vainqueur devant l'ennemi} ». C'est par ses succès militaires que César avait conquis le pouvoir. Auguste, son successeur, était chef des armées. Sa stratégie de conquête du pouvoir exigeait des victoires militaires, et d'abord contre ses rivaux. Accepter une armée faible était consentir à sa propre défaite à plus ou moins proche échéance. Il ne pouvait donc se satisfaire d'une armée d'affranchis. Il ne pouvait pas non plus envisager de confier son armée à des magistrats (officiers) dont on ne connaissait pas le père. Il lui fallait un maximum de citoyens ingénus et beaucoup de citoyens de naissance légitime. 

 Un autre facteur a pu avoir une importance significative dans les décisions d'Auguste : chez les aristocrates qu'il avait dépouillés d'une grande part de leur pouvoir antérieur, la morale stoïcienne occupait à cette époque une position dominante. Chez les personnes de qualité la maîtrise des comportements était devenue une fin en soi, ce qui avait entraîné des effets jusque dans la morale conjugale : en effet selon les stoïciens le mariage est le lieu unique de la sexualité autorisée et vertueuse (pure). Tous les comportements sexuels qui n'y trouvent pas place sont moralement condamnables. La chasteté pré conjugale des garçons est aussi souhaitable que celle des filles, tout comme le maintien de leur virginité jusqu'au mariage. Pour Sénèque, contemporain de Néron : « \emph{l'injustice la plus grave envers une épouse est d'avoir une maîtresse}%
% [3]
\footnote{\emph{Lettres à Lucilius}, XV, 94, 26, et 95, 37, \fsc{SENEQUE}}
». Pour lui \emph{« c'est être adultère envers sa propre femme que de l'aimer d'un amour trop ardent »}. Quant à Épictète il accusait l'adultère d'être contraire à la nature, car les êtres humains sont nés pour être fidèles%
% [4]
\footnote{\fsc{Épictète}, 2, 4 ; 3,7 ; 16, 21.}%
. 

 Pour Musonius Rufus, moraliste et philosophe du premier siècle : « \emph{il faut que ceux qui ne sont ni voluptueux ni mauvais considèrent que les plaisirs amoureux sont justes dans le seul mariage et pour procréer, parce que licites, tandis que ceux qui assouvissent le seul plaisir sont injustes et criminels, même dans le mariage}%
% [5]
\footnote{Musonius Rufus, VII}
». Pour lui encore : « \emph{s'il semble} [à quelqu'un qu'il n'est] \emph{ni honteux ni déplacé qu'un maître ait des relations avec sa propre esclave, surtout s'il se trouve qu'elle n'est pas mariée, que celui-ci considère comment il apprécierait que son épouse ait des relations avec un esclave mâle. Cela ne paraîtrait-il pas totalement intolérable} [...] ? [...] \emph{Quel besoin y a-t-il ici de dire que c'est un acte de dévergondage et rien d'autre pour un maître d'avoir des relations avec un esclave ? Tout le monde le sait}%
%[6]
\footnote{Musonius Rufus, VII}
». Comme on peut s'y attendre compte tenu de ces prémisses, il pensait que les relations sexuelles entre hommes sont « \emph{une chose monstrueuse et contraire à la nature} ». Jusqu'à quel point ces idées s'incarnaient-elles dans les comportements ? Même les stoïciens les plus dévots pouvaient à l'occasion s'écarter de leurs propres règles. Cela ne veut pas dire qu'elles n'avaient pas d'importance. 

\tempnote{{ }~{ } \\Qui cite qui dans ce paragraphe ?}%
 Selon Géraldine \fsc{PUCCINI-DELBEY} (\emph{La vie sexuelle à Rome}, 2007, p. 69-70) : « l'homme romain doit se montrer officiellement un bon époux et respecter son épouse ... ». D'une morale civique les romains passent ainsi à une « morale du couple », changement qui se produit en un siècle ou deux. Paul \fsc{VEYNE} résume cette mutation de la manière suivante : la première morale disait : « se marier est un des devoirs du citoyen » ; la seconde : « si l'on veut être un homme de bien, il ne faut faire l'amour que pour avoir des enfants ; l'état de mariage ne sert pas à des plaisirs vénériens ». La seconde morale postule une amitié, une affection durable entre deux personnes de bien qui ne font l'amour que pour perpétuer l'espèce. C'est, aux yeux de l'historien, la naissance du mythe du couple, du « mythe de l'amour conjugal ». La femme, jusque là nécessaire pour faire des enfants et arrondir le patrimoine, devient une amie, « la compagne de toute une vie ». Son époux doit la respecter comme une amie qui lui est inférieure. Paul \fsc{VEYNE} résume (\emph{Histoire de la vie privée}, Tome I, p. 47) cette évolution en affirmant que les romains sont passés d'une « bisexualité de viol » à une « hétérosexualité de reproduction », une morale de la conjugalité que le christianisme reprendra par la suite à son compte.

 Si l'on en croit Emmanuel \fsc{TODD} il s'agissait du déclin du patriarcat romain et de la résurgence de la famille nucléaire originelle, souterrainement agissante dans les représentations romaines malgré le poids du modèle patriarcal incarné dans le droit romain, en particulier dans les couches de la populations qui détenaient peu de pouvoir et de biens à transmettre. 

%PROMOTION DES NAISSANCES INGENUES
\section{Promotion des naissances ingénues}

 Un maître romain pouvait affranchir les enfants qu'une esclave lui avait donnés. Ils ne seraient jamais ses enfants légitimes, puisqu'ils avaient été esclaves, mais ils devenaient des citoyens. Ils étaient ses affranchis, et il était leur patron, conservant de ce fait une forte autorité sur eux. Il pouvait enfin les désigner comme héritiers. S'il reconnaissait officiellement dès leur naissance les enfants de ses esclaves concubines, ils étaient considérés comme des ingénus qui n'avaient jamais été esclaves, et comme ses enfants, même s'ils étaient illégitimes. 

 Auguste voulait augmenter le nombre des jeunes ingénus aux dépens de celui des jeunes affranchis, aussi a-t-il décidé%
% [7]
\footnote{\emph{Leges Juliae de adulteriis coercendis} et \emph{de maritandis ordinibus} de 18 av. J.-C., et \emph{lex Papia Poppaea} de 9 ap. J.-C. L'empereur était l'une des sources principales du droit. Il était donc de coutume d'écrire et de dire que « \emph{Auguste a décidé que...} » , même quand il se bornait à entériner une proposition que lui avaient faite les juristes qui travaillaient dans ses bureaux (parmi les meilleurs de son époque).}
que :
\begin{enumerate}
%A)
\item seuls pourraient hériter de leurs parents éloignés ou de personnes non apparentées les citoyens (mariés ou non) qui avaient donné le jour à au moins un enfant ingénu, les citoyennes (mariées ou non) qui avaient donné le jour à trois enfants ingénus, et les affranchies qui, après leur affranchissement, avaient donné le jour à quatre ingénus, enfants déclarés, reconnus, et non abandonnés (légitimes ou non, vivants ou morts, nés viables ou non, garçons ou filles). On peut penser que cette loi a donné de fortes chances d'être élevés aux trois premiers enfants de chaque citoyen, même aux filles ;
% B)
\item les citoyennes qui avaient donné le jour à trois enfants ingénus (quatre pour les affranchies), et qui n'avaient pas de \emph{pater familias} (père ou patron), seraient dispensées de tutelle. À partir de leurs 25 ans (l'âge de la majorité légale des hommes \emph{sui juris}), elles pouvaient gérer elles-mêmes leurs affaires et se marier ou se remarier à leur gré : elles aussi pouvaient donc devenir juridiquement majeures. 
\end{enumerate}

%REPRESSION DE L'ADULTERE FEMININ
\section{Répression de l'adultère féminin}

 Sans la fidélité des épouses la paternité des maris est incertaine. Les adultères c'est-à-dire les relations \emph{passagères} entre une femme mariée et un autre homme que son mari dérangeaient le bon ordre une société construite sur la base des lignées et de leurs alliances (ce qui en soi n'implique pas le patriarcat). Une femme mariée qui avait des relations sexuelles avec son propre esclave était aussi adultère que si son amant avait été un homme libre. Dans un ordre d'idée assez proche, si une femme libre (célibataire ou mariée, citoyenne ou non) prenait pour amant un esclave qui ne lui appartenait pas, elle était elle aussi coupable d'\emph{adulterium} alors qu'aucun mari n'était en jeu. Si malgré les avertissements du maître de l'esclave la coupable persistait dans ses relations elle devenait elle-même l'esclave du maître de son amant%
% [8]
\footnote{\emph{Sénatus-consulte} de Claude, 52 ap J.-C. … comme si le fait qu'une femme aille vers un homme pour lui demander un rapport sexuel signifiait qu'elle se met sous son autorité.}%
.

 Par contre le fait qu'une épouse quitte son mari pour aller vivre \emph{durablement} avec \emph{un seul} autre homme libre ne tombait pas sous le coup de la loi. C'était un droit que possédait toute citoyenne romaine, mais jusqu'à Auguste la dot de la femme qui agissait ainsi restait dans les mains de celui qu'avait choisi son père, ce qui pouvait l'empêcher de trouver un nouveau parti avantageux, et donc de concevoir de nouveaux enfants légitimes. C'est pourquoi Auguste a dans le même mouvement facilité le divorce et sanctionné lourdement les adultères. Il a décidé que les citoyennes (les actes des autres femmes n'importaient pas puisqu'ils n'avaient pas d'effet sur l'ordre social) qui quitteraient leur mari pour une relation durable (un nouveau mariage ou un concubinage stable) garderaient le bénéfice de leur dot. 

 Une femme convaincue d'adultère devenait infâme. Elle perdait le droit au manteau des matrones, ce qui rendait public son statut. Une infâme n'était pas astreinte au devoir de fidélité et ses relations sexuelles ne regardaient plus qu'elle, mais elle ne pouvait pas s'appuyer sur la loi pour protéger son intégrité corporelle contre les agressions sexuelles. Celui qui la violait n'attentait plus l'honneur d'aucun homme. Son mariage était automatiquement rompu et elle n'avait pas le droit d'en contracter un autre. La moitié de sa dot et le tiers de ses biens passaient au fisc impérial, et elle était exilée, assignée à résidence dans une île%
% [9]
\footnote{… par exemple en Corse.}%
. Son complice masculin était également frappé d'infamie et puni de la même façon : confiscation de la moitié de ses biens et relégation dans une autre île. Une femme accusée d'adultère pouvait éteindre les poursuites engagées contre elle en prenant l'initiative de se faire inscrire sur la liste des prostituées : elle n'en était que plus sûrement frappée d'infamie. 

 Tout mari trompé devait sans tarder envoyer à son épouse une lettre de répudiation, sans quoi il était considéré comme proxénète et devenait infâme, et ses enfants aussi. Cela s'accompagnait de la confiscation de tout ou partie de leur fortune. L'obligation de dénonciation s'appliquait aussi au père et aux frères de l'épouse : tous ceux qui avaient connaissance d'un adultère se devaient de le dénoncer sous peine d'être eux aussi condamnés à l'infamie.

 Un mari trompé n'avait pas le droit de tuer son épouse, puisqu'il n'avait pas autorité complète sur elle (elle dépendait de son propre \emph{pater familias}). Par contre il pouvait tuer l'amant en toute impunité si celui-ci était un infâme, ou un esclave, ou un affranchi attaché à sa famille (son affranchi ou celui de son épouse, ou celui de leurs pères ou mères respectifs, ou de l'un ou l'autre de leurs enfants). Dans tous les autres cas de figure, ou bien s'il tuait son épouse, il commettait certes un homicide, mais la loi lui accordait de larges circonstances atténuantes. 

 Contrairement au mari et à la condition qu'il soit aussi son \emph{pater familias}, le père de l'épouse infidèle avait le droit de tuer sa fille, à la condition de tuer son amant aussi. Sinon il pouvait être accusé de meurtre.

 Jusqu'à quel point cette législation a-t-elle été appliquée ? Dans quelle mesure a-t-elle seulement eu un rôle dissuasif à l'encontre de la prostitution clandestine ? On pouvait compter sur le fisc pour tout mettre en œuvre pour faire rentrer de l'argent dans les caisses de l'état, et celui qui venait des adultères, quoique impur, était aussi bon à prendre que celui des taxes sur les vespasiennes ou sur les affranchissements d'esclaves, aussi les dénonciations étaient-elles encouragées financièrement. Elles avaient d'autant plus d'intérêt pour les finances de l'empereur et celles des dénonciateurs qu'elles concernaient des personnes plus riches. Mais le droit romain prévoyait que les accusateurs qui ne parvenaient pas à prouver leurs accusations étaient très sévèrement punis, ce qui pouvait mettre un frein aux accusations gratuites lorsque les juges n'étaient pas corruptibles . 

%PROMOTION DU MARIAGE
\section{Promotion du mariage}

 Puisque le concubinage pouvait accroître le nombre des citoyens ingénus, Auguste n'a rien fait pour l'interdire aux hommes mariés. Par contre pour augmenter le nombre des enfants légitimes il a promu le mariage. Il a décidé que : 
\begin{enumerate}
% A)
\item tous les citoyens qui n'étaient pas mariés entre 25 et 60 ans et toutes les citoyennes qui n'étaient pas mariées entre 20 et 50 ans devaient payer un impôt spécial. Au-delà de ces âges le remariage n'était pas encouragé%
%[10]
\footnote{\emph{La vie sexuelle à Rome}, Géraldine \fsc{PUCCINI-DELBEY}, 2007, p. 48.}
: il n'était plus convenable de convoler alors que la conception des enfants était hors d'atteinte pour les femmes, et que les hommes étaient trop âgés pour assumer l'éducation de leurs fils. Cela devenait un motif de scandale ;
% B)
\item les citoyennes qui prendraient l'initiative de divorcer pourraient garder l'intégralité de leur dot, de telle façon qu'elles retrouvent un mari sans difficulté et continuent d'enfanter des citoyens légitimes ;
% C)
\item pour parer à l'éventualité qu'il n'y ait pas suffisamment de citoyennes nubiles pour tous les citoyens célibataires, Auguste a promu une solution de remplacement : il reconnaissait comme valide l'union des citoyens avec une de leurs propres esclaves, à la condition qu'ils l'aient affranchie pour l'épouser avant l'âge de douze ans, âge minimum légal du mariage pour les filles (même si l'âge moyen au mariage des femmes était beaucoup plus tardif). Affranchies pour être épousées ces fillettes devenaient aussitôt des citoyennes, ce qui était un grand privilège. De cette façon leurs enfants à venir naîtraient de deux citoyens libres et mariés. Ils seraient de plein droit citoyens, ingénus, et de naissance légitime. 
\end{enumerate}

 Comme toutes les autres femmes libres ces affranchies avaient le droit de divorcer le plus simplement du monde, en quittant leur mari. Mais Auguste pénalisait l'exercice de ce droit : si elles quittaient leur époux \emph{contre son gré} elles perdaient le \emph{connubium}, le droit d'épouser un (autre) citoyen romain%
% [11]
\footnote{... parce que ce faisant elles se soustrayaient à l'autorité de leur tuteur. Le maître d'une esclave devenait en effet de droit son tuteur en même temps que son patron quand il l'affranchissait.}%
, ce qui implique que leurs enfants à venir ne seraient pas des citoyens. Elles étaient donc incitées à s'acquitter de leur tâche d'épouse conformément aux désirs de leur « maître et seigneur ».

 Le concubinage monogame stable entre deux citoyens produisait à peu près les mêmes effets qu'un mariage. Pour contracter mariage il n'y avait pas de cérémonie officielle non plus. La différence la plus significative entre le concubinage et le mariage était l'absence de dot, même si chez les romains il existait des mariages en bonne et due forme sans dot (à la différence des grecs). Pendant un bon millénaire encore la seule preuve d'une union, en dehors de la parole des témoins, sera le contrat notarié qui enregistrait la dot%
% [12]
\footnote{On peut en rapprocher la \emph{Kétouba}, contrat de mariage écrit sans lequel un mariage juif n'est pas valide.}%
. Le concubinage était selon \fsc{VEYNE} le mariage des gens de peu, des pauvres, des gens sans importance, de ceux qui n'appartenaient pas à une famille aristocratique et qui n'avaient aucun espoir d'arriver un jour à des positions en vue. Les enfants (reconnus par leur père) qui en naissaient n'étaient pas considérés comme illégitimes tant que leurs parents n'étaient pas frappés d'un interdit de mariage (inceste, infamie, etc.). 

 En ce qui concerne les interdits de mariage cette période innove sur un point : sous le règne de l'Empereur Claude (41--54 de notre ère) un sénatus-consulte (une réponse du Sénat, consulté sur un point de droit) autorise le mariage d'une nièce et de son oncle paternel. Cette exception aux interdits de mariage était un cadeau fait à l'empereur, désireux d'épouser sa nièce. Elle semble n'avoir jamais complètement perdu son aura de scandale. La tendance ensuite s'inversera et on constate un élargissement du cercle des parents prohibés: à l'époque des juristes classiques (de la fin du \siecle{2} au début du \siecle{3}). Le droit positif étend les interdictions matrimoniales aux \emph{adfines} (parents par alliance), et au milieu du \siecle{4} la \emph{fratris filia} puis la \emph{consobrina} sont à nouveau interdites, ainsi que les \emph{adfines} de même génération.

 Une preuve du fait qu'Auguste est sans doute parvenu à peu près aux buts qu'il s'était fixés (... mais peut-être aussi à d'autres objectifs non recherchés initialement, mais appréciés tout de même ?) est que ses lois sur le mariage vont être appliquées pendant trois siècles et demi. 

 Cela dit pour obtenir les enfants réglementaires les romains et les romaines ne s'y prenaient pas toujours comme on s'y attend lorsqu'on a l'imaginaire façonné par les siècles d'indissolubilité du mariage ultérieurs. La fréquence des divorces et des remariages était grande, et selon \fsc{VEYNE} (2001) c'était souvent à plusieurs épouses que les hommes demandaient les enfants dont ils avaient un impérieux besoin. Des maris divorçaient pour prêter celles qui avaient prouvé leur fécondité à leurs amis, qui les épousaient à leur tour pour légitimer les enfants attendus. Les premiers pouvaient ré-épouser à nouveau les femmes ainsi prêtées une fois accomplies leur service génésique. Lorsque la fécondité du futur époux était incertaine ils pouvaient même les prêter déjà enceintes. 

%LIMITATION DES AFFRANCHISSEMENTS 
\section{Limitation des affranchissements}

 Pour que le corps social ne soit pas trop vite envahi par les affranchis (comme par autant d'immigrés mal assimilés et provoquant chez les « vieux romains » le sentiment désagréable de n'être plus chez eux à Rome), Auguste a restreint le droit des maîtres d'affranchir leurs esclaves%
% [13]
\footnote{... et rendu plus difficiles les fraudes à la citoyenneté romaine (statut qui était avantageux) ? Pour faire à coup sûr de son fils mineur un citoyen romain un pérégrin pouvait le vendre à un citoyen romain non infâme (par exemple à un associé en affaires). Après quoi celui-ci n'avait plus qu'à l'affranchir. Avant cette réglementation, cela pouvait aller vite. Par contre s'il fallait vivre sous le statut d'esclave jusqu'à l'âge de trente ans c'était nettement moins intéressant.}%
. Il a décidé que ceux-ci ne pourraient obtenir la citoyenneté romaine que s'ils étaient affranchis après leurs 30 ans, à la condition que leur maître ait plus de 20 ans, et qu'il ne les affranchisse pas en masse. 

 Quand l'une ou l'autre de ces conditions n'était pas remplie leur affranchissement leur demeurait acquis, mais ils n'obtenaient que le droit \emph{latin}, nettement moins favorable que la citoyenneté romaine. L'objectif de la mesure était de limiter l'acquisition de la nationalité romaine à ceux qui avaient eu suffisamment de temps pour s'assimiler. Il est vrai que pour devenir un citoyen romain il suffisait aux hommes de statut latin de faire un enfant à une citoyenne.
 
 