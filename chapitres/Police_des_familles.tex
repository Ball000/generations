
\section{Organisation d'une police des pauvres}

 À la fin du Moyen-Âge il était courant que les mendiants représentent 10~\% de la population%
% [1]
\footnote{José \fsc{CUBERO}, \emph{Histoire du vagabondage}, 1998, p. 8}% 
. Les solutions en vigueur depuis la fin de l'antiquité pour traiter l'indigence et les malheurs individuels, pensées pour de petites communautés rurales où tous se connaissent%
%[2]
\footnote{José \fsc{CUBERO}, 1998, p. 42 et suivantes.}% 
, n'étaient plus à l'échelle des problèmes en un temps où les villes débordaient de leurs murailles anciennes, et où les États modernes se constituaient, imposant plus d'ordre, de rigueur et de contrôles, et rognant peu à peu les larges marges jusque là consenties entre les principes et les pratiques réelles. 

 Face à la pauvreté dès 1350 apparaissent les signes avant-coureurs d'un changement des mentalités et des pratiques. On commence à parler de « {bons} » et de « {mauvais pauvres} ». Les vagabonds ne sont plus assimilés aux pèlerins mais sont de plus en plus considérés comme des fauteurs de trouble. Les « {bons pauvres} » ou « {pauvres honteux} » ont le droit moral de mendier parce qu'il leur est impossible de travailler, et parce qu'ils restent rattachés à leur cadre villageois, à leur paroisse d'origine : enfants, infirmes, malades, vieillards... Ils ne se soustraient pas au contrôle de leur communauté. Les \emph{mauvais pauvres} sont ceux qui ont force et santé mais qui fuient le travail par paresse ou par goût de l'errance%
% [3] 
\footnote{... ou par refus de conditions de travail par trop inacceptables (mais cela c'est notre point de vue du \siecle{21}, ce n'était pas celui des décideurs d'alors).} 
loin de tous les cadres sociaux, sans aveu. On soupçonne les vagabonds de vivre dans la débauche et de commettre nombre de délits (notamment des vols). On a peur de leur nombre qui favorise la mendicité agressive et qui intimide les personnes sans défense (enfants, jeunes filles, femmes, vieillards). On les accuse de contrefaire maladies ou infirmités, d'enlever des enfants pour exciter la pitié des passants%
%[4]
\footnote{José \fsc{CUBERO}, 1998, p. 70.}% 
, et même de mutiler ces derniers pour obtenir plus d'aumônes%
%[5]
\footnote{Bronislaw \fsc{GEREMEK} fait état de procès tenus dans la région parisienne en 1449 où ont été condamnés des criminels qui avaient successivement enlevé plusieurs enfants à leurs parents, enfants auxquels ils avaient crevé les yeux et coupé bras ou jambes, pour en tirer profit en mendiant. (in \emph{Les marginaux parisiens aux \crmieme{14} et \crmieme{15} siècles}, Paris, 1976).}% 
. 

 Dès le \siecle{14} les hôpitaux refusent de plus en plus souvent les vagabonds%
% [6]
\footnote{José \fsc{CUBERO}, 1998, p. 68.}% 
, tandis que de nombreuses mesures de police tentent de les contrôler et surtout de les chasser. À l'intention des petits délinquants, des vagabonds et autres chômeurs sans ressources avouables on fait des expériences multiples de travaux « forcés », travaux d'utilité publique, ou même galères du roi%
%[7]
\footnote{En 1456 les États du Languedoc prévoient cette peine pour les vagabonds invétérés. \emph{En 1486, Charles~VIII étend cette mesure à l'ensemble du royaume.} La condamnation aux galères, résurgence de la condamnation antique aux mines, \emph{ad metallas}, se substitue alors dans la plupart des cas à la peine de mort, jusque là appliquée largement en l'absence de peines plus adaptées : \emph{Avec la peine des galères... le Moyen-Âge renoue avec la notion antique de l'esclavage... Seul le travail rédempteur peut éviter les galères[7] à ces mendiants valides et vagabonds qui menacent la paix}, José \fsc{CUBERO}, p. 78 idem.}% 
. Le fait que ces décisions d'expulsion aient été périodiquement reformulées montre et leur relative inefficacité, et la persistance des représentations qui les sous-tendent.

 Les cités, en expansion, sont dirigées par leurs bourgeois, commerçants, artisans, juristes et autres détenteurs d'offices. Leur expérience personnelle les porte à tenir pour synonymes les vertus familiales et « bourgeoises » : fidélité, économie, sens de l'effort, contrôle de soi et prévoyance. Pour eux un sou est un sou : contrairement aux aristocrates ils ne valorisent ni le panache, ni le faste, ni la prodigalité. Les anathèmes des religieux contre la richesse, traditionnels, ne les impressionnent plus, sauf lorsqu'ils sont à l'article de la mort, et ils sont fiers de leur fortune. Ils ont la tranquille assurance de ceux qui ont réussi. À leurs yeux les autres n'ont qu'à en faire autant, et ils se font forts de le leur enseigner : dans la plus grande partie des sociétés européennes c'est à la suite des réformes protestante et catholique que les écarts seront les plus faibles entre la morale sexuelle et conjugale officielle et les pratiques réelles. Ce sera le moment où tous les laïcs ou presque se marieront et feront des enfants. Ce sera le moment où les taux de naissances illégitimes et de conceptions pré conjugales seront au plus bas de toute l'histoire européenne : de 1650 à 1750, Normandie : 2 à 3~\% d'enfants illégitimes ; bassin parisien : 1~\% ; Languedoc et Bretagne : 1 à 2~\%. Angleterre sous Cromwell : moins de 1~\% ; en 1600 : 3,2~\%. Ces taux impliquent un haut degré de contrôle social, exercé conjointement par les familles, par les autorités civiles et par les autorités religieuses.

 Au moment où les peuples d'Europe sont en train de se cliver entre catholiques et protestants apparaissent simultanément dans tous les grands États européens des mesures très semblables pour contrôler pauvres et vagabonds. Pas un seul instant la marche vers la rationalisation du contrôle des pauvres et l'organisation de leur mise au travail, forcé si nécessaire, n'a été entravée ou modifiée par les guerres de religion, et tous les États concernés connaissent des évolutions à peu de choses près superposables : mêmes représentations, mêmes solutions, mêmes réussites et mêmes impuissances. 

 Le 22 Avril 1532 le Parlement de Paris ordonne une fois de plus que tous ceux qui dans cette ville peuvent travailler et n'ont ni emploi ni revenus avouables seront contraints à entrer dans les ateliers publics qu'il organise pour eux. Ils travailleront enchaînés deux à deux, gardés rigoureusement et employés aux travaux d'utilité publique les plus rudes. On reconnaît là les pratiques des bagnes%
% [8]
\footnote{... décrites par exemple par Philippe \fsc{HENWOOD} dans \emph{Bagnards à Brest} : « l'accouplement » des bagnards enchaînés, deux par deux, p. 40, 41 et 42, etc.}% 
. \emph{Mais l'Ordonnance royale du 22 avril 1532 est fondamentale en ceci qu'elle ordonne le placement d'office des enfants des vagabonds arrêtés.} L'autorité parentale peut désormais être disqualifiée en l'absence de tout autre délit que le vagabondage. Ce n'était pas la première fois que des essais de ce genre étaient tentés (exemple : Reims, 1454) mais cette fois il s'agit de le faire à Paris, où se trouve la plus grande concentration de vagabonds du royaume (environ un tiers) et l'Ordonnance est signée par le roi. Elle donne aux \emph{bureaux des pauvres}, où siègent des représentants des autorités ecclésiastiques et judiciaires, une part de l'autorité de l'État. Ils exercent une fonction d'autorité sur tous les pauvres, dont ils peuvent et doivent contrôler non seulement l'incapacité de travailler, mais aussi la correction des pratiques conjugales, éducatives et religieuses. Contrôle et assistance sont désormais liés, et les assujettis ont peu de recours judiciaires possibles : ils subissent une justice d'exception. 

 L'Ordonnance royale de 1566 étend l'interdiction de la mendicité à tout le royaume de France et met les pauvres à la charge de leur paroisse d'origine (\emph{domicile de secours} : seul lieu où l'indigent a droit aux secours) ce qui leur interdit de vagabonder. Elle prévoit que les \emph{bureaux des pauvres} et autres \emph{aumônes générales} doivent si nécessaire organiser et financer des ateliers pour donner du travail aux indigents valides. Entre 1550 et 1600, des forces de police spéciales placées sous l'autorité directe des {bureaux des pauvres} (souvent appelées \emph{archers de l'Hôpital}) sont chargées de traquer la mendicité, de poursuivre hors de l'hôpital et d'arrêter les vagabonds, de récupérer les enfants placés par les bureaux des pauvres lorsqu'ils ont fugué de leur lieu de placement, et de faire régner l'ordre dans les hospices et hôpitaux. 

 Au \siecle{17} les expériences réalisées et les réflexions entretenues par les divers acteurs de l'assistance et du contrôle social confluent dans l'idée qu'il convient de regrouper en une seule administration centralisée les hôpitaux et les hospices, et d'y renfermer tous les indigents qui ne peuvent se prendre en charge seuls, en raison de leur immaturité, de leurs infirmités ou maladies, ou bien en raison de leurs comportements%
% [9]
\footnote{Sources principales :
\\Collectif sous la direction de Jean \fsc{IMBERT}, \emph{L'histoire des hôpitaux en France,} 1982.
\\Maurice \fsc{CAPUL}, \emph{Internat et internement sous l'ancien régime, contribution à l'histoire de l'éducation spéciale}, Thèse d'État, 4 tomes, Tomes 1 et 2, \emph{Les enfants placés}, Tome 3 et 4, \emph{La pédagogie des maisons d'assistance}, 1983-1984.
\\Michel \fsc{FOUCAULT}, \emph{Folie et déraison : histoire de la folie à l'âge classique}, 1961.
\\Michel \fsc{FOUCAULT}, \emph{Surveiller et punir, naissance de la prison,} 1975.
\\Bronislaw \fsc{GEREMEK}, \emph{La potence ou la pitié, l'Europe et les pauvres du Moyen-Âge à nos jours,} 1987.
\\Jean \fsc{IMBERT}, \emph{Le droit hospitalier de l'ancien régime}, 1993.
\\Jacques \fsc{TENON}, \emph{Mémoires sur les hôpitaux de Paris,} 1788.}% 
. 

 Louis~XIV ordonne en 1656 la création d'un \emph{Hôpital Général} dans toutes les grandes villes du royaume, et le 14 juin 1662 l'établissement d'un hôpital général dans \emph{toutes les villes et gros bourgs}. Les directeurs, nommés à vie, reçoivent des pouvoirs administratifs et de police pour accomplir leurs missions : \emph{tout pouvoir d'autorité, de direction, d'administration, commerce, police, juridiction, corrections et châtiments sur tous les pauvres de Paris, tant en dehors qu'au-dedans de l'hôpital général \emph{[...]} sans que l'appel puisse être reçu des ordonnances qui seront par eux rendues} [...] Les administrateurs de l'hôpital jugent sans appel, à charge pour eux \emph{si lesdits pauvres méritent peine afflictive plus grande que le fouet, de le mettre es mains du juge ordinaire pour à la requête du procureur d'office leur procez estre fait et parfait}. 

 Que le mouvement de création des Hôpitaux généraux se soit poursuivi à la demande des autorités locales, et pas seulement en France, jusqu'à la fin du \siecle{18} montre que cette formule de l'institution fermée et à l'écart du monde correspondait%
% [10] 
\footnote{Maurice \fsc{CAPUL}, idem, T III, p 301.} 
bien aux conceptions de l'époque : partout en Europe on observait à cette période le même mouvement. Les \emph{Poor Laws} anglaises ordonnaient en 1661 ou 1662 l'enfermement des pauvres dans des \emph{Workhouses} qui sont l'exact pendant (en plus dur ?) des hôpitaux généraux. Il en était de même à Berlin, etc. 

 Les contemporains essayaient de ne pas avoir personnellement affaire à ces institutions dont le régime n'était pas fait pour être désirable. Par contre ils approuvaient leur utilisation pour mettre à l'écart les indésirables et pour éviter les catastrophes en cas de disette ou de crise de l'emploi. 

 Et pourtant il y avait des listes d'attente pour entrer à l'hôpital et il fallait souvent patienter avant d'y être admis. Une recommandation était ordinairement nécessaire (très souvent celle de son curé). Un certain nombre de personnes, pauvres mais non indigentes, acceptaient même de payer pension pour y entrer, ce qui laisse à penser que même si les conditions de vie y étaient rudes (mais ces personnes-là n'étaient pas astreintes au travail forcé) il y avait encore pire ailleurs. Pour elles l'Hôpital Général fonctionnait comme une maison de retraite (cf. les « petites maisons » dans le cadre de celui de Paris), et assumait une forme de prise en charge qui existait déjà avant sa propre création.

 Quant à ceux des mendiants et vagabonds qui troublaient l'ordre public par leurs débordements, ils ne venaient pas à l'hôpital de leur plein gré et leurs comportements le traduisaient, aussi les employés des hôpitaux généraux ne faisaient-ils aucun effort pour les garder. Au bout d'un siècle d'expériences cela conduira les Intendants du roi à créer à partir de 1768 à l'intention de cette population les \emph{dépôts de mendicité}, dépôts qui seront à l'origine des futures \emph{prisons départementales}%
% [11]
\footnote{Leur histoire est complexe et s'étend sur une bonne part du \siecle{19} : cf. entre autres \emph{Lieux d'hospitalité : hospices, hôpital, hostellerie}, ouvrage collectif sous la direction d'Alain \fsc{MONTANDON}, P.U. Blaise Pascal, 2001.}% 
.


\section{Les enfants illégitimes}

 Alors que les grossesses légitimes n'avaient pas à être déclarées, à partir de 1556 obligation est faite par Henri~II de déclarer toutes les grossesses illégitimes, sous peine pour les filles non mariées et les femmes veuves depuis plus d'un an qui seraient enceintes d'être accusées d'infanticide si leur enfant décédait avant son baptême%
% [12] 
\footnote{... qui avait valeur officielle de déclaration de naissance puisque les curés avaient reçu peu de temps auparavant l'obligation de tenir les \emph{registres de catholicité}, ou registres de baptême, ancêtres directs des registres d'état civil.} 
(crime en principe puni de mort). Dans la déclaration devait figurer le nom du père allégué par la mère, sauf refus de celle-ci. Cette déclaration renforçait la position de la mère face à l'homme qui l'avait engrossée, et celle de son enfant, et permettait les actions en justice. Cette décision royale a été rappelée par Henri~III en 1585, et renforcée par Louis~XIV. C'est ainsi qu'en 1708 ce dernier ordonnait encore aux curés de la rappeler en chaire tous les trois mois. 

 Si elle l'a été si souvent, c'est qu'elle n'a jamais été observée de manière rigoureuse. Il semble même que la majorité des grossesses illégitimes n'aient jamais été déclarées. En dépit de la sévérité des peines annoncées les mères préféraient oublier de se signaler à l'attention des autorités lorsqu'elles pensaient pouvoir mieux défendre leurs intérêts et leur réputation (et ceux de leur enfant) par un arrangement discret avec le géniteur (ex. : mariage, pension alimentaire, octroi d'une dot, etc.) ou par un abandon discret. Combien parmi les veuves et filles dont l'enfant est décédé sans baptême ont-elles effectivement subi les peines prévues ? Il ne semble pas que les autorités aient poursuivi ce genre d'infraction avec beaucoup d'énergie : le plus souvent les tribunaux accordaient de larges circonstances atténuantes aux « coupables » déférées devant elles%
% [13]
\footnote{Frédéric \fsc{Chauvaud}, Jacques-Guy \fsc{Petit}, Jean-Jacques \fsc{Yvorel}, \emph{Histoire de la justice de la Révolution à nos jours}, Presses universitaires de Rennes, 2007.}% 
. 

 Tout enfant, même illégitime, avait le droit d'exiger de ses auteurs des « aliments » c'est-à-dire des moyens de vivre. Un vieil adage juridique, toujours cité, disait en effet que \emph{qui fait l'enfant doit le nourrir}. Le représentant naturel de l'enfant né hors mariage est sa mère, et \emph{protéger celle-ci était aussi protéger l'enfant}. Les actions de la mère%
% [14] 
\footnote{Nommée « fille-mère », et n'ayant droit qu'au titre de « mademoiselle » jusqu'au milieu du \siecle{20}. Ce n'est pas un enfant qui pouvait faire d'elle une femme, une « dame », mais un époux en règle.} 
contre le géniteur qui l'avait délaissée étaient encouragées et soutenues, notamment par les hôpitaux, qui en cas d'abandon de l'enfant devaient en assumer seuls la charge. Elle pouvait entreprendre une \emph{actio provisionis} : demande de provisions pour frais de grossesse ou d'accouchement. Si plusieurs hommes avaient partagé à la même période son intimité ils pouvaient être solidairement responsables de l'enfant. Elle pouvait aussi tenter une \emph{actio susceptionnis partus} ou \emph{actio captionis} : action qui demandait de condamner le géniteur à assumer les frais de l'éducation de l'enfant, sur lequel il ne recevait pour autant aucune autorité. 

 L'\emph{actio dotis} prévoyait que le coupable d'un viol épouse la célibataire qu'il avait déflorée, surtout s'il l'avait engrossée. S'il refusait de l'épouser, ce qui était son droit, il devait payer une dot à la mère et financer l'entretien de l'enfant. Il en était de même si le géniteur était déjà engagé ailleurs (mariage, vœux religieux, ordination sacerdotale). Quel que soit son statut (célibataire, marié, clerc, moine, noble, roturier ou serf) il était et demeurait responsable de la vie de l'enfant et devait donc le nourrir. Même si le géniteur n'était pas père légal il restait \emph{nutritor}.

 Par contre les enfants adultérins étaient toujours traités comme des enfants abandonnés, qui n'avaient ni père ni mère, ni \emph{nutritor}. Ils n'avaient aucun droit vis-à-vis de leurs deux géniteurs, dans la famille desquels ils n'entraient pas et auxquels ils ne pouvaient pas réclamer des aliments%
% [15]
\footnote{Ceci dit la loi n'interdisait pas à leurs auteurs de prendre librement l'initiative de pourvoir à leur éducation.}% 
. Ils étaient exclus de toute possibilité de légitimation, même par mariage, puisque leurs géniteurs ne pourraient pas se marier, même après la mort de l'époux qui faisait obstacle à leur mariage. 

 Étaient encore plus rigoureusement exclus de toute légitimation les enfants nés d'une relation incestueuse.


\section{Protection des nouveaux-nés abandonnés}

 Les grandes villes ont toujours enregistré des taux de naissances illégitimes plus élevés que les petites et les campagnes : les domestiques, femmes et hommes, y étaient nombreux, presque toujours contraints au célibat par leur emploi et par leur pauvreté, donc condamnés à abandonner les enfants nés de leur activité sexuelle (surtout dans les cas où le géniteur était l'employeur ou un membre de sa famille). C'est dans les villes que se réfugiaient aussi toutes celles qui voulaient accoucher clandestinement, et les filles chassées par leur famille ou par leur patron à cause de leur grossesse.

 Vers 1635 l'attention de Vincent de Paul (1581-1660), aumônier général des galères et spécialiste de l'assistance, possédant l'oreille du roi Louis~XIII, a été attirée par le chapitre de Notre-Dame de Paris et par les \emph{dames de l'Hôtel-Dieu}%
% [16] 
\footnote{... c'est-à-dire les religieuses qui assumaient le fonctionnement de l'hôtel-Dieu, voisin immédiat de Notre Dame et de la Couche.} 
sur la situation « effroyable » des enfants de \emph{La Couche} (maison où vivaient et mouraient les enfants abandonnés dans l'hôpital), ce qui l'a conduit à fonder en 1638 l'\emph{œuvre des enfants trouvés}. Après quelques tâtonnements il a repris les recettes éprouvées, celles qui avaient toujours marché dans le passé, même s'il l'a fait à l'échelle d'une grande capitale et avec beaucoup de détermination. Ce qu'il a apporté de véritablement nouveau, c'est qu'il a affirmé haut et fort que même s'ils étaient (peut-être) de naissance illégitime les enfants trouvés avaient le même droit de vivre que les autres enfants. Il a refusé la situation d'infanticide déguisé qui était celle des nouveaux-nés abandonnés et il a agi pour qu'ils bénéficient \emph{au même titre que les autres} enfants des soins et du lait d'une nourrice. C'est pourquoi il a créé une institution capable de mettre en présence \emph{rapidement} nourrices et nourrissons et mis au point un service de nourrices rurales efficace avec une surveillance effective%
%[17]
\footnote{Voici comment en 1788, un siècle et demi après, \fsc{TENON} raconte dans ses Mémoires l'histoire des enfants abandonnés (p. 89) : \emph{Dès l'an 1180, à l'Hôpital du Saint-Esprit à Montpellier, on avait ouvert des secours pour les enfans exposés. Les Hôpitaux des Enfans-Trouvés à Paris sont plus modernes : ils datent de 1638 : on les doit au zèle éclairé et infatigable de S. Vincent de Paul. Il faut se transporter à cette époque pour juger du mérite de leur institution.}
\emph{ En 1638, une Dame veuve, charitable, se chargeoit officieusement des enfans exposés : elle demeuroit près Saint-Landry ; sa maison fut nommée Maison de la Couche, comme on nomme aujourd'hui celle des Enfans-Trouvés, près Notre-Dame.}
\emph{La tâche qu'elle avait entreprise, excéda ses facultés ; ses servantes, fatiguées des soins qu'elles donnaient aux enfans, en firent un commerce scandaleux : elles les vendoient à des mendiantes, qui s'en servoient, afin d'exciter la charité du public ; des nourrices, dont les enfans étoient morts, en achetoient, s'en faisoient teter ; plusieurs d'entre'elles leur donnoient un lait corrompu ; on en prenoit pour en supposer dans les familles : ils ne coûtoient que vingt sols. Dès que ces désordres furent connus, on cessa de recourir à un hospice si dangereux : les enfans déposés furent transportés près Saint-Victor ; les dons de quelques personnes vertueuses ne suffisoient pas à leur subsistance ; le nombre de ces enfans devenu trop grand, on tira au sort ceux qui seroient élevés : les autres étoient abandonnés.}
\emph{Dans ces circonstances, St. Vincent de Paul, en 1640, convoqua une assemblée de Dames, distinguées par leur naissance, leur piété : il en obtint des secours. Le choix du sort des enfans à élever, fut aboli, la vie conservée à tous. Louis~XIII entra dans ces vues charitables : il accorda le château de Bicêtre pour les retirer ; on se persuada que la vivacité de l'air s'opposait à leur conservation : ils furent ramenés dans le fauxbourg Saint-Lazare, où ils demeurèrent sous les yeux de Mlle de Marillac, veuve Le Gras, jusqu'en 1670, époque de leur translation dans la Maison de la Couche}.}% 
.

 Ceci dit la majorité de ses contemporains n'a été que fort peu ébranlée dans ses certitudes par son exemple et ses arguments : Maître \fsc{Ducros}, cité plus haut, qui écrivait en 1659, soit plus de vingt ans après la fondation de l'œuvre des enfants trouvés, n'avait rien entendu. Jusqu'au milieu du \siecle{18} (au moins) ceux qui mettaient à l'écart les enfants illégitimes ou supposés tels, et qui réservaient le meilleur des ressources de l'assistance aux nouveaux-nés légitimes pensaient faire pour le mieux. 


\section{Les « enfants de l'hôpital »}

 En ce qui concerne les enfants les plus jeunes la croyance en la vertu éducatrice et rééducatrice de l'internat est à cette époque à son apogée. Les décideurs n'ont pas encore compris l'importance des relations interpersonnelle (corps à corps et cœur à cœur) dans la construction d'une personnalité d'enfant. Ils n'ont pas plus compris combien est déterminante, pour l'investissement de quelque enseignement que ce soit, la différence entre le placement en internat scolaire choisi par les parents, et l'internement d'office ordonné contre leur gré par une instance administrative ou judiciaire. Ils n'ont pas compris non plus la différence qui existe entre la prise en charge des enfants sans famille (orphelins ou abandonnés) qui ni les uns ni les autres n'ont plus de parents, et celle des enfants qui connaissent leurs parents mais à qui on prétend interdire de s'identifier à eux. 


\subsection{Enfants trouvés et abandonnés}

Les enfants abandonnés pris en charge par les institutions d'assistance pouvaient avoir été déposés dans un lieu public ou dans le « tour » d'un hôpital, ou confiés par leur père ou leur mère, ou volontairement « perdus » par eux dans un lieu inconnu%
%[18]
\footnote{L'histoire du \emph{Petit Poucet}, racontée par \fsc{Perrault} dans les \emph{Contes de ma mère l'oye} (1697) a parfois correspondu à une réalité, pour des enfants très jeunes incapables de dire de quelle commune ils venaient ni comment s'appelaient leurs parents.}% 
. Beaucoup de nouveaux-nés étaient abandonnés par leurs mères dans les services d'accouchement des hôpitaux, que seules fréquentaient les indigentes qui ne pouvaient accoucher à leur propre domicile ni chez une sage-femme. D'autres tout-petits n'étaient pas abandonnés à proprement parler. Il s'agissait par exemple d'enfants dont les pères ou/et mères étaient incarcérés dans les « \emph{lieux de force} » (dont la prison pour femmes de \emph{La Force} qui faisait partie de l'hôpital de la Salpêtrière) pour vagabondage, prostitution ou autres actes de délinquance, et qui ne pouvaient donc pour un temps s'occuper d'eux. Dès que l'incarcération durait un temps significatif (un an ?) la restauration des droits parentaux devenait impossible. 

 À part ce cas les enfants abandonnés pouvaient être repris par leurs parents. Il fallait évidemment que leur abandon n'ait pas été anonyme pour que ce retour soit possible. En fait ces \emph{retours en famille}étaient rares, les causes de l'abandon, et d'abord la misère, persistant dans la plupart des cas.
 De nombreux enfants entraient à l'Hôpital bien après leur petite enfance : « \emph{... dans la généralité de Lyon, le plus grand nombre d'enfants présentés aux hôpitaux par leurs parents ont une dizaine d'années…} » À cet âge la plupart des enfants « de famille » travaillaient déjà. Ceux qui étaient confiés à l'hôpital étaient donc souvent ceux qui étaient jugés inaptes au travail. Certains d'entre eux se présentaient d'eux-mêmes à l'hôpital. 

 Au-dessous de 4 à 5 ans les enfants de l'Hôpital sont placés en nourrice. Une fois finie la petite enfance, le placement en institution est préféré. Les administrateurs croient que leurs Hôpitaux offrent des possibilités d'éducation nettement supérieures à une famille nourricière, pour des raisons variées, dont la modestie du niveau culturel des nourrices et de leur maris, qui sont le plus souvent paysans ou ouvriers agricoles, et parce que l'hôpital fournit une scolarité qu'on ne trouve pas à la campagne. Ils estiment aussi que les possibilités de trouver un emploi sont plus grandes en ville. Peut-être ne se sentent-ils pas non plus le droit de déraciner pour toujours des jeunes citadins en les laissant vivre à la campagne, surtout s'ils ont de la parenté dans la ville ? Mais il faut aussi tenir compte du fait que le prix de journée de l'hôpital est à l'époque nettement inférieur au salaire d'une nourrice.

 Tous les enfants de 6 ans et plus, non placés chez un maître artisan ou un nourricier, vivent dans les murs de l'hôpital. Même quand ils ont une famille, les enfants placés en sont plus ou moins radicalement coupés, \emph{même quand leurs parents sont placés dans le même établissement}. Les clôtures internes de l'hôpital sont aussi hautes que son mur d'enceinte%
% [19]
\footnote{Il n'est pour en être persuadé que de visiter la chapelle de l'Hôpital de La Salpêtrière.} 
. Pour nombre d'enfants cette coupure est définitive. 

 En dépit d'un souci éducatif certain%
% [20] 
\footnote{... manifesté à Paris par 5 heures 30 d'enseignement par jour, durant six jours par semaine, ce qui n'a rien à envier aux écoles primaires d'aujourd'hui... mais aussi un nombre d'élèves très élevé pour un seul maitre.} 
l'encadrement humain des jeunes placés est extrêmement réduit (d'où la modestie du prix de journée), ce qui contraint les relations entre les jeunes et les adultes à être formelles, distantes et souvent impersonnelles%
%[21]
\footnote{Selon l'expression de Maurice \fsc{CAPUL} : \emph{Pour les pauvres, les moyens de la pédagogie étaient pauvres}.}% 
. Contrairement aux jeunes « de famille » inscrits par leurs parents dans les collèges contemporains, il ne s'agit pas d'intégrer ces jeunes à la « grande » culture ni de leur donner les moyens de penser plus ou moins librement : il s'agit seulement, comme dans les petites écoles, de leur donner les rudiments de la lecture et de l'écriture, et d'en faire de bons pauvres.

\subsection{« Correctionnaires »}

Les mineurs « correctionnaires » sont les jeunes qu'il faut « corriger », ceux dont les comportements font problème, c'est-à-dire les délinquants, rebelles et opposants : mineurs condamnés par décision de justice, faux saulniers de moins de 14 ans, vagabonds, mendiants, prostitué(e)s, « enfants de bohême ». Les enfants au dessus de 6 ans sont soumis aux mêmes règles de droit que les adultes. Dès l'âge de 8 ou 10 ans la peine de mort peut leur être appliquée si une « malignité » exceptionnelle justifie de les exclure du bénéfice de l'excuse de minorité. Les jeunes délinquants sont ordinairement condamnés à un temps d'incarcération déterminé : de quelques mois à 20 ans et plus. Mais ils peuvent aussi être enfermés pour une durée indéterminée : aussi longtemps que l'administration estimera qu'ils ne seront pas suffisamment amendés, jusqu'à leurs 25 ans et plus. Les jeunes garçons condamnés aux galères pour des délits commis sans l'excuse de minorité ne peuvent y être envoyés avant leurs 15 ou 16 ans. Ils attendent donc à l'hôpital d'avoir atteint l'âge d'aller au bagne, soumis au régime des autres correctionnaires, mais le temps qu'ils passent à l'hôpital ne compte pas comme temps d'exécution de la peine

\subsection{« Religionnaires »}

À partir de la \emph{Révocation de l'Édit de Nantes} (1685) ce terme désigne les enfants des protestants rebelles à la conversion au catholicisme qu'on exige d'eux%
%[22]
\footnote{L'Angleterre avait précédé la France dans la persécution des dissidents religieux et leur exclusion de toutes les charges et fonctions officielles. C'était l'application stricte du principe \emph{cujus regio, cujus religio}, « {un roi, une foi, une loi} ». Il faudra attendre le \siecle{18} pour que la tolérance apparaisse comme une vertu et non comme une faiblesse.}% 
. La légitimité des mariages des protestants n'est plus reconnue, ce qui fait de leurs enfants des bâtards incapables d'hériter. Ils se voient retirer leurs droits parentaux. Pour cette raison dès l'âge de sept ans leurs enfants leur sont enlevés. 

 À partir de cette date il est demandé aux hôpitaux généraux d'enfermer et rééduquer les membres de la « \emph{religion prétendue réformée} » (RPR) si aucune autre solution n'est possible. Les enfants de ceux qui ne peuvent payer sont placés en hôpital général, avec les correctionnaires. Les autres sont placés aux frais de leurs parents dans une section de correction d'un collège (catholique comme tous les collèges du royaume à partir de la Révocation), avec les enfants indisciplinés ou récalcitrants des mêmes milieux sociaux qu'eux. Ils y sont soumis à une pression morale ouverte ou insidieuse, brutale ou habile, pour les pousser à abjurer la religion de leurs parents et à se convertir au catholicisme. Leur sortie de l'hôpital ou du collège dépend en grande partie de leur « conversion ». 

 Selon Maurice \fsc{CAPUL}, cette politique a été poursuivie activement de 1685 au milieu du \siecle{18}, en dépit du fait qu'elle ne donnait que des résultats insatisfaisants : selon les observateurs du temps elle produisait des adultes peu consistants, qui ne savaient plus à quoi ils croyaient, ou des sceptiques qui ne croyaient plus à rien. D'autre part elle jetait la discorde au sein des familles et la brouille entre les parents et les enfants. Elle va se déliter peu à peu après le milieu du \siecle{18}, mais ce n'est qu'en 1787 que \emph{l'Édit de Versailles} y met un terme en créant un état-civil laïque, qui rend aux enfants protestants leur légitimité, et en prenant officiellement acte de la tolérance dont le culte protestant avait fini par bénéficier à cette date%
% [23] 
\footnote{Depuis l'affaire Calas (condamné par le parlement de Toulouse à être roué, exécuté en 1762) et l'intervention de Voltaire (qui avait entrainé sa réhabilitation en 1765) la répression du protestantisme s'était adoucie : dans l'opinion publique la légitimité avait changé de camp.} 
dans la réalité quotidienne. Les dispositions de cet édit concernent aussi les français de confession juive.


