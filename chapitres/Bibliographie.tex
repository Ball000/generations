M, BIBLIOGRAPHIE
 
ALEXANDRE-BIDON Danièle et LETT Didier, \emph{Les enfants au Moyen-Âge, V-XVème siècles,} Hachette, Paris, 1997, 281 p.
ARENDT Hannah, \emph{La crise de la culture, huit exercices de pensée politique}, premières parutions de 1954 à 1968, pour la traduction française : Éditions Gallimard, 1972 ; Collection Folio-essais, 2002, 380 p.
ASSAILLY Jean Pascal, CORBILLON Michel, DUYME Michel, « le placement à l'Aide Sociale à l'Enfance, la reproduction intergénérationnelle », in \emph{Les cahiers du CTNERHI}, n° 37, janvier-mars, 1987, p. 1 à 33.
ASSAILLY Jean Pascal, CORBILLON Michel, DUYME Michel, « Transmission intergénérationnelle et comportement parental, étude longitudinale d'enfants placés », in \emph{Neuropsychiatrie de l'enfance}, 37 (7), 1989, p. 285 à 290.
ASSAILLY Jean Pascal, CORBILLON Michel, DUYME Michel, \emph{L'enfant placé, de l'Assistance Publique à l'Aide Sociale à l'Enfance}, Ministère de la solidarité, de la santé et de la protection sociale, 1988, 196 p.
BASLEZ Marie Françoise, \emph{L'Étranger dans la Grèce antique}, Les Belles Lettres, Paris, 1984, 361 p.
BIANCO Jean-louis., LAMY P., \emph{L'aide à l'enfance demain, contribution à une politique de réduction des inégalités,}étude R.C.B, rapport établi à la demande du Ministère de la santé et de la sécurité sociale, 1980, 217 p.
BIGOT F., \emph{Les enjeux de l'assistance à l'enfance, Étude des configurations et mutations depuis 1811 à partir d'un département nourricier (Loir-et-Cher),} tomes I et II, Université de Tours, U.F.R. Arts et Sciences Humaines, Thèse de doctorat, 1986, 437 p.
BORSA S., MICHEL, C.-R., \emph{La vie quotidienne des hôpitaux en France au XIXème siècle}, Hachette, Paris, 247 p.
BOSWELL John, \emph{Au bon cœur des inconnus, les enfants abandonnés de l'Antiquité à la Renaissance}, Gallimard, Paris, 1993, 518 p.
BROWN Peter, \emph{Genèse de l'Antiquité tardive}, Harvard, 1978, NRF, Gallimard, Paris, 1999, 195 p.
BROWN Peter, \emph{Le renoncement à la chair, virginité, célibat et continence dans le christianisme primitif}, Gallimard, Paris, 2002, 599 p.
BROWN Peter, \emph{Pouvoir et persuasion dans l'antiquité tardive, vers un empire chrétien}, Seuil ; Paris, 1992, 253 p.
BRUN Jean, \emph{Le stoïcisme}, Que sais-je ? PUF, Paris, 126 p.
CAPUL Maurice, \emph{Abandon et marginalité, les enfants placés sous l'ancien régime} (tome I), préface de Michel Serres, Privat, Toulouse, 1989, 215 p.
CAPUL Maurice, \emph{Infirmités et hérésies, les enfants placés sous l'ancien régime} (tome II), Privat, Toulouse, 1989, 1990, 179 p.
CAPUL Maurice, \emph{Internat et internement sous l'ancien régime, contribution à l'histoire de l'éducation spéciale,}Thèse d'état, 4 tomes, Tomes 1 et 2, les enfants placés, Tome 3 et 4, la pédagogie des maisons d'assistance, CTNERHI-PUF, Paris, 1983-1984.
CARCOPINO Jérôme, \emph{La vie quotidienne à Rome à l'apogée de l'Empire}, Hachette, Paris, 1939, 1978, 351 p.
CARRIE Jean-Michel, ROUSSELLE Aline, \emph{L'Empire romain en mutation, des Sévères à Constantin,} 192-337, Éditions du Seuil, 1999, 584 p.
CHOURAQUI, André, \emph{La vie quotidienne des hommes de la Bible}, Hachette, Paris, 1978, 412 p.
COHEN A., \emph{Le Talmud, exposé synthétique du Talmud et de l'enseignement des rabbins sur l'éthique, la religion les coutumes et la jurisprudence}, Payot, Paris, 1980, 467 p.
Collectif, sous la direction d'Alain BURGUIERE, Christine KLAPISH-ZUBER, Martine SEGALEN, Françoise ZONABEND, \emph{Histoire de la famille, II, Temps médiévaux : Orient / Occident}, Armand Colin, Paris, 1986, 479 p.
Collectif, sous la direction d'Alain BURGUIERE, Christine KLAPISH-ZUBER, Martine SEGALEN, Françoise ZONABEND, \emph{Histoire de la famille, I, Mondes lointains,} Armand Colin, Paris, 1986, 448 p.
Collectif, sous la direction d'Alain BURGUIERE, Christine KLAPISH-ZUBER, Martine SEGALEN, Françoise ZONABEND, Histoire \emph{de la famille, III, Le choc des modernités}, Armand Colin, Paris, 1986, 736 p.
Collectif, sous la direction d'Yves LEHMANN, \emph{Religions de l'Antiquité}, PUF, Paris, 1999, 592 p.
Collectif, sous la direction de Cécile LEFEVRE et Alexandre FILHON, \emph{Histoires de familles, histoires familiales, Les résultats de l'enquête famille de 1999,} INED, Paris, 2005, 639 p.
Collectif, sous la direction de Didier HOUZEL, \emph{Les enjeux de la parentalité,}Ministère de l'emploi et de la solidarité, Direction de l'action sociale, ERES, 1999, 198 p.
Collectif, sous la direction de Doudou DIENE, \emph{La chaîne et le lien, une vision de la traite négrière,} Éditions Unesco, 1998, 591 p.
Collectif, sous la direction de Georges DUBY et Michelle PERROT, \emph{Histoires des femmes en Occident, I, l'Antiquité,}Plon, Perrin, 2002, 726 p.
Collectif, sous la direction de Georges DUBY et Michelle PERROT, \emph{Histoire des femmes en Occident, II, le Moyen-Age,}Plon, Perrin, 2002, 691 p.
Collectif, sous la direction de Georges DUBY et Michelle PERROT, \emph{Histoire des femmes en Occident, III, XVIeme-XVIIIème siècle,} Plon, Perrin, 2002, 657 p.
Collectif, sous la direction de Germaine DULAC et Nadine LEFAUCHEUR, « \emph{Politiques du père} », numéro spécial (n° 37) de \emph{Lien social et politiques}, printemps 1997, 192 p.
Collectif, sous la direction de Jean DELUMEAU et Daniel ROCHE, \emph{Histoire des pères et de la paternité,} Larousse, 1990, édition 2000, 535 p.
Collectif, sous la direction de Jean IMBERT, \emph{Histoire des hôpitaux en France}, Privat, Toulouse, 1982, 559 p.
Collectif, sous la direction de l'Administration générale de l'Assistance Publique, \emph{L'Assistance Publique en 1900}, (écrit à l'occasion de l'Exposition Universelle de 1900), Paris, 1900 (au Musée Social, Paris, VIIème).
Collectif, sous la direction de Michel CHAUVIERE, Pierre LENOËL, Éric PIERRE, \emph{Protéger l'enfant, Raison juridique et pratiques socio-judiciaires (XIXème et XXème siècles}), Presses Universitaires de Rennes, 1996, 183 pages.
Collectif, sous la direction de Michel CORBILLON, \emph{L'enfant placé, actualité de la recherche française et internationale, Actes du colloque international, Paris, 31 mai-1er juin 1989,} Unesco, Groupe de Recherche sur la Reproduction et l'Innovation Sociales (GERIS) et CREAI de la région Centre, CTNERHI, PUF, 1989, 350 p.
Collectif, sous la direction de Michel CORBILLON, \emph{Suppléance familiale, nouvelles approches, nouvelles pratiques,} Éditions Matrice, 2001, 241 p.
Collectif, sous la direction de Pascale PLANCHE : L'aide sociale à l'enfance de l'antiquité à nos jours, Hervé TIGREAT, Pascale PLANCHE, Jean-Luc GOASCOZ, préface de Pascal DAVID, Tikinagan, 2010.
Collectif, sous la direction de Pierre GEOLTRAIN, \emph{Aux origines du christianisme}, Gallimard, Paris, 2000, 601 p.
Commission Nationale d'études et de recherches de l'ANPASE, \emph{L'Aide... Sociale... à l'Enfance ? Interrogations}, ESF, Paris, 1980, 158 p.
CUBERO José, \emph{Histoire du vagabondage du moyen-âge à nos jours}, Imago, Paris, 1998, 294 p.
DEKEUWER-DEFOSSEZ Françoise, \emph{Les droits de l'enfant}, Que sais-je ? PUF, Paris, 127 p.
DELACAMPAGNE Christian, \emph{Une histoire de l'esclavage, de l'antiquité à nos jours}, Le livre de poche, Librairie générale française, Paris, 2002, 319 p.
DONZELOT Jacques, \emph{L'invention du social, essai sur le déclin des passions politiques,}Seuil, Paris, 1994, 263 p.
DONZELOT Jacques, \emph{La police des familles}, Les Éditions de minuit, Paris, 1977, 221 p.
DUBY, \emph{Le chevalier, la femme et le prêtre}, Hachette, 1981, Paris. 
DUMAS Didier, \emph{La Bible et ses fantômes}, Desclée de Brouwer, 2001, Paris.
DUPOUX Albert, \emph{Sur les pas de Monsieur Vincent, 300 ans d'histoire parisienne de l'enfance abandonnée}, Édité par la Revue de l'Assistance Publique, Paris, 1958, 407 p (Musée Social, Paris, VIIème).
DURLIAT Jean, \emph{De l'antiquité au Moyen-âge, l'Occident de 313 à 800}, Ellipses, Paris, 2002, 192 p.
FAIVRE Alexandre, \emph{Naissance d'une hiérarchie, les premières étapes du cursus clérical}, Éditions Beauchesne, Paris, 1977, 443 p.
FLACELIERE Robert, \emph{La Grèce au siècle de Périclès, Vème siècle avant J.-C.,} Hachette, Paris, 1959, 1996, 375 p.
FLANDRIN Jean-louis, \emph{Familles, parenté, maison, sexualité dans l'ancienne société}, Seuil, Paris, 1984, 332 p.
FOUCAULT Michel, \emph{Folie et déraison : histoire de la folie à l'âge classique,}Gallimard, Paris, 1961. 
FOUCAULT Michel, \emph{Surveiller et punir, naissance de la prison}, Gallimard, Paris, 1975, 323 p.
FURET et OZOUF, \emph{Lire et écrire, l'alphabétisation des français de Calvin à Jules Ferry,} Les éditions de minuit, Paris, 1977, 390 p.
GARAUD Marcel, \emph{La révolution française et la famille}, manuscrit mis à jour et complété par Romuald SZRAMKIEWICZ, PUF, Paris, 1978, 270 p.
GARLAN Yvon, \emph{Les esclaves en Grèce ancienne}, Éditions de la découverte, 1995 (Première édition, 1982) 214 p.
GARNSEY Peter, SALLER Richard, \emph{L'empire romain, Économie, société, culture}, La Découverte, Paris, 2001, 340 p.
GAUDENS Bernard, \emph{Archéologie et idéologie de la rééducation, genèse des structures et évolution des concepts}, Thèse pour le Doctorat de troisième cycle en sciences de l'éducation, Université de Bordeaux II, U.E.R. de sciences sociales et psychologiques, Section des sciences de l'éducation, 1978, 400 p.
GEREMEK Bronislaw, \emph{La potence ou la pitié, l'Europe et les pauvres du moyen-âge à nos jours}, traduit du polonais par Joanna
GEREMEK Bronislaw, \emph{Les marginaux parisiens aux XIVème et XVème siècles}, traduit du polonais par Daniel BEAUVOIS, première édition française, 1976, Gallimard, Paris, 1991, 376 p.
GOODY Jack, \emph{L'évolution de la famille et du mariage en Europe}, Armand Colin, Paris, 1985, 303 p.
HAMMAN A., \emph{La vie quotidienne des premiers chrétiens}, 95-197, Hachette, Paris, 1971, 301 p.
HEERS Jacques, \emph{Esclaves et domestiques au Moyen-Age dans le monde méditerranéen}, première édition : Fayard, 1981, Hachette-Pluriel, Paris, 1996, 289 p.
IMBERT Jean, \emph{Le droit hospitalier de l'ancien régime}, PUF, 1993, 307 p.
LANCON Bertrand, \emph{Le monde romain tardif, IIIème-VIIème siècle ap. J.C}., Éditeur Armand Colin, Paris, 1992, 191 p. 
LE ROUX Patrick, \emph{Le Haut-Empire romain en Occident, d'Auguste aux Sévères}, Éditions du Seuil, 1998, 497 p.
LEBECQ Stéphane, \emph{Les origines franques, Vème-IXème siècles}, Nouvelle histoire de la France Médiévale, Seuil, Paris, 1990, 321 p.
LEFEBVRE-TEILLARD Anne, \emph{Introduction historique au droit des personnes et de la famille}, PUF, PARIS, 1996, 475 p
LENOIR Rémi, \emph{Généalogie de la morale familiale}, Seuil, 2003, 587 p.
LEVY Jean-Philippe et CASTALDO André, \emph{Histoire du droit civil}, DALLOZ, Paris, 2002, 1554 p.
MARROU Henri-Irénée, \emph{Décadence romaine ou antiquité tardive ? IIIème-VIeme siècle}, Éditions du Seuil, 1977, 1999, 179 p.
MARROU Henri-Irénée, \emph{Église de l'Antiquité tardive, 303-604}, 1963 (contribution à la Nouvelle histoire de Église), Éditions du Seuil, 1985, 1996, 292 p.
MATTEI Paul, \emph{Le christianisme antique (du premier au cinquième siècle}), Ellipses, Paris, 176 p.
MEILLASSOUX Claude, \emph{Anthropologie de l'esclavage}, PUF, Paris, 1986, 375 p.
MENDEL Gérard, \emph{pour décoloniser l'enfant, socio psychanalyse de l'autorité}, UNESCO, Payot, Paris, 1971, 267 p.
MENDEL Gérard, \emph{une histoire de l'autorité, permanences et variations}, La découverte, poche, Paris, 2003, 281 p.
MINOIS Georges, \emph{Les religieux en Bretagne sous l'Ancien Régime}, Éditions Ouest France Université, 1989, 332 p.
MOLLAT Michel, \emph{Les pauvres au moyen-âge, étude sociale}, Hachette, 1978, 395 p.
MOULIN Léo, \emph{La vie quotidienne des religieux au Moyen-Age, Xème au XVème siècle}, Hachette, Paris, 1978, 383 p.
PARISSE Michel, \emph{Les nonnes au Moyen-Age}, Éditions Christian Bonneton, 1983, 270 p.
PETOT Pierre, \emph{La famille}, texte établi par Claude BONTEMPS, Loysel, Paris, 1992, 528 p. 
POLY Jean-Pierre, \emph{Le chemin des amours barbares, Genèse} \emph{médiévale de la sexualité européenne}, Perrin, 2003, 607 p. 
PUCCINI-DELBEY Géraldine, \emph{La vie sexuelle à Rome}, Taillandier, Paris, 2007, 383 p.
ROUSSELLE Aline, \emph{La contamination spirituelle, science, droit et religion dans l'antiquité}, Les belles lettres, Paris, 1998, 368 p. 
SALLES Catherine, \emph{Les bas-fonds de l'antiquité}, première édition : 1982, Payot, 2004, 359 p. 
SEGALEN Martine, \emph{A qui appartiennent les enfants ?}, Tallandier, 2010.
STEINSALTZ Adin, \emph{Introduction au Talmud}, (édition anglaise : 1976), Albin Michel, Paris, 2002, 326 p.
TENON Jacques, \emph{Mémoires sur les hôpitaux de Paris, Fac-similé de l'édition originale de 1788}, édité par l'Assistance Publique, Hôpitaux de Paris, 1998, 472 p.
TESTART Alain, \emph{L'esclave, la dette et le pouvoir}, Éditions Errance, 2001, 238 p
THERY Irène, « Peut-on parler d'une crise de la famille ? Un point de vue sociologique », in \emph{Neuropsychiatrie de l'enfance et de l'adolescence}, Décembre 2001, Vol. 49, n° 8, pp. 492-501.
TODD Emmanuel, \emph{L'origine des systèmes familiaux, Tome I}, Gallimard, Paris, 2011, 788 p.
UNTERMAN Alan, \emph{Dictionnaire du judaïsme, Histoires, mythes, traditions}, Thames et Hudson, Paris, 1997
VERDIER Pierre, \emph{L'autorité parentale, le droit en plus}, Bayard, 1993, 121 p. 
VEYNE Paul, \emph{L'empire gréco-romain}, Éditions du Seuil, Paris, 2005, 874 p.
VEYNE Paul, \emph{La société romaine}, Éditions du Seuil, Paris, 1991, 2001, 342 p.
VEYNE Paul, \emph{Le pain et le cirque, sociologie historique d'un pluralisme politique}, Éditions du Seuil, 1976, 890 p.
VEYNE Paul, \emph{Sexe et pouvoir à Rome}, Taillandier, Paris, 2005, 208 p.

TABLE DES MATIERES
 
 GENERATIONS ET HISTOIRE	1
AO PROLOGUE	5
A1 CITES ANTIQUES INEGALITAIRES ET PATRIARCALES	8
 \emph{Le pater familias romain	14}
 \emph{Le patriarcat	16}
A2 CITES ESCLAVAGISTES	19
 \emph{Un esclave pour quoi faire ?	20}
 Une main-d'œuvre à bon marché	20
 Un corps sans défenses	21
 \emph{La fabrique des esclaves	24}
 La violence	24
 La sanction pénale	25
 Le surendettement	25
 La naissance	26
 L'abandon	27
 La vente par un parent	27
 La vente par soi-même	28
 \emph{Qui peut-on asservir légitimement ?	29}
A3 LES REPRESENTATIONS ANTIQUES	31
 \emph{Des hommes et des dieux	31}
 \emph{La vie bonne	33}
 \emph{Piété et solidarité familiale	34}
 \emph{Morale d'esclaves	36}
 \emph{Vertu virile	37}
 \emph{Pudeur féminine	40}
B1 L'EXCEPTION JUIVE	43
 \emph{Un dieu à part	43}
 \emph{Le problème du mal	45}
 \emph{Un culte spirituel	47}
B2 MŒURS JUIVES	49
 \emph{Le pur et l'impur	49}
 \emph{Morale et société	52}
 \emph{Exaltation de la sexualité conjugale	53}
 \emph{Naissances impures	56}
 \emph{Clergé et familles	58}
B3 LES JUIFS, LE TRAVAIL ET LES ESCLAVES	60
C1 LES LOIS D'AUGUSTE	64
 \emph{Promotion des naissances ingénues	66}
 \emph{Répression de l'adultère féminin	67}
 \emph{Promotion du mariage	69}
 \emph{limitation des affranchissements	70
}
 
 
C2 EVOLUTION DES DROITS PERSONNELS SOUS L'EMPIRE	72
 \emph{Des droits pour les femmes	72}
 \emph{Des droits pour les enfants	72}
 \emph{Des droits pour les esclaves	73}
 \emph{Tous romains, tous égaux ?	75}
D1 LA GENERATION CHEZ LES CHRETIENS	77
 \emph{Indissolubilité du mariage	77}
 \emph{Valorisation du célibat et de la continence	81}
 \emph{Désacralisation de la fécondité, valorisation des enfants	83}
D2 UNE CONTRE-SOCIETE CHRETIENNE	86
 \emph{Laïcs et laïques consacrés	86}
 \emph{Clergé et continence	88}
 \emph{Il n'y a plus ni esclave ni homme libre ?	90}
 \emph{Le service des pauvres	91}
 Veuves	93
 Orphelins	94
 Malades	95
 Énergumènes	95
 Voyageurs, vagabonds, mendiants	97
 Captifs	97
 Morts sans sépulture	98
 Enfants trouvés	98
E1 CONSTANTIN ET LE CHRISTIANISME	102
E2 CONSTANTIN ET LE DROIT DES PERSONNES	107
F1 ENTREE EN SCENE DES BARBARES	115
F2 BAS-EMPIRE ET HAUT-MOYEN-AGE	118
F3 L'ESCLAVAGE DU BAS-EMPIRE ET DU HAUT-MOYEN-AGE	121
F4 A LE CLERGE CHRETIEN	125
F4 B LES RELIGIEUX	129
F5 LE MARIAGE CONSTANTINIEN	132
F6 FAMILLES DE CHAIR	138
 \emph{Mépris de la chair ?	138}
 \emph{Disparition de l'adoption	140}
 \emph{Divorces et remariages	140}
 \emph{Phobie de l'inceste	141}
 \emph{Enfants en trop, enfants « irréguliers »	144}
 \emph{Les éducations	146}
F7 FAMILLES SPIRITUELLES	149
G1 LES FAMILLES DE L'ANCIEN REGIME ENTRE AUTORITES CIVILES ET RELIGIEUSES	154
 \emph{La Réforme Grégorienne	154}
 \emph{Monopole de l'Eglise sur le droit familial	155}
 \emph{Conflits de juridictions	156}
 \emph{Résistances aux règles canoniques du mariage	158}
 \emph{« Bâtards »	159
}
 
G2 LES DEVOIRS DES PERES DE L'ANCIEN REGIME	163
 \emph{Montée en puissance de l'enseignement	164}
 \emph{La Correction paternelle	168}
 \emph{Enfants « adoptifs »	169}
G3 LA POLICE DES FAMILLES	171
 \emph{Organisation d'une police des pauvres	171}
 \emph{Les enfants illégitimes	174}
 \emph{Protection des nouveaux-nés abandonnés	175}
 \emph{Les « enfants de l'hôpital »	176}
 Enfants trouvés et abandonnés	176
 « Correctionnaires »	178
 « Religionnaires »	178
H1 LES LUMIERES CONTRE LES FAMILLES TRADITIONNELLES	181
 \emph{Contestation de l'autonomie de l'Eglise	181}
 \emph{Contrôle des autorités civiles sur les congrégations religieuses	182}
 \emph{Contestation de la puissance des pères	185}
 \emph{Banalisation des abandons	186}
 \emph{Valorisation de l'éducation familiale et maternelle	187}
H2 LA REVOLUTION FRANÇAISE	191
 \emph{Limitation de la puissance paternelle	191}
 \emph{Privatisation des vœux perpétuels et droit au divorce	192}
 \emph{Libération des enfants majeurs	192}
 \emph{« Nul ne peut être parent contre son gré »	192}
 \emph{Démembrement de l'hôpital général	194}
I1 LA FAMILLE DU CODE NAPOLEON	195
 \emph{Suppression du divorce	195}
 \emph{Restauration de l'autorité des pères	196}
 \emph{Interdiction des recherches en paternité	198}
I2 LA POLICE DES FAMILLES DU XIX EME SIECLE	200
 \emph{Les enfants trouvés et abandonnés	200}
 \emph{La prévention des abandons	202}
 \emph{Des clivages idéologiques durables autour des familles	203}
J1 LEGISLATION REPUBLICAINE (1880-1946)	205
J2 SEPARATION DES EGLISES ET DE L'ETAT	209
J3 CONTESTATION DU CODE NAPOLEON	212
 \emph{Critiques théoriques	212}
 \emph{Evolutions du droit.	213}
 \emph{Erosion du droit de correction	214}
J4 L'ETAT, PROVIDENCE DES FAMILLES ?	219
K1 DEMANTELEMENT DEPUIS 1965 DE LA FAMILLE TRADITIONNELLE	222
 \emph{Lois principales	222}
 \emph{Le sens des évolutions	224}
 \emph{Séparation des familles et de l'État ?	227
}
 
K2 VICTOIRE DE LA LIBERTE INDIVIDUELLE ET DU MARIAGE D'AMOUR	229
K3 « LE CORPS DES FEMMES EST A ELLES »	233
K4 INCESTES ET PARADOXES	236
K5 PERPLEXITES EDUCATIVES	238
K6 DESARROIS MASCULINS	241
K7 INERTIE DES PRATIQUES	246
L1 UN ENFANT POUR QUOI ? POUR QUI ?	249
L2 QUI POUR ACCUEILLIR L'ENFANT ?	252
L3 DROIT A L'ENFANT ?	256
L4 PROGRES OU REGRESSIONS ?	259
L5 CONCLUSION	265
M, BIBLIOGRAPHIE	269
TABLE DES MATIERES	273
 
 
 
 
 


 
 
 
 4
12-1-2015 GENERATIONS ET HISTOIRE
 
