
\chapter[Démantèlement de la famille traditionnelle]{Démantèlement\\de la famille traditionnelle}


 En ce qui concerne le sexe, la conjugalité et la procréation, les évolutions du droit et des mœurs depuis 1960 ont été fulgurantes comparées aux rythmes antérieurs de l'histoire.


\section{Lois principales}

\begin{description}

\item[1961] Une mesure administrative qui sur le moment n'a pas frappé beaucoup d'esprits, mais dont l'importance symbolique n'en est pas moins significative (les grandes fractures commencent souvent par une fissure imperceptible à l'œil nu) : le ministère de l'éducation nationale supprime le caractère obligatoire de l'enseignement du droit romain dans le programme de la licence de droit, obligation qui datait de la création des études de droit dans les universités aux \crmieme{11} et \siecle{12}s. Ce corpus n'est plus qu'une option facultative parmi d'autres. 

\item[1965] La loi du 13 juillet lève les derniers obstacles à l'exercice d'une activité commerciale par les femmes mariées sans la tutelle de leurs maris. Ceux-ci ne gèrent plus de droit les biens propres de leurs épouses (dot, etc.). Elles n'ont plus à obtenir leur autorisation pour exercer une profession séparée, quelle qu'elle soit.

\item[1966] La loi du 11 juillet sur l'adoption assimile les enfants adoptés aux enfants légitimes non adoptés (adoption plénière). 

%\item[1966] la (même)
Cette
loi du 11 juillet ouvre le droit à l'adoption plénière à une personne seule, qu'elle soit célibataire ou non et quelles que soient ses préférences sexuelles, d'au moins 28 ans.

\item[1967] La loi \fsc{NEUWIRTH} dépénalise la prévention des naissances : elle autorise la publicité concernant les méthodes anticonceptionnelles (interdite depuis les années 20), et elle autorise leur mise à disposition du public :
%la première visée et la principale était la « pilule » anticonceptionnelle 
la « pilule » anticonceptionnelle était principalement visée,
qui venait d'être mise au point. L'accord du mari n'est pas nécessaire, son refus n'a pas d'effet.

\item[1972] La loi du 3 janvier fait entrer les enfants naturels dans la famille du ou des parents qui les ont reconnus. À quelques restrictions près -- enfants adultérins, elle leur ouvre un droit à l'héritage égal à celui des enfants légitimes.

%\item[1972] 
La puissance paternelle est abolie au profit de l'autorité parentale. En cas de séparation, cette autorité est conservée à égalité par chacun des deux parents. 

%\item[1972] 
Une loi ordonne l'égalité des salaires féminins avec les salaires masculins.

\item[1974] L'âge de la majorité légale est abaissé de 21 à 18 ans. 

\item[1975] Loi du 30 juin relative aux institutions sociales et médicosociale : les usagers et les familles doivent être associés au fonctionnement de l'établissement qui les prend en charge (il doit les « prendre en compte » ). 

%\item[1975] 
À côté du divorce pour faute, la loi du 11 juillet ouvre la possibilité de divorcer par consentement mutuel ou pour rupture de la vie commune. Par ailleurs, cette loi met les deux époux à égalité en matière de choix résidentiel et en matière de contribution aux charges du mariage.

%\item[1975] 
La loi \fsc{WEILL} dépénalise l'avortement (\emph{interruption volontaire de grossesse} ou IVG). La loi ne demande pas l'avis des maris éventuels.

%\item[1975] 
Les épouses ne sont plus tenues de faire usage du nom de leur mari dans la vie quotidienne et les relations avec l'administration.

%\item[1975] 
L'adultère féminin est dépénalisé. Ce n'est plus un délit qui concerne la société, ce n'est qu'un affront privé qui ne concerne que le mari.

\item[1976] Loi du 22 décembre relative aux conditions d'adoption : la présence d'enfants légitimes n'est plus un obstacle à l'adoption, même si leur avis est entendu. 

\item[1978] La loi du 6 janvier donne à tout individu majeur le droit de connaître le contenu de tout dossier administratif le concernant. Cela concerne notamment tous les enfants abandonnés.

\item[1983] Un arrêt de la cour de cassation du 21 mars 1983 admet la légalité de la garde conjointe de l'enfant après divorce.

%\item[1983] 
Loi sur l'égalité professionnelle entre femmes et hommes.

\item[1984] La loi du 6 juin relative aux \emph{droits des familles dans leurs rapports avec les services chargés de la protection de la famille et de l'enfance}, et au statut des pupilles de l'État, donne aux parents des droits plus étendus face à l'administration. L'autorité des parents sur leurs enfants placés à l'ASE est confortée dans tous les domaines (sauf limites définies expressément par un juge).

\item[1985] La loi du 23 décembre 1985 met les deux parents à égalité dans la gestion des biens de l'enfant : ils exercent cette tâche conjointement quand ils exercent en commun l'autorité parentale. Sinon l'un des deux l'exerce sous le contrôle du juge.

\item[1987] L'autorité parentale est redéfinie par la loi du 22 juillet (loi \fsc{MALHURET}) en termes de \emph{responsabilité parentale ordonnée à l'intérêt de l'enfant}. Elle est à égalité assumée par chacun des deux parents, qu'ils cohabitent ou pas.

\item[1989] La \emph{Convention Internationale des Droits de l'Enfant}, promulguée dans le cadre de l'ONU le 20 novembre, reconnaît le droit de tout mineur à une famille, et ses droits face à sa propre famille.

%\item[1989] 
Loi du 10 juillet \emph{relative à la prévention des mauvais traitements à l'égard des mineurs et à la protection de l'enfance}. Elle prévoit que le délai de prescription ne court qu'à partir de la majorité pour les mineurs victimes de violences.

\item[1993] L'autorité parentale conjointe devient la règle pour les couples de concubins comme pour les couples mariés.

\item[1996] Convention européenne du 25 janvier sur l'exercice des droits de l'enfant. Elle donne le droit aux enfants mineurs de donner leur avis sur les mesures qui les concernent lors du divorce de leurs parents.

\item[1999] Création du PACS : pacte civil de solidarité, ouvert aux couples hétérosexuels et aux couples homosexuels.

\item[2000] La pilule {\emph{du lendemain}} est en vente libre dans les pharmacies, et distribuée gratuitement aux mineures par les infirmières scolaires sur simple demande de la mineure, sans demander l'avis de ses parents, et sans qu'ils en soient informés.

\item[2002] Sur décision de la Cour Européenne de Justice les dernières discriminations juridiques que subissaient en matière d'héritage les enfants adultérins et incestueux sont effacées. Seuls sont distingués les enfants nés des incestes parent--enfant, qui ne peuvent être reconnus que par un seul de leurs deux parents. Ils doivent néanmoins être traités absolument en tout le reste, et d'abord en ce qui concerne l'héritage, comme leurs éventuels demi-frères ou sœurs. 

%\item[2002] 
Loi du 4 mars : {[...] \emph{les parents associent l'enfant aux décisions qui le concernent, selon son âge et son degré de maturité}}. L'administration de la famille par les deux parents doit être démocratique.

\item[2005] La loi autorise les femmes à donner à leurs enfants leur propre nom à égalité avec leur mari à compter de janvier 2005. 

%\item[2005] 
Interdiction du mariage des filles avant dix-huit ans (traditionnel âge au mariage des garçons).

\item[2013] Loi \fsc{Taubira} : ouverture du mariage aux couples de même sexe.

\item[2013] Remboursement à 100~\% de l'IVG.

\item[2014] Suppression de l'exigence d'une « détresse » pour reconnaître à une femme enceinte son droit à un avortement. 
\end{description}
 
 
\section{Le sens des évolutions}

 Jusqu'à notre présent, la prééminence du masculin allait de soi. Le mâle \emph{(vir)} était le modèle accompli du genre humain \emph{(homo)}, l'homme véritable. La femme était son exception et n'était en principe que cela. Pleine de charme et de mystère et belle à troubler les plus chastes, elle n'en était pas moins caractérisée par le manque. Toutes les sociétés faisaient d'elle un être comme de second rang (le {\emph{deuxième sexe}}), presque toujours limité dans ses droits et dans son autonomie, un peu à la manière d'un enfant, et toujours exclue des postes de pouvoir. Les hommes remerciaient leurs dieux de n'être pas nés femmes. Celles-ci se montraient d'ailleurs autrement comblées par la naissance d'un garçon que par celle d'une fille, et exerçaient une pression redoutablement efficace sur leurs filles pour qu'elles ne s'écartent pas du rôle que société et familles attendaient d'elles.

 Jusqu'au milieu du \siecle{20}, le statut légal des femmes françaises était plus proche de celui des femmes de l'antiquité tardive que de celui de leurs petites-filles de l'an 2010. Jusqu'aux années soixante du \siecle{20}, tout mari était le chef de sa famille et avait à ce titre autorité, sur ses enfants mineurs certes, mais aussi sur sa femme, puisque celle-ci avait abdiqué une bonne part de sa capacité juridique en se mariant. C'est lui qui détenait ces droits, et même s'il l'autorisait à les exercer, c'était en son nom à lui. C'est l'homme qui donnait son nom aux enfants du couple ; sauf contrat de mariage particulier c'est lui qui gérait tous leurs biens ; il signait seul la déclaration de revenus, et pouvait laisser son épouse dans l'ignorance sur le montant des ressources du couple ; il était censé être le principal pourvoyeur financier même quand par son travail ou par sa dot son épouse contribuait autant ou plus que lui aux dépenses du ménage ; au nom de l'unité de commandement nécessaire à toute entreprise il pouvait lui interdire d'exercer un emploi salarié, d'ouvrir un commerce, de prendre une gérance en son nom propre, et même de posséder un compte en banque personnel ; lorsqu'elle travaillait avec lui elle était censée lui être subordonnée, et son travail était rarement reconnu et individualisé. En droit c'est aux pères que les gains des membres de leur famille revenaient. S'ils le jugeaient bon ils plaçaient leurs enfants chez le maître de leur choix dès que leur âge le leur permettait%
% [1]
\footnote{Quand ils le pouvaient, ils les plaçaient chez leur propre employeur, ce qui leur permettait de les garder sous leur autorité directe, ce sur quoi les patrons étaient d'accord avec eux.}% 
. De la même façon, ils étaient censés choisir l'employeur de leur épouse et négocier son salaire et ses conditions de travail. 

 Lorsque les deux époux s'entendaient bien, la réalité de leurs rapports et l'influence de chacun sur les décisions du couple et la vie de leurs enfants pouvaient être fort différentes du modèle que la loi postulait, et il en a toujours été ainsi, mais en cas de conflit c'est dans le cadre de celui-ci que statuaient les juges.

 Si l'on en croit les observateurs du \siecle{19}, de nombreux prolétaires chefs de famille dilapidaient les gains des uns et des autres, même si dans la plupart des couples populaires c'étaient en réalité les épouses qui tenaient les cordons de la bourse. En tout cas c'est en principe au nom de ces abus, et non pour favoriser la parité homme--femme, que les femmes mariées ont reçu au début du \siecle{20} le droit de percevoir et de gérer elles-mêmes leurs gains et leurs économies personnelles.

 Certes depuis Constantin (au plus tard), les femmes n'étaient pas considérées comme incapables ou déficientes par nature, mais en cas de mariage c'est le masculin qui avait la préséance. Si leur « seigneur et maître » venait à mourir elles retrouvaient le plein exercice de leurs droits personnels, ceux qu'elles lui avaient délégués en se donnant à lui par le mariage. Une veuve non remariée remplaçait donc de plein droit son époux dans tous les actes juridiques ou commerciaux, comme dans l'exercice de l'autorité éducative sur leurs enfants communs. Mais aujourd'hui l'idée même d'une préséance du masculin n'a plus de sens. Elle n'est plus défendable dans les pays développés, quelles que soient les difficultés que l'on peut rencontrer dans la recherche de l'égalité absolue des deux sexes, difficultés qui ne manquent pas étant donné les différences entre les corps masculins et féminins et leurs performances respectives selon les domaines.

 Les brimades, interdits et humiliations subis par les « mal nés » du passé avaient pour but ultime de prévenir la conception d'enfants en dehors du cadre du mariage légitime, en dehors de l'alliance de la mère avec un homme nommément désigné. Pour défendre l'institution familiale et la société comprise comme un réseau de familles alliées (un filet de relations noué par les mariages) il fallait décourager les femmes de faire naître des enfants sans pères, et interdire aux hommes de se procurer des héritiers en dehors d'une alliance avec une autre famille. Aujourd'hui tout se passe comme s'il n'existait plus que des enfants légitimes : tout enfant a vocation à faire partie de la famille de chacun de ses deux géniteurs quelles que soient les circonstances de sa conception. Tout enfant a vocation à hériter de ses deux parents à égalité avec ses éventuels demi-frères et demi-sœurs. Tout enfant est un « enfant de famille ». On peut aussi bien dire que tous les enfants sont devenus « naturels » et que la notion même de légitimité s'est évaporée, réduite à un mot sans épaisseur puisqu'il n'a plus de prise sur rien, puisque les effets concrets de la légitimité sont les mêmes que ceux de l'illégitimité, et inversement. 

 {\emph{Cette distinction légitimité--illégitimité était totalement structurante de la société. Aujourd'hui il n'est pas un pays qui n'ait soit complètement aboli cette distinction, soit s'apprête à l'abolir. C'est un changement majeur des rapports entre famille et société qui montre que nous sommes face à des changements de la structure sociale elle-même}.%
% [2]
\footnote{Irène \fsc{THERY}, « peut-on parler d'une crise de la famille ? Un point de vue sociologique » , \emph{Neuropsychiatrie de l'enfance et de l'adolescence}, 2001, 49, 492-501, p. 403.}% 
}

 La cheville qui depuis \nombre{1600} ans tenait ensemble tout le système de la famille constantinienne a été retirée, et cela se passe apparemment à la satisfaction de tous. Puisque ni les parents ni les enfants ne risquent plus aucun désagrément du fait d'une naissance illégitime, à quoi bon le mariage, surtout quand on est convaincu que le seul couple légitime c'est celui qui repose sur l'accord quotidien de deux volontés. Depuis plus d'une génération, le nombre d'enfants nés hors mariage a donc progressé en même temps que croissait leur assimilation aux enfants nés dans le mariage. Aujourd'hui ils représentent la moitié des naissances. 

 Depuis la fin du Moyen-Âge, la coutume en France était de nommer les enfants du nom de leur père. Dans ce contexte porter le nom de sa mère signifiait que l'on était né hors mariage et qu'on n'avait pas de père légitime. Dans la France d'aujourd'hui, l'illégitimité a cessé d'être honteuse et il n'est plus socialement utile que le nom que l'on porte atteste qu'on a été reconnu par un homme. Une loi de 2001 autorise (donc ?) les couples mariés à donner à leurs enfants le patronyme de la mère à la place de celui du père, ou bien à côté de lui (texte complété par la loi \fsc{Taubira} de 2013). Ce changement est significatif, puisque la pratique traditionnelle n'était pas celle-là, contrairement à l'Espagne, entre autres exemples, tout comme est significatif le moment où il a été institué. 


\section{Séparation des familles et de l'État}

 Sauf crime caractérisé les membres dépendants d'une famille (enfants, serviteurs libres ou esclaves) ne pouvaient faire appel d'aucune des décisions des parents devant un magistrat extérieur à la famille. Certes chez les grecs et les romains l'unité de commandement était au moins aussi nécessaire à la famille qu'à toute autre institution : c'est donc le mari qui tranchait en cas de conflit entre les deux époux, comme dans toutes les sociétés patriarcales. Mais selon Aristote c'est comme un couple royal que les deux époux régnaient sur leur maison, et qu'ils exerçaient sur leurs enfants et leurs esclaves ce qu'il appelle la \emph{justice domestique}. Cette conception royale du couple parental a trouvé son achèvement lorsqu'à la fin du siècle de Constantin, en 390, les veuves non remariées \emph{(sui juris)} se sont vu reconnaître le droit d'exercer elles-mêmes la tutelle de leurs enfants mineurs, en lieu et place de leur mari défunt. Quand Thomas d'Aquin a réintroduit Aristote dans les universités du Moyen-Âge, il a repris cette doctrine du pouvoir royal des parents sur leur maison. 

 Du haut Moyen-Âge aux Lumières, la continuité entre le pouvoir du roi et celui des pères allait sans dire et n'avait pas à être démontrée. Au début du \siecle{19} le Code Civil reprenait encore à son compte cette conception monarchique du rôle parental. Il n'imaginait pas un instant un fonctionnement familial démocratique associant les enfants aux décisions qui les concernaient. Jusqu'à leur majorité leurs parents avaient pleine autorité sur eux, parlaient pour eux, et si nécessaire le père, \emph{chef de la famille}, tranchait en dernier ressort.

 Aujourd'hui le pouvoir royal des pères sur leurs enfants est mort, et celui des mères l'est tout autant. Nous avons assisté à la délégitimation de la justice domestique, du droit des deux parents à régler eux-mêmes sans tiers extérieur tous les conflits intra familiaux. Lorsqu'ils ne réussissent pas à se mettre d'accord entre eux ou avec leurs enfants, ils sont désormais contraints (par leur égalité elle-même) à recourir à un tiers extérieur pour arbitrer leur différend. 

 De nouveaux personnages se sont imposés au sein des familles. Les juges et leurs collaborateurs, les travailleurs sociaux, sont entrés dans le champ, jusque là bien clos, des familles ordinaires, des familles non stigmatisées au préalable comme défaillantes%
% [3]
\footnote{... en ce qui concerne les familles reconnues officiellement comme incompétentes, déficientes, ou délinquantes, c'est depuis toujours que les représentants de la société y avaient leurs entrées.}% 
. {\emph{Ce qui caractérise la loi de 1970 (qui substitue l'autorité parentale à la puissance paternelle) ce sont trois concepts au centre de la réforme, celui « d'égalité » des époux et parents, celui « d'intérêt de l'enfant » et enfin celui de « contrôle judiciaire » devenu nécessaire pour arbitrer d'éventuels conflits entre les parents, entre parents et enfants. On assiste à un recentrage des positions de chacun des membres de la famille. Au centre l'enfant, en face de lui, responsables de lui, ses parents. Entre les deux des médiateurs, les spécialistes judiciaires.}%
%[4]
\footnote{Françoise \fsc{HURSTEL}, \emph{La déchirure paternelle}, p. 117.}% 
}

 Quelqu'un a dit que nous assistions à la \frquote{séparation de la famille et de l'État%
\tempnote{J'ai trouvé ceci : \url{https://www.liberaux.org/index.php/topic/50889-motivation-du-mariage-gay/?p=892196} Apparemment c'est l'administrateur du site, dans ce cas il y a sans doute moyen de le contacter pour avoir son nom et s'assurer de la paternité de la formule.}%
%[5]
\footnote{J'ai malheureusement égaré le nom du l'auteur de cette formule choc, qui pose si bien le problème. Qu'il ou elle me le pardonne.}%
} ? 

 Assiste-t-on à la dissolution de la famille en tant qu'institution juridique ? 

 Assiste-t-on à la disparition de la sphère privée, cette sphère de la vie de chacun qui se définit par le fait que tant qu'il n'enfreint aucune loi, il n'a aucun compte à rendre sur ce qui s'y passe, et surtout pas à l'État et à ses représentants ? 

 Tout ce qui concerne les enfants est-il entré dans le domaine public, alignant le traitement de l'ensemble des familles sur celui qui était autrefois réservé aux seuls « cas sociaux », et mettant en cause l'aptitude de tous les parents à défendre suffisamment bien l'intérêt de leurs propres enfants ?



