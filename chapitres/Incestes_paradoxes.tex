

 L'abondance actuelle, depuis les années 1985-1990, des discours sur ce qu'on est convenu d'appeler les \emph{abus sexuels intra familiaux}%
% [1] 
\footnote{... comme s'il y avait un usage normal, correct du sexe entre générations différentes au sein des familles ?} 
signifie-t-elle qu'il s'en commet plus qu'autrefois ? Si l'on en croit le témoignage de Jeannine \fsc{NOEL} (1965) il est permis d'en douter : selon elle entre le quart et le tiers des adolescentes placées à l'Hôpital Hospice Saint Vincent de Paul%
%[2]
\footnote{À cette époque c'était encore le Foyer de l'Enfance de Paris (anciennement « dépôt de l'Assistance Publique ») recevant (souvent avant une orientation ailleurs) tous les jeunes dont les parents ne pouvaient pas s'occuper ou de l'autorité desquels ils avaient été soustraits par décision de justice. On plaçait et place toujours dans les foyers de l'enfance les jeunes qui n'ont pas d'autre lieu où aller, quelle que soit la raison qui les a mis dans cette situation.} 
 au cours des années cinquante du \siecle{20} avaient été confrontées à des problèmes de ce genre : la situation ne semble pas être pire aujourd'hui. 

 Les travailleurs sociaux d'autrefois, quel que soit le titre qu'on leur donnait, ont toujours su que même dans les « meilleures » familles il pouvait se passer des choses « pas très catholiques ». Durant tout le \siecle{19} les visiteurs des pauvres et les médecins n'ont pas arrêté pas de dénoncer la promiscuité des logements des indigents et des ouvriers. Ils n'ont pas arrêté de plaider pour qu'à défaut d'une chambre par enfant il y ait au moins une chambre pour les garçons et une chambre pour les filles, et un lit par enfant, et d'abord et avant tout une chambre pour le couple parental. Cela dépassait évidemment le souci d'hygiène. S'ils n'en disaient pas plus, tous comprenaient ce que ces propos pudiques impliquaient : que la promiscuité des familles allait souvent jusqu'à la promiscuité sexuelle. Et ils ne croyaient pas que ces faits concernaient seulement les familles mal logées. 

 Dans le passé, les paniers de linge sale et les cadavres des placards familiaux étaient protégés par la rigueur du secret absolu auquel étaient tenus médecins, ecclésiastiques, infirmières visiteuses,~etc. Pourquoi n'est-ce qu'aujourd'hui que le caractère absolu des secrets professionnels a été mis en question ? Pourquoi n'est-ce qu'aujourd'hui que l'évocation des sévices intra familiaux à l'encontre des enfants obtient un tel effet ? Naguère la justice ne voulait rien entendre non plus des violences conjugales tant qu'il n'y avait pas de lésions physiques sérieuses%
% [4]
\footnote{\frquote{\emph{entre l'arbre et l'écorce il ne faut pas mettre le doigt}}.}%
. On n'avait semble-t-il pas conscience de la gravité des effets à moyen et à long terme de ces maltraitances. Nous est-il devenu moins difficile de nous identifier aux souffrances des victimes ? Pourquoi ? Pensait-on jusque là que ces délits et ces crimes, aussi condamnables qu'ils aient pu être, étaient impossibles à traiter pénalement, et qu'il était préférable de les recouvrir du \emph{manteau de Noé} ? Avait-on peur d'ébranler l'autorité des parents (du père en fait) et la représentation idéale d'une institution sacralisée ? 

 Aujourd'hui nous ne pouvons plus investir les pères comme idéaux : ils ne sont plus pensés comme les relais du pouvoir d'un Dieu, de la Cité, de la République, de l'Empereur, du Roi ou de l'État, relais qu'on jugeait nécessaire naguère de défendre par le déni ou la disqualification de la parole de l'enfant qui dénonçait la confusion des générations, ou encore par le rejet de celui dont la naissance intempestive révélait cette confusion. Aucun des adultes en position éducative n'est plus investi d'une fonction sacrée dont il faudrait protéger l'image, au prix si nécessaire de la vérité et du sacrifice silencieux des victimes. À celui qui accuse un père, un enseignant, un éducateur, un prêtre, etc. de l'avoir agressé sexuellement, nous n'imaginons plus imposer le silence, ni suggérer le « pardon » et « l'oubli ». Nous tenons désormais à ce que chaque délinquant réponde personnellement de tous les actes qu'il a commis sur ses propres enfants comme sur ceux des autres. C'est un progrès %tout à fait 
 incontestable%
% [5]
\footnote{Il nous reste encore à prendre la pleine mesure de l'emprise totalitaire exercée sur leurs enfants par ceux des pères et mères qui, sans pour autant les utiliser comme des partenaires sexuels, les privent néanmoins activement de leur autre parent de naissance. Cela \emph{peut} avoir sur les enfants des effets psychologiques aussi destructeurs que des « abus » ouvertement sexuels.}% 
. 

 Ceci dit on n'a jamais tant parlé d'inceste que depuis que les enfants des couples incestueux ne sont plus pénalisés par le droit. Le Code Napoléon avait, sans le nommer, refusé de criminaliser l'inceste entre \emph{majeurs} consentants, mais il n'en refusait pas moins toute reconnaissance aux enfants qui pouvaient en naître. Depuis une génération, les enfants nés d'un inceste ne sont plus discriminables. S'ils ne peuvent (pour l'instant ?) être reconnus que par un seul de leurs géniteurs, ils peuvent au moins l'être par l'un des deux, mais \emph{surtout} ils peuvent hériter %simultanément
 de \emph{chacun} des deux, ce qui a toujours été \emph{in fine} la forme la plus substantielle de la reconnaissance. Ils ont donc bien deux parents, et ne sont plus hors famille. 

 Cela veut-il dire qu'il n'y a réellement plus d'enfants mal nés ? Cela veut-il dire qu'il n'y a plus d'unions interdites ? Cela veut-il dire que tout désir quel qu'il soit est avouable dès lors que c'est un désir réciproque ? 

 Selon Irène \fsc{THERY} {\emph{nous commençons de mesurer l'incertitude normative qui est la nôtre dès lors que le droit ignore l'inceste \emph{[...]} il convient de se poser la question : notre société est-elle toujours fondée sur le principe de prohibition de l'inceste ? On ne s'est jamais autant soucié de l'inceste, de lever le voile sur la réalisation de l'inceste, de punir l'inceste. Cependant la prohibition de l'inceste est devenue floue dès lors que l'interdit sexuel s'est délié de la question matrimoniale... N'est-ce pas aujourd'hui une autre catégorie, celle du viol, qui devient le cadre à l'intérieur duquel vient s'inscrire l'inceste ? L'inceste n'est-il pas considéré comme un viol sur mineur ?}%
% [6]
\footnote{Irène \fsc{THERY}, idem p. 499-501.}% 
}


