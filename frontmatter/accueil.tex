Après plus de quinze siècles de stabilité la reproduction humaine
est soumise depuis les années soixante du \siecle{20} à de tels bouleversements
qu'il ne s'agit plus d'une évolution, mais d'une révolution. Il
ne semble pas que soient enfin parvenus à leur terme les changements en
cours dans le droit et dans les mœurs. C'est pour cela que ne sont pas
encore pleinement reconnaissables les procédés que les individus emploieront
à l'avenir pour mettre en œuvre leurs désirs à travers le filtre
de ces changements et des institutions qu'ils généreront ou infléchiront.
Dans cette situation instable, déroutante et difficile à penser, pour
comprendre le point où nous en sommes j'ai éprouvé la nécessité de faire
un retour sur le passé.

La situation actuelle de la reproduction des humains ne prend en
effet tout son sens que par ses écarts avec les pratiques des siècles antérieurs.
La famille traditionnelle, celle qui a disparu du droit entre 1960 et
1980, même si elle n'est pas encore tout à fait morte dans les mœurs, est
née d'une synthèse entre les pratiques de l'Empire de Rome, celles des
juifs et celles des chrétiens de l'antiquité. Ces pratiques et les représentations
qui les sous-tendaient étaient elles-mêmes le point d'aboutissement
d'autant d'évolutions particulières.

Depuis la mise en place de ses fondations juridiques sous le règne
de l'empereur Constantin la famille « traditionnelle » a mis de nombreux
siècles à s'imposer, non sans résistances ni déformations multiples par
rapport aux desseins initiaux. Et depuis son apogée des siècles « classiques »
de la fin de l'Ancien Régime elle ne s'efface pas sans résistances, en
même temps que de nouvelles formes d'union et de parentalité font leur
apparition.

C'est ce panorama historique que je me propose de déployer, en
en dégageant les enjeux, en soulignant les articulations et les ruptures, les
conflits et les crises.

L'histoire de la reproduction humaine recouvre partiellement celle
de la prise en charge des personnes faibles, malades, âgées, infirmes ou
démunies, la famille ayant été jusqu'au \siecle{20} la principale institution
d'assistance, sinon la seule. Cet aspect de l'histoire sera donc évoqué
au passage, aussi succinctement que possible. Pour aller plus loin on
pourra se reporter (entre autres) à \emph{L'aide sociale à l'enfance de l'antiquité à
nos jours},%
\tempnote{- Changement du style par rapport à l'original

- Gérer la bibliographie !}
Hervé \fsc{Tigréat}, Pascale \fsc{Planche}, Jean-Luc \fsc{Goascoz},
préface de Pascal \fsc{David}, Tikinagan, 2010.

Avant d'aller plus loin je propose un bref retour sur les sociétés
que l'on appelle primitives, même quand elles sont nos contemporaines,
sociétés dont on peut penser qu'elles fournissent au moins une image
approchée de ce qui se passait avant le monde des sociétés antiques dont
nous sommes issus. De ce point de départ où tout se résume presque à la
famille et aux relations de parenté, en dehors desquelles il n'y a rien, en
dehors desquelles on ne peut pas vivre, nous irons jusqu'à l'époque actuelle
où les individus attendent beaucoup de la collectivité et peu de leur
parentèle, mais où la famille, quelle que soient les formes nouvelles
qu'elle prend, n'en est pas moins l'objet d'un investissement affectif qui
ne faiblit pas.

