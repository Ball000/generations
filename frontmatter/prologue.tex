Les membres des sociétés primitives%
\footnote{Pour rédiger cette présentation, je me suis particulièrement appuyé sur
Quentin \fsc{Meillassoux} : \emph{Anthropologie de l'esclavage}, 1986.}%
\tempnote{Cette note de bas de page ne respecte pas l'original...}
se sentent liées par une 
continuité organique avec leur territoire et avec l'univers matériel dans 
son ensemble, avec les esprits dans (ou de) la nature, avec leurs ancêtres, 
avec le monde des esprits et du ou des dieux. Souvent ils se pensent 
comme les seuls vraiment humains : chacun dans sa langue, ils se 
désignent alors eux-mêmes comme « les humains par excellence », ce
qui implique que les autres, ceux qui leur sont étrangers, ne sont pas humains, 
ou pas vraiment humains, ou pas au même degré qu'eux. Pour eux la 
bonne vie, la seule vie vivable, n'est possible que sur le territoire dont ils 
ont hérité. Partout ailleurs c'est l'inconnu, l'étrange, l'étranger, le
non-humain, l'inhumain.

Le plus souvent ces sociétés ne connaissent aucune forme
d'écriture. Elles se caractérisent d'abord par l'absence d'échanges marchands et 
de moyens de paiement, comme par la faiblesse ou l'absence de leurs 
structures étatiques. Elles vivent d'une économie de subsistance tournée 
vers l'auto consommation. L'accumulation des biens n'est pas pensée par 
elles comme la constitution d'un capital susceptible d'être réinvesti dans 
des opérations économiques nouvelles. Il s'agit plutôt d'acquérir des
objets à haute valeur symbolique (religieuse, esthétique, magique, etc.), ou 
de constituer des réserves destinées à être consommées de manière
festive ou/et ostentatoire. La nature leur donne ses fruits (maternellement). 
C'est la fécondité de leur territoire qui limite la récolte et non le nombre
de bras ou celui des heures de travail disponibles.

Dans ces sociétés la famille est le cadre essentiel, et parfois le
cadre unique des rapports entre individus. Le chef, presque toujours un 
homme, a pour première tâche de veiller à la pérennité de son groupe 
familial. La vie de chacun appartient au groupe, et il n'est pas question 
d'opposer à celui-ci les droits d'un individu particulier ni de mettre
l'ensemble du groupe en danger pour un seul de ses membres. Si trop de 
naissances mettent en danger l'intérêt collectif le don des nouveaux nés 
excédentaires, leur abandon ou leur infanticide sont des pratiques
ordinaires. En cas de disette il arrive que des vieillards se laissent mourir 
pour que les jeunes survivent.

La parenté assigne à chacun une fonction précise : des obligations 
mais aussi des droits sur les ressources du groupe. Aucun membre de la 
famille n'est exclu des redistributions, mais les parts peuvent être très 
inégalement distribuées, sans tenir compte de la contribution de chaque 
membre du groupe au volume des biens à répartir, ni de ses besoins
réels, mais plutôt de son rang et de sa place symbolique. Il est fréquent 
qu'à ce compte les femmes soient mal loties, mais ce n'est pas
systématique. La règle de base est que les adultes travaillent pour nourrir les plus 
jeunes et les plus vieux. Les plus jeunes reçoivent plus qu'ils ne donnent, 
jusqu'à ce qu'ils soient à leur tour capables de nourrir tous ceux qui les 
ont nourris. Les plus vieux sont directement ou indirectement nourris 
par ceux qu'ils ont élevés : chacun investit dans une descendance pour 
préparer ses vieux jours. Quand tout se passe normalement c'est au fil 
d'une vie entière que les tâches et les droits s'équilibrent pour chaque
individu.

Dans un tel système aucun garçon (et a fortiori aucune fille) ne 
possède rien en propre : ni terres, ni troupeaux, etc. Si son groupe familial refuse de 
lui procurer une femme, ou de lui donner les moyens d'en acquérir une, 
il reste bloqué dans un statut de dépendance (juvénile). Condamné à
travailler toute sa vie pour les enfants des autres, il n'accèdera jamais au
statut avantageux et respecté de ceux qui ont de grands enfants productifs.

Dans ces sociétés il n'y a pas de sens à faire une place à un
étranger : à quel titre, au nom de quoi ? Et quel rôle lui donner ? Comment 
l'accueillir sans déséquilibrer le réseau compliqué et tendu des échanges 
et des obligations réciproques ? Pour un guerrier l'étranger ou l'ennemi qu'il a
capturé vivant est une preuve de sa valeur, mais il ne peut être un moyen 
d'entretenir et d'accroître sa puissance, un moyen de production de
richesses. Il n'est bon qu'à être rapidement consommé d'une façon
ostentatoire : il n'est bon qu'à être sacrifié : accepter de ses proches une
rançon serait déjà entrer dans le monde marchand, ce monde où une vie humaine a un 
coût et peut se monnayer, ce qui ne fait pas partie de leurs représentations.

Par contre s'il y a une place vacante dans une famille, celle-ci peut 
adopter un étranger ou une étrangère pour occuper cette place, afin que 
la vie continue, afin que les prestations masculines et féminines
continuent d'être procurées, afin que les enfants continuent de naître, que les 
vieillards ne soient pas à l'abandon, que les ancêtres continuent d'être 
honorés, et que le monde continue sa course, etc. S'il n'y a pas assez 
d'épouses pour tous les garçons, on peut enlever des filles dans un autre 
groupe, ou leur en acheter. Un ennemi prisonnier peut d'autant plus 
facilement remplacer un mari ou un fils mort, que c'est ordinairement à ses 
voisins, à ceux que l'on pourrait épouser, qu'on fait la guerre.

En cas de conflit, de délit ou de crime, la mise au ban du groupe 
est d'autant plus fréquemment choisie qu'elle présente sur la mise à mort 
l'avantage d'éviter la souillure du territoire familial par un meurtre, ainsi 
que le ressentiment des ancêtres ou des dieux contre le ou les exécuteurs 
éventuels. Celui qui est condamné à l'exil est comme mort pour son 
groupe d'origine. S'il tombe aux mains d'un autre groupe, s'il est asservi 
(et a fortiori s'il leur était vendu) sa famille ne cherchera ni à le racheter 
ni à le délivrer. Ainsi le livre de la Genèse raconte comment Joseph,
benjamin de Jacob, a été vendu par ses frères parce qu'ils étaient jaloux de 
voir qu'il était le préféré de son père. Leur première intention était de le 
tuer, mais comme une caravane de marchands passait par là cela leur a 
évité d'avoir à assumer la culpabilité de sa mort, et par dessus le marché 
la vente leur a rapporté de l'argent (l'Asie Mineure où cela se passait était 
déjà en partie entrée dans le monde des marchands).

Si un individu banni est tué ses parents ne chercheront pas à le 
venger. Le jour où il mourra, les rites et sacrifices funéraires nécessaires 
au repos de son esprit ne seront pas exécutés dans les règles. Il ne pourra 
pas rejoindre le monde de ses ancêtres et il ne sera pas rituellement
nourri par les vivants. Son souvenir ne sera pas honoré. Cela l'exclura de son 
clan une deuxième fois. Aux yeux des intéressés l'errance et l'exil
valent-ils toujours mieux qu'une mort immédiate, mais au milieu des siens ?

Aucune société primitive ne faisait le poids militairement face 
aux sociétés plus développées, telles que les cités antiques. Elles leur 
ont servi de réserves de futurs esclaves à capturer périodiquement (dès 
qu'ils avaient surmonté la crise démographique et morale créée par le précédent prélèvement). Ce 
système a perduré longtemps : au \siecle{19} on l'observait encore
en bien des endroits écartés des centres où se concentraient les hommes 
et où s'inventait l'avenir.

%NB. Pour rédiger cette présentation je me suis particulièrement appuyé sur Quentin \fsc{Meillassoux} :
%\emph{Anthropologie de l'esclavage}, 1986.
%    %Passé en note au début du chapitre


